\documentclass{article}
\usepackage{amsmath}
\usepackage{tcolorbox}
\usepackage[margin=0.5in]{geometry} 
\usepackage{amsmath,amsthm,amssymb,amsfonts, fancyhdr, color, comment, graphicx, environ}
\usepackage{float}
\usepackage{xcolor}
\usepackage{mdframed}
\usepackage[shortlabels]{enumitem}
\usepackage{indentfirst}
\usepackage{mathrsfs}
\usepackage{hyperref}
\graphicspath{./}
\makeatletter
\newcommand*{\rom}[1]{\expandafter\@slowromancap\romannumeral #1@}
\makeatother
% Change enumerate labels to (a), (b), (c), ...
% Define a new environment for problems
\newcounter{problemCounter}
\newtcolorbox{problem}[2][]{colback=white, colframe=black, boxrule=0.5mm, arc=4mm, auto outer arc, title={\ifstrempty{#1}{Problem \stepcounter{problemCounter}\theproblemCounter}{#1}}}

% \renewcommand{\labelenumi}{\alph{enumi})}
\def\zz{{\mathbb Z}}
\def\rr{{\mathbb R}}
\def\qq{{\mathbb Q}}
\def\cc{{\mathbb C}}
\def\nn{{\mathbb N}}
\def\ss{{\mathbb S}}
\def\ff{{\mathbb F}}

\newtheorem{theorem}{Theorem}[section]
\newtheorem{corollary}{Corollary}[theorem]
\newtheorem{lemma}[theorem]{Lemma}
\newtcolorbox{proposition}[1][]{colback=white, colframe=blue, boxrule=0.5mm, arc=4mm, auto outer arc, title={Proposition #1}}
\newtcolorbox{definition}[1][]{colback=white, colframe=violet, boxrule=0.5mm, arc=4mm, auto outer arc, title={Definition #1}}
\newcommand{\Zmod}[1]{\zz/#1\zz}
\newcommand{\partFrac}[2]{\frac{\partial #1}{\partial #2}}

\newcommand\Mydiv[2]{%
$\strut#1$\kern.25em\smash{\raise.3ex\hbox{$\big)$}}$\mkern-8mu
        \overline{\enspace\strut#2}$}

\begin{document}

\begin{center}
    Math 540
    \hfill Homework 10
    \hfill \textit{Stephen Cornelius}
\end{center}
% \textbf{Remarks:} \\
% \begin{enumerate}[A)]
%     \item Definition is just a definition, there is no need to jjustify or explain it.
%     \item Answers to questions with proofs should be written, as much as you can, in the following format: \\
%     \begin{enumerate}[i)]
%         \item Statement
%         \item Main points that will appear in your proof
%         \item The actual proof
%     \end{enumerate}
%     Answers to questions with computations should be written, as much as possible, in the following format:
%     \begin{enumerate}[i)]
%         \item Statement and Result
%         \item Main points that will appear in your computation.
%         \item The actual computation
%     \end{enumerate}
% \end{enumerate}



% % Start of problems

\begin{problem} \\
    \textit{Minimal Polynomials and Diagonalizability.} Let $T : V \to V$ be a linear transformation on a finite imensional vcetor space $V$ over a field $\ff$. \\ 
    \begin{enumerate}[(a)]
        \item Define the minimal polynomial of $T$, denoted $m_T(x)$.
        \item Prove the following:
        \begin{theorem}
            $T$ is diagonalizable if and only if $m_T(x)$ is a product of different linear factors of the form $x - \lambda$, for some  $\lambda \in \ff$.
        \end{theorem}
    \end{enumerate}
\end{problem}



\begin{enumerate}[(a)]
    \item The minimal polynomial of $T$, denoted $m_T(x)$, is the monic polynomial of smallest degree with coefficients in $\ff$ such that $m_T(T) = 0$.

    \item \begin{proof}
        ($\Rightarrow$) Suppose $T$ is diagonalizable. Then there exists a basis $\{v_1, \ldots, v_n\}$ of $V$ consisting of eigenvectors of $T$ with corresponding eigenvalues $\lambda_1, \ldots, \lambda_k$ (distinct). Let $p(x) = (x - \lambda_1) \cdots (x - \lambda_k)$. For any $v_i$, we have $p(T)(v_i) = (T - \lambda_1 I) \cdots (T - \lambda_k I)(v_i) = 0$ since $v_i$ is an eigenvector. Thus $p(T) = 0$, so $m_T(x)$ divides $p(x)$. Since $m_T(x)$ is monic and has no repeated factors, $m_T(x) = (x - \lambda_1) \cdots (x - \lambda_m)$ for distinct $\lambda_i \in \ff$.

        ($\Leftarrow$) Suppose $m_T(x) = (x - \lambda_1) \cdots (x - \lambda_k)$ with distinct $\lambda_i$. Then $V = \ker(T - \lambda_1 I) \oplus \cdots \oplus \ker(T - \lambda_k I)$ by the Primary Decomposition Theorem. Each kernel consists of eigenvectors, so $V$ has a basis of eigenvectors of $T$. Thus $T$ is diagonalizable.
    \end{proof}

    
\end{enumerate}



\begin{problem} \\
    \textit{Simultaneously diagonalizable operators.} Let $S,T$ be two operators defined on a finite-dimensional vector space $V$.
    \begin{enumerate}[(a)]
        \item Define when we say that $S$ and $T$ are simultaneously diagonalizable.
        \item Suppose $S$ and $T$ are both diagonalizable. Show that TFAE:
        \begin{enumerate}[1.]
            \item $S$ and $T$ commute, i.e., $S \circ T = T \circ S$.
            \item $S$ and $T$ are simultaneously diagonalizable.
        \end{enumerate}
        \item Can you extend the result above to arbitrary collection of diagonalizable operators on $V$?
    \end{enumerate}
\end{problem}




\begin{enumerate}[(a)]
    \item Two operators $S$ and $T$ on a finite-dimensional vector space $V$ are \textit{simultaneously diagonalizable} if there exists a single basis $\{v_1, \ldots, v_n\}$ of $V$ such that both $S$ and $T$ are diagonal with respect to this basis. Equivalently, $S$ and $T$ share the same eigenbasis.
    \item \begin{proof}
        ($1 \Rightarrow 2$) Suppose $S \circ T = T \circ S$. Since $S$ is diagonalizable, there exists a basis $\{v_1, \ldots, v_n\}$ of eigenvectors of $S$ with eigenvalues $\lambda_1, \ldots, \lambda_k$. Let $V_i = \ker(S - \lambda_i I)$. Since $S$ and $T$ commute, $T(V_i) \subseteq V_i$ for each $i$. Thus $T$ restricted to each $V_i$ is diagonalizable. Therefore, each $V_i$ has a basis of common eigenvectors of both $S$ and $T$. The union of these bases is a basis of $V$ diagonalizing both $S$ and $T$.

        ($2 \Rightarrow 1$) Suppose $S$ and $T$ are simultaneously diagonalizable. Then there exists a basis where both are diagonal. Diagonal matrices commute, so $S \circ T = T \circ S$.
    \end{proof}

    \item Yes. The result extends to any finite collection of diagonalizable operators. If $T_1, \ldots, T_m$ are diagonalizable operators on $V$, then they are simultaneously diagonalizable if and only if they pairwise commute, i.e., $T_i \circ T_j = T_j \circ T_i$ for all $i, j$.
\end{enumerate}



\end{document}
