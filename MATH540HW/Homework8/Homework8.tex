\documentclass{article}
\usepackage{amsmath}
\usepackage{tcolorbox}
\usepackage[margin=0.5in]{geometry} 
\usepackage{amsmath,amsthm,amssymb,amsfonts, fancyhdr, color, comment, graphicx, environ}
\usepackage{float}
\usepackage{xcolor}
\usepackage{mdframed}
\usepackage[shortlabels]{enumitem}
\usepackage{indentfirst}
\usepackage{mathrsfs}
\usepackage{hyperref}
\graphicspath{./}
\makeatletter
\newcommand*{\rom}[1]{\expandafter\@slowromancap\romannumeral #1@}
\makeatother
% Change enumerate labels to (a), (b), (c), ...
% Define a new environment for problems
\newcounter{problemCounter}
\newtcolorbox{problem}[2][]{colback=white, colframe=black, boxrule=0.5mm, arc=4mm, auto outer arc, title={\ifstrempty{#1}{Problem \stepcounter{problemCounter}\theproblemCounter}{#1}}}

% \renewcommand{\labelenumi}{\alph{enumi})}
\def\zz{{\mathbb Z}}
\def\rr{{\mathbb R}}
\def\qq{{\mathbb Q}}
\def\cc{{\mathbb C}}
\def\nn{{\mathbb N}}
\def\ss{{\mathbb S}}

\newtheorem{theorem}{Theorem}[section]
\newtheorem{corollary}{Corollary}[theorem]
\newtheorem{lemma}[theorem]{Lemma}
\newtcolorbox{proposition}[1][]{colback=white, colframe=blue, boxrule=0.5mm, arc=4mm, auto outer arc, title={Proposition #1}}
\newtcolorbox{definition}[1][]{colback=white, colframe=violet, boxrule=0.5mm, arc=4mm, auto outer arc, title={Definition #1}}
\newcommand{\Zmod}[1]{\zz/#1\zz}
\newcommand{\partFrac}[2]{\frac{\partial #1}{\partial #2}}

\newcommand\Mydiv[2]{%
$\strut#1$\kern.25em\smash{\raise.3ex\hbox{$\big)$}}$\mkern-8mu
        \overline{\enspace\strut#2}$}

\begin{document}

\begin{center}
    Math 540
    \hfill Homework 8
    \hfill \textit{Stephen Cornelius}
\end{center}
% \textbf{Remarks:} \\
% \begin{enumerate}[A)]
%     \item Definition is just a definition, there is no need to jjustify or explain it.
%     \item Answers to questions with proofs should be written, as much as you can, in the following format: \\
%     \begin{enumerate}[i)]
%         \item Statement
%         \item Main points that will appear in your proof
%         \item The actual proof
%     \end{enumerate}
%     Answers to questions with computations should be written, as much as possible, in the following format:
%     \begin{enumerate}[i)]
%         \item Statement and Result
%         \item Main points that will appear in your computation.
%         \item The actual computation
%     \end{enumerate}
% \end{enumerate}



% % Start of problems

\begin{problem} \\ 
    \textit{The ring of integers is a PID.} \\
    \begin{enumerate}[(a)]
        \item Define when a ring is called a principal ideal domain (PID).
        \item Prove that the ring of integers $\zz$ i a principal ideal domain. That is, show that every idear $I$ of $\zz$ is generated by a single element, i.e., $I = (d) = \{ kd; k \in \zz \}$ for some $d \in \zz$.
        \item Take two integers $m,n \in \zz$. The ideal generate by them is defined to be 
        \[
            (m,n) = \{ am + bn; a,b \in \zz \}.
        \]
        Find the integer $d$ such that $(d) = (6,15)$.
    \end{enumerate}
\end{problem}


\begin{enumerate}[(a)]
    \item A principal ideal domain (PID) is an integral domain in which every ideal is principal, meaning that it can be generated by a single element. In other words, for any ideal $I$ in a PID, there exists an element $d$ in the ring such that $I = (d) = \{ rd; r \in R \}$, where $R$ is the ring.
    \item \begin{proof}
        To prove that the ring of integers $\zz$ is a principal ideal domain, we need to show that every ideal $I$ in $\zz$ can be generated by a single integer. Let $I$ be a non-zero ideal in $\zz$. Since $I$ is non-empty, it contains some non-zero integers. Let $d$ be the smallest positive integer in $I$ (such a $d$ exists by the well-ordering principle). We will show that $I = (d)$. \\
        First, we show that $(d) \subseteq I$. By definition of $(d)$, any element in $(d)$ can be written as $kd$ for some integer $k$. Since $d \in I$ and $I$ is an ideal, it follows that $kd \in I$ for all integers $k$. Thus, every element of $(d)$ is in $I$, so $(d) \subseteq I$. \\
        Next, we show that $I \subseteq (d)$. Let $a$ be any element in $I$. By the division algorithm, we can write $a$ as:
        \[
            a = qd + r,
        \]
        where $q$ is an integer and $0 \leq r < d$. Since $a \in I$ and $qd \in I$ (because $d \in I$ and $I$ is an ideal), it follows that $r = a - qd \in I$. However, since $d$ is the smallest positive integer in $I$, the only way for $r$ to be in $I$ and satisfy $0 \leq r < d$ is if $r = 0$. Therefore, we have:
        \[
            a = qd.
        \]
        This shows that every element $a$ in $I$ can be expressed as a multiple of $d$, so $a \in (d)$. Thus, we have $I \subseteq (d)$. \\
        Combining both inclusions, we conclude that $I = (d)$. Therefore, every ideal in $\zz$ is principal, and hence $\zz$ is a principal ideal domain.
    \end{proof}
    \item To find the integer $d$ such that $(d) = (6,15)$, we need to determine the greatest common divisor (gcd) of 6 and 15. The gcd of 6 and 15 is 3, since 3 is the largest integer that divides both 6 and 15 without leaving a remainder. Therefore, we have that $(d) = (6,15)$ with $d = 3$.
\end{enumerate}




\begin{problem} \\ 
    \textit{Factoring real polynomials in over $\cc$.} \\
    \begin{enumerate}[(a)]
        \item Define what is a linear polynomial. \\
        Let $f(x) = x^2 + bx + c \in \rr[x]$.
        \item Factor $f(x)$ into product of linear polynomials over $\cc$, i.e., into linear factors from $\cc[x]$. Hint: Try the quadratic formula.
        \item Factorize the polynomial $f(x) = x^3 + 1$ into product of linear factors (polynomials) over $\cc$.
    \end{enumerate}
\end{problem}


\begin{enumerate}[(a)]
    \item A linear polynomial is a polynomial of degree one, which can be expressed in the form $f(x) = ax + b$, where $a$ and $b$ are constants and $a \neq 0$.
    \item To factor the polynomial $f(x) = x^2 + bx + c$ into linear factors over $\cc$, we can use the quadratic formula to find its roots. The roots of the polynomial are given by:
    \[
        x = \frac{-b \pm \sqrt{b^2 - 4c}}{2}.
    \]
    Let the roots be denoted as $\alpha_1$ and $\alpha_2$. Then, we can express the polynomial as:
    \[
        f(x) = (x - \alpha_1)(x - \alpha_2).
    \]
    \item To factor the polynomial $f(x) = x^3 + 1$ into linear factors over $\cc$, we can use the fact that $x^3 + 1$ can be factored using the sum of cubes formula:
    \[
        x^3 + 1 = (x + 1)(x^2 - x + 1).
    \]
    Next, we need to factor the quadratic $x^2 - x + 1$ over $\cc$. Using the quadratic formula, we find the roots:
    \[
        x = \frac{1 \pm \sqrt{(-1)^2 - 4 \cdot 1 \cdot 1}}{2 \cdot 1} = \frac{1 \pm \sqrt{-3}}{2} = \frac{1 \pm i\sqrt{3}}{2}.
    \]
    Let the roots be denoted as $\alpha_1 = \frac{1 + i\sqrt{3}}{2}$ and $\alpha_2 = \frac{1 - i\sqrt{3}}{2}$. Thus, we can express the quadratic as:
    \[
        x^2 - x + 1 = (x - \alpha_1)(x - \alpha_2).
    \]
    Therefore, the complete factorization of $f(x) = x^3 + 1$ into linear factors over $\cc$ is:
    \[
        f(x) = (x + 1)(x - \alpha_1)(x - \alpha_2) = (x + 1)\left(x - \frac{1 + i\sqrt{3}}{2}\right)\left(x - \frac{1 - i\sqrt{3}}{2}\right).
    \]
\end{enumerate}



\begin{problem} \\
    \textit{Factoring in $\rr[x]$.} \\
    \begin{enumerate}[(a)]
        \item Let $\mathbb{F}$ be a field. Define when we say that a polynomial in $\mathbb{F}[x]$ is called \underline{irreducible}.
        \item Let $f(x) \in \rr[x]$
        \begin{enumerate}[1.]
            \item if $f(x) \in \rr[x]$, and $\alpha \in \cc$ is a root of $f$, then so is $\overline{\alpha}$.
            \item Show that every non-constant irreducible polynomial in $\rr[x]$ is of degree 1 or 2.
        \end{enumerate}

        \item Factor the following polynomials from $\rr[x]$ into product of irreducibles over $\rr$:
        \begin{enumerate}[1.]
            \item $x^3 - 1$
            \item $x^4 + 1$
            \item $x^6 - 1$
        \end{enumerate}
    \end{enumerate}
\end{problem}


\begin{enumerate}[(a)]
    \item A polynomial $f(x) \in \mathbb{F}[x]$ is called irreducible over the field $\mathbb{F}$ if it cannot be factored into the product of two non-constant polynomials in $\mathbb{F}[x]$. In other words, if $f(x) = g(x)h(x)$ for some polynomials $g(x), h(x) \in \mathbb{F}[x]$, then either $g(x)$ or $h(x)$ must be a constant polynomial.
    \item \begin{enumerate}[1.]
        \item \begin{proof}
            Let $f(x) \in \rr[x]$ and suppose $\alpha \in \cc$ is a root of $f(x)$. Since the coefficients of $f(x)$ are real numbers, we can express $f(x)$ as:
            \[
                f(x) = a_n x^n + a_{n-1} x^{n-1} + \ldots + a_1 x + a_0,
            \]
            where $a_i \in \rr$ for all $i$. Taking the complex conjugate of both sides, we have:
            \[
                \overline{f(x)} = \overline{a_n} \overline{x}^n + \overline{a_{n-1}} \overline{x}^{n-1} + \ldots + \overline{a_1} \overline{x} + \overline{a_0}.
            \]
            Since the coefficients are real, we have $\overline{a_i} = a_i$ for all $i$. Thus, we can rewrite this as:
            \[
                \overline{f(x)} = a_n \overline{x}^n + a_{n-1} \overline{x}^{n-1} + \ldots + a_1 \overline{x} + a_0.
            \]
            Now, if $\alpha$ is a root of $f(x)$, then $f(\alpha) = 0$. Taking the complex conjugate, we get:
            \[
                \overline{f(\alpha)} = f(\overline{\alpha}) = 0.
            \]
            Therefore, $\overline{\alpha}$ is also a root of $f(x)$.
        \end{proof}
        \item \begin{proof}
            Let $f(x) \in \rr[x]$ be a non-constant irreducible polynomial. We will show that the degree of $f(x)$ must be either 1 or 2. Suppose, for the sake of contradiction, that the degree of $f(x)$ is greater than 2. Then, by the Fundamental Theorem of Algebra, $f(x)$ has at least one complex root $\alpha$. By part (1), its complex conjugate $\overline{\alpha}$ is also a root of $f(x)$. Thus, we can factor $f(x)$ as:
            \[
                f(x) = (x - \alpha)(x - \overline{\alpha})g(x),
            \]
            where $g(x) \in \cc[x]$ is a polynomial of degree at least 1. The product $(x - \alpha)(x - \overline{\alpha})$ is a quadratic polynomial with real coefficients, since:
            \[
                (x - \alpha)(x - \overline{\alpha}) = x^2 - (\alpha + \overline{\alpha})x + \alpha\overline{\alpha} = x^2 - 2\text{Re}(\alpha)x + |\alpha|^2.
            \]
            Therefore, we can express $f(x)$ as:
            \[
                f(x) = q(x)g(x),
            \]
            where $q(x) = (x - \alpha)(x - \overline{\alpha})$ is a quadratic polynomial with real coefficients. Since $g(x)$ has degree at least 1, this means that $f(x)$ can be factored into the product of two non-constant polynomials in $\rr[x]$, contradicting the assumption that $f(x)$ is irreducible. Therefore, the degree of $f(x)$ must be either 1 or 2.
        \end{proof}
        \item \begin{enumerate}[1.]
            \item $x^3 - 1 = (x - 1)(x^2 + x + 1)$, where $x - 1$ is linear and $x^2 + x + 1$ is irreducible over $\rr$.
            \item $x^4 + 1 = (x^2 + \sqrt{2}x + 1)(x^2 - \sqrt{2}x + 1)$, where both quadratics are irreducible over $\rr$.
            \item $x^6 - 1 = (x - 1)(x + 1)(x^2 - x + 1)(x^2 + x + 1)$, where $x - 1$ and $x + 1$ are linear, and both quadratics are irreducible over $\rr$.
        \end{enumerate}
    \end{enumerate}
\end{enumerate}




\begin{problem} \\ 
    \textit{Irreducibles need not be primes.} \\ 
    \begin{enumerate}[(a)]
        \item Recall that in an integral domain $\rr$, a nonzero non-unit $q \in \rr$ is prime if whenever $q | ab$ then either $q | a$ or $q | b$. A nonzero non-unit $q \in R$ is irreducible if whenever $q = ab$ then either $a$ or $b$ is a unit. Show that in an integral domain, every prime is irreducible.
        \item Consider the subring $S$ of $\cc$,
        \[
            S = \{a + b \sqrt{-3} | a,b \in \zz \}.
        \]
        Show that in this integral domain, $2 \in S$ is irreducible but not prime.

        \item Give an example of a commutative unital ring $R$, and prime element in $R$ which is not irreducible.
    \end{enumerate}
\end{problem}


\begin{enumerate}[(a)]
    \item \begin{proof}
        Let $\rr$ be an integral domain, and let $q \in \rr$ be a prime element. We need to show that $q$ is irreducible. Suppose, for the sake of contradiction, that $q$ is not irreducible. Then, we can write $q = ab$ for some non-unit elements $a, b \in \rr$. Since $q$ is prime, it follows that if $q | ab$, then either $q | a$ or $q | b$. However, since $q = ab$, we have that $q | ab$ trivially. Therefore, either $q | a$ or $q | b$. Without loss of generality, assume that $q | a$. This means there exists some element $c \in \rr$ such that $a = qc$. Substituting this back into the equation for $q$, we have:
        \[
            q = ab = (qc)b = q(cb).
        \]
        Since $\rr$ is an integral domain and $q \neq 0$, we can cancel $q$ from both sides to obtain:
        \[
            1 = cb.
        \]
        This implies that $b$ is a unit in $\rr$, which contradicts our assumption that both $a$ and $b$ are non-units. Therefore, our initial assumption that $q$ is not irreducible must be false. Hence, every prime element in an integral domain is irreducible.
    \end{proof}

    \item \begin{proof}
        Let $S = \{a + b \sqrt{-3} | a,b \in \zz \}$ be the subring of $\cc$. We will show that $2 \in S$ is irreducible but not prime. \\
        First, we show that $2$ is irreducible. Suppose that $2 = ab$ for some $a, b \in S$. We can express $a$ and $b$ as:
        \[
            a = x + y\sqrt{-3}, \quad b = u + v\sqrt{-3},
        \]
        where $x, y, u, v \in \zz$. Then, we have:
        \[
            2 = (x + y\sqrt{-3})(u + v\sqrt{-3}) = (xu - 3yv) + (xv + yu)\sqrt{-3}.
        \]
        Equating the real and imaginary parts, we get the system of equations:
        \[
            xu - 3yv = 2,
        \]
        \[
            xv + yu = 0.
        \]
        From the second equation, we can express $yu$ in terms of $xv$:
        \[
            yu = -xv.
        \]
        Substituting this into the first equation, we have:
        \[
            xu + 3xv^2/y = 2.
        \]
        Since $x, y, u, v$ are integers, it follows that either $x$ or $y$ must be a unit in $\zz$, which means either $a$ or $b$ is a unit in $S$. Therefore, $2$ is irreducible in $S$. \\
        Next, we show that $2$ is not prime. Consider the elements $1 + \sqrt{-3}$ and $1 - \sqrt{-3}$ in $S$. We have:
        \[
            (1 + \sqrt{-3})(1 - \sqrt{-3}) = 1 - (-3) = 4.
        \]
        Since $4 = 2 \cdot 2$, it follows that $2 | 4$. However, neither $2 | (1 + \sqrt{-3})$ nor $2 | (1 - \sqrt{-3})$, since dividing either by $2$ does not yield an element in $S$. Therefore, $2$ is not prime in $S$. \\
    \end{proof}
    \item Consider the ring $R = \zz/6\zz$. In this ring, the element $2 + 6\zz$ is prime because if $2 + 6\zz | ab + 6\zz$, then either $2 + 6\zz | a + 6\zz$ or $2 + 6\zz | b + 6\zz$. However, $2 + 6\zz$ is not irreducible because we can write:
    \[
        2 + 6\zz = (2 + 6\zz)(1 + 6\zz),
    \]
    where neither $2 + 6\zz$ nor $1 + 6\zz$ is a unit in $R$. Thus, we have found a prime element in $R$ that is not irreducible.
\end{enumerate}



\begin{problem} \\ 
    \textit{Eigenvalues over $\rr$ and $\cc$.} \\
    For each of the folowing linear transformations, find all eigenvalues. For each eigenvalue, find the corresponding eigenspace. In each case, do the problem first with $\mathbb{F} = \rr$ and then again with $\mathbb{F} = \cc$.
    \begin{enumerate}[(a)]
        \item $T: \mathbb{F}^3 \to \mathbb{F}^3$,
        \[
            T(x_1,x_2,x_3) = (x_1 + x_2,\, x_2 + x_3,\, x_1 + x_3).
        \]
        \item $T: \mathbb{F}^2 \to \mathbb{F}^2$,
        \[
            T(x_1,x_2) = (x_1 + x_2, \, x_1 - x_2).
        \]
        \item $T: \mathbb{F}^4 \to \mathbb{F}^4$,
        \[
            T(x_1,x_2,x_3,x_4) = (x_2,\, 2x_3,\, 3x_4,\, 0).
        \]
    \end{enumerate}
\end{problem}

\begin{enumerate}[(a)]
\item
Matrix of \(T\) (standard basis): \(A=\begin{pmatrix}1&1&0\\0&1&1\\1&0&1\end{pmatrix}\).
Characteristic polynomial:
\[
\det(A-\lambda I)=(1-\lambda)^3+1=0.
\]
Over \(\mathbb{R}\) the only real root is \(\lambda=2\). Solving \((A-2I)v=0\) gives
\[
v=(1,1,1)^T,
\]
so eigenspace for \(\lambda=2\) is \(\operatorname{span}\{(1,1,1)\}\).

Over \(\mathbb{C}\) the other two eigenvalues are
\(\lambda=\tfrac12\pm \tfrac{i\sqrt3}{2}\). Let \(r=1-\lambda\) (so \(r^3=-1\)). For each such \(\lambda\) an eigenvector is
\[
v=(1,-r,r^2)^T,
\]
and the conjugate eigenvalue has the conjugate eigenvector. Thus the three eigenspaces are
\(\operatorname{span}\{(1,1,1)\}\) and \(\operatorname{span}\{(1,-r,r^2)\},\operatorname{span}\{(1,-\overline r,\overline r^{2})\}\).

\item
Matrix \(A=\begin{pmatrix}1&1\\1&-1\end{pmatrix}\).
Characteristic polynomial: \(\lambda^2-2=0\), so \(\lambda=\pm\sqrt2\) (real).

For a general eigenvalue \(\lambda\) an eigenvector satisfies \((1-\lambda)x_1+x_2=0\), so we may take
\[
v=(1,\;\lambda-1)^T.
\]
Thus for \(\lambda=\sqrt2\) an eigenvector is \((1,\sqrt2-1)^T\), and for \(\lambda=-\sqrt2\) an eigenvector is \((1,-\sqrt2-1)^T\). The same answers hold over \(\mathbb{C}\).

\item
Matrix of \(T\) is
\(\begin{pmatrix}0&1&0&0\\0&0&2&0\\0&0&0&3\\0&0&0&0\end{pmatrix}\).
Characteristic polynomial \(\lambda^4=0\), so the only eigenvalue (over \(\mathbb{R}\) and \(\mathbb{C}\)) is \(\lambda=0\).
Solve \(T(v)=0\): \(x_2=0,\;x_3=0,\;x_4=0\), so the eigenspace is
\(\operatorname{span}\{(1,0,0,0)\}\).
\end{enumerate}



\end{document}
