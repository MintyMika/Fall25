\documentclass{article}
\usepackage{amsmath}
\usepackage{tcolorbox}
\usepackage[margin=0.5in]{geometry} 
\usepackage{amsmath,amsthm,amssymb,amsfonts, fancyhdr, color, comment, graphicx, environ}
\usepackage{float}
\usepackage{xcolor}
\usepackage{mdframed}
\usepackage[shortlabels]{enumitem}
\usepackage{indentfirst}
\usepackage{mathrsfs}
\usepackage{hyperref}
\graphicspath{./}
\makeatletter
\newcommand*{\rom}[1]{\expandafter\@slowromancap\romannumeral #1@}
\makeatother

% Define a new environment for problems
\newcounter{problemCounter}
\newtcolorbox{problem}[2][]{colback=white, colframe=black, boxrule=0.5mm, arc=4mm, auto outer arc, title={\ifstrempty{#1}{Problem \stepcounter{problemCounter}\theproblemCounter}{#1}}}

% \renewcommand{\labelenumi}{\alph{enumi})}
\def\zz{{\mathbb Z}}
\def\rr{{\mathbb R}}
\def\qq{{\mathbb Q}}
\def\cc{{\mathbb C}}
\def\nn{{\mathbb N}}
\def\ss{{\mathbb S}}

\newtheorem{theorem}{Theorem}[section]
\newtheorem{corollary}{Corollary}[theorem]
\newtheorem{lemma}[theorem]{Lemma}
\newtcolorbox{proposition}[1][]{colback=white, colframe=blue, boxrule=0.5mm, arc=4mm, auto outer arc, title={Proposition #1}}
\newtcolorbox{definition}[1][]{colback=white, colframe=violet, boxrule=0.5mm, arc=4mm, auto outer arc, title={Definition #1}}
\newcommand{\Zmod}[1]{\zz/#1\zz}
\newcommand{\partFrac}[2]{\frac{\partial #1}{\partial #2}}

\newcommand\Mydiv[2]{%
$\strut#1$\kern.25em\smash{\raise.3ex\hbox{$\big)$}}$\mkern-8mu
        \overline{\enspace\strut#2}$}

\begin{document}

\begin{center}
    Math 540
    \hfill Homework 1
    \hfill \textit{Stephen Cornelius}
\end{center}
\textbf{Remarks:} \\
\begin{enumerate}
    \item Definition is just a definition, there is no need to jjustify or explain it.
    \item Answers to questions with proofs should be written, as much as you can, in the following format: \\
    \begin{enumerate}
        \item Statement
        \item Main points that will appear in your proof
        \item The actual proof
    \end{enumerate}
    Answers to questions with computations should be written, as much as possible, in the following format:
    \begin{enumerate}
        \item Statement and Result
        \item Main points that will appear in your computation.
        \item The actual computation
    \end{enumerate}
\end{enumerate}
\begin{problem}
    \textit{Vector Spaces}. Suppose $\mathbb{F}$ is a field.
    \begin{enumerate}
        \item Define when we say that a vector space $V$ overa  field $\mathbb{F}$ is \textit{finite dimensional}.
        \item Consider the vector space
        \[
            V = \mathbb{F}[x]
        \]
        of all polynomials with coefficients in $\mathbb{F}$. Show that $V$ is not finite dimensional.
        \item Suppose $X$ is a finite set. Consider the vector space $V$, of all functions from $X$ to $\mathbb{F}$,
        \[
            V = \mathbb{F}(X):= \{\text{all }f: X \to \mathbb{F}; \text{s.t. $f$ is a function}\},
        \]
        with the standard addition and multiplication by scalars from $\mathbb{F}$. Show that $V$ is finite dimensional.
    \end{enumerate}
\end{problem}



\begin{problem}
    \textit{Short exact sequences}. Suppose $U,V,W$ are three vector spaces over $\mathbb{F}$. Consider the following seequence of spaces and linear transformations between them:
    \begin{equation}
        0 \xrightarrow{} U \xrightarrow{\iota} V \xrightarrow{\epsilon} W \xrightarrow{} 0,
    \end{equation}
    where $0 \to U$, are the obvious maps from the zero space into $U$, and from the space $W$ onto the zero space, respectively. \\
    \begin{enumerate}
        \item Define when we say that the sequence (1) is \underline{short exact sequence} (s.e.s.).
        \item Given two subspaces $U,V < V$, such that $V = U \oplus W$, Show that there is a natural s.e.s. associated with the spaces of functions $U = \mathbb{F}(U), V = \mathbb{F}(V)$ and $W = \mathbb{F}(Y \backslash X)$, where $Y \backslash X$ denotes set-minus, i.e., the set of elements which are in $Y$ and are not in $X$.
    \end{enumerate}
\end{problem}


\begin{problem}
    \textit{Dimension}. Denote by $\operatorname{Vect}_{}\mathbb{F}^{fd}$ the collection of finite-dimensional vector spaces over $\mathbb{F}$, with linear transformations between them.
    \begin{enumerate}
        \item State the fact about uniqueness and existence of unique dimenstion function
        \[
            \dim: \operatorname{Vect}_\mathbb{F}^{fd} \to \mathbb{N},
        \]
        that satisfies certain desired properties. \\
        \textbf{Def.} For $V$ finite dimensional, the integer $\dim(V)$ is called the \underline{dimension} of $V$.
        \item Show that $\dim(M_n(\mathbb{F})) = n^2$.
        \item Suppose $1 + 1 \neq 0$ in $\mathbb{F}$. Consider the spaces $U = A_n(\mathbb{F})$, $V = M_n(\mathbb{F})$, $W = S_n(\mathbb{F})$, of anti-symmetric matrices ($A^T = -A$), all matrices, and symmetric matrices (sastisfy $A^T = A$), respectively.
        \begin{enumerate}
            \item Show that, they form in a natural why a s.e.s.
            \item Deduce that $\dim(A_n(\mathbb{F})) = \frac{n(n-1)}{2}$ and $\dim(S_n(\mathbb{F})) = \frac{n(n+1)}{2}$.
        \end{enumerate}
    \end{enumerate}
\end{problem}

\end{document}