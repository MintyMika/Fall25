\documentclass{article}
\usepackage{amsmath}
\usepackage{tcolorbox}
\usepackage[margin=0.5in]{geometry} 
\usepackage{amsmath,amsthm,amssymb,amsfonts, fancyhdr, color, comment, graphicx, environ}
\usepackage{float}
\usepackage{xcolor}
\usepackage{mdframed}
\usepackage[shortlabels]{enumitem}
\usepackage{indentfirst}
\usepackage{mathrsfs}
\usepackage{hyperref}
\graphicspath{./}
\makeatletter
\newcommand*{\rom}[1]{\expandafter\@slowromancap\romannumeral #1@}
\makeatother
% Change enumerate labels to (a), (b), (c), ...
% Define a new environment for problems
\newcounter{problemCounter}
\newtcolorbox{problem}[2][]{colback=white, colframe=black, boxrule=0.5mm, arc=4mm, auto outer arc, title={\ifstrempty{#1}{Problem \stepcounter{problemCounter}\theproblemCounter}{#1}}}

% \renewcommand{\labelenumi}{\alph{enumi})}
\def\zz{{\mathbb Z}}
\def\rr{{\mathbb R}}
\def\qq{{\mathbb Q}}
\def\cc{{\mathbb C}}
\def\nn{{\mathbb N}}
\def\ss{{\mathbb S}}

\newtheorem{theorem}{Theorem}[section]
\newtheorem{corollary}{Corollary}[theorem]
\newtheorem{lemma}[theorem]{Lemma}
\newtcolorbox{proposition}[1][]{colback=white, colframe=blue, boxrule=0.5mm, arc=4mm, auto outer arc, title={Proposition #1}}
\newtcolorbox{definition}[1][]{colback=white, colframe=violet, boxrule=0.5mm, arc=4mm, auto outer arc, title={Definition #1}}
\newcommand{\Zmod}[1]{\zz/#1\zz}
\newcommand{\partFrac}[2]{\frac{\partial #1}{\partial #2}}

\newcommand\Mydiv[2]{%
$\strut#1$\kern.25em\smash{\raise.3ex\hbox{$\big)$}}$\mkern-8mu
        \overline{\enspace\strut#2}$}

\begin{document}

\begin{center}
    Math 540
    \hfill Homework 1
    \hfill \textit{Stephen Cornelius}
\end{center}
% \textbf{Remarks:} \\
% \begin{enumerate}[A)]
%     \item Definition is just a definition, there is no need to jjustify or explain it.
%     \item Answers to questions with proofs should be written, as much as you can, in the following format: \\
%     \begin{enumerate}[i)]
%         \item Statement
%         \item Main points that will appear in your proof
%         \item The actual proof
%     \end{enumerate}
%     Answers to questions with computations should be written, as much as possible, in the following format:
%     \begin{enumerate}[i)]
%         \item Statement and Result
%         \item Main points that will appear in your computation.
%         \item The actual computation
%     \end{enumerate}
% \end{enumerate}



% % Start of problems




\begin{problem}
    \textit{Vector Spaces}. Suppose $\mathbb{F}$ is a field.
    \begin{enumerate}[a)]
        \item Define when we say that a vector space $V$ over a field $\mathbb{F}$ is \textit{finite dimensional}.
        \item Consider the vector space
        \[
            V = \mathbb{F}[x]
        \]
        of all polynomials with coefficients in $\mathbb{F}$. Show that $V$ is not finite dimensional.
        \item Suppose $X$ is a finite set. Consider the vector space $V$, of all functions from $X$ to $\mathbb{F}$,
        \[
            V = \mathbb{F}(X):= \{\text{all }f: X \to \mathbb{F}; \text{s.t. $f$ is a function}\},
        \]
        with the standard addition and multiplication by scalars from $\mathbb{F}$. Show that $V$ is finite dimensional.
    \end{enumerate}
\end{problem}

\begin{enumerate}[a)]
    \item We say that a vector space $V$ over a field $\mathbb{F}$ is finite dimensional if there exists $S \subseteq V$ such that $\#(S) < \infty$ and $\operatorname{span}(S) = V$.
    \item \begin{enumerate}[i)]
        \item Statement: Show that $V = \mathbb{F}[x]$ is not finite dimensional.
        \item Main Points:
        \begin{itemize}
            \item Suppose $V$ is finite dimensional with a finite spanning set $S$.
            \item Let $m$ be the maximum degree of the polynomials in $S$ and find a polynomial $q(x)$ with degree greater than $m$.
            \item Show that $q(x)$ cannot be written as a linear combination of the polynomials in $S$, leading to a contradiction.
        \end{itemize}
        \item Proof: 
        \begin{proof}
            Suppose $V$ is finite dimensional. Then there exists $S \subseteq V$ such that $\#(S) < \infty$ and $\operatorname{span}(S) = V$. Let $S = \{p_1(x), p_2(x), \ldots, p_n(x)\}$. Let $m = \max(\deg(p_i(x)))$ for $1 \leq i \leq n$. Then consider the polynomial $q(x) = x^{m+1}$. Since $\deg(q(x)) > m$, $q(x)$ cannot be written as a linear combination of the polynomials in $S$. This contradicts the fact that $\operatorname{span}(S) = V$. Thus, $V$ is not finite dimensional.
        \end{proof}
        \end{enumerate}

        \item \begin{enumerate}[i)]
            \item Statement: Show that $V = \mathbb{F}(X)$ is finite dimensional.
            \item Main Points:
                \begin{itemize}
                    \item Since $X$ is finite, say $X = \{x_1, x_2, \ldots, x_n\}$, consider the set of functions $\{\delta_{x_i}\}_{i=1}^n$, where
                    \[
                        \delta_{x_i}(x_j) = \begin{cases}
                            1 & \text{if } i = j \\
                            0 & \text{if } i \neq j
                        \end{cases}
                    \]
                    \item Show that $\{\delta_{x_i}\}_{i=1}^n$ forms a basis for $V$.
                \end{itemize}
            \item Proof:
            \begin{proof}
                Since $X$ is finite, say $X = \{x_1, x_2, \ldots, x_n\}$, consider the set of functions $\{\delta_{x_i}\}_{i=1}^n$, where
                \[
                    \delta_{x_i}(x_j) = \begin{cases}
                        1 & \text{if } i = j \\
                        0 & \text{if } i \neq j
                    \end{cases}
                \]
                We will show that $\{\delta_{x_i}\}_{i=1}^n$ forms a basis for $V$. First, we show that they span $V$. Let $f \in V$. Then we can write
                \[
                    f(x) = \sum_{i=1}^n f(x_i) \delta_{x_i}(x)
                \]
                for all $x \in X$. Thus, $\{\delta_{x_i}\}_{i=1}^n$ spans $V$. Next, we show that they are linearly independent. Suppose
                \[
                    \sum_{i=1}^n a_i \delta_{x_i}(x) = 0
                \]
                for some scalars $a_i \in \mathbb{F}$. Evaluating at $x = x_j$, we get
                \[
                    a_j = 0
                \]
                for all $j = 1, 2, \ldots, n$. Thus, all $a_i = 0$, and $\{\delta_{x_i}\}_{i=1}^n$ are linearly independent. Therefore, $\{\delta_{x_i}\}_{i=1}^n$ forms a basis for $V$, and hence $V$ is finite dimensional with $\dim(V) = n$.
            \end{proof}
        \end{enumerate}
\end{enumerate}





\begin{problem}
    \textit{Short exact sequences}. Suppose $U,V,W$ are three vector spaces over $\mathbb{F}$. Consider the following sequence of spaces and linear transformations between them:
    \begin{equation}
        0 \xrightarrow{} U \xrightarrow{\iota} V \xrightarrow{\epsilon} W \xrightarrow{} 0,
    \end{equation}
    where $0 \to U$, are the obvious maps from the zero space into $U$, and from the space $W$ onto the zero space, respectively. \\
    \begin{enumerate}[a)]
        \item Define when we say that the sequence (1) is \underline{short exact sequence} (s.e.s.).
        \item Given two subspaces $U,V < V$, such that $V = U \oplus W$, Show that there is a natural s.e.s.associated with the spaces of functions $U = \mathbb{F}(U), V = \mathbb{F}(V)$ and $W = \mathbb{F}(Y \backslash X)$, where $Y \backslash X$ denotes set-minus, i.e., the set of elements which are in $Y$ and are not in $X$.
    \end{enumerate}
\end{problem}


\begin{enumerate}[a)]
    \item A sequence (1) is a short exact sequence (s.e.s.) if the image of each map is equal to the kernel of the next map, i.e.,
    \[
        \operatorname{im}(0 \to U) = \ker(\iota), \quad \operatorname{im}(\iota) = \ker(\epsilon), \quad \operatorname{im}(\epsilon) = \ker(0 \to W).
    \]
    Since the maps from and to the zero space are trivial, this reduces to
    \[
        \iota \text{ is injective}, \quad \epsilon \text{ is surjective}, \quad \operatorname{im}(\iota) = \ker(\epsilon).
    \]
    \item \begin{enumerate}[i)]
        \item Statement: There is a natural s.e.s. associated with the spaces of functions $U = \mathbb{F}(U), V = \mathbb{F}(V)$ and $W = \mathbb{F}(Y \backslash X)$.
        \item Main Points:
            \begin{itemize}
                \item Define the inclusion map $\iota: U \to V$ by extending functions by zero outside $U$.
                \item Define the projection map $\epsilon: V \to W$ by restricting functions to $W$.
                \item Show that $\iota$ is injective.
                \item Show that $\epsilon$ is surjective.
                \item Show that $\operatorname{im}(\iota) = \ker(\epsilon)$.
            \end{itemize}
        \item Proof:
        \begin{proof}
            Since $V = U \oplus W$, every element $v \in V$ can be uniquely written as $v = u + w$ for some $u \in U$ and $w \in W$. Define the inclusion map $\iota: U \to V$ by
            \[
                \iota(f)(x) = \begin{cases}
                    f(x) & \text{if } x \in U \\
                    0 & \text{if } x \notin U
                \end{cases}
            \]
            for all $f \in U$. This map is injective because if $\iota(f) = 0$, then $f(x) = 0$ for all $x \in U$, which implies $f = 0$. Next, define the projection map $\epsilon: V \to W$ by
            \[
                \epsilon(g)(x) = g(x)
            \]
            for all $g \in V$ and $x \in W$. This map is surjective because for any $h \in W$, we can define a function $g \in V$ by
            \[
                g(x) = \begin{cases}
                    h(x) & \text{if } x \in W \\
                    0 & \text{if } x \notin W
                \end{cases}
            \]
            such that $\epsilon(g) = h$. Finally, we need to show that $\operatorname{im}(\iota) = \ker(\epsilon)$. If $f \in U$, then $\epsilon(\iota(f)) = 0$ since $\iota(f)$ is zero outside of $U$. Conversely, if $g \in V$ and $\epsilon(g) = 0$, then $g(x) = 0$ for all $x \in W$, which means that $g$ must be in the image of $\iota$. Therefore, we have shown that the sequence
            \[
                0 \xrightarrow{} U \xrightarrow{\iota} V \xrightarrow{\epsilon} W \xrightarrow{} 0
            \]
            is a short exact sequence.
        \end{proof}
    \end{enumerate}
    \item[] Remark(s): Since the problem stated that $U = \mathbb{F}(U)$, $V = \mathbb{F}(V)$ and $W = \mathbb{F}(Y \backslash X)$, I assumed that $\iota$ and $\epsilon$ were defined in terms of functions instead of elements of the vector spaces. If this is not the case, please let me know and I can adjust the proof accordingly.
\end{enumerate}


\begin{problem}
    \textit{Dimension}. Denote by $\operatorname{Vect}_{\mathbb{F}}^{fd}$ the collection of finite-dimensional vector spaces over $\mathbb{F}$, with linear transformations between them.
    \begin{enumerate}[a)]
        \item State the fact about uniqueness and existence of unique dimenstion function
        \[
            \dim: \operatorname{Vect}_{\mathbb{F}}^{fd} \to \mathbb{N},
        \]
        that satisfies certain desired properties. \\
        \textbf{Def.} For $V$ finite dimensional, the integer $\dim(V)$ is called the \underline{dimension} of $V$.
        \item Show that $\dim(M_n(\mathbb{F})) = n^2$.
        \item Suppose $1 + 1 \neq 0$ in $\mathbb{F}$. Consider the spaces $U = A_n(\mathbb{F})$, $V = M_n(\mathbb{F})$, $W = S_n(\mathbb{F})$, of anti-symmetric matrices ($A^T = -A$), all matrices, and symmetric matrices (sastisfy $A^T = A$), respectively. 
        \begin{enumerate}[i)]
            \item Show that, they form in a natural why a s.e.s.
            \item Deduce that $\dim(A_n(\mathbb{F})) = \frac{n(n-1)}{2}$ and $\dim(S_n(\mathbb{F})) = \frac{n(n+1)}{2}$.
        \end{enumerate}
    \end{enumerate}
\end{problem}

\begin{enumerate}[a)]
    \item There exists a unique function $\dim: \operatorname{Vect}_{\mathbb{F}}^{fd} \to \mathbb{N}$ such that
    \begin{enumerate}[i)]
        \item for a short exact sequence like (1), we have
        \[
            \dim(V) = \dim(U) + \dim(W),
        \]
        \item $\dim(\mathbb{F}) = 1$.
    \end{enumerate}

    \item \begin{enumerate}[i)]
        \item Statement: We have that $\dim(M_n(\mathbb{F})) = n^2$.
        \item Main Points:
        \begin{itemize}
            \item Consider the standard basis for $M_n(\mathbb{F})$ consisting of matrices with a single entry of 1 and all other entries 0.
            \item Count the number of such basis matrices.
        \end{itemize}
        \item Computation: \\
        The standard basis for $M_n(\mathbb{F})$ consists of matrices $E_{ij}$ where the $(i,j)$-th entry is 1 and all other entries are 0, for $1 \leq i,j \leq n$. There are $n$ choices for $i$ and $n$ choices for $j$, giving a total of $n \times n = n^2$ basis matrices. Therefore, $\dim(M_n(\mathbb{F})) = n^2$.
    \end{enumerate}
    \item \begin{enumerate}[i)]
        \item \begin{itemize}
            \item Statement: We have that $U = A_n(\mathbb{F})$, $V = M_n(\mathbb{F})$, and $W = S_n(\mathbb{F})$ form a short exact sequence.
            \item Main Points:
            \begin{itemize}
                \item Define the inclusion map $\iota: U \to V$ and the projection map $\epsilon: V \to W$.
                \item Show that $\iota$ is injective.
                \item Show that $\epsilon$ is surjective.
                \item Show that $\operatorname{im}(\iota) = \ker(\epsilon)$.
            \end{itemize}
            \item Proof:
            \begin{proof}
                Define the inclusion map $\iota: U \to V$ by $\iota(A) = A$ for all $A \in U$. This map is injective because if $\iota(A) = 0$, then $A = 0$. Next, define the projection map $\epsilon: V \to W$ by
                \[
                    \epsilon(B) = \frac{1}{2}(B + B^T)
                \]
                for all $B \in V$. This map is surjective because for any $C \in W$, we can take $B = C$ and have $\epsilon(B) = C$. Finally, we need to show that $\operatorname{im}(\iota) = \ker(\epsilon)$. If $A \in U$, then $\epsilon(\iota(A)) = 0$ since $A^T = -A$. Conversely, if $B \in V$ and $\epsilon(B) = 0$, then $B^T = -B$, which means that $B$ must be in the image of $\iota$. Therefore, we have shown that the sequence
                \[
                    0 \xrightarrow{} U \xrightarrow{\iota} V \xrightarrow{\epsilon} W \xrightarrow{} 0
                \]
                is a short exact sequence.
            \end{proof}
        \end{itemize} \newpage
        \item \begin{itemize}
            \item Statement: We have that $\dim(A_n(\mathbb{F})) = \frac{n(n-1)}{2}$ and $\dim(S_n(\mathbb{F})) = \frac{n(n+1)}{2}$.
            \item Main Points:
            \begin{itemize}
                \item Use the short exact sequence from part (i).
                \item Apply the dimension function to the sequence.
                \item Use the fact that $\dim(M_n(\mathbb{F})) = n^2$.
                \item Solve for the dimensions of $A_n(\mathbb{F})$ and $S_n(\mathbb{F})$.
            \end{itemize}
            \item Computation: \\
            From the short exact sequence in part (i), we have
            \[
                \dim(V) = \dim(U) + \dim(W).
            \]
            Substituting the known dimension of $V = M_n(\mathbb{F})$, we get
            \[
                n^2 = \dim(A_n(\mathbb{F})) + \dim(S_n(\mathbb{F})).
            \]
            Next, we count the dimensions of $A_n(\mathbb{F})$ and $S_n(\mathbb{F})$. A matrix in $A_n(\mathbb{F})$ is determined by its entries above the main diagonal, since the entries below the main diagonal are determined by the anti-symmetry condition. There are $\frac{n(n-1)}{2}$ such entries, so $\dim(A_n(\mathbb{F})) = \frac{n(n-1)}{2}$. Similarly, a matrix in $S_n(\mathbb{F})$ is determined by its entries on and above the main diagonal. There are $\frac{n(n+1)}{2}$ such entries, so $\dim(S_n(\mathbb{F})) = \frac{n(n+1)}{2}$. Therefore, we have
            \[
                n^2 = \frac{n(n-1)}{2} + \frac{n(n+1)}{2},
            \]
            which confirms our calculations. Thus, $\dim(A_n(\mathbb{F})) = \frac{n(n-1)}{2}$ and $\dim(S_n(\mathbb{F})) = \frac{n(n+1)}{2}$.
        \end{itemize}
        \end{enumerate}
    \end{enumerate}
    General Remark(s): I had plenty of extra time on this homework, so I typed it up and tried to make it look nice. If there are any issues with the formatting or if you would like me to change anything, please let me know. Depending on my homework load in the future, I may not be able to type up future homeworks, but I will do my best to make them look nice if I can. This homework was fun, and I look forward to the rest of the course!

\end{document}