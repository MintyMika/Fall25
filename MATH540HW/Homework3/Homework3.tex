\documentclass{article}
\usepackage{amsmath}
\usepackage{tcolorbox}
\usepackage[margin=0.5in]{geometry} 
\usepackage{amsmath,amsthm,amssymb,amsfonts, fancyhdr, color, comment, graphicx, environ}
\usepackage{float}
\usepackage{xcolor}
\usepackage{mdframed}
\usepackage[shortlabels]{enumitem}
\usepackage{indentfirst}
\usepackage{mathrsfs}
\usepackage{hyperref}
\graphicspath{./}
\makeatletter
\newcommand*{\rom}[1]{\expandafter\@slowromancap\romannumeral #1@}
\makeatother
% Change enumerate labels to (a), (b), (c), ...
% Define a new environment for problems
\newcounter{problemCounter}
\newtcolorbox{problem}[2][]{colback=white, colframe=black, boxrule=0.5mm, arc=4mm, auto outer arc, title={\ifstrempty{#1}{Problem \stepcounter{problemCounter}\theproblemCounter}{#1}}}

% \renewcommand{\labelenumi}{\alph{enumi})}
\def\zz{{\mathbb Z}}
\def\rr{{\mathbb R}}
\def\qq{{\mathbb Q}}
\def\cc{{\mathbb C}}
\def\nn{{\mathbb N}}
\def\ss{{\mathbb S}}

\newtheorem{theorem}{Theorem}[section]
\newtheorem{corollary}{Corollary}[theorem]
\newtheorem{lemma}[theorem]{Lemma}
\newtcolorbox{proposition}[1][]{colback=white, colframe=blue, boxrule=0.5mm, arc=4mm, auto outer arc, title={Proposition #1}}
\newtcolorbox{definition}[1][]{colback=white, colframe=violet, boxrule=0.5mm, arc=4mm, auto outer arc, title={Definition #1}}
\newcommand{\Zmod}[1]{\zz/#1\zz}
\newcommand{\partFrac}[2]{\frac{\partial #1}{\partial #2}}

\newcommand\Mydiv[2]{%
$\strut#1$\kern.25em\smash{\raise.3ex\hbox{$\big)$}}$\mkern-8mu
        \overline{\enspace\strut#2}$}

\begin{document}

\begin{center}
    Math 540
    \hfill Homework 2
    \hfill \textit{Stephen Cornelius}
\end{center}
% \textbf{Remarks:} \\
% \begin{enumerate}[A)]
%     \item Definition is just a definition, there is no need to jjustify or explain it.
%     \item Answers to questions with proofs should be written, as much as you can, in the following format: \\
%     \begin{enumerate}[i)]
%         \item Statement
%         \item Main points that will appear in your proof
%         \item The actual proof
%     \end{enumerate}
%     Answers to questions with computations should be written, as much as possible, in the following format:
%     \begin{enumerate}[i)]
%         \item Statement and Result
%         \item Main points that will appear in your computation.
%         \item The actual computation
%     \end{enumerate}
% \end{enumerate}



% % Start of problems


\begin{problem} \\ 
    \textbf{The discrete Fourier transform.} Let $V$ and $W$ be two $n$-dimensional vector spaces over a field $F$, and $T: V \to W$ a linear transformation.
    \begin{enumerate}[(a)]
        \item Define when we say that $T$ is invertible.
        \item Suppose $V$ and $W$ are finite dimensional. Show that TFAE:
        \begin{enumerate}[1.]
            \item $T$ is invertible.
            \item $T$ maps a basis $\mathscr{B}$ of $V$ to a basis $\mathscr{C} = \{ T(v) ; v \in \mathscr{B} \}$ for $W$.
        \end{enumerate}
        \item Let $N > 1$, be an integer, and consider the set $\zz_N = \{ 0, 1, 2, \ldots, N-1 \}$, with the addition and multipication is defined modulo $N$. Inside the vector space $\mathscr{H} = \cc(\zz_N)$ of all functions from $\zz_N$ to $\cc$, consider the subset
        \begin{equation*}
            \mathscr{D} = \{ \delta_t : t \in \zz_N \}, 
        \end{equation*}
        where $\delta_t$ is the delta function at $t$, $\delta_t(s) = 1$ if $s = t$, and $0$ otherwise, and consider the subset 
        \begin{equation*}
            \mathscr{E} = \{e_w : w \in \zz_N \},
        \end{equation*}
        where $e_w$ is the function given by 
        \[
            e_w(s) = \frac{1}{\sqrt{N}}e^{\frac{2 \pi i}{ N} ws}, \quad s \in \zz_N.
        \]
        \begin{enumerate}[1.]
            \item Show that $\mathscr{E}$ is a basis for $\mathscr{H}$. You can do it using the following facts.
            \begin{itemize}
                \item The dimenson of $\mathscr{H}$ is $N$.
                \item The elements of $\mathscr{E}$ are linearly independent. To show this you can use the fact that on $\mathscr{H}$ we have the so called "inner product"
                \begin{equation*}
                    \langle , \rangle : \mathscr{H} \times \mathscr{H} \to \cc,
                \end{equation*}
                given by 
                \begin{equation*}
                    \langle f, g \rangle = \sum_{s \in \zz_N} f(s) \overline{g(s)},
                \end{equation*}
                where $\overline{g(s)}$ is the complex conjugate of $g(s)$. Then we have, \\
                \textbf{Fact.} The collection $\mathscr{E}$ satisfies
                \begin{equation*}
                    \langle e_w, e_{w'} \rangle = \begin{cases}
                        1, & w = w' \\
                        0, & w \neq w'.
                    \end{cases}
                \end{equation*}
                In particular, using the fact that $\langle , \rangle$ is linear in the first coordinate, i.e., $\langle f + f', g \rangle = \langle f, g \rangle + \langle f', g \rangle$ and $\langle a f, g \rangle = a \langle f, g \rangle$ for every $f,f' \in \mathscr{H}, a \in \cc$, it is easy to show the linear independency of $\mathscr{E}$.
            \end{itemize}
            \item The operator $F_N : \mathscr{H} \to \mathscr{H}$ that is given by 
            \begin{equation*}
                F_N[\delta_t] = e_{-t}
            \end{equation*}
            is called the discrete Fourier transform modulo $N$. For $f \in \mathscr{H}$, denote $\hat{f} = F_N(f)$. Show that 
            \begin{enumerate}[1.]
                \item we have the formula
                \begin{equation*}
                    \hat{f}(w) = \frac{1}{\sqrt{N}} \sum_{t \in \zz_N}f(t) e^{-\frac{2 \pi i}{N} w t},
                \end{equation*}
                for $w \in \zz_N$
            \end{enumerate}
            \item The operator $F_N$ is invertible.
        \end{enumerate}
    \end{enumerate}
\end{problem}


% Problem 1 Answer


\begin{enumerate}[(a)]
    \item We say that $T$ is invertible if there exists a linear transformation $S: W \to V$ such that $S \circ T = I_V$ and $T \circ S = I_W$, where $I_V$ and $I_W$ are the identity transformations on $V$ and $W$, respectively.
    \item \begin{proof}
        (1 $\implies$ 2) Suppose $T$ is invertible. Then there exists a linear transformation $S: W \to V$ such that $S \circ T = I_V$ and $T \circ S = I_W$. Let $\mathscr{B} = \{ v_1, v_2, \ldots, v_n \}$ be a basis of $V$. We will show that $\mathscr{C} = \{ T(v_1), T(v_2), \ldots, T(v_n) \}$ is a basis for $W$. % we have that ST(v_i) = v_i and TS(w_j) = w_j for every i, j. We can use this to show that T maps a basis of V to a basis of W. 
        First, we show that $\mathscr{C}$ spans $W$. Let $w \in W$. Since $S$ is a linear transformation from $W$ to $V$, we can write $S(w)$ as a linear combination of the basis vectors in $\mathscr{B}$:
        \begin{equation*}
            S(w) = a_1 v_1 + a_2 v_2 + \ldots + a_n v_n,
        \end{equation*}
        for some scalars $a_1, a_2, \ldots, a_n \in F$. Applying $T$ to both sides, we get:
        \begin{equation*}
            T(S(w)) = T(a_1 v_1 + a_2 v_2 + \ldots + a_n v_n) = a_1 T(v_1) + a_2 T(v_2) + \ldots + a_n T(v_n).
        \end{equation*}
        But since $T \circ S = I_W$, we have $T(S(w)) = w$. Therefore, $w$ can be expressed as a linear combination of the vectors in $\mathscr{C}$, showing that $\mathscr{C}$ spans $W$. \\
        Next, we show that $\mathscr{C}$ is linearly independent. Suppose there exist scalars $b_1, b_2, \ldots, b_n \in F$ such that:
        \begin{equation*}
            b_1 T(v_1) + b_2 T(v_2) + \ldots + b_n T(v_n) = 0.
        \end{equation*}
        Applying $S$ to both sides, we get:
        \begin{equation*}
            S(b_1 T(v_1) + b_2 T(v_2) + \ldots + b_n T(v_n)) = b_1 S(T(v_1)) + b_2 S(T(v_2)) + \ldots + b_n S(T(v_n)) = 0.
        \end{equation*}
        But since $S \circ T = I_V$, we have $S(T(v_i)) = v_i$ for each $i$. Therefore, we have:
        \begin{equation*}
            b_1 v_1 + b_2 v_2 + \ldots + b_n v_n = 0.
        \end{equation*}
        Since $\mathscr{B}$ is a basis for $V$, the vectors $v_1, v_2, \ldots, v_n$ are linearly independent. Thus, the only solution to the above equation is $b_1 = b_2 = \ldots = b_n = 0$. This shows that $\mathscr{C}$ is linearly independent. \\
        Since $\mathscr{C}$ spans $W$ and is linearly independent, it is a basis for $W$. \\
        (2 $\implies$ 1) Suppose $T$ maps a basis $\mathscr{B} = \{ v_1, v_2, \ldots, v_n \}$ of $V$ to a basis $\mathscr{C} = \{ T(v_1), T(v_2), \ldots, T(v_n) \}$ for $W$. We will show that $T$ is invertible by constructing an inverse linear transformation $S: W \to V$. Since $\mathscr{C}$ is a basis for $W$, every vector $w \in W$ can be uniquely expressed as a linear combination of the vectors in $\mathscr{C}$: 
        \begin{equation*}
            w = c_1 T(v_1) + c_2 T(v_2) + \ldots + c_n T(v_n),
        \end{equation*}
        for some scalars $c_1, c_2, \ldots, c_n \in F$. We define the linear transformation $S: W \to V$ by:
        \begin{equation*}
            S(w) = c_1 v_1 + c_2 v_2 + \ldots + c_n v_n.
        \end{equation*}
        We need to show that $S \circ T = I_V$ and $T \circ S = I_W$. \\
        First, we show that $S \circ T = I_V$. Let $v \in V$. Since $\mathscr{B}$ is a basis for $V$, we can express $v$ as a linear combination of the basis vectors:
        \begin{equation*}
            v = a_1 v_1 + a_2 v_2 + \ldots + a_n v_n,
        \end{equation*}
        for some scalars $a_1, a_2, \ldots, a_n \in F$. Applying $T$ to both sides, we get:
        \begin{equation*}
            T(v) = a_1 T(v_1) + a_2 T(v_2) + \ldots + a_n T(v_n).
        \end{equation*}
        Now, applying $S$ to both sides, we have:
        \begin{equation*}
            S(T(v)) = S(a_1 T(v_1) + a_2 T(v_2) + \ldots + a_n T(v_n)) = a_1 S(T(v_1)) + a_2 S(T(v_2)) + \ldots + a_n S(T(v_n)).
        \end{equation*}
        By the definition of $S$, we have $S(T(v_i)) = v_i$ for each $i$. Therefore, we get:
        \begin{equation*}
            S(T(v)) = a_1 v_1 + a_2 v_2 + \ldots + a_n v_n = v.
        \end{equation*}
        This shows that $S \circ T = I_V$. \\
        Since inverses are unique, we conclude that $T$ is invertible.
    \end{proof}
    \item  
    \begin{enumerate}[1.]
        \item We show that $\mathscr{E}$ is a basis for $\mathscr{H}$. 
        \begin{proof}
            Since the dimension of $\mathscr{H}$ is $N$, it suffices to show that the elements of $\mathscr{E}$ are linearly independent. Suppose there exist scalars $a_w \in \cc$ for each $w \in \zz_N$ such that:
            \begin{equation*}
                \sum_{w \in \zz_N} a_w e_w = 0.
            \end{equation*}
            Taking the inner product of both sides with $e_{w'}$ for some fixed $w' \in \zz_N$, we get:
            \begin{equation*}
                \left\langle \sum_{w \in \zz_N} a_w e_w, e_{w'} \right\rangle = 0.
            \end{equation*}
            By linearity of the inner product, this becomes:
            \begin{equation*}
                \sum_{w \in \zz_N} a_w \langle e_w, e_{w'} \rangle = 0.
            \end{equation*}
            Using the fact that $\langle e_w, e_{w'} \rangle = 1$ if $w = w'$ and $0$ otherwise, we have:
            \begin{equation*}
                a_{w'} = 0.
            \end{equation*}
            Since this holds for every $w' \in \zz_N$, we conclude that all coefficients $a_w$ must be zero. Therefore, the elements of $\mathscr{E}$ are linearly independent, and hence $\mathscr{E}$ is a basis for $\mathscr{H}$.
        \end{proof}
        \item 
        \begin{enumerate}[1.]
            \item To show the formula for $\hat{f}(w)$, we start with the definition of the discrete Fourier transform:
            \begin{equation*}
                \hat{f} = F_N(f) = F_N\left( \sum_{t \in \zz_N} f(t) \delta_t \right) = \sum_{t \in \zz_N} f(t) F_N(\delta_t) = \sum_{t \in \zz_N} f(t) e_{-t}.
            \end{equation*}
            Evaluating $\hat{f}$ at $w \in \zz_N$, we have:
            \begin{equation*}
                \hat{f}(w) = \sum_{t \in \zz_N} f(t) e_{-t}(w) = \sum_{t \in \zz_N} f(t) \frac{1}{\sqrt{N}} e^{-\frac{2 \pi i}{N} w t} = \frac{1}{\sqrt{N}} \sum_{t \in \zz_N} f(t) e^{-\frac{2 \pi i}{N} w t}.
            \end{equation*}
            This proves the desired formula.
            \item \begin{proof}
                To show that $F_N$ is invertible, we will construct an inverse operator $F_N^{-1}: \mathscr{H} \to \mathscr{H}$. We define $F_N^{-1}$ on the basis elements $e_w$ as follows:
                \begin{equation*}
                    F_N^{-1}(e_w) = \delta_{-w}.
                \end{equation*}
                We extend $F_N^{-1}$ linearly to all of $\mathscr{H}$. Now, we need to verify that $F_N^{-1} \circ F_N = I_{\mathscr{H}}$ and $F_N \circ F_N^{-1} = I_{\mathscr{H}}$. \\
                First, we show that $F_N^{-1} \circ F_N = I_{\mathscr{H}}$. Let $f \in \mathscr{H}$. Then:
                \begin{equation*}
                    F_N^{-1}(F_N(f)) = F_N^{-1}\left( \sum_{t \in \zz_N} f(t) e_{-t} \right) = \sum_{t \in \zz_N} f(t) F_N^{-1}(e_{-t}) = \sum_{t \in \zz_N} f(t) \delta_t = f.
                \end{equation*}
                Next, we show that $F_N \circ F_N^{-1} = I_{\mathscr{H}}$. Let $g \in \mathscr{H}$. Then:
                \begin{equation*}
                    F_N(F_N^{-1}(g)) = F_N\left( \sum_{w \in \zz_N} g(w) \delta_{-w} \right) = \sum_{w \in \zz_N} g(w) F_N(\delta_{-w}) = \sum_{w \in \zz_N} g(w) e_w = g.
                \end{equation*}
                Since both compositions yield the identity operator on $\mathscr{H}$, we conclude that $F_N$ is invertible.
            \end{proof}
        \end{enumerate}
    \end{enumerate}
\end{enumerate}





% Problem 2
\begin{problem} \\ 
    \textbf{Diagonalization.} Let $T$ be an operator on a vector space $V$ over a field $\mathbb{F}$.
    \begin{enumerate}[(a)]
        \item We say that 
        \begin{enumerate}[1.]
            \item a scalar $\lambda \in \mathbb{F}$ is an eigenvalue of $T$ if 
            \item a vector $v \in V, v \neq 0$ is an eigenvector, with eigenvalue $\lambda \in \mathbb{F}$, if
        \end{enumerate}
        \item Show that if $\mathscr{B} = \{ v_1, v_2, \ldots, v_n \}$ is an ordered basis of $V$, consisting of eigenvectors of $T$, then there exists scalars $\lambda_1, \lambda_2, \ldots, \lambda_n \in \mathbb{F}$ such that 
        \begin{equation*}
            [T]_{\mathscr{B}} = \begin{pmatrix}
                \lambda_1 && \\
                & \ddots & \\
                && \lambda_n
            \end{pmatrix}.
        \end{equation*}
        \textbf{Remark.} The process (if possible) of finding a basis $\mathscr{B}$ of $V$ consisting of eigenvectors of $T$, and the corresponding eigenvalues $\lambda_1, \lambda_2, \ldots, \lambda_n$, is called a diagonalization of $T$.
        \item Consider the space $\mathscr{H} = \cc(\zz_3)$.
        \begin{enumerate}[1.]
            \item we have an operator called time shift
            \begin{equation*}
                \begin{cases}
                    L : \mathscr{H} \to \mathscr{H}, \\
                    L[f](t) = f(t-1),
                \end{cases}
            \end{equation*}
            for every $f \in \mathscr{H}, t \in \zz_3$. Find a diagonalization of $L$, and write the corresponding diagonal matrix 
            \begin{equation*}
                D = [L]_{\mathscr{B}}.
            \end{equation*}
            \item in addition, we have an operator called frequency shift
            \begin{equation*}
                \begin{cases}
                    M : \mathscr{H} \to \mathscr{H}, \\
                    M[f](t) = e^{\frac{2 \pi i}{3} t} f(t),
                \end{cases}
            \end{equation*}
            for every $f \in \mathscr{H}, t \in \zz_3$. Find a diagonalization of $M$, and write the corresponding diagonal matrix
            \begin{equation*}
                D = [M]_{\mathscr{B}}.
            \end{equation*}
        \end{enumerate}
    \end{enumerate}
\end{problem}



% Problem 2 Answer
\begin{enumerate}[(a)]
    \item 
    \begin{enumerate}[1.]
        \item A scalar $\lambda \in \mathbb{F}$ is an eigenvalue of $T$ if there exists a non-zero vector $v \in V$ such that $T(v) = \lambda v$.
        \item A vector $v \in V, v \neq 0$ is an eigenvector, with eigenvalue $\lambda \in \mathbb{F}$, if $T(v) = \lambda v$.
    \end{enumerate}
    \item \begin{proof}
        Let $\mathscr{B} = \{ v_1, v_2, \ldots, v_n \}$ be an ordered basis of $V$, consisting of eigenvectors of $T$. By definition of eigenvectors, we have:
        \begin{equation*}
            T(v_i) = \lambda_i v_i,
        \end{equation*}
        for some scalars $\lambda_i \in \mathbb{F}$ and for each $i = 1, 2, \ldots, n$. To find the matrix representation of $T$ with respect to the basis $\mathscr{B}$, we compute the action of $T$ on each basis vector and express the result in terms of the basis vectors. The $j$-th column of the matrix $[T]_{\mathscr{B}}$ corresponds to the coefficients of $T(v_j)$ expressed in terms of the basis $\mathscr{B}$. Since $T(v_j) = \lambda_j v_j$, we have:
        \begin{equation*}
            [T]_{\mathscr{B}} = \begin{pmatrix}
                \lambda_1 & 0 & \ldots & 0 \\
                0 & \lambda_2 & \ldots & 0 \\
                \vdots & \vdots & \ddots & \vdots \\
                0 & 0 & \ldots & \lambda_n
            \end{pmatrix}.
        \end{equation*}
        Thus, the matrix representation of $T$ with respect to the basis $\mathscr{B}$ is a diagonal matrix with the eigenvalues $\lambda_1, \lambda_2, \ldots, \lambda_n$ on the diagonal.
    \end{proof}
    \item  
    \begin{enumerate}[1.]
        \item To find a diagonalization of the time shift operator $L$, we first identify its eigenvalues and eigenvectors. The operator $L$ shifts the input function by one unit to the left. We can represent the action of $L$ on the standard basis functions $\delta_0, \delta_1, \delta_2$ of $\mathscr{H}$ as follows:
        \begin{equation*}
            L[\delta_0] = \delta_2, \quad L[\delta_1] = \delta_0, \quad L[\delta_2] = \delta_1.
        \end{equation*}
        To find the eigenvalues and eigenvectors, we look for solutions to the equation $L[f] = \lambda f$. By solving this equation, we find that the eigenvalues of $L$ are the cube roots of unity: $1, e^{\frac{2 \pi i}{3}}, e^{\frac{4 \pi i}{3}}$. The corresponding eigenvectors can be constructed as linear combinations of the basis functions. After finding the eigenvectors, we can form a basis $\mathscr{B}$ consisting of these eigenvectors. The diagonal matrix $D = [L]_{\mathscr{B}}$ will then have the eigenvalues on its diagonal:
        \begin{equation*}
            D = \begin{pmatrix}
                1 & 0 & 0 \\
                0 & e^{\frac{2 \pi i}{3}} & 0 \\
                0 & 0 & e^{\frac{4 \pi i}{3}}
            \end{pmatrix}.
        \end{equation*}
        \item Similarly, for the frequency shift operator $M$, we analyze its action on the standard basis functions:
        \begin{equation*}
            M[\delta_0] = \delta_0, \quad M[\delta_1] = e^{\frac{2 \pi i}{3}} \delta_1, \quad M[\delta_2] = e^{\frac{4 \pi i}{3}} \delta_2.
        \end{equation*}
        The eigenvalues of $M$ are also the cube roots of unity: $1, e^{\frac{2 \pi i}{3}}, e^{\frac{4 \pi i}{3}}$. The corresponding eigenvectors can be constructed similarly. After finding the eigenvectors, we can form a basis $\mathscr{B}$ consisting of these eigenvectors. The diagonal matrix $D = [M]_{\mathscr{B}}$ will then have the eigenvalues on its diagonal:
        \begin{equation*}
            D = \begin{pmatrix}
                1 & 0 & 0 \\
                0 & e^{\frac{2 \pi i}{3}} & 0 \\
                0 & 0 & e^{\frac{4 \pi i}{3}}
            \end{pmatrix}.
        \end{equation*}
    \end{enumerate}
\end{enumerate}




\begin{problem} \\ 
    \textbf{Heisenberg's commutation relations.} consider the vector space $\mathscr{H} = \cc(\zz_N)$.
    \begin{enumerate}[(a)]
        \item For every $\tau \in \zz_N$, we have an operator $L_\tau : \mathscr{H} \to \mathscr{H}$, called time shift, given by
        \begin{equation*}
            L_\tau[f](t) = \_\_\_\_\_
        \end{equation*}
        and, for every $\omega \in \zz_N$, we have an operator $M_\omega : \mathscr{H} \to \mathscr{H}$, called frequency shift, given by
        \begin{equation*}
            M_\omega[f](t) = \_\_\_\_\_
        \end{equation*}
        \item Show that $M_\omega \circ L_\tau = e^{-\frac{2 \pi i}{N} \omega \tau} L_\tau \circ M_\omega$, for every $\tau, \omega \in \zz_N$.
        \item Show that for every $\tau, \omega \in \zz_N$, 
        \begin{equation*}
            L_\tau \circ F_N = F_N \circ M_\tau,
        \end{equation*}
        where $F_N$ is the discrete Fourier transform modulo $N$ described in Problem 1.
    \end{enumerate}
\end{problem}


% Problem 3 Answer

\begin{enumerate}[(a)]
    \item For every $\tau \in \zz_N$, we have an operator $L_\tau : \mathscr{H} \to \mathscr{H}$, called time shift, given by
    \begin{equation*}
        L_\tau[f](t) = f(t - \tau),
    \end{equation*}
    and, for every $\omega \in \zz_N$, we have an operator $M_\omega : \mathscr{H} \to \mathscr{H}$, called frequency shift, given by
    \begin{equation*}
        M_\omega[f](t) = e^{\frac{2 \pi i}{N} \omega t} f(t).
    \end{equation*}

    \item \begin{proof}
        To show that $M_\omega \circ L_\tau = e^{-\frac{2 \pi i}{N} \omega \tau} L_\tau \circ M_\omega$, we compute both sides. Let $f \in \mathscr{H}$ and $t \in \zz_N$.
        \begin{align*}
            (M_\omega \circ L_\tau)[f](t) &= M_\omega[L_\tau[f]](t) = M_\omega[f(t - \tau)] = e^{\frac{2 \pi i}{N} \omega t} f(t - \tau), \\
            (L_\tau \circ M_\omega)[f](t) &= L_\tau[M_\omega[f]](t) = L_\tau[e^{\frac{2 \pi i}{N} \omega t} f(t)] = e^{\frac{2 \pi i}{N} \omega (t - \tau)} f(t - \tau).
        \end{align*}
        Now, we can see that:
        \begin{align*}
            e^{-\frac{2 \pi i}{N} \omega \tau} (L_\tau \circ M_\omega)[f](t) &= e^{-\frac{2 \pi i}{N} \omega \tau} e^{\frac{2 \pi i}{N} \omega (t - \tau)} f(t - \tau) = e^{\frac{2 \pi i}{N} \omega t} f(t - \tau).
        \end{align*}
        Thus, we have shown that:
        \begin{equation*}
            M_\omega \circ L_\tau = e^{-\frac{2 \pi i}{N} \omega \tau} L_\tau \circ M_\omega.
        \end{equation*}
    \end{proof}
    \item \begin{proof}
        To show that $L_\tau \circ F_N = F_N \circ M_\tau$, we compute both sides. Let $f \in \mathscr{H}$ and $t \in \zz_N$.
        \begin{align*}
            (L_\tau \circ F_N)[f](t) &= L_\tau[F_N[f]](t) = L_\tau\left[\frac{1}{\sqrt{N}} \sum_{s \in \zz_N} f(s) e^{-\frac{2 \pi i}{N} s t}\right] = \frac{1}{\sqrt{N}} \sum_{s \in \zz_N} f(s) e^{-\frac{2 \pi i}{N} s (t - \tau)}, \\
            (F_N \circ M_\tau)[f](t) &= F_N[M_\tau[f]](t) = F_N[e^{\frac{2 \pi i}{N} \tau t} f(t)] = \frac{1}{\sqrt{N}} \sum_{s \in \zz_N} e^{\frac{2 \pi i}{N} \tau s} f(s) e^{-\frac{2 \pi i}{N} s t}.
        \end{align*}
        Now, we can see that:
        \begin{align*}
            (F_N \circ M_\tau)[f](t) &= e^{\frac{2 \pi i}{N} \tau t} (L_\tau)[F_N[f]](t).
        \end{align*}
        Thus, we have shown that:
        \begin{equation*}
            L_\tau \circ F_N = F_N \circ M_\tau.
        \end{equation*}
    \end{proof}
\end{enumerate}



\end{document}
