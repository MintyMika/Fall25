\documentclass{article}
\usepackage{amsmath}
\usepackage{tcolorbox}
\usepackage[margin=0.5in]{geometry} 
\usepackage{amsmath,amsthm,amssymb,amsfonts, fancyhdr, color, comment, graphicx, environ}
\usepackage{float}
\usepackage{xcolor}
\usepackage{mdframed}
\usepackage[shortlabels]{enumitem}
\usepackage{indentfirst}
\usepackage{mathrsfs}
\usepackage{hyperref}
\graphicspath{./}
\makeatletter
\newcommand*{\rom}[1]{\expandafter\@slowromancap\romannumeral #1@}
\makeatother
% Change enumerate labels to (a), (b), (c), ...
% Define a new environment for problems
\newcounter{problemCounter}
\newtcolorbox{problem}[2][]{colback=white, colframe=black, boxrule=0.5mm, arc=4mm, auto outer arc, title={\ifstrempty{#1}{Problem \stepcounter{problemCounter}\theproblemCounter}{#1}}}

% \renewcommand{\labelenumi}{\alph{enumi})}
\def\zz{{\mathbb Z}}
\def\rr{{\mathbb R}}
\def\qq{{\mathbb Q}}
\def\cc{{\mathbb C}}
\def\nn{{\mathbb N}}
\def\ss{{\mathbb S}}

\newtheorem{theorem}{Theorem}[section]
\newtheorem{corollary}{Corollary}[theorem]
\newtheorem{lemma}[theorem]{Lemma}
\newtcolorbox{proposition}[1][]{colback=white, colframe=blue, boxrule=0.5mm, arc=4mm, auto outer arc, title={Proposition #1}}
\newtcolorbox{definition}[1][]{colback=white, colframe=violet, boxrule=0.5mm, arc=4mm, auto outer arc, title={Definition #1}}
\newcommand{\Zmod}[1]{\zz/#1\zz}
\newcommand{\partFrac}[2]{\frac{\partial #1}{\partial #2}}

\newcommand\Mydiv[2]{%
$\strut#1$\kern.25em\smash{\raise.3ex\hbox{$\big)$}}$\mkern-8mu
        \overline{\enspace\strut#2}$}

\begin{document}

\begin{center}
    Math 540
    \hfill Homework 2
    \hfill \textit{Stephen Cornelius}
\end{center}
% \textbf{Remarks:} \\
% \begin{enumerate}[A)]
%     \item Definition is just a definition, there is no need to jjustify or explain it.
%     \item Answers to questions with proofs should be written, as much as you can, in the following format: \\
%     \begin{enumerate}[i)]
%         \item Statement
%         \item Main points that will appear in your proof
%         \item The actual proof
%     \end{enumerate}
%     Answers to questions with computations should be written, as much as possible, in the following format:
%     \begin{enumerate}[i)]
%         \item Statement and Result
%         \item Main points that will appear in your computation.
%         \item The actual computation
%     \end{enumerate}
% \end{enumerate}



% % Start of problems


\begin{problem} \\ 
    \textbf{The discrete Fourier transform.} Let $V$ and $W$ be two $n$-dimensional vector spaces over a field $F$, and $T: V \to W$ a linear transformation.
    \begin{enumerate}[(a)]
        \item Define when we say that $T$ is invertible.
        \item Suppose $V$ and $W$ are finite dimensional. Show that TFAE:
        \begin{enumerate}[1.]
            \item $T$ is invertible.
            \item $T$ maps a basis $\mathscr{B}$ of $V$ to a basis $\mathscr{C} = \{ T(v) ; v \in \mathscr{B} \}$ for $W$.
        \end{enumerate}
        \item Let $N > 1$, be an integer, and consider the set $\zz_N = \{ 0, 1, 2, \ldots, N-1 \}$, with the addition and multipication is defined modulo $N$. Inside the vector space $\mathscr{H} = \cc(\zz_N)$ of all functions from $\zz_N$ to $\cc$, consider the subset
        \begin{equation*}
            \mathscr{D} = \{ \delta_t : t \in \zz_N \}, 
        \end{equation*}
        where $\delta_t$ is the delta function at $t$, $\delta_t(s) = 1$ if $s = t$, and $0$ otherwise, and consider the subset 
        \begin{equation*}
            \mathscr{E} = \{e_w : w \in \zz_N \},
        \end{equation*}
        where $e_w$ is the function given by 
        \[
            e_w(s) = \frac{1}{\sqrt{N}}e^{\frac{2 \pi i}{ N} ws}, \quad s \in \zz_N.
        \]
        \begin{enumerate}[1.]
            \item Show that $\mathscr{E}$ is a basis for $\mathscr{H}$. You can do it using the following facts.
            \begin{itemize}
                \item The dimenson of $\mathscr{H}$ is $N$.
                \item The elements of $\mathscr{E}$ are linearly independent. To show this you can use the fact that on $\mathscr{H}$ we have the so called "inner product"
                \begin{equation*}
                    \langle , \rangle : \mathscr{H} \times \mathscr{H} \to \cc,
                \end{equation*}
                given by 
                \begin{equation*}
                    \langle f, g \rangle = \sum_{s \in \zz_N} f(s) \overline{g(s)},
                \end{equation*}
                where $\overline{g(s)}$ is the complex conjugate of $g(s)$. Then we have, \\
                \textbf{Fact.} The collection $\mathscr{E}$ satisfies
                \begin{equation*}
                    \langle e_w, e_{w'} \rangle = \begin{cases}
                        1, & w = w' \\
                        0, & w \neq w'.
                    \end{cases}
                \end{equation*}
                In particular, using the fact that $\langle , \rangle$ is linear in the first coordinate, i.e., $\langle f + f', g \rangle = \langle f, g \rangle + \langle f', g \rangle$ and $\langle a f, g \rangle = a \langle f, g \rangle$ for every $f,f' \in \mathscr{H}, a \in \cc$, it is easy to show the linear independency of $\mathscr{E}$.
            \end{itemize}
            \item The operator $F_N : \mathscr{H} \to \mathscr{H}$ that is given by 
            \begin{equation*}
                F_N[\delta_t] = e_{-t}
            \end{equation*}
            is called the discrete Fourier transform modulo $N$. For $f \in \mathscr{H}$, denote $\hat{f} = F_N(f)$. Show that 
            \begin{enumerate}[1.]
                \item we have the formula
                \begin{equation*}
                    \hat{f}(w) = \frac{1}{\sqrt{N}} \sum_{t \in \zz_N}f(t) e^{-\frac{2 \pi i}{N} w t},
                \end{equation*}
                for $w \in \zz_N$
            \end{enumerate}
            \item The operator $F_N$ is invertible.
        \end{enumerate}
    \end{enumerate}
\end{problem}




\begin{problem} \\ 
    \textbf{Diagonalization.} Let $T$ be an operator on a vector space $V$ over a field $\mathbb{F}$.
    \begin{enumerate}[(a)]
        \item We say that 
        \begin{enumerate}[1.]
            \item a scalar $\lambda \in \mathbb{F}$ is an eigenvalue of $T$ if 
            \item a vector $v \in V, v \neq 0$ is an eigenvector, with eigenvalue $\lambda \in \mathbb{F}$, if
        \end{enumerate}
        \item Show that if $\mathscr{B} = \{ v_1, v_2, \ldots, v_n \}$ is an ordered basis of $V$, consisting of eigenvectors of $T$, then there exists scalars $\lambda_1, \lambda_2, \ldots, \lambda_n \in \mathbb{F}$ such that 
        \begin{equation*}
            [T]_{\mathscr{B}} = \begin{pmatrix}
                \lambda_1 && \\
                & \ddots & \\
                && \lambda_n
            \end{pmatrix}.
        \end{equation*}
        \textbf{Remark.} The process (if possible) of finding a basis $\mathscr{B}$ of $V$ consisting of eigenvectors of $T$, and the corresponding eigenvalues $\lambda_1, \lambda_2, \ldots, \lambda_n$, is called a diagonalization of $T$.
        \item Consider the space $\mathscr{H} = \cc(\zz_3)$.
        \begin{enumerate}[1.]
            \item we have an operator called time shift
            \begin{equation*}
                \begin{cases}
                    L : \mathscr{H} \to \mathscr{H}, \\
                    L[f](t) = f(t-1),
                \end{cases}
            \end{equation*}
            for every $f \in \mathscr{H}, t \in \zz_3$. Find a diagonalization of $L$, and write the corresponding diagonal matrix 
            \begin{equation*}
                D = [L]_{\mathscr{B}}.
            \end{equation*}
            \item in addition, we have an operator called frequency shift
            \begin{equation*}
                \begin{cases}
                    M : \mathscr{H} \to \mathscr{H}, \\
                    M[f](t) = e^{\frac{2 \pi i}{3} t} f(t),
                \end{cases}
            \end{equation*}
            for every $f \in \mathscr{H}, t \in \zz_3$. Find a diagonalization of $M$, and write the corresponding diagonal matrix
            \begin{equation*}
                D = [M]_{\mathscr{B}}.
            \end{equation*}
        \end{enumerate}
    \end{enumerate}
\end{problem}




\begin{problem} \\ 
    \textbf{Heisenberg's commutation relations.} consider the vector space $\mathscr{H} = \cc(\zz_N)$.
    \begin{enumerate}[(a)]
        \item For every $\tau \in \zz_N$, we have an operator $L_\tau : \mathscr{H} \to \mathscr{H}$, called time shift, given by
        \begin{equation*}
            L_\tau[f](t) = \_\_\_\_\_
        \end{equation*}
        and, for every $\omega \in \zz_N$, we have an operator $M_\omega : \mathscr{H} \to \mathscr{H}$, called frequency shift, given by
        \begin{equation*}
            M_\omega[f](t) = \_\_\_\_\_
        \end{equation*}
        \item Show that $M_\omega \circ L_\tau = e^{-\frac{2 \pi i}{N} \omega \tau} L_\tau \circ M_\omega$, for every $\tau, \omega \in \zz_N$.
        \item Show that for every $\tau, \omega \in \zz_N$, 
        \begin{equation*}
            L_\tau \circ F_N = F_N \circ M_\tau,
        \end{equation*}
        where $F_N$ is the discrete Fourier transform modulo $N$ described in Problem 1.
    \end{enumerate}
\end{problem}


\end{document}
