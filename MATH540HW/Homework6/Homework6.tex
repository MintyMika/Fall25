\documentclass{article}
% Common LaTeX preamble. Save as preamble.tex and % Common LaTeX preamble. Save as preamble.tex and % Common LaTeX preamble. Save as preamble.tex and \input{preamble} after \documentclass.
% Engine detection
\usepackage{iftex}
\ifPDFTeX
    \usepackage[T1]{fontenc}
    \usepackage[utf8]{inputenc} % harmless on newer LaTeX
    \usepackage{lmodern}
    \usepackage[english]{babel}
    \usepackage{inconsolata} % better mono font
\else
    \usepackage{fontspec}
    \defaultfontfeatures{Ligatures=TeX,Scale=MatchLowercase}
    \setmainfont{Latin Modern Roman}
    \setsansfont{Latin Modern Sans}
    \setmonofont{Latin Modern Mono}
    \usepackage{polyglossia}
    \setmainlanguage{english}
\fi

% Page geometry and layout
\usepackage[margin=1in]{geometry}
\usepackage{microtype}
\usepackage{setspace}
% \onehalfspacing % uncomment for 1.5 line spacing

% Graphics and color
\usepackage{xcolor}
\usepackage{graphicx}
\graphicspath{{figures/}}
\usepackage{tikz}
\usetikzlibrary{arrows.meta,positioning,calc,fit,patterns,decorations.pathmorphing}
\usepackage{pgfplots}
\pgfplotsset{compat=1.18}

% Tables
\usepackage{booktabs}
\usepackage{tabularx}
\usepackage{array}
\newcolumntype{Y}{>{\centering\arraybackslash}X}

% Lists
\usepackage{enumitem}
\setlist{itemsep=0.25em, topsep=0.5em}

% Math packages
\usepackage{amsmath,amssymb,mathtools,bm, nicefrac}
\allowdisplaybreaks
\usepackage{siunitx}
\sisetup{detect-all=true, per-mode=symbol, group-minimum-digits=4}

% Theorems
\usepackage{amsthm}
\numberwithin{equation}{section}
\newtheorem{theorem}{Theorem}[section]
\newtheorem{lemma}[theorem]{Lemma}
\newtheorem{proposition}[theorem]{Proposition}
\theoremstyle{definition}
\newtheorem{definition}[theorem]{Definition}
\theoremstyle{remark}
\newtheorem{remark}[theorem]{Remark}
\newtheorem{example}[theorem]{Example}

% Code listings
\usepackage{listings}
\lstdefinestyle{code}{
    basicstyle=\ttfamily\small,
    numbers=left,
    numberstyle=\scriptsize\color{gray},
    stepnumber=1,
    numbersep=8pt,
    showstringspaces=false,
    breaklines=true,
    frame=lines,
    tabsize=2,
    captionpos=b,
    keywordstyle=\color{blue!70!black}\bfseries,
    commentstyle=\color{green!50!black}\itshape,
    stringstyle=\color{orange!60!black},
}
\lstset{style=code}

% Captions and subfigures
\usepackage[labelfont=bf]{caption}
\usepackage{subcaption}

% Hyperlinks and clever references
\usepackage[hidelinks]{hyperref}
\hypersetup{
    pdftitle={},
    pdfauthor={},
    pdfsubject={},
    pdfcreator={LaTeX},
    pdfkeywords={}
}
\usepackage[capitalize,noabbrev]{cleveref}

% Citations (BibLaTeX)
\usepackage{csquotes}
\usepackage[
    backend=biber,
    style=authoryear,
    natbib=true,
    giveninits=true,
    maxbibnames=99,
    uniquename=init
]{biblatex}
% \addbibresource{references.bib}

% Headers/footers
\usepackage{fancyhdr}
\pagestyle{fancy}
\fancyhf{}
\lhead{\leftmark}
\rhead{\thepage}

% Utility packages
\usepackage{lastpage}
\usepackage[textsize=footnotesize]{todonotes} % \todo{...}

% Paragraph spacing
\setlength{\parskip}{0.5em}
\setlength{\parindent}{0pt}

% TOC and numbering depth
\setcounter{secnumdepth}{3}
\setcounter{tocdepth}{2}

% Common macros
\newcommand{\R}{\mathbb{R}}
\newcommand{\N}{\mathbb{N}}
\newcommand{\Z}{\mathbb{Z}}
\newcommand{\Q}{\mathbb{Q}}
\newcommand{\E}{\mathbb{E}}
\newcommand{\Var}{\operatorname{Var}}
\newcommand{\Cov}{\operatorname{Cov}}
\newcommand{\argmin}{\operatorname*{arg\,min}}
\newcommand{\argmax}{\operatorname*{arg\,max}}
\newcommand{\calC}{\mathcal{C}}
\newcommand{\calF}{\mathcal{F}}
\newcommand{\norm}[1]{\left\lVert #1 \right\rVert}
\newcommand{\abs}[1]{\left\lvert #1 \right\rvert}
\newcommand{\set}[1]{\left\{ #1 \right\}}
\newcommand{\ip}[2]{\left\langle #1,\, #2 \right\rangle}
\DeclarePairedDelimiter{\ceil}{\lceil}{\rceil}
\DeclarePairedDelimiter{\floor}{\lfloor}{\rfloor}

% Problem environment
\usepackage{tcolorbox}
\newcounter{problemCounter}
\newtcolorbox{problem}[2][]{colback=white, colframe=black, boxrule=0.5mm, arc=4mm, auto outer arc, title={\ifstrempty{#1}{Problem \stepcounter{problemCounter}\theproblemCounter}{#1}}}


% Draft helpers
\newif\ifdraft
\draftfalse
% \drafttrue
\ifdraft
    \overfullrule=2pt
    \usepackage[inline]{showlabels}
    \usepackage{refcheck}
\fi

% Usage:
% \documentclass{article}
% \input{preamble}
% \title{Title}\author{Author}\date{\today}
% \begin{document}\maketitle
% ...
% \printbibliography % if using BibLaTeX
% \end{document} after \documentclass.
% Engine detection
\usepackage{iftex}
\ifPDFTeX
    \usepackage[T1]{fontenc}
    \usepackage[utf8]{inputenc} % harmless on newer LaTeX
    \usepackage{lmodern}
    \usepackage[english]{babel}
    \usepackage{inconsolata} % better mono font
\else
    \usepackage{fontspec}
    \defaultfontfeatures{Ligatures=TeX,Scale=MatchLowercase}
    \setmainfont{Latin Modern Roman}
    \setsansfont{Latin Modern Sans}
    \setmonofont{Latin Modern Mono}
    \usepackage{polyglossia}
    \setmainlanguage{english}
\fi

% Page geometry and layout
\usepackage[margin=1in]{geometry}
\usepackage{microtype}
\usepackage{setspace}
% \onehalfspacing % uncomment for 1.5 line spacing

% Graphics and color
\usepackage{xcolor}
\usepackage{graphicx}
\graphicspath{{figures/}}
\usepackage{tikz}
\usetikzlibrary{arrows.meta,positioning,calc,fit,patterns,decorations.pathmorphing}
\usepackage{pgfplots}
\pgfplotsset{compat=1.18}

% Tables
\usepackage{booktabs}
\usepackage{tabularx}
\usepackage{array}
\newcolumntype{Y}{>{\centering\arraybackslash}X}

% Lists
\usepackage{enumitem}
\setlist{itemsep=0.25em, topsep=0.5em}

% Math packages
\usepackage{amsmath,amssymb,mathtools,bm, nicefrac}
\allowdisplaybreaks
\usepackage{siunitx}
\sisetup{detect-all=true, per-mode=symbol, group-minimum-digits=4}

% Theorems
\usepackage{amsthm}
\numberwithin{equation}{section}
\newtheorem{theorem}{Theorem}[section]
\newtheorem{lemma}[theorem]{Lemma}
\newtheorem{proposition}[theorem]{Proposition}
\theoremstyle{definition}
\newtheorem{definition}[theorem]{Definition}
\theoremstyle{remark}
\newtheorem{remark}[theorem]{Remark}
\newtheorem{example}[theorem]{Example}

% Code listings
\usepackage{listings}
\lstdefinestyle{code}{
    basicstyle=\ttfamily\small,
    numbers=left,
    numberstyle=\scriptsize\color{gray},
    stepnumber=1,
    numbersep=8pt,
    showstringspaces=false,
    breaklines=true,
    frame=lines,
    tabsize=2,
    captionpos=b,
    keywordstyle=\color{blue!70!black}\bfseries,
    commentstyle=\color{green!50!black}\itshape,
    stringstyle=\color{orange!60!black},
}
\lstset{style=code}

% Captions and subfigures
\usepackage[labelfont=bf]{caption}
\usepackage{subcaption}

% Hyperlinks and clever references
\usepackage[hidelinks]{hyperref}
\hypersetup{
    pdftitle={},
    pdfauthor={},
    pdfsubject={},
    pdfcreator={LaTeX},
    pdfkeywords={}
}
\usepackage[capitalize,noabbrev]{cleveref}

% Citations (BibLaTeX)
\usepackage{csquotes}
\usepackage[
    backend=biber,
    style=authoryear,
    natbib=true,
    giveninits=true,
    maxbibnames=99,
    uniquename=init
]{biblatex}
% \addbibresource{references.bib}

% Headers/footers
\usepackage{fancyhdr}
\pagestyle{fancy}
\fancyhf{}
\lhead{\leftmark}
\rhead{\thepage}

% Utility packages
\usepackage{lastpage}
\usepackage[textsize=footnotesize]{todonotes} % \todo{...}

% Paragraph spacing
\setlength{\parskip}{0.5em}
\setlength{\parindent}{0pt}

% TOC and numbering depth
\setcounter{secnumdepth}{3}
\setcounter{tocdepth}{2}

% Common macros
\newcommand{\R}{\mathbb{R}}
\newcommand{\N}{\mathbb{N}}
\newcommand{\Z}{\mathbb{Z}}
\newcommand{\Q}{\mathbb{Q}}
\newcommand{\E}{\mathbb{E}}
\newcommand{\Var}{\operatorname{Var}}
\newcommand{\Cov}{\operatorname{Cov}}
\newcommand{\argmin}{\operatorname*{arg\,min}}
\newcommand{\argmax}{\operatorname*{arg\,max}}
\newcommand{\calC}{\mathcal{C}}
\newcommand{\calF}{\mathcal{F}}
\newcommand{\norm}[1]{\left\lVert #1 \right\rVert}
\newcommand{\abs}[1]{\left\lvert #1 \right\rvert}
\newcommand{\set}[1]{\left\{ #1 \right\}}
\newcommand{\ip}[2]{\left\langle #1,\, #2 \right\rangle}
\DeclarePairedDelimiter{\ceil}{\lceil}{\rceil}
\DeclarePairedDelimiter{\floor}{\lfloor}{\rfloor}

% Problem environment
\usepackage{tcolorbox}
\newcounter{problemCounter}
\newtcolorbox{problem}[2][]{colback=white, colframe=black, boxrule=0.5mm, arc=4mm, auto outer arc, title={\ifstrempty{#1}{Problem \stepcounter{problemCounter}\theproblemCounter}{#1}}}


% Draft helpers
\newif\ifdraft
\draftfalse
% \drafttrue
\ifdraft
    \overfullrule=2pt
    \usepackage[inline]{showlabels}
    \usepackage{refcheck}
\fi

% Usage:
% \documentclass{article}
% % Common LaTeX preamble. Save as preamble.tex and \input{preamble} after \documentclass.
% Engine detection
\usepackage{iftex}
\ifPDFTeX
    \usepackage[T1]{fontenc}
    \usepackage[utf8]{inputenc} % harmless on newer LaTeX
    \usepackage{lmodern}
    \usepackage[english]{babel}
    \usepackage{inconsolata} % better mono font
\else
    \usepackage{fontspec}
    \defaultfontfeatures{Ligatures=TeX,Scale=MatchLowercase}
    \setmainfont{Latin Modern Roman}
    \setsansfont{Latin Modern Sans}
    \setmonofont{Latin Modern Mono}
    \usepackage{polyglossia}
    \setmainlanguage{english}
\fi

% Page geometry and layout
\usepackage[margin=1in]{geometry}
\usepackage{microtype}
\usepackage{setspace}
% \onehalfspacing % uncomment for 1.5 line spacing

% Graphics and color
\usepackage{xcolor}
\usepackage{graphicx}
\graphicspath{{figures/}}
\usepackage{tikz}
\usetikzlibrary{arrows.meta,positioning,calc,fit,patterns,decorations.pathmorphing}
\usepackage{pgfplots}
\pgfplotsset{compat=1.18}

% Tables
\usepackage{booktabs}
\usepackage{tabularx}
\usepackage{array}
\newcolumntype{Y}{>{\centering\arraybackslash}X}

% Lists
\usepackage{enumitem}
\setlist{itemsep=0.25em, topsep=0.5em}

% Math packages
\usepackage{amsmath,amssymb,mathtools,bm, nicefrac}
\allowdisplaybreaks
\usepackage{siunitx}
\sisetup{detect-all=true, per-mode=symbol, group-minimum-digits=4}

% Theorems
\usepackage{amsthm}
\numberwithin{equation}{section}
\newtheorem{theorem}{Theorem}[section]
\newtheorem{lemma}[theorem]{Lemma}
\newtheorem{proposition}[theorem]{Proposition}
\theoremstyle{definition}
\newtheorem{definition}[theorem]{Definition}
\theoremstyle{remark}
\newtheorem{remark}[theorem]{Remark}
\newtheorem{example}[theorem]{Example}

% Code listings
\usepackage{listings}
\lstdefinestyle{code}{
    basicstyle=\ttfamily\small,
    numbers=left,
    numberstyle=\scriptsize\color{gray},
    stepnumber=1,
    numbersep=8pt,
    showstringspaces=false,
    breaklines=true,
    frame=lines,
    tabsize=2,
    captionpos=b,
    keywordstyle=\color{blue!70!black}\bfseries,
    commentstyle=\color{green!50!black}\itshape,
    stringstyle=\color{orange!60!black},
}
\lstset{style=code}

% Captions and subfigures
\usepackage[labelfont=bf]{caption}
\usepackage{subcaption}

% Hyperlinks and clever references
\usepackage[hidelinks]{hyperref}
\hypersetup{
    pdftitle={},
    pdfauthor={},
    pdfsubject={},
    pdfcreator={LaTeX},
    pdfkeywords={}
}
\usepackage[capitalize,noabbrev]{cleveref}

% Citations (BibLaTeX)
\usepackage{csquotes}
\usepackage[
    backend=biber,
    style=authoryear,
    natbib=true,
    giveninits=true,
    maxbibnames=99,
    uniquename=init
]{biblatex}
% \addbibresource{references.bib}

% Headers/footers
\usepackage{fancyhdr}
\pagestyle{fancy}
\fancyhf{}
\lhead{\leftmark}
\rhead{\thepage}

% Utility packages
\usepackage{lastpage}
\usepackage[textsize=footnotesize]{todonotes} % \todo{...}

% Paragraph spacing
\setlength{\parskip}{0.5em}
\setlength{\parindent}{0pt}

% TOC and numbering depth
\setcounter{secnumdepth}{3}
\setcounter{tocdepth}{2}

% Common macros
\newcommand{\R}{\mathbb{R}}
\newcommand{\N}{\mathbb{N}}
\newcommand{\Z}{\mathbb{Z}}
\newcommand{\Q}{\mathbb{Q}}
\newcommand{\E}{\mathbb{E}}
\newcommand{\Var}{\operatorname{Var}}
\newcommand{\Cov}{\operatorname{Cov}}
\newcommand{\argmin}{\operatorname*{arg\,min}}
\newcommand{\argmax}{\operatorname*{arg\,max}}
\newcommand{\calC}{\mathcal{C}}
\newcommand{\calF}{\mathcal{F}}
\newcommand{\norm}[1]{\left\lVert #1 \right\rVert}
\newcommand{\abs}[1]{\left\lvert #1 \right\rvert}
\newcommand{\set}[1]{\left\{ #1 \right\}}
\newcommand{\ip}[2]{\left\langle #1,\, #2 \right\rangle}
\DeclarePairedDelimiter{\ceil}{\lceil}{\rceil}
\DeclarePairedDelimiter{\floor}{\lfloor}{\rfloor}

% Problem environment
\usepackage{tcolorbox}
\newcounter{problemCounter}
\newtcolorbox{problem}[2][]{colback=white, colframe=black, boxrule=0.5mm, arc=4mm, auto outer arc, title={\ifstrempty{#1}{Problem \stepcounter{problemCounter}\theproblemCounter}{#1}}}


% Draft helpers
\newif\ifdraft
\draftfalse
% \drafttrue
\ifdraft
    \overfullrule=2pt
    \usepackage[inline]{showlabels}
    \usepackage{refcheck}
\fi

% Usage:
% \documentclass{article}
% \input{preamble}
% \title{Title}\author{Author}\date{\today}
% \begin{document}\maketitle
% ...
% \printbibliography % if using BibLaTeX
% \end{document}
% \title{Title}\author{Author}\date{\today}
% \begin{document}\maketitle
% ...
% \printbibliography % if using BibLaTeX
% \end{document} after \documentclass.
% Engine detection
\usepackage{iftex}
\ifPDFTeX
    \usepackage[T1]{fontenc}
    \usepackage[utf8]{inputenc} % harmless on newer LaTeX
    \usepackage{lmodern}
    \usepackage[english]{babel}
    \usepackage{inconsolata} % better mono font
\else
    \usepackage{fontspec}
    \defaultfontfeatures{Ligatures=TeX,Scale=MatchLowercase}
    \setmainfont{Latin Modern Roman}
    \setsansfont{Latin Modern Sans}
    \setmonofont{Latin Modern Mono}
    \usepackage{polyglossia}
    \setmainlanguage{english}
\fi

% Page geometry and layout
\usepackage[margin=1in]{geometry}
\usepackage{microtype}
\usepackage{setspace}
% \onehalfspacing % uncomment for 1.5 line spacing

% Graphics and color
\usepackage{xcolor}
\usepackage{graphicx}
\graphicspath{{figures/}}
\usepackage{tikz}
\usetikzlibrary{arrows.meta,positioning,calc,fit,patterns,decorations.pathmorphing}
\usepackage{pgfplots}
\pgfplotsset{compat=1.18}

% Tables
\usepackage{booktabs}
\usepackage{tabularx}
\usepackage{array}
\newcolumntype{Y}{>{\centering\arraybackslash}X}

% Lists
\usepackage{enumitem}
\setlist{itemsep=0.25em, topsep=0.5em}

% Math packages
\usepackage{amsmath,amssymb,mathtools,bm, nicefrac}
\allowdisplaybreaks
\usepackage{siunitx}
\sisetup{detect-all=true, per-mode=symbol, group-minimum-digits=4}

% Theorems
\usepackage{amsthm}
\numberwithin{equation}{section}
\newtheorem{theorem}{Theorem}[section]
\newtheorem{lemma}[theorem]{Lemma}
\newtheorem{proposition}[theorem]{Proposition}
\theoremstyle{definition}
\newtheorem{definition}[theorem]{Definition}
\theoremstyle{remark}
\newtheorem{remark}[theorem]{Remark}
\newtheorem{example}[theorem]{Example}

% Code listings
\usepackage{listings}
\lstdefinestyle{code}{
    basicstyle=\ttfamily\small,
    numbers=left,
    numberstyle=\scriptsize\color{gray},
    stepnumber=1,
    numbersep=8pt,
    showstringspaces=false,
    breaklines=true,
    frame=lines,
    tabsize=2,
    captionpos=b,
    keywordstyle=\color{blue!70!black}\bfseries,
    commentstyle=\color{green!50!black}\itshape,
    stringstyle=\color{orange!60!black},
}
\lstset{style=code}

% Captions and subfigures
\usepackage[labelfont=bf]{caption}
\usepackage{subcaption}

% Hyperlinks and clever references
\usepackage[hidelinks]{hyperref}
\hypersetup{
    pdftitle={},
    pdfauthor={},
    pdfsubject={},
    pdfcreator={LaTeX},
    pdfkeywords={}
}
\usepackage[capitalize,noabbrev]{cleveref}

% Citations (BibLaTeX)
\usepackage{csquotes}
\usepackage[
    backend=biber,
    style=authoryear,
    natbib=true,
    giveninits=true,
    maxbibnames=99,
    uniquename=init
]{biblatex}
% \addbibresource{references.bib}

% Headers/footers
\usepackage{fancyhdr}
\pagestyle{fancy}
\fancyhf{}
\lhead{\leftmark}
\rhead{\thepage}

% Utility packages
\usepackage{lastpage}
\usepackage[textsize=footnotesize]{todonotes} % \todo{...}

% Paragraph spacing
\setlength{\parskip}{0.5em}
\setlength{\parindent}{0pt}

% TOC and numbering depth
\setcounter{secnumdepth}{3}
\setcounter{tocdepth}{2}

% Common macros
\newcommand{\R}{\mathbb{R}}
\newcommand{\N}{\mathbb{N}}
\newcommand{\Z}{\mathbb{Z}}
\newcommand{\Q}{\mathbb{Q}}
\newcommand{\E}{\mathbb{E}}
\newcommand{\Var}{\operatorname{Var}}
\newcommand{\Cov}{\operatorname{Cov}}
\newcommand{\argmin}{\operatorname*{arg\,min}}
\newcommand{\argmax}{\operatorname*{arg\,max}}
\newcommand{\calC}{\mathcal{C}}
\newcommand{\calF}{\mathcal{F}}
\newcommand{\norm}[1]{\left\lVert #1 \right\rVert}
\newcommand{\abs}[1]{\left\lvert #1 \right\rvert}
\newcommand{\set}[1]{\left\{ #1 \right\}}
\newcommand{\ip}[2]{\left\langle #1,\, #2 \right\rangle}
\DeclarePairedDelimiter{\ceil}{\lceil}{\rceil}
\DeclarePairedDelimiter{\floor}{\lfloor}{\rfloor}

% Problem environment
\usepackage{tcolorbox}
\newcounter{problemCounter}
\newtcolorbox{problem}[2][]{colback=white, colframe=black, boxrule=0.5mm, arc=4mm, auto outer arc, title={\ifstrempty{#1}{Problem \stepcounter{problemCounter}\theproblemCounter}{#1}}}


% Draft helpers
\newif\ifdraft
\draftfalse
% \drafttrue
\ifdraft
    \overfullrule=2pt
    \usepackage[inline]{showlabels}
    \usepackage{refcheck}
\fi

% Usage:
% \documentclass{article}
% % Common LaTeX preamble. Save as preamble.tex and % Common LaTeX preamble. Save as preamble.tex and \input{preamble} after \documentclass.
% Engine detection
\usepackage{iftex}
\ifPDFTeX
    \usepackage[T1]{fontenc}
    \usepackage[utf8]{inputenc} % harmless on newer LaTeX
    \usepackage{lmodern}
    \usepackage[english]{babel}
    \usepackage{inconsolata} % better mono font
\else
    \usepackage{fontspec}
    \defaultfontfeatures{Ligatures=TeX,Scale=MatchLowercase}
    \setmainfont{Latin Modern Roman}
    \setsansfont{Latin Modern Sans}
    \setmonofont{Latin Modern Mono}
    \usepackage{polyglossia}
    \setmainlanguage{english}
\fi

% Page geometry and layout
\usepackage[margin=1in]{geometry}
\usepackage{microtype}
\usepackage{setspace}
% \onehalfspacing % uncomment for 1.5 line spacing

% Graphics and color
\usepackage{xcolor}
\usepackage{graphicx}
\graphicspath{{figures/}}
\usepackage{tikz}
\usetikzlibrary{arrows.meta,positioning,calc,fit,patterns,decorations.pathmorphing}
\usepackage{pgfplots}
\pgfplotsset{compat=1.18}

% Tables
\usepackage{booktabs}
\usepackage{tabularx}
\usepackage{array}
\newcolumntype{Y}{>{\centering\arraybackslash}X}

% Lists
\usepackage{enumitem}
\setlist{itemsep=0.25em, topsep=0.5em}

% Math packages
\usepackage{amsmath,amssymb,mathtools,bm, nicefrac}
\allowdisplaybreaks
\usepackage{siunitx}
\sisetup{detect-all=true, per-mode=symbol, group-minimum-digits=4}

% Theorems
\usepackage{amsthm}
\numberwithin{equation}{section}
\newtheorem{theorem}{Theorem}[section]
\newtheorem{lemma}[theorem]{Lemma}
\newtheorem{proposition}[theorem]{Proposition}
\theoremstyle{definition}
\newtheorem{definition}[theorem]{Definition}
\theoremstyle{remark}
\newtheorem{remark}[theorem]{Remark}
\newtheorem{example}[theorem]{Example}

% Code listings
\usepackage{listings}
\lstdefinestyle{code}{
    basicstyle=\ttfamily\small,
    numbers=left,
    numberstyle=\scriptsize\color{gray},
    stepnumber=1,
    numbersep=8pt,
    showstringspaces=false,
    breaklines=true,
    frame=lines,
    tabsize=2,
    captionpos=b,
    keywordstyle=\color{blue!70!black}\bfseries,
    commentstyle=\color{green!50!black}\itshape,
    stringstyle=\color{orange!60!black},
}
\lstset{style=code}

% Captions and subfigures
\usepackage[labelfont=bf]{caption}
\usepackage{subcaption}

% Hyperlinks and clever references
\usepackage[hidelinks]{hyperref}
\hypersetup{
    pdftitle={},
    pdfauthor={},
    pdfsubject={},
    pdfcreator={LaTeX},
    pdfkeywords={}
}
\usepackage[capitalize,noabbrev]{cleveref}

% Citations (BibLaTeX)
\usepackage{csquotes}
\usepackage[
    backend=biber,
    style=authoryear,
    natbib=true,
    giveninits=true,
    maxbibnames=99,
    uniquename=init
]{biblatex}
% \addbibresource{references.bib}

% Headers/footers
\usepackage{fancyhdr}
\pagestyle{fancy}
\fancyhf{}
\lhead{\leftmark}
\rhead{\thepage}

% Utility packages
\usepackage{lastpage}
\usepackage[textsize=footnotesize]{todonotes} % \todo{...}

% Paragraph spacing
\setlength{\parskip}{0.5em}
\setlength{\parindent}{0pt}

% TOC and numbering depth
\setcounter{secnumdepth}{3}
\setcounter{tocdepth}{2}

% Common macros
\newcommand{\R}{\mathbb{R}}
\newcommand{\N}{\mathbb{N}}
\newcommand{\Z}{\mathbb{Z}}
\newcommand{\Q}{\mathbb{Q}}
\newcommand{\E}{\mathbb{E}}
\newcommand{\Var}{\operatorname{Var}}
\newcommand{\Cov}{\operatorname{Cov}}
\newcommand{\argmin}{\operatorname*{arg\,min}}
\newcommand{\argmax}{\operatorname*{arg\,max}}
\newcommand{\calC}{\mathcal{C}}
\newcommand{\calF}{\mathcal{F}}
\newcommand{\norm}[1]{\left\lVert #1 \right\rVert}
\newcommand{\abs}[1]{\left\lvert #1 \right\rvert}
\newcommand{\set}[1]{\left\{ #1 \right\}}
\newcommand{\ip}[2]{\left\langle #1,\, #2 \right\rangle}
\DeclarePairedDelimiter{\ceil}{\lceil}{\rceil}
\DeclarePairedDelimiter{\floor}{\lfloor}{\rfloor}

% Problem environment
\usepackage{tcolorbox}
\newcounter{problemCounter}
\newtcolorbox{problem}[2][]{colback=white, colframe=black, boxrule=0.5mm, arc=4mm, auto outer arc, title={\ifstrempty{#1}{Problem \stepcounter{problemCounter}\theproblemCounter}{#1}}}


% Draft helpers
\newif\ifdraft
\draftfalse
% \drafttrue
\ifdraft
    \overfullrule=2pt
    \usepackage[inline]{showlabels}
    \usepackage{refcheck}
\fi

% Usage:
% \documentclass{article}
% \input{preamble}
% \title{Title}\author{Author}\date{\today}
% \begin{document}\maketitle
% ...
% \printbibliography % if using BibLaTeX
% \end{document} after \documentclass.
% Engine detection
\usepackage{iftex}
\ifPDFTeX
    \usepackage[T1]{fontenc}
    \usepackage[utf8]{inputenc} % harmless on newer LaTeX
    \usepackage{lmodern}
    \usepackage[english]{babel}
    \usepackage{inconsolata} % better mono font
\else
    \usepackage{fontspec}
    \defaultfontfeatures{Ligatures=TeX,Scale=MatchLowercase}
    \setmainfont{Latin Modern Roman}
    \setsansfont{Latin Modern Sans}
    \setmonofont{Latin Modern Mono}
    \usepackage{polyglossia}
    \setmainlanguage{english}
\fi

% Page geometry and layout
\usepackage[margin=1in]{geometry}
\usepackage{microtype}
\usepackage{setspace}
% \onehalfspacing % uncomment for 1.5 line spacing

% Graphics and color
\usepackage{xcolor}
\usepackage{graphicx}
\graphicspath{{figures/}}
\usepackage{tikz}
\usetikzlibrary{arrows.meta,positioning,calc,fit,patterns,decorations.pathmorphing}
\usepackage{pgfplots}
\pgfplotsset{compat=1.18}

% Tables
\usepackage{booktabs}
\usepackage{tabularx}
\usepackage{array}
\newcolumntype{Y}{>{\centering\arraybackslash}X}

% Lists
\usepackage{enumitem}
\setlist{itemsep=0.25em, topsep=0.5em}

% Math packages
\usepackage{amsmath,amssymb,mathtools,bm, nicefrac}
\allowdisplaybreaks
\usepackage{siunitx}
\sisetup{detect-all=true, per-mode=symbol, group-minimum-digits=4}

% Theorems
\usepackage{amsthm}
\numberwithin{equation}{section}
\newtheorem{theorem}{Theorem}[section]
\newtheorem{lemma}[theorem]{Lemma}
\newtheorem{proposition}[theorem]{Proposition}
\theoremstyle{definition}
\newtheorem{definition}[theorem]{Definition}
\theoremstyle{remark}
\newtheorem{remark}[theorem]{Remark}
\newtheorem{example}[theorem]{Example}

% Code listings
\usepackage{listings}
\lstdefinestyle{code}{
    basicstyle=\ttfamily\small,
    numbers=left,
    numberstyle=\scriptsize\color{gray},
    stepnumber=1,
    numbersep=8pt,
    showstringspaces=false,
    breaklines=true,
    frame=lines,
    tabsize=2,
    captionpos=b,
    keywordstyle=\color{blue!70!black}\bfseries,
    commentstyle=\color{green!50!black}\itshape,
    stringstyle=\color{orange!60!black},
}
\lstset{style=code}

% Captions and subfigures
\usepackage[labelfont=bf]{caption}
\usepackage{subcaption}

% Hyperlinks and clever references
\usepackage[hidelinks]{hyperref}
\hypersetup{
    pdftitle={},
    pdfauthor={},
    pdfsubject={},
    pdfcreator={LaTeX},
    pdfkeywords={}
}
\usepackage[capitalize,noabbrev]{cleveref}

% Citations (BibLaTeX)
\usepackage{csquotes}
\usepackage[
    backend=biber,
    style=authoryear,
    natbib=true,
    giveninits=true,
    maxbibnames=99,
    uniquename=init
]{biblatex}
% \addbibresource{references.bib}

% Headers/footers
\usepackage{fancyhdr}
\pagestyle{fancy}
\fancyhf{}
\lhead{\leftmark}
\rhead{\thepage}

% Utility packages
\usepackage{lastpage}
\usepackage[textsize=footnotesize]{todonotes} % \todo{...}

% Paragraph spacing
\setlength{\parskip}{0.5em}
\setlength{\parindent}{0pt}

% TOC and numbering depth
\setcounter{secnumdepth}{3}
\setcounter{tocdepth}{2}

% Common macros
\newcommand{\R}{\mathbb{R}}
\newcommand{\N}{\mathbb{N}}
\newcommand{\Z}{\mathbb{Z}}
\newcommand{\Q}{\mathbb{Q}}
\newcommand{\E}{\mathbb{E}}
\newcommand{\Var}{\operatorname{Var}}
\newcommand{\Cov}{\operatorname{Cov}}
\newcommand{\argmin}{\operatorname*{arg\,min}}
\newcommand{\argmax}{\operatorname*{arg\,max}}
\newcommand{\calC}{\mathcal{C}}
\newcommand{\calF}{\mathcal{F}}
\newcommand{\norm}[1]{\left\lVert #1 \right\rVert}
\newcommand{\abs}[1]{\left\lvert #1 \right\rvert}
\newcommand{\set}[1]{\left\{ #1 \right\}}
\newcommand{\ip}[2]{\left\langle #1,\, #2 \right\rangle}
\DeclarePairedDelimiter{\ceil}{\lceil}{\rceil}
\DeclarePairedDelimiter{\floor}{\lfloor}{\rfloor}

% Problem environment
\usepackage{tcolorbox}
\newcounter{problemCounter}
\newtcolorbox{problem}[2][]{colback=white, colframe=black, boxrule=0.5mm, arc=4mm, auto outer arc, title={\ifstrempty{#1}{Problem \stepcounter{problemCounter}\theproblemCounter}{#1}}}


% Draft helpers
\newif\ifdraft
\draftfalse
% \drafttrue
\ifdraft
    \overfullrule=2pt
    \usepackage[inline]{showlabels}
    \usepackage{refcheck}
\fi

% Usage:
% \documentclass{article}
% % Common LaTeX preamble. Save as preamble.tex and \input{preamble} after \documentclass.
% Engine detection
\usepackage{iftex}
\ifPDFTeX
    \usepackage[T1]{fontenc}
    \usepackage[utf8]{inputenc} % harmless on newer LaTeX
    \usepackage{lmodern}
    \usepackage[english]{babel}
    \usepackage{inconsolata} % better mono font
\else
    \usepackage{fontspec}
    \defaultfontfeatures{Ligatures=TeX,Scale=MatchLowercase}
    \setmainfont{Latin Modern Roman}
    \setsansfont{Latin Modern Sans}
    \setmonofont{Latin Modern Mono}
    \usepackage{polyglossia}
    \setmainlanguage{english}
\fi

% Page geometry and layout
\usepackage[margin=1in]{geometry}
\usepackage{microtype}
\usepackage{setspace}
% \onehalfspacing % uncomment for 1.5 line spacing

% Graphics and color
\usepackage{xcolor}
\usepackage{graphicx}
\graphicspath{{figures/}}
\usepackage{tikz}
\usetikzlibrary{arrows.meta,positioning,calc,fit,patterns,decorations.pathmorphing}
\usepackage{pgfplots}
\pgfplotsset{compat=1.18}

% Tables
\usepackage{booktabs}
\usepackage{tabularx}
\usepackage{array}
\newcolumntype{Y}{>{\centering\arraybackslash}X}

% Lists
\usepackage{enumitem}
\setlist{itemsep=0.25em, topsep=0.5em}

% Math packages
\usepackage{amsmath,amssymb,mathtools,bm, nicefrac}
\allowdisplaybreaks
\usepackage{siunitx}
\sisetup{detect-all=true, per-mode=symbol, group-minimum-digits=4}

% Theorems
\usepackage{amsthm}
\numberwithin{equation}{section}
\newtheorem{theorem}{Theorem}[section]
\newtheorem{lemma}[theorem]{Lemma}
\newtheorem{proposition}[theorem]{Proposition}
\theoremstyle{definition}
\newtheorem{definition}[theorem]{Definition}
\theoremstyle{remark}
\newtheorem{remark}[theorem]{Remark}
\newtheorem{example}[theorem]{Example}

% Code listings
\usepackage{listings}
\lstdefinestyle{code}{
    basicstyle=\ttfamily\small,
    numbers=left,
    numberstyle=\scriptsize\color{gray},
    stepnumber=1,
    numbersep=8pt,
    showstringspaces=false,
    breaklines=true,
    frame=lines,
    tabsize=2,
    captionpos=b,
    keywordstyle=\color{blue!70!black}\bfseries,
    commentstyle=\color{green!50!black}\itshape,
    stringstyle=\color{orange!60!black},
}
\lstset{style=code}

% Captions and subfigures
\usepackage[labelfont=bf]{caption}
\usepackage{subcaption}

% Hyperlinks and clever references
\usepackage[hidelinks]{hyperref}
\hypersetup{
    pdftitle={},
    pdfauthor={},
    pdfsubject={},
    pdfcreator={LaTeX},
    pdfkeywords={}
}
\usepackage[capitalize,noabbrev]{cleveref}

% Citations (BibLaTeX)
\usepackage{csquotes}
\usepackage[
    backend=biber,
    style=authoryear,
    natbib=true,
    giveninits=true,
    maxbibnames=99,
    uniquename=init
]{biblatex}
% \addbibresource{references.bib}

% Headers/footers
\usepackage{fancyhdr}
\pagestyle{fancy}
\fancyhf{}
\lhead{\leftmark}
\rhead{\thepage}

% Utility packages
\usepackage{lastpage}
\usepackage[textsize=footnotesize]{todonotes} % \todo{...}

% Paragraph spacing
\setlength{\parskip}{0.5em}
\setlength{\parindent}{0pt}

% TOC and numbering depth
\setcounter{secnumdepth}{3}
\setcounter{tocdepth}{2}

% Common macros
\newcommand{\R}{\mathbb{R}}
\newcommand{\N}{\mathbb{N}}
\newcommand{\Z}{\mathbb{Z}}
\newcommand{\Q}{\mathbb{Q}}
\newcommand{\E}{\mathbb{E}}
\newcommand{\Var}{\operatorname{Var}}
\newcommand{\Cov}{\operatorname{Cov}}
\newcommand{\argmin}{\operatorname*{arg\,min}}
\newcommand{\argmax}{\operatorname*{arg\,max}}
\newcommand{\calC}{\mathcal{C}}
\newcommand{\calF}{\mathcal{F}}
\newcommand{\norm}[1]{\left\lVert #1 \right\rVert}
\newcommand{\abs}[1]{\left\lvert #1 \right\rvert}
\newcommand{\set}[1]{\left\{ #1 \right\}}
\newcommand{\ip}[2]{\left\langle #1,\, #2 \right\rangle}
\DeclarePairedDelimiter{\ceil}{\lceil}{\rceil}
\DeclarePairedDelimiter{\floor}{\lfloor}{\rfloor}

% Problem environment
\usepackage{tcolorbox}
\newcounter{problemCounter}
\newtcolorbox{problem}[2][]{colback=white, colframe=black, boxrule=0.5mm, arc=4mm, auto outer arc, title={\ifstrempty{#1}{Problem \stepcounter{problemCounter}\theproblemCounter}{#1}}}


% Draft helpers
\newif\ifdraft
\draftfalse
% \drafttrue
\ifdraft
    \overfullrule=2pt
    \usepackage[inline]{showlabels}
    \usepackage{refcheck}
\fi

% Usage:
% \documentclass{article}
% \input{preamble}
% \title{Title}\author{Author}\date{\today}
% \begin{document}\maketitle
% ...
% \printbibliography % if using BibLaTeX
% \end{document}
% \title{Title}\author{Author}\date{\today}
% \begin{document}\maketitle
% ...
% \printbibliography % if using BibLaTeX
% \end{document}
% \title{Title}\author{Author}\date{\today}
% \begin{document}\maketitle
% ...
% \printbibliography % if using BibLaTeX
% \end{document}

\begin{document}
\begin{center}
    Math 540
    \hfill Homework 6
    \hfill \textit{Stephen Cornelius}
\end{center}

\begin{enumerate} % [1.)]
    \item Let $V$ be a vector space and $V_j < V$, $j = 1, \dots, k$ subspaces.
    \begin{enumerate} % [(a)]
        \item \textbf{Definition: }We say that $V$ is a direct sum (denoted $V = V_1 \oplus V_2 \oplus \cdots \oplus V_k$) of the subspaces $V_j$ if every $v \in V$ can be written uniquely as $v = v_1 + v_2 + \cdots + v_k$ with $v_j \in V_j$. 
        \item 
        \begin{proof}
            ($1 \implies 2$): Let $V$ be the direct sum of the subspaces $V_j$. Then for any $v \in V$, there exist unique $v_j \in V_j$ such that $v = v_1 + v_2 + \cdots + v_k$. Define the projectors $P_j: V \to V_j$ by $P_j(v) = v_j$. Then we have
            \[
                (P_1 + P_2 + \cdots + P_k)(v) = P_1(v) + P_2(v) + \cdots + P_k(v) = v_1 + v_2 + \cdots + v_k = v,
            \]
            so $Id_V = P_1 + P_2 + \cdots + P_k$. Furthermore, for $i \neq j$, we have
            \[
                P_i(P_j(v)) = P_i(v_j) = 0,
            \]
            since $v_j \in V_j$ and $P_i$ projects onto $V_i$. Thus, $P_i \circ P_j = 0$ for $i \neq j$.

            ($2 \implies 1$): Now suppose there exist projectors $P_i$ on the subspaces $V_i$ such that $Id_V = P_1 + P_2 + \cdots + P_k$ and $P_i \circ P_j = 0$ for $i \neq j$. For any $v \in V$, we can write
            \[
                v = Id_V(v) = (P_1 + P_2 + \cdots + P_k)(v) = P_1(v) + P_2(v) + \cdots + P_k(v).
            \]
            Since each $P_i(v) \in V_i$, this expresses $v$ as a sum of elements from the subspaces $V_i$. To show uniqueness, suppose
            \[
                v = v_1 + v_2 + \cdots + v_k = w_1 + w_2 + \cdots + w_k,
            \]
            with $v_i, w_i \in V_i$. Applying $P_j$ to both sides, we get
            \[
                P_j(v) = P_j(v_1 + \cdots + v_k) = v_j,
            \]
            and similarly
            \[
                P_j(v) = P_j(w_1 + \cdots + w_k) = w_j.
            \]
            Hence, $v_j = w_j$ for all $j$, proving uniqueness. Therefore, $V = V_1 \oplus V_2 \oplus \cdots \oplus V_k$.
        \end{proof}
        \item % Suppose $V$ if f.d. and T: V -> V, linear transf. Show that TFAE: 
        % 1) V = \bigoplus_{\lambda \in spec(T)} V_\lambda direct sum of eigenspaces of T (That is T is diagonalizable)
        % 2) there are projectors P_\lambda for some collection \Lambda of \lambda's from F, such that 1) Id_V = \sum_{\lambda \in \Lambda} P_\lambda, 2) P_\lambda \circ P_\mu = 0 for \lambda \neq \mu, 3) T = \sum_{\lambda \in \Lambda} \lambda P_\lambda
        % Moreover show thta in this case \Lambda = spec(T) and for each \mu \in spec(T) we have P_\mu = \prod_{\mu \neq \lambda \in spec(t)}^{} \left( \frac{T - \lambda \cdot Id}{\mu - \lambda} \right)
        \begin{proof}
            ($1 \implies 2$): Suppose $V = \bigoplus_{\lambda \in \text{spec}(T)} V_\lambda$. For each eigenvalue $\lambda$, define the projector $P_\lambda: V \to V_\lambda$ by projecting onto the eigenspace $V_\lambda$. Then, by the direct sum property, we have
            \[
                Id_V = \sum_{\lambda \in \text{spec}(T)} P_\lambda,
            \]
            and for $\lambda \neq \mu$, $P_\lambda \circ P_\mu = 0$. Furthermore, for any $v \in V$,
            \[
                T(v) = T\left(\sum_{\lambda} P_\lambda(v)\right) = \sum_{\lambda} T(P_\lambda(v)) = \sum_{\lambda} \lambda P_\lambda(v),
            \]
            so
            \[
                T = \sum_{\lambda} \lambda P_\lambda.
            \]

            ($2 \implies 1$): Now suppose there exist projectors $P_\lambda$ satisfying the given conditions. For any $v \in V$, we can write
            \[
                v = Id_V(v) = \sum_{\lambda} P_\lambda(v).
            \]
            Each $P_\lambda(v) \in V_\lambda$, so this expresses $v$ as a sum of elements from the eigenspaces. To show uniqueness, suppose
            \[
                v = v_{\lambda_1} + v_{\lambda_2} + \cdots + v_{\lambda_k} = w_{\lambda_1} + w_{\lambda_2} + \cdots + w_{\lambda_k},
            \]
            with $v_{\lambda_i}, w_{\lambda_i} \in V_{\lambda_i}$. Applying $P_{\mu}$ to both sides, we get
            \[
                P_{\mu}(v) = P_{\mu}(v_{\lambda_1} + \cdots + v_{\lambda_k}) = v_{\mu},
            \]
            and similarly
            \[
                P_{\mu}(v) = P_{\mu}(w_{\lambda_1} + \cdots + w_{\lambda_k}) = w_{\mu}.
            \]
            Hence, $v_{\mu} = w_{\mu}$ for all $\mu$, proving uniqueness. Therefore, $V = \bigoplus_{\lambda \in \text{spec}(T)} V_\lambda$.
            Moreover, for each $\mu \in \text{spec}(T)$, we can express $P_\mu$ as
            \[
                P_\mu = \prod_{\substack{\lambda \in \text{spec}(T) \\ \lambda \neq \mu}} \left( \frac{T - \lambda \cdot Id}{\mu - \lambda} \right),
            \]
            which follows from the properties of the projectors and the definition of the eigenspaces.
        \end{proof}
    \end{enumerate}


    \item Let $\F$ be a field.
    \begin{enumerate}
        \item % Define the ring $\F[X]$ of polynomials with coefficients in $\F$
        \textbf{Definition: }The ring $\F[X]$ of polynomials with coefficients in $\F$ is the set of all expressions of the form
        \[
            p(X) = a_n X^n + a_{n-1} X^{n-1} + \cdots + a_1 X + a_0,
        \]
        where $n \geq 0$, $a_i \in \F$, and $X$ is an indeterminate. Addition and multiplication are defined in the usual way for polynomials.
        
        \item % Show that $\dim(\F[X])$ is infinite
        \begin{proof}
            To show that $\dim(\F[X])$ is infinite, we need to demonstrate that there is no finite basis for the vector space $\F[X]$. Consider the set of monomials $\{1, X, X^2, X^3, \ldots\}$. This set is linearly independent because no finite linear combination of these monomials can equal zero unless all coefficients are zero. 

            Furthermore, any polynomial in $\F[X]$ can be expressed as a finite linear combination of these monomials. Since we can find infinitely many linearly independent vectors (the monomials), it follows that the dimension of $\F[X]$ is infinite.
        \end{proof}
        \item % Now, let $R$ be a ring and $R[X]$ the ring of polynomials with coefficiens in $R$. Recall that the degree $\deg(f)$ of a polynomial $f \in R[X]$ is defined to be the $\deg(f) = d$ if $f = a_d X^d + a_{d-1} X^{d-1} + \cdots + a_1 X + a_0$ with $a_d \neq 0$, and $\deg(0) = -\infty$ if $f = 0$. Show that for every $f, g \in R[X]$ we have 
        % 1) $\deg(fg) \leq \deg(f) + \deg(g)$
        % 2) $\deg(f + g) \leq \max(\deg(f), \deg(g))$
        % Moreover, show that if $R$ is an integral domain than the inequality in 1) is an equality.
        \begin{proof}
            Let $R$ be a ring and $R[X]$ the ring of polynomials with coefficients in $R$.
            \begin{enumerate}
                \item Let $f, g \in R[X]$ with $\deg(f) = m$ and $\deg(g) = n$. Then we can write
                \[
                    f = a_m X^m + a_{m-1} X^{m-1} + \cdots + a_0,
                \]
                \[
                    g = b_n X^n + b_{n-1} X^{n-1} + \cdots + b_0,
                \]
                where $a_m, b_n \neq 0$. The product $fg$ is given by
                \[
                    fg = (a_m X^m + \cdots)(b_n X^n + \cdots) = a_m b_n X^{m+n} + \text{(lower degree terms)}.
                \]
                If the characteristic of $R$ is such that $a_m b_n \neq 0$, then $\deg(fg) = m + n$. However, if $a_m b_n = 0$, then the highest degree term may cancel out, leading to $\deg(fg) < m + n$. Thus, we have
                \[
                    \deg(fg) \leq \deg(f) + \deg(g).
                \]
                In an integral $a_m \times b_n = 0$ implies that either $a_m = 0$ or $b_n = 0$ by definition, which contradicts our assumption. Therefore, in an integral domain, we have equality:
                \[
                    \deg(fg) = \deg(f) + \deg(g).
                \]

                \item Let $f, g \in R[X]$ with $\deg(f) = m$ and $\deg(g) = n$. Without loss of generality, assume $m \geq n$. Then we can write
                \begin{align*}
                    f &= a_m X^m + a_{m-1} X^{m-1} + \cdots + a_0, \\
                    g &= b_n X^n + b_{n-1} X^{n-1} + \cdots + b_0, \\
                \end{align*}
                The sum $f + g$ is given by
                \begin{align*}
                    f + g &= (a_m X^m + \cdots) + (b_n X^n + \cdots) \\
                          &= a_m X^m + \cdots + b_n X^n + \cdots.
                \end{align*}
                If $a_m + b_n \neq 0$, then $\deg(f + g) = m$. If $a_m + b_n = 0$, then the highest degree term cancels out, and we need to consider the next highest degree terms. In any case, we have $\deg(f + g) \leq \max(\deg(f), \deg(g))$, as desired.
            \end{enumerate}
        \end{proof}
    \end{enumerate}


    \item Let $\varphi : R \to S$ be a homomorphism of rings.
    \begin{enumerate}
        \item % Define the kernel of $\varphi$, denote $\ker(\varphi) by $\ker(\varphi) = \{a \in R | \varphi(a) = 0\}.$
        \textbf{Definition: }The kernel of a ring homomorphism $\varphi : R \to S$ is defined as $\ker(\varphi) = \{a \in R \mid \varphi(a) = 0\}$.
        \item % Show that $\ker(\varphi)$ is injective if and only if $\ker(\varphi) = \{0\}.$
        \begin{proof}
            To show that $\varphi$ is injective if and only if $\ker(\varphi) = \{0\}$, we proceed as follows:

            ($\implies$) Suppose $\varphi$ is injective. Then for any $a \in \ker(\varphi)$, we have $\varphi(a) = 0$. Since $\varphi$ is injective, the only element that maps to $0$ in $S$ is $0$ itself. Therefore, $a = 0$, and thus $\ker(\varphi) = \{0\}$.

            ($\impliedby$) Now suppose $\ker(\varphi) = \{0\}$. To show that $\varphi$ is injective, let $a, b \in R$ such that $\varphi(a) = \varphi(b)$. Then,
            \[
                \varphi(a) - \varphi(b) = 0 \implies \varphi(a - b) = 0.
            \]
            Since $a - b \in \ker(\varphi)$ and $\ker(\varphi) = \{0\}$, it follows that $a - b = 0$, or equivalently, $a = b$. Thus, $\varphi$ is injective.

            Therefore, we conclude that $\varphi$ is injective if and only if $\ker(\varphi) = \{0\}$.
        \end{proof}
    \end{enumerate}


    \item Suppose $R$ is a ring.
    \begin{enumerate}
        \item % Define when we say that $R$ is a ring with unit.
        \textbf{Definition: }A ring $R$ is said to be a ring with unit (or unital ring) if there exists an element $1_R \in R$ such that for all $a \in R$, we have $1_R \cdot a = a \cdot 1_R = a$.
        \item % Suppose $R$ is a ring with unit. Show that for $a \in R$ invertible, i.e., there exists at most one inverse $b \in R$, i.e., such that $ab = ba = 1_R$.
        \begin{proof}
            Suppose that $a \in R$ is invertible. Suppose, for a contradiction, that there exist $b, c \in R$ inverses for $a$ with $b \neq c$. Then we have that 
            \begin{align*}
                1_R &= a \cdot b \\
                \implies c \cdot 1_R &= c \cdot (a \cdot b) \\
                c &= (c \cdot a) \cdot b \\
                c &= 1_R \cdot b \\
                \implies c &= b \\
            \end{align*}
            a contradiction as we assumed that $b \neq c$. Therefore we have that for $a \in R$ invertible there exists a unique inverse.
        \end{proof}
        \item % Compute all the invertible elements in the ring $\Zmod{12}$
        The invertible elements of $\Z_{12}$ will be all of the elements that are coprime to 12. These elements are 1, 5, 7, and 11. Thus, the invertible elements in the ring $\Z_{12}$ are $\{1, 5, 7, 11\}$. The inverses are as follows:
        \begin{center}
            \begin{tabular}{c | c}
                Element & Inverse \\
                \hline
                1 & 1 \\
                5 & 5 \\
                7 & 7 \\
                11 & 11 \\
            \end{tabular}
        \end{center}
    \end{enumerate}

\newpage
    \item Let $\varphi : R \to S$ be a homomorphism of rings.
    \begin{enumerate}
        \item % Define when $R'$ is a subring of $R$.
        % We define the image of $\varphi$, denoted $\text{Im}(\varphi)$ by $\text{Im}(\varphi) = \{\varphi(a) | a \in R\}.$
        \textbf{Definition: }A subset $R' \subseteq R$ is called a subring of $R$ if $R'$ is itself a ring under the operations of addition and multiplication defined on $R$ and is nonempty. \\
        \item % Show that $\operatorname{Im}(\varphi)$ is a subring of $S$ and that $\ker(\varphi)$ is a subring of $R$.
        \begin{proof}
            First we show that $\operatorname{Im}(\varphi)$ is a subring of $S$. To do this we will show that for any $x, y \in \operatorname{Im}(\varphi)$,
            $x - y \in \operatorname{Im}(\varphi)$ and $xy \in \operatorname{Im}(\varphi)$. \\ 
            Let $x, y \in \operatorname{Im}(\varphi)$. Then there exist $a, b \in R$ such that $\varphi(a) = x$ and $\varphi(b) = y$. We have that 
            \[
                x - y = \varphi(a) - \varphi(b)\quad  \stackrel{\varphi \text{ is homomorphism}}{=} \quad \varphi(a - b) \in \operatorname{Im}(\varphi),
            \]
            and
            \[
                xy = \varphi(a) \varphi(b) \quad \stackrel{\varphi \text{ is homomorphism}}{=} \quad \varphi(ab) \in \operatorname{Im}(\varphi).
            \]
            Thus, $\operatorname{Im}(\varphi)$ is closed under subtraction and multiplication, and since it is nonempty (it contains $\varphi(0_R) = 0_S$), it is a subring of $S$.

            Next, we show that $\ker(\varphi)$ is a subring of $R$. Recall that
            \[
                \ker(\varphi) = \{a \in R \mid \varphi(a) = 0_S\}.
            \]
            Let $a, b \in \ker(\varphi)$. Then
            \[
                \varphi(a - b) = \varphi(a) - \varphi(b) = 0_S - 0_S = 0_S,
            \]
            so $a - b \in \ker(\varphi)$. Also,
            \[
                \varphi(ab) = \varphi(a) \varphi(b) = 0_S \cdot 0_S = 0_S,
            \]
            so $ab \in \ker(\varphi)$. Since $\ker(\varphi)$ contains $0_R$ and is closed under subtraction and multiplication, it is a subring of $R$.
        \end{proof}
    \end{enumerate}

\end{enumerate}



\end{document}