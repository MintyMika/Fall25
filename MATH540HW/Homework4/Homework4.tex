\documentclass{article}
\usepackage{amsmath}
\usepackage{tcolorbox}
\usepackage[margin=0.5in]{geometry} 
\usepackage{amsmath,amsthm,amssymb,amsfonts, fancyhdr, color, comment, graphicx, environ}
\usepackage{float}
\usepackage{xcolor}
\usepackage{mdframed}
\usepackage[shortlabels]{enumitem}
\usepackage{indentfirst}
\usepackage{mathrsfs}
\usepackage{hyperref}
\graphicspath{./}
\makeatletter
\newcommand*{\rom}[1]{\expandafter\@slowromancap\romannumeral #1@}
\makeatother
% Change enumerate labels to (a), (b), (c), ...
% Define a new environment for problems
\newcounter{problemCounter}
\newtcolorbox{problem}[2][]{colback=white, colframe=black, boxrule=0.5mm, arc=4mm, auto outer arc, title={\ifstrempty{#1}{Problem \stepcounter{problemCounter}\theproblemCounter}{#1}}}

% \renewcommand{\labelenumi}{\alph{enumi})}
\def\zz{{\mathbb Z}}
\def\rr{{\mathbb R}}
\def\qq{{\mathbb Q}}
\def\cc{{\mathbb C}}
\def\nn{{\mathbb N}}
\def\ss{{\mathbb S}}

\newtheorem{theorem}{Theorem}[section]
\newtheorem{corollary}{Corollary}[theorem]
\newtheorem{lemma}[theorem]{Lemma}
\newtcolorbox{proposition}[1][]{colback=white, colframe=blue, boxrule=0.5mm, arc=4mm, auto outer arc, title={Proposition #1}}
\newtcolorbox{definition}[1][]{colback=white, colframe=violet, boxrule=0.5mm, arc=4mm, auto outer arc, title={Definition #1}}
\newcommand{\Zmod}[1]{\zz/#1\zz}
\newcommand{\partFrac}[2]{\frac{\partial #1}{\partial #2}}

\newcommand\Mydiv[2]{%
$\strut#1$\kern.25em\smash{\raise.3ex\hbox{$\big)$}}$\mkern-8mu
        \overline{\enspace\strut#2}$}

\begin{document}

\begin{center}
    Math 540
    \hfill Homework 2
    \hfill \textit{Stephen Cornelius}
\end{center}
% \textbf{Remarks:} \\
% \begin{enumerate}[A)]
%     \item Definition is just a definition, there is no need to jjustify or explain it.
%     \item Answers to questions with proofs should be written, as much as you can, in the following format: \\
%     \begin{enumerate}[i)]
%         \item Statement
%         \item Main points that will appear in your proof
%         \item The actual proof
%     \end{enumerate}
%     Answers to questions with computations should be written, as much as possible, in the following format:
%     \begin{enumerate}[i)]
%         \item Statement and Result
%         \item Main points that will appear in your computation.
%         \item The actual computation
%     \end{enumerate}
% \end{enumerate}



% % Start of problems

\begin{problem} \\
    \textit{Diagonalizability - geometric definition.} \\
    Let $V$ be an $n$-dimensional vector space over a field $\mathbb{F}$ and let $T: V \to V$ a transformation.
    \begin{enumerate}[(a)]
        \item Write down the geometric definition (that we gave in class interms of direct sum decomposition of $V$) for when $T$ is diagonalizable.
        \item Define what does it mean for $\lambda \in \mathbb{F}$ to be an eigenvalue of $T$. Denote $\operatorname{Spec}(T)$ the set of eigenvalues of $T$ in $\mathbb{F}$. For each $\lambda \in \operatorname{Spec}(T)$, define the eigenspace $V_{\lambda}$. Show that the following are equivalent:
        \begin{enumerate}[(i)]
            \item $T$ is diagonalizable.
            \item $V = \bigoplus_{\lambda \in \operatorname{Spec}(T)} V_{\lambda}$.
        \end{enumerate}

        \item A linear transformation $P: V \to V$ is called a projector of $P^2 = P$. Show that any projector is diagonalizable.
    \end{enumerate}
\end{problem}


\begin{enumerate}
    \item We say that $T$ is diagonalizable if there exists $\lambda_1, \ldots, \lambda_k \in \mathbb{F}$ distinct and subspaces $V_1, \ldots, V_k < V$ such that
    \[
    V = \bigoplus_{i=1}^k V_i,
    \]
    and $T$ preserves each $V_i$, and $T|_{V_i} = \lambda_i Id_{V_i}$ for all $i = 1, \ldots, k$.
\end{enumerate}


\begin{problem} \\
    \textit{Diagonalizability - computational definition.} \\
    \begin{enumerate}[(a)]
        \item Let $V$ be an $n$-dimensional vector space over a field $\mathbb{F}$ and let $T: V \to V$ a linear transformation. Write down the computational definition (that we gave in class in terms of a basis $\mathscr{B}$ and the corresponding matrix $[T]_{\mathscr{B}}$) for when $T$ is diagonalizable.
        \item For a matrix $A \in M_n(\mathbb{F})$, consider the linear transformation $T_A: \mathbb{F}^n \to \mathbb{F}^n$ given by $v \mapsto Av$. Show that the following are equivalent:
        \begin{enumerate}[(i)]
            \item $T_A$ is diagonalizable (in this case we also say that $A$ is diagonalizable).
            \item There exists a diagonal matrix $D \in M_n(\mathbb{F})$ and an invertible matrix $C \in M_n(\mathbb{F})$ such that $C^{-1} A C = D$.
        \end{enumerate}
        \item Consider the operator $T_A: \mathbb{R}^2 \to \mathbb{R}^2$ where $A = \begin{pmatrix}
            0 & 1 \\
            1 & 0
        \end{pmatrix}$. 
        \begin{enumerate}
            \item Show that $T_A$ is diagonalizable
            \item find its eigenvalues
            \item find the direct sum decomposition for eigenspaces
            \item find a basis of eigenvectors
            \item find $D,C \in M_2(\mathbb{R})$ with $D$ diagonal and $C$ invertible such that $D = C^{-1} A C$.
        \end{enumerate}
    \end{enumerate}
\end{problem}


\end{document}
