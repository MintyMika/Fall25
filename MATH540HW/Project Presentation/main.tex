\documentclass{beamer}
% The "beamerthemeMadison.sty" file is in ./uw-beamer-template directory
%\usepackage{./uw-beamer-template/beamerthemeMadison} 
% \usetheme[compactlogo,white]{Madison}

\usefonttheme[onlymath]{serif}


\usepackage{amsmath}
\usepackage{amssymb,amsfonts,amsthm}
\usepackage{graphicx}
\graphicspath{{figs/}}
\usepackage{natbib}


\title{Learning With Errors (LWE)}
\subtitle{an explanation for beginners}

\author{Stephen Cornelius, Liam Thomassen, and Divyesh Pandey }

\date[]{
  \\
  A presentation for
  \emph{MATH540: Linear Algebra II}\\
  University of Wisconsin--Madison\\
  \today
}

\begin{document}

\small

\begin{frame}[plain]
  \maketitle
\end{frame}

\begin{frame}{Learning With Errors (LWE)}
  \begin{itemize}
    \item Sample uniformly random matrix $A \in \mathbb{Z}_q^{m \times n}$
    \item Sample secret key $\mathbf{x} \in \mathbb{Z}_q^n$
    \item Sample small error vector $\mathbf{e} \in \mathbb{Z}_q^m$ from discrete Gaussian
    \item Compute $\mathbf{b} = A\mathbf{x} + \mathbf{e} \pmod{q}$
    \item LWE assumption: $(\mathbf{b}, A)$ is computationally indistinguishable from random
  \end{itemize}
\end{frame}

\begin{frame}{One-Time Pad}
  \begin{itemize}
    \item Message $m$ and random key $r$ in vector space $V$ over finite field
    \item Ciphertext: $c = m \oplus r$
    \item Decryption: $m = c \oplus r$ (since $r \oplus r = 0$)
    \item Information-theoretic security: unbreakable even with infinite computing power
    \item Used as baseline for security proofs
  \end{itemize}
\end{frame}

\begin{frame}{Our Construction}
  \begin{itemize}
    \item Replace random vector $r$ with $A\mathbf{x} + \mathbf{e}$
    \item Ciphertext: $\mathbf{c} = m \oplus (A\mathbf{x} + \mathbf{e})$
    \item Store message in most significant digits
    \item Receiver recovers $m$ by removing least significant digits (error)
  \end{itemize}
\end{frame}

\begin{frame}{Security Proof}
  \begin{itemize}
    \item Assume attacker can break our construction
    \item Alice randomly chooses: One-Time Pad or LWE construction
    \item Attacker cannot break One-Time Pad (information-theoretic security)
    \item Therefore, attacker must break LWE ciphertexts
    \item Attacker's success distinguishes $(A\mathbf{x} + \mathbf{e}, A)$ from $(r, A)$
    \item This breaks LWE assumption—contradiction!
    \item Conclusion: No such attacker exists
  \end{itemize}
\end{frame}



\bibliographystyle{plainnat}

\begin{frame}[t,allowframebreaks]
  \frametitle{References}
  \bibliography{main}
\end{frame}


\end{document}
