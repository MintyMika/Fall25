\documentclass{article}
\usepackage{amsmath}
\usepackage{tcolorbox}
\usepackage[margin=0.5in]{geometry} 
\usepackage{amsmath,amsthm,amssymb,amsfonts, fancyhdr, color, comment, graphicx, environ}
\usepackage{float}
\usepackage{xcolor}
\usepackage{mdframed}
\usepackage[shortlabels]{enumitem}
\usepackage{indentfirst}
\usepackage{mathrsfs}
\usepackage{hyperref}
\usepackage{extarrows}
\graphicspath{./}
\makeatletter
\newcommand*{\rom}[1]{\expandafter\@slowromancap\romannumeral #1@}
\makeatother

% Define a new environment for problems
\newcounter{problemCounter}
\newtcolorbox{problem}[2][]{colback=white, colframe=black, boxrule=0.5mm, arc=4mm, auto outer arc, title={\ifstrempty{#1}{Problem \stepcounter{problemCounter}\theproblemCounter}{#1}}}

% \renewcommand{\labelenumi}{\alph{enumi})}
\def\zz{{\mathbb Z}}
\def\rr{{\mathbb R}}
\def\qq{{\mathbb Q}}
\def\cc{{\mathbb C}}
\def\nn{{\mathbb N}}
\def\ss{{\mathbb S}}

\newtheorem{theorem}{Theorem}[section]
\newtheorem{corollary}{Corollary}[theorem]
\newtheorem{lemma}[theorem]{Lemma}
\newtcolorbox{proposition}[1][]{colback=white, colframe=blue, boxrule=0.5mm, arc=4mm, auto outer arc, title={Proposition #1}}
\newtcolorbox{definition}[1][]{colback=white, colframe=violet, boxrule=0.5mm, arc=4mm, auto outer arc, title={Definition #1}}
\newcommand{\Zmod}[1]{\zz/#1\zz}
\newcommand{\partFrac}[2]{\frac{\partial #1}{\partial #2}}

\newcommand\Mydiv[2]{%
$\strut#1$\kern.25em\smash{\raise.3ex\hbox{$\big)$}}$\mkern-8mu
        \overline{\enspace\strut#2}$}

\begin{document}

\begin{center}
    Math 741
    \hfill Homework 5
    \hfill \textit{Stephen Cornelius}
\end{center}



\begin{problem} \\
    In this problem, no explanation is required. All parts are worth 2 points.
    \begin{enumerate}[(a)]
        \item True or false: In a free abelian group of finite rank, every linearly independent set can be completed to a basis.
        \item How many different (up to isomorphism) abelian groups of order $300$ are there?
        \item True or false: For any action of a finite group $G$ on a set $X$, the cardinality $|X|$ divides $|G|$.
        \item Give an example of an infinite group $G$ such that every element of $G$ has finite order.
        \item Let $F_2$ be the free group on two generators. True or false: For every $n$, there exists a normal subgroup $H_n \subset F_2$ such that $F_2/H_n \cong S_n$?
    \end{enumerate}
\end{problem}


\begin{problem} \\
    Let $\qq^\times$ be the group of non-zero rational numbers under multiplication.
    \begin{enumerate}
        \item[(a)] Show that $\qq^\times$ is isomorphic to the product of $\zz/2\zz$ and a free abelian group. % , 4pts\\
        \item[(b)] Describe all group homomorphisms $\zz/2\zz \to \qq^\times$. % , 2pts\\
        \item[(c)] Describe all group homomorphisms $\qq^\times \to \zz/2\zz$. % , 2pts \\   
    \end{enumerate}

\end{problem}


\begin{problem} \\
    Let $G$ be a group of order $2017 \times 2027 \times 2029$ (these are all prime numbers). Show that $G$ is cyclic.
\end{problem}


\begin{problem} \\
    Let $G$ be a finite group, and let $A = \operatorname{Aut}(G)$ be the group of automorphisms $\phi : G \to G$. Consider the natural action of $A$ on $G$, and take the quotient $G/A$.
    \begin{enumerate}
        \item[(a)] What is $|G/A|$ if $G = \zz/6\zz$? % , 3pts \\
        \item[(b)] Show that if $|G/A| = 2$, then $G \cong (\zz/p\zz)^n$ for a prime $p$ and $n > 0$. % , 7pts \\
    \end{enumerate}
\end{problem}




\begin{problem} \\
    A finite group $G$ acts transitively (that is, with a single orbit) on a finite set $X$ such that $|X| > 1$. Show that there exists an element $g \in G$ which does not fix any element of $X$.
\end{problem}



\begin{problem} \\
    A map $\phi : \rr \to \rr$ is said to be an \textit{affine-linear bijection} if it is of the form 
    \[
        \phi(x) = ax + b \quad (a,b \in \rr : a \neq 0).
    \]
    \begin{enumerate}
        \item[(a)] Show that the set of affine-linear bijections forms a group $G$ under composition. % , 3pts \\
        \item[(b)] Show that $G$ is isomorphic to semidirect product of \textit{abelian} groups $A$ and $B$. Make sure to identify the groups $A$ and $B$, as well as the action of one on the other used in the semidirect product. % , 7pts \\
    \end{enumerate}
\end{problem}


\end{document}