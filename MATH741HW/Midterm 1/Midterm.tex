\documentclass{article}
\usepackage{amsmath}
\usepackage{tcolorbox}
\usepackage[margin=0.5in]{geometry} 
\usepackage{amsmath,amsthm,amssymb,amsfonts, fancyhdr, color, comment, graphicx, environ}
\usepackage{float}
\usepackage{xcolor}
\usepackage{mdframed}
\usepackage[shortlabels]{enumitem}
\usepackage{indentfirst}
\usepackage{mathrsfs}
\usepackage{hyperref}
\usepackage{extarrows}
\graphicspath{./}
\makeatletter
\newcommand*{\rom}[1]{\expandafter\@slowromancap\romannumeral #1@}
\makeatother

% Define a new environment for problems
\newcounter{problemCounter}
\newtcolorbox{problem}[2][]{colback=white, colframe=black, boxrule=0.5mm, arc=4mm, auto outer arc, title={\ifstrempty{#1}{Problem \stepcounter{problemCounter}\theproblemCounter}{#1}}}

% \renewcommand{\labelenumi}{\alph{enumi})}
\def\zz{{\mathbb Z}}
\def\rr{{\mathbb R}}
\def\qq{{\mathbb Q}}
\def\cc{{\mathbb C}}
\def\nn{{\mathbb N}}
\def\ss{{\mathbb S}}

\newtheorem{theorem}{Theorem}[section]
\newtheorem{corollary}{Corollary}[theorem]
\newtheorem{lemma}[theorem]{Lemma}
\newtcolorbox{proposition}[1][]{colback=white, colframe=blue, boxrule=0.5mm, arc=4mm, auto outer arc, title={Proposition #1}}
\newtcolorbox{definition}[1][]{colback=white, colframe=violet, boxrule=0.5mm, arc=4mm, auto outer arc, title={Definition #1}}
\newcommand{\Zmod}[1]{\zz/#1\zz}
\newcommand{\partFrac}[2]{\frac{\partial #1}{\partial #2}}

\newcommand\Mydiv[2]{%
$\strut#1$\kern.25em\smash{\raise.3ex\hbox{$\big)$}}$\mkern-8mu
        \overline{\enspace\strut#2}$}

\begin{document}

\begin{center}
    Math 741
    \hfill Homework 5
    \hfill \textit{Stephen Cornelius}
\end{center}



\begin{problem} \\
    In this problem, no explanation is required. All parts are worth 2 points.
    \begin{enumerate}[(a)]
        \item True or false: In a free abelian group of finite rank, every linearly independent set can be completed to a basis.
        \item How many different (up to isomorphism) abelian groups of order $300$ are there?
        \item True or false: For any action of a finite group $G$ on a set $X$, the cardinality $|X|$ divides $|G|$.
        \item Give an example of an infinite group $G$ such that every element of $G$ has finite order.
        \item Let $F_2$ be the free group on two generators. True or false: For every $n$, there exists a normal subgroup $H_n \subset F_2$ such that $F_2/H_n \cong S_n$?
    \end{enumerate}
\end{problem}

\begin{enumerate}[(a)]
    % TODO: Look over these answers
    \item True.
    \item There are 4 abelian groups of order 300 up to isomorphism. % This is because $300 = 2^2 \cdot 3^1 \cdot 5^2$, and the number of abelian groups of order $n$ is given by the product of the number of partitions of the exponents in its prime factorization. The partitions are: for $2^2$ (2), for $3^1$ (1), and for $5^2$ (2). Thus, the total number is $2 \times 1 \times 2 = 4$.
    \item False.
    \item An example of an infinite group where every element has finite order is the group of all roots of unity in the complex numbers, denoted by $\{e^{2\pi i k/n} \mid k \in \mathbb{Z}, n \in \mathbb{N}\}$.
    \item True. % Why? 
\end{enumerate}


\begin{problem} \\
    Let $\qq^\times$ be the group of non-zero rational numbers under multiplication.
    \begin{enumerate}
        \item[(a)] Show that $\qq^\times$ is isomorphic to the product of $\zz/2\zz$ and a free abelian group. % , 4pts\\
        \item[(b)] Describe all group homomorphisms $\zz/2\zz \to \qq^\times$. % , 2pts\\
        \item[(c)] Describe all group homomorphisms $\qq^\times \to \zz/2\zz$. % , 4pts \\   
    \end{enumerate}
\end{problem}


\begin{enumerate}[(a)]
    %TODO: Look over these answers and extrapolate
    \item \begin{proof}
        By the Fundamental Theorem of Arithmetic, every non-zero rational number can be uniquely expressed as a product of prime numbers raised to integer powers. Specifically, any $q \in \qq^\times$ can be written as
        \[
            q = \pm p_1^{a_1} p_2^{a_2} \cdots p_k^{a_k},
        \]
        where $p_i$ are distinct prime numbers and $a_i \in \zz$. The sign of $q$ can be captured by the factor $\pm 1$, which corresponds to the group $\zz/2\zz$. To see this, note that the group $\zz/2\zz$ has two elements: the identity element $0$ (which corresponds to $+1$) and the non-identity element $1$ (which corresponds to $-1$). Thus, we can separate the sign from the rest of the rational number. \\
        The remaining part, $p_1^{a_1} p_2^{a_2} \cdots p_k^{a_k}$, forms a free abelian group generated by the primes. To see that this is a free abelian group, note that the exponents $a_i$ can be any integers, and the multiplication of rational numbers corresponds to the addition of these exponents. Thus, we can express $\qq^\times$ as the direct product
        \[
            \qq^\times \cong \zz/2\zz \times F,
        \]
        where $F$ is the free abelian group generated by the primes. Therefore, we conclude that $\qq^\times$ is isomorphic to the product of $\zz/2\zz$ and a free abelian group.
    \end{proof}
    \item The group $\zz/2\zz$ has two elements: $0$ and $1$. The image of the identity element $0$ must be the identity element in $\qq^\times$, which is $1$. The image of the non-identity element $1$ can either be $1$ or $-1$. Thus, there are two possible homomorphisms: the trivial homomorphism sending both elements to $1$, and the homomorphism sending $0$ to $1$ and $1$ to $-1$.
    \item Any homomorphism $\varphi : \qq^\times \to \zz/2\zz$ must satisfy $\varphi(xy) = \varphi(x) + \varphi(y)$ for all $x, y \in \qq^\times$. We have found that $\qq^\times$ is generated by $-1$ and the prime numbers so any homomorphism id determined by its values on these generators. \\
    Then we have that $\varphi(-1) \in \zz/2\zz$ can be either $0$ or $1$. For any prime number $p$, we have that $\varphi(p^n) = n \varphi(p)$ for any integer $n$. Since $\zz/2\zz$ has only two elements, $\varphi(p)$ can also be either $0$ or $1$. Thus, for each prime number, we have two choices for its image under $\varphi$. \\
    Therefore, the group homomorphisms from $\qq^\times$ to $\zz/2\zz$ are determined by the choices of images for $-1$ and each prime number, leading to a large number of possible homomorphisms. $\operatorname{Hom}(\qq^\times, \zz/2\zz) \cong \bigoplus_{p \text{ prime}} \zz/2\zz \oplus \zz/2\zz$. The additional $\zz/2\zz$ factor corresponds to the choice of image (sign) for $-1$.
\end{enumerate}




\newpage 
\begin{problem} \\
    Let $G$ be a group of order $2017 \times 2027 \times 2029$ (these are all prime numbers). Show that $G$ is cyclic.
\end{problem}


\begin{proof}
    We have that the order of $G$ is the product of three distinct primes: $2017$, $2027$, and $2029$. Then, by the first Sylow theorem, for each prime $p$ dividing the order of $G$, there exists a Sylow $p$-subgroup of $G$. Let $n_p$ denote the number of Sylow $p$-subgroups of $G$. By the third Sylow theorem, we have that $n_p \equiv 1 \mod p$ and $n_p$ divides the order of $G$. Since the primes are distinct and large, the only divisors of the order of $G$ that are congruent to 1 modulo $p$ are 1 itself. Therefore, each Sylow $p$-subgroup is unique and hence normal in $G$. Since the Sylow subgroups are normal and their orders are pairwise relatively prime, $G$ is isomorphic to the direct product of its Sylow subgroups, each of which is cyclic of prime order. Thus, we have that $G = \zz/2017\zz \times \zz/2027\zz \times \zz/2029\zz$ is cyclic.
\end{proof}



\begin{problem} \\
    Let $G$ be a finite group, and let $A = \operatorname{Aut}(G)$ be the group of automorphisms $\phi : G \to G$. Consider the natural action of $A$ on $G$, and take the quotient $G/A$.
    \begin{enumerate}
        \item[(a)] What is $|G/A|$ if $G = \zz/6\zz$? % , 3pts \\
        \item[(b)] Show that if $|G/A| = 2$, then $G \cong (\zz/p\zz)^n$ for a prime $p$ and $n > 0$. % , 7pts \\
    \end{enumerate}
\end{problem}


\begin{enumerate}[(a)]
    \item For $G = \zz/6\zz$, the automorphism group $\operatorname{Aut}(G)$ consists of all group automorphisms of $\zz/6\zz$. The elements of $\zz/6\zz$ are $\{0, 1, 2, 3, 4, 5\}$. The automorphisms are determined by the images of the generator $1$. The possible images are $1$ and $5$ (since they are coprime to $6$). Thus, there are two automorphisms: the identity and the one sending $1$ to $5$. The orbits under this action are $\{0\}$, $\{1, 5\}$, $\{2, 4\}$, and $\{3\}$. Therefore, there are 4 distinct orbits, so $|G/A| = 4$.
    \item \begin{proof}
        We have that $|G/A| = 2$ implies that there are exactly two orbits under the action of $\operatorname{Aut}(G)$ on $G$. One orbit must be the identity element $\{e\}$, and the other orbit must contain all other elements of $G$. This means that for any non-identity element $g \in G$, there exists an automorphism $\phi \in \operatorname{Aut}(G)$ such that $\phi(g) = h$ for any other non-identity element $h \in G$. This property implies that all non-identity elements of $G$ have the same order. Let this common order be $p$. Since $G$ is finite, $p$ must be a prime number. Thus, every non-identity element of $G$ has order $p$, and $G$ is a $p$-group. Furthermore, since all non-identity elements have the same order, $G$ must be isomorphic to a direct product of copies of $\zz/p\zz$. Therefore, we conclude that $G \cong (\zz/p\zz)^n$ for some prime $p$ and integer $n > 0$.
    \end{proof}
\end{enumerate}




\begin{problem} \\
    A finite group $G$ acts transitively (that is, with a single orbit) on a finite set $X$ such that $|X| > 1$. Show that there exists an element $g \in G$ which does not fix any element of $X$.
\end{problem}

% \begin{theorem}[II.4.3]
%     If a group $G$ acts on a set $S$, then the cardinal number of the orbit of $x \in S$ is the index $[G : G_x]$.
% \end{theorem}

\begin{proof}
    Let $G$ act transitively on the set $X$. By Theorem II.4.3, the size of the orbit of any element $x \in X$ under the action of $G$ is given by the index $[G : G_x]$, where $G_x$ is the stabilizer of $x$ in $G$. Since the action is transitive, there is only one orbit, which means that the size of the orbit is equal to the size of the set $X$, denoted as $|X|$.

    Now, since $|X| > 1$, we have $|X| = [G : G_x] > 1$. This implies that the index $[G : G_x]$ is greater than 1, meaning that the stabilizer $G_x$ is a proper subgroup of $G$. 

    By Lagrange's theorem, the order of $G$ is equal to the order of the stabilizer $G_x$ multiplied by the size of the orbit, i.e., 
    \[
    |G| = |G_x| \cdot |X|.
    \]
    Since $|X| > 1$, it follows that $|G| > |G_x|$. 

    Now, consider the action of $G$ on the set $X$. If every element of $G$ fixed every element of $X$, then the action would be trivial, meaning that every element of $G$ would act as the identity on $X$. However, this contradicts the fact that the action is transitive and that $|X| > 1$. 

    Therefore, there must exist at least one element $g \in G$ such that $g$ does not fix any element of $X$. This means that for every $x \in X$, we have $g \cdot x \neq x$. 

    Thus, we conclude that there exists an element $g \in G$ which does not fix any element of $X$.
\end{proof}


\begin{problem} \\
    A map $\phi : \rr \to \rr$ is said to be an \textit{affine-linear bijection} if it is of the form 
    \[
        \phi(x) = ax + b \quad (a,b \in \rr : a \neq 0).
    \]
    \begin{enumerate}
        \item[(a)] Show that the set of affine-linear bijections forms a group $G$ under composition. % , 3pts \\
        \item[(b)] Show that $G$ is isomorphic to semidirect product of \textit{abelian} groups $A$ and $B$. Make sure to identify the groups $A$ and $B$, as well as the action of one on the other used in the semidirect product. % , 7pts \\
    \end{enumerate}
\end{problem}


\begin{enumerate}[(a)]
    \item  \begin{proof}
        To show that the set of affine-linear bijections forms a group under composition, we show closure, associativity, identity, and inverses. \\
        First we show closure. Let $\phi(x) = a x + b$ and $\psi(x) = c x + d$ be two affine-linear bijections. Then their composition is given by
        \[
            (\phi \circ \psi)(x) = \phi(\psi(x)) = a (c x + d) + b = (a c) x + (a d + b),
        \]
        which is again of the form $e x + f$ with $e = a c \neq 0$. Thus, the composition of two affine-linear bijections is again an affine-linear bijection, establishing closure. \\
        Next we show associativity. Let $\phi(x) = a x + b$, $\psi(x) = c x + d$, and $\theta(x) = e x + f$ be three affine-linear bijections. Then we have
        \begin{align*}
            ((\phi \circ \psi) \circ \theta)(x) &= (\phi \circ \psi)(\theta(x)) = \phi(\psi(e x + f)) = \phi(c (e x + f) + d) = a (c (e x + f) + d) + b \\
            &= a c e x + a c f + a d + b, \\
            (\phi \circ (\psi \circ \theta))(x) &= \phi((\psi \circ \theta)(x)) = \phi(\psi(e x + f)) = \phi(c (e x + f) + d) = a (c (e x + f) + d) + b \\
            &= a c e x + a c f + a d + b.
        \end{align*}
        Clearly $((\phi \circ \psi) \circ \theta)(x) = (\phi \circ (\psi \circ \theta))(x)$, thus we have associativity. \\
        Next we show the identity element. From intuition, we can see that the identity element should be $\operatorname{id}(x) = x$. To verify this, let $\phi(x) = a x + b$ be an affine-linear bijection. Then we have
        \[
            (\phi \circ \operatorname{id})(x) = \phi(\operatorname{id}(x)) = \phi(x) = a x + b,
        \]
        and
        \[
            (\operatorname{id} \circ \phi)(x) = \operatorname{id}(\phi(x)) = \phi(x) = a x + b.
        \]
        Thus, $\operatorname{id}$ is the identity element. \\
        Finally, we show the existence of inverses. For $\phi(x) = a x + b$, the inverse is given by
        \[
            \phi^{-1}(x) = \frac{1}{a} x - \frac{b}{a},
        \]
        which can be verified as follows:
        \[
            (\phi \circ \phi^{-1})(x) = \phi\left(\frac{1}{a} x - \frac{b}{a}\right) = a\left(\frac{1}{a} x - \frac{b}{a}\right) + b = x - b + b = x = \operatorname{id}(x),
        \]
        and
        \[
            (\phi^{-1} \circ \phi)(x) = \phi^{-1}(a x + b) = \frac{1}{a}(a x + b) - \frac{b}{a} = x + \frac{b}{a} - \frac{b}{a} = x = \operatorname{id}(x).
        \]
        Thus, every affine-linear bijection has an inverse that is also an affine-linear bijection. \\
        Since we have shown closure, associativity, identity, and inverses, we conclude that the set of affine-linear bijections forms a group under composition.
    \end{proof}
    
    \item \begin{proof}
        We can identify the group $A$ as the group of translations, which consists of all affine-linear bijections of the form $\phi(x) = x + b$ for $b \in \rr$. This group is isomorphic to $(\rr, +)$, which is abelian. \\
        The group $B$ can be identified as the group of dilations, which consists of all affine-linear bijections of the form $\psi(x) = a x$ for $a \in \rr^\times$ (the non-zero real numbers). This group is also abelian under multiplication. \\
        The action of $B$ on $A$ is given by conjugation. Specifically, for $\psi(x) = a x \in B$ and $\phi(x) = x + b \in A$, we have
        \[
            \psi \circ \phi \circ \psi^{-1}(x) = a (x + b/a) = a x + b,
        \]
        which shows that the action of $B$ on $A$ scales the translation by the factor $a$. \\
        Therefore, we can express the group $G$ of affine-linear bijections as the semidirect product of $A$ and $B$, denoted by $G \cong A \rtimes B$. This establishes that $G$ is isomorphic to the semidirect product of the abelian groups $A$ and $B$.
        \end{proof}
\end{enumerate}


\end{document}