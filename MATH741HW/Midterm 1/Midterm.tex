\documentclass{article}
\usepackage{amsmath}
\usepackage{tcolorbox}
\usepackage[margin=0.5in]{geometry} 
\usepackage{amsmath,amsthm,amssymb,amsfonts, fancyhdr, color, comment, graphicx, environ}
\usepackage{float}
\usepackage{xcolor}
\usepackage{mdframed}
\usepackage[shortlabels]{enumitem}
\usepackage{indentfirst}
\usepackage{mathrsfs}
\usepackage{hyperref}
\usepackage{extarrows}
\graphicspath{./}
\makeatletter
\newcommand*{\rom}[1]{\expandafter\@slowromancap\romannumeral #1@}
\makeatother

% Define a new environment for problems
\newcounter{problemCounter}
\newtcolorbox{problem}[2][]{colback=white, colframe=black, boxrule=0.5mm, arc=4mm, auto outer arc, title={\ifstrempty{#1}{Problem \stepcounter{problemCounter}\theproblemCounter}{#1}}}

% \renewcommand{\labelenumi}{\alph{enumi})}
\def\zz{{\mathbb Z}}
\def\rr{{\mathbb R}}
\def\qq{{\mathbb Q}}
\def\cc{{\mathbb C}}
\def\nn{{\mathbb N}}
\def\ss{{\mathbb S}}

\newtheorem{theorem}{Theorem}[section]
\newtheorem{corollary}{Corollary}[theorem]
\newtheorem{lemma}[theorem]{Lemma}
\newtcolorbox{proposition}[1][]{colback=white, colframe=blue, boxrule=0.5mm, arc=4mm, auto outer arc, title={Proposition #1}}
\newtcolorbox{definition}[1][]{colback=white, colframe=violet, boxrule=0.5mm, arc=4mm, auto outer arc, title={Definition #1}}
\newcommand{\Zmod}[1]{\zz/#1\zz}
\newcommand{\partFrac}[2]{\frac{\partial #1}{\partial #2}}

\newcommand\Mydiv[2]{%
$\strut#1$\kern.25em\smash{\raise.3ex\hbox{$\big)$}}$\mkern-8mu
        \overline{\enspace\strut#2}$}

\begin{document}

\begin{center}
    Math 741
    \hfill Homework 5
    \hfill \textit{Stephen Cornelius}
\end{center}



\begin{problem} \\
    In this problem, no explanation is required. All parts are worth 2 points.
    \begin{enumerate}[(a)]
        \item True or false: In a free abelian group of finite rank, every linearly independent set can be completed to a basis.
        \item How many different (up to isomorphism) abelian groups of order $300$ are there?
        \item True or false: For any action of a finite group $G$ on a set $X$, the cardinality $|X|$ divides $|G|$.
        \item Give an example of an infinite group $G$ such that every element of $G$ has finite order.
        \item Let $F_2$ be the free group on two generators. True or false: For every $n$, there exists a normal subgroup $H_n \subset F_2$ such that $F_2/H_n \cong S_n$?
    \end{enumerate}
\end{problem}

\begin{enumerate}[(a)]
    % TODO: Look over these answers
    \item True.
    \item There are 5 different abelian groups of order 300 up to isomorphism. This is because $300 = 2^2 \cdot 3^1 \cdot 5^2$, and the number of abelian groups of order $n$ is given by the product of the number of partitions of the exponents in its prime factorization. The partitions are: for $2^2$ (2), for $3^1$ (1), and for $5^2$ (2). Thus, the total number is $2 \times 1 \times 2 = 4$.
    \item False.
    \item An example of an infinite group where every element has finite order is the group of all roots of unity in the complex numbers, denoted by $\{e^{2\pi i k/n} \mid k \in \mathbb{Z}, n \in \mathbb{N}\}$.
    \item True. % Why? 
\end{enumerate}


\begin{problem} \\
    Let $\qq^\times$ be the group of non-zero rational numbers under multiplication.
    \begin{enumerate}
        \item[(a)] Show that $\qq^\times$ is isomorphic to the product of $\zz/2\zz$ and a free abelian group. % , 4pts\\
        \item[(b)] Describe all group homomorphisms $\zz/2\zz \to \qq^\times$. % , 2pts\\
        \item[(c)] Describe all group homomorphisms $\qq^\times \to \zz/2\zz$. % , 2pts \\   
    \end{enumerate}

\end{problem}


\begin{enumerate}[(a)]
    %TODO: Look over these answers and extrapolate
    \item \begin{proof}
        Every non-zero rational number can be uniquely expressed in the form 
        \[
            r = \pm p_1^{a_1} p_2^{a_2} \cdots p_k^{a_k},
        \]
        where $p_i$ are distinct prime numbers and $a_i \in \zz$. The sign of $r$ corresponds to an element of $\zz/2\zz$, while the exponents $a_i$ correspond to elements of a free abelian group generated by the primes. Thus, we can define an isomorphism
        \[
            \phi : \qq^\times \to \zz/2\zz \times F,
        \]
        where $F$ is the free abelian group generated by the primes. This shows that $\qq^\times$ is isomorphic to the product of $\zz/2\zz$ and a free abelian group.
    \end{proof}
    \item The group homomorphisms from $\zz/2\zz$ to $\qq^\times$ are determined by the image of the non-identity element of $\zz/2\zz$. Since $\qq^\times$ contains elements of order 2 (specifically, $-1$), there are two homomorphisms: the trivial homomorphism sending everything to $1$, and the homomorphism sending the non-identity element to $-1$.
    \item The group homomorphisms from $\qq^\times$ to $\zz/2\zz$ are determined by the kernel of the homomorphism. The only non-trivial homomorphism is the one that sends all positive rational numbers to the identity element of $\zz/2\zz$ and all negative rational numbers to the non-identity element. Thus, there are two homomorphisms: the trivial homomorphism and the sign homomorphism.
\end{enumerate}




\newpage 
\begin{problem} \\
    Let $G$ be a group of order $2017 \times 2027 \times 2029$ (these are all prime numbers). Show that $G$ is cyclic.
\end{problem}


\begin{proof}
    We have that the order of $G$ is the product of three distinct primes: $2017$, $2027$, and $2029$. Then, by the first Sylow theorem, for each prime $p$ dividing the order of $G$, there exists a Sylow $p$-subgroup of $G$. Let $n_p$ denote the number of Sylow $p$-subgroups of $G$. By the third Sylow theorem, we have that $n_p \equiv 1 \mod p$ and $n_p$ divides the order of $G$. Since the primes are distinct and large, the only divisors of the order of $G$ that are congruent to 1 modulo $p$ are 1 itself. Therefore, each Sylow $p$-subgroup is unique and hence normal in $G$. Since the Sylow subgroups are normal and their orders are pairwise relatively prime, $G$ is isomorphic to the direct product of its Sylow subgroups, each of which is cyclic of prime order. Thus, we have that $G = \zz/2017\zz \times \zz/2027\zz \times \zz/2029\zz$ is cyclic.
\end{proof}



\begin{problem} \\
    Let $G$ be a finite group, and let $A = \operatorname{Aut}(G)$ be the group of automorphisms $\phi : G \to G$. Consider the natural action of $A$ on $G$, and take the quotient $G/A$.
    \begin{enumerate}
        \item[(a)] What is $|G/A|$ if $G = \zz/6\zz$? % , 3pts \\
        \item[(b)] Show that if $|G/A| = 2$, then $G \cong (\zz/p\zz)^n$ for a prime $p$ and $n > 0$. % , 7pts \\
    \end{enumerate}
\end{problem}


\begin{enumerate}[(a)]
    \item For $G = \zz/6\zz$, the automorphism group $\operatorname{Aut}(G)$ consists of all group automorphisms of $\zz/6\zz$. The elements of $\zz/6\zz$ are $\{0, 1, 2, 3, 4, 5\}$. The automorphisms are determined by the images of the generator $1$. The possible images are $1$ and $5$ (since they are coprime to $6$). Thus, there are two automorphisms: the identity and the one sending $1$ to $5$. The orbits under this action are $\{0\}$, $\{1, 5\}$, $\{2, 4\}$, and $\{3\}$. Therefore, there are 4 distinct orbits, so $|G/A| = 4$.
    \item \begin{proof}
        We have that $|G/A| = 2$ implies that there are exactly two orbits under the action of $\operatorname{Aut}(G)$ on $G$. One orbit must be the identity element $\{e\}$, and the other orbit must contain all other elements of $G$. This means that for any non-identity element $g \in G$, there exists an automorphism $\phi \in \operatorname{Aut}(G)$ such that $\phi(g) = h$ for any other non-identity element $h \in G$. This property implies that all non-identity elements of $G$ have the same order. Let this common order be $p$. Since $G$ is finite, $p$ must be a prime number. Thus, every non-identity element of $G$ has order $p$, and $G$ is a $p$-group. Furthermore, since all non-identity elements have the same order, $G$ must be isomorphic to a direct product of copies of $\zz/p\zz$. Therefore, we conclude that $G \cong (\zz/p\zz)^n$ for some prime $p$ and integer $n > 0$.
    \end{proof}
\end{enumerate}




\begin{problem} \\
    A finite group $G$ acts transitively (that is, with a single orbit) on a finite set $X$ such that $|X| > 1$. Show that there exists an element $g \in G$ which does not fix any element of $X$.
\end{problem}



\begin{problem} \\
    A map $\phi : \rr \to \rr$ is said to be an \textit{affine-linear bijection} if it is of the form 
    \[
        \phi(x) = ax + b \quad (a,b \in \rr : a \neq 0).
    \]
    \begin{enumerate}
        \item[(a)] Show that the set of affine-linear bijections forms a group $G$ under composition. % , 3pts \\
        \item[(b)] Show that $G$ is isomorphic to semidirect product of \textit{abelian} groups $A$ and $B$. Make sure to identify the groups $A$ and $B$, as well as the action of one on the other used in the semidirect product. % , 7pts \\
    \end{enumerate}
\end{problem}


\end{document}