\documentclass{article}
\usepackage{amsmath}
\usepackage{tcolorbox}
\usepackage[margin=0.5in]{geometry} 
\usepackage{amsmath,amsthm,amssymb,amsfonts, fancyhdr, color, comment, graphicx, environ}
\usepackage{float}
\usepackage{xcolor}
\usepackage{mdframed}
\usepackage[shortlabels]{enumitem}
\usepackage{indentfirst}
\usepackage{mathrsfs}
\usepackage{hyperref}
\usepackage{extarrows}
\graphicspath{./}
\makeatletter
\newcommand*{\rom}[1]{\expandafter\@slowromancap\romannumeral #1@}
\makeatother

% Define a new environment for problems
\newcounter{problemCounter}
\newtcolorbox{problem}[2][]{colback=white, colframe=black, boxrule=0.5mm, arc=4mm, auto outer arc, title={\ifstrempty{#1}{Problem \stepcounter{problemCounter}\theproblemCounter}{#1}}}

% \renewcommand{\labelenumi}{\alph{enumi})}
\def\zz{{\mathbb Z}}
\def\rr{{\mathbb R}}
\def\qq{{\mathbb Q}}
\def\cc{{\mathbb C}}
\def\nn{{\mathbb N}}
\def\ss{{\mathbb S}}

\newtheorem{theorem}{Theorem}[section]
\newtheorem{corollary}{Corollary}[theorem]
\newtheorem{lemma}[theorem]{Lemma}
\newtcolorbox{proposition}[1][]{colback=white, colframe=blue, boxrule=0.5mm, arc=4mm, auto outer arc, title={Proposition #1}}
\newtcolorbox{definition}[1][]{colback=white, colframe=violet, boxrule=0.5mm, arc=4mm, auto outer arc, title={Definition #1}}
\newcommand{\Zmod}[1]{\zz/#1\zz}
\newcommand{\partFrac}[2]{\frac{\partial #1}{\partial #2}}

\newcommand\Mydiv[2]{%
$\strut#1$\kern.25em\smash{\raise.3ex\hbox{$\big)$}}$\mkern-8mu
        \overline{\enspace\strut#2}$}

\begin{document}

\begin{center}
    Math 741
    \hfill Homework 5
    \hfill \textit{Stephen Cornelius}
\end{center}


\begin{problem}[II.5.2] \\
    If $G$ is a finite $p$-group, $H \triangleleft G$ and $H \neq \langle e \rangle$, then $H \cap C(G) \neq \langle e \rangle$.
\end{problem}

\begin{proof}
   Let $H$ be a nontrivial normal subgroup of $G$. Then we have that for each conjugacy class $C$ of $G$, either $C \subseteq H$ or $C \cap H = \emptyset$ because $H$ is normal. Pick representatives of the conjugacy classes of $G$:
   \[
        a_1, a_2, \ldots, a_r,
   \]
   with $a_1, \dots, a_k \in H$ and $a_{k+1}, \ldots, a_r \notin H$. Let $C_i$ be the conjugacy class of $a_i$ in $G$, for all $i$. Thus,
   \[
        C_i \subseteq H, \quad i = 1, 2, \ldots, k, \quad \text{and} \quad C_i \cap H = \emptyset, \quad i = k+1, k+2, \ldots, r.
   \]
   By renumbering $a_1, \dots, a_k$ if necessary, we may assume $a_1, \ldots, a_s$ represent classes of size 1 (i.e., are in the center of $G$) and $a_{s+1}, \ldots, a_k$ represent classes of size greater than 1. Since $H$ is the disjoint union of these we have that 
   \[
        |H| = |H \cap C(G)| + \sum_{i=s+1}^k \frac{|G|}{|C_G(a_i)|}.
   \]
   Now $p$ divides $|H|$ and $p$ divides each term in the sum $\sum_{i=s+1}^k [G : C_G(a_i)]$. So $p$ divides their difference: $|H \cap C(G)|$. This proves $H \cap C(G) \neq \langle e \rangle$. If $|H| = p$, since $H \cap C(G) \neq \langle e \rangle$, we must have $H \leq C(G)$.
\end{proof}


\begin{problem}[II.5.5] \\
    If $P$ is a normal Sylow $p$-subgroup of a finite group $G$ and $f: G \to G$ is an endomorphism, then $f(P) \leq P$.
\end{problem}

\begin{proof}
    Since $P$ is a Sylow $p$-subgroup of $G$, we know that $|P| = p^k$ for some integer $k \geq 0$, and that $P$ is maximal with respect to this property. Since $P$ is normal in $G$, we have that for any $g \in G$, the conjugate $gPg^{-1} = P$. 

    Now, consider the endomorphism $f: G \to G$. The image of $P$ under $f$, denoted by $f(P)$, is a subgroup of $G$. We need to show that $f(P) \leq P$. 

    First, note that since $f$ is a homomorphism, it preserves the group operation. Therefore, for any elements $x, y \in P$, we have
    \[
        f(xy) = f(x)f(y).
    \]
    This shows that the image of the product of two elements in $P$ is the product of their images, which means that $f(P)$ is closed under the group operation.

    Next, we need to show that the order of $f(P)$ divides the order of $P$. Since $P$ is a finite group of order $p^k$, any subgroup of $P$ must have an order that is a power of $p$. Therefore, the order of $f(P)$ must be of the form $p^m$ for some integer $m \leq k$. 

    Now, since $P$ is normal in $G$, for any element $g \in G$, we have
    \[
        g f(P) g^{-1} = f(gPg^{-1}) = f(P),
    \]
    which shows that $f(P)$ is also normal in $G$. 

    Finally, since both $P$ and $f(P)$ are Sylow $p$-subgroups of $G$, and Sylow's theorems state that all Sylow $p$-subgroups are conjugate to each other (Second Sylow Theorem), it follows that there exists some element $g \in G$ such that
    \[
        f(P) = gPg^{-1}.
    \]
    However, since $P$ is normal in $G$, we have
    \[
        gPg^{-1} = P.
    \]
    Therefore, we conclude that
    \[
        f(P) \leq P.
    \]
\end{proof}


\begin{problem}[II.5.7] \\
    Find the Sylow $2$-subgroups and Sylow $3$-subgroups of $S_3, S_4,$ and $S_5$.
\end{problem}

\begin{enumerate}[(a)]
    \item $S_3$: 
        \begin{itemize}
            \item Sylow 2-subgroups: There are three Sylow 2-subgroups, each of order 2. They are generated by the transpositions:
            \[
                \langle (1\ 2) \rangle, \quad \langle (1\ 3) \rangle, \quad \langle (2\ 3) \rangle.
            \]
            \item Sylow 3-subgroup: There is one Sylow 3-subgroup, which is of order 3. It is generated by the 3-cycles:
            \[
                \langle (1\ 2\ 3) \rangle.
            \]
        \end{itemize}
    \item $S_4$:
        \begin{itemize}
            \item From Proposition II.6.3 we have that there are (up to isomorphism) exactly two distinct nonabelian groups of order 8: $D_4$ and $Q_8$. We have that $|S_4| = 24 = 2^3 \cdot 3$, so the Sylow 2-subgroups of $S_4$ are of order 8. The Sylow 2-subgroups of $S_4$ are isomorphic to $D_4$. There are three such Sylow 2-subgroups, which can be described as follows:
            \[
                \langle (1\ 2), (1\ 3)(2\ 4) \rangle, \quad \langle (1\ 3), (1\ 2)(3\ 4) \rangle, \quad \langle (1\ 4), (1\ 2)(3\ 4) \rangle.
            \]
            These are the only Sylow 2-subgroups of $S_4$ since any other subgroup of order 8 would have to be isomorphic to $Q_8$, which cannot be embedded in $S_4$.
            \\
            \item Sylow 3-subgroup: There are four Sylow 3-subgroups, each of order 3. This is because the number of Sylow 3-subgroups, denoted by $n_3$, must satisfy the congruence $n_3 \equiv 1 \pmod{3}$ and also divide the order of the group. Assuming the group order is such that these conditions are met, we find that $n_3 = 4$ satisfies both requirements. They are generated by the 3-cycles:
            \[
                \langle (1\ 2\ 3) \rangle, \quad \langle (1\ 2\ 4) \rangle, \quad \langle (1\ 3\ 4) \rangle, \quad \langle (2\ 3\ 4) \rangle.
            \]
            There are only four 3-cycles in $S_4$, so these are all the Sylow 3-subgroups. There are no other subgroups of order 3 in $S_4$.
            \\
        \end{itemize}
    \item $S_5$:
        \begin{itemize}
            \item Sylow 2-subgroups: There are fifteen Sylow 2-subgroups, each of order 8 as $|S_5| = 120 = 2^3 \cdot 3 \cdot 5$ and by Sylow's Theorems we have that $15 \equiv 1 \pmod{2}$ and $15| 120$. They can be generated by various combinations of transpositions and products of disjoint transpositions. For example, one such Sylow 2-subgroup is:
            \[
                \langle (1\ 2), (3\ 4), (1\ 3)(2\ 4) \rangle.
            \]
            Other Sylow 2-subgroups can be found by considering different sets of transpositions and their products. They are all isomorphic to $D_4$. There are no Sylow 2-subgroups isomorphic to $Q_8$ in $S_5$.
            \\
            \item Sylow 3-subgroups: There are ten Sylow 3-subgroups, each of order 3. This is because the number of Sylow 3-subgroups, denoted by $n_3$, must satisfy the congruence $n_3 \equiv 1 \pmod{3}$ and also divide the order of the group. Assuming the group order is such that these conditions are met, we find that $n_3 = 10$ satisfies both requirements. They are generated by the 3-cycles. 
        \end{itemize}
\end{enumerate}

\newpage
\begin{problem}[II.5.8] \\
    If every Sylow $p$-group of a finite group $G$ is normal for every prime $p$, then $G$ is the direct product of its Sylow subgroups.
\end{problem}

\begin{proof}
    %TODO: Look this over
    Let $|G| = p_1^{k_1} p_2^{k_2} \cdots p_m^{k_m}$ be the prime factorization of the order of $G$, where $p_1, p_2, \ldots, p_m$ are distinct primes and $k_1, k_2, \ldots, k_m$ are positive integers. Let $P_i$ be a Sylow $p_i$-subgroup of $G$ for each $i = 1, 2, \ldots, m$. By assumption, each $P_i$ is normal in $G$.

    Since each $P_i$ is normal in $G$, we have that for any $g \in G$ and any $x \in P_i$, the conjugate $g x g^{-1} \in P_i$. This implies that the product of any two elements from different Sylow subgroups commutes. Specifically, for any $x \in P_i$ and $y \in P_j$ with $i \neq j$, we have
    \[
        xy = yx.
    \]
    % Is the above necessary?

    Now, consider the product of all Sylow subgroups:
    \[
        H = P_1 P_2 \cdots P_m.
    \]
    Since the orders of the Sylow subgroups are relatively prime (i.e., $\gcd(|P_i|, |P_j|) = 1$ for $i \neq j$), it follows that the order of $H$ is given by
    \[
        |H| = |P_1| |P_2| \cdots |P_m| = p_1^{k_1} p_2^{k_2} \cdots p_m^{k_m} = |G|.
    \]
    Therefore, we have that $H = G$.

    To show that $G$ is the direct product of its Sylow subgroups, we need to verify that the intersection of any two distinct Sylow subgroups is trivial. Suppose there exists an element $x \in P_i \cap P_j$ for some $i \neq j$. Then the order of $x$ must divide both $|P_i|$ and $|P_j|$. However, since the orders of these subgroups are powers of distinct primes, the only element that can satisfy this condition is the identity element. Thus, we have
    \[
        P_i \cap P_j = \{e\} \quad \text{for } i \neq j.
    \]
    Therefore, we conclude that
    \[
        G \cong P_1 \times P_2 \times \cdots \times P_m,
    \]
    where $P_1, P_2, \ldots, P_m$ are the Sylow subgroups of $G$. Hence, $G$ is the direct product of its Sylow subgroups.
\end{proof}


\begin{problem}[II.5.9] \\ 
    If $|G| = p^n q$ with $p > q$ primes, then $G$ contains a unique normal subgroup of index $q$.
\end{problem}

\begin{proof}
    Let $|G| = p^n q$ where $p$ and $q$ are distinct primes with $p > q$. By Sylow's theorems, the number of Sylow $q$-subgroups of $G$, denoted by $n_q$, satisfies the following conditions:
    \begin{enumerate}
        \item $n_q \equiv 1 \mod q$
        \item $n_q$ divides $p^n$
    \end{enumerate}

    Since $p^n$ is a power of the prime $p$ and does not contain the prime factor $q$, the only divisors of $p^n$ are powers of $p$. Therefore, the possible values for $n_q$ are limited to powers of $p$. The smallest power of $p$ is 1, which satisfies both conditions:
    \[
        n_q = 1 \equiv 1 \mod q
    \]
    and
    \[
        1 \text{ divides } p^n.
    \]

    Since $n_q = 1$, there is exactly one Sylow $q$-subgroup in $G$. Let this unique Sylow $q$-subgroup be denoted by $Q$. Because there is only one such subgroup, it must be normal in $G$. 

    The index of this subgroup in $G$ is given by
    \[
        [G : Q] = \frac{|G|}{|Q|} = \frac{p^n q}{q} = p^n.
    \]
    Thus, we have found a unique normal subgroup of index $q$ in $G$, which is the Sylow $q$-subgroup $Q$. 

    Therefore, we conclude that if $|G| = p^n q$ with $p > q$ primes, then $G$ contains a unique normal subgroup of index $q$.
\end{proof}



\begin{problem} \\
    Let $G$ be a group and $H_1$ and $H_2$ be two subgroups. Construct bijections between the following sets:
    \begin{enumerate}[(a)]
        \item The quotient of $G/H_1$ by the action of $H_2$ (on the left).
        \item The quotient of $G/H_2$ by the action of $H_1$ (on the left).
        \item The quotient of $G$ by the action of $H_1 \times H_2$ with $H_1$ action on the left and $H_2$ acting on the right (i.e., $H_1 \times H_2$ as a subgroup of $G \times G$).
        \item The quotient of $(G/H_1) \times (G/H_2)$ with $G$ acting on two copies simultaneously (this is called the \textit{diagonal action}).
    \end{enumerate}
    (Going between definitions sometimes requires inverting elements of $g$.) The resulting set is the \textit{double coset} space $H_1 \backslash G / H_2$; it can be interpreted as the set of \textit{double cosets} $H_1 g H_2$. %= \{ h_1 g h_2 : h_1 \in H_1, h_2 \in H_2 \}$.
\end{problem}

Bijection between (a) and (b):
Define $\phi_1 : (G/H_1)/H_2 \to (G/H_2)/H_1$ by $\phi_1(H_2(gH_1)) = H_1(gH_2)$. This map is well-defined because if $gH_1 = g'H_1$ for some $g, g' \in G$, then $g' = gh$ for some $h \in H_1$, and thus $H_1(g'H_2) = H_1(ghH_2) = H_1(gH_2)$. Surjectivity follows since for any $H_1(gH_2) \in (G/H_2)/H_1$, we can find a corresponding $H_2(gH_1) \in (G/H_1)/H_2$. Injectivity follows from the fact that if $\phi_1(H_2(gH_1)) = \phi_1(H_2(g'H_1))$, then by the definition of $\phi_1$, we have $H_1(gH_2) = H_1(g'H_2)$. This equality implies that the cosets $gH_2$ and $g'H_2$ are the same, since $H_1$ is well-defined and respects the equivalence relation. Consequently, $gH_2 = g'H_2$ leads to $gH_1 = g'H_1$, as $g$ and $g'$ must belong to the same coset with respect to $H_1$. Therefore, $\phi_1$ is injective.

Bijection between (b) and (c):
Define $\phi_2 : (G/H_2)/H_1 \to G/(H_1 \times H_2)$ by $\phi_2(H_1(gH_2)) = (H_1 \times H_2)(g)$. This map is well-defined because if $gH_2 = g'H_2$ for some $g, g' \in G$, then $g' = gh$ for some $h \in H_2$, and thus $(H_1 \times H_2)(g') = (H_1 \times H_2)(gh) = (H_1 \times H_2)(g)$. Surjectivity follows since for any $(H_1 \times H_2)(g) \in G/(H_1 \times H_2)$, we can find a corresponding $H_1(gH_2) \in (G/H_2)/H_1$. Injectivity follows from the fact that if $\phi_2(H_1(gH_2)) = \phi_2(H_1(g'H_2))$, then $(H_1 \times H_2)(g) = (H_1 \times H_2)(g')$, which implies $g' = h_1 g h_2$ for some $h_1 \in H_1$ and $h_2 \in H_2$, and hence $gH_2 = g'H_2$.

Bijection between (c) and (d):
Define $\phi_3 : G/(H_1 \times H_2) \to ((G/H_1) \times (G/H_2))/G$ by $\phi_3((H_1 \times H_2)(g)) = G(gH_1, gH_2)$. This map is well-defined because if $(H_1 \times H_2)(g) = (H_1 \times H_2)(g')$ for some $g, g' \in G$, then $g' = h_1 g h_2$ for some $h_1 \in H_1$ and $h_2 \in H_2$, and thus $G(g'H_1, g'H_2) = G(h_1 g h_2 H_1, h_1 g h_2 H_2) = G(gH_1, gH_2)$. Surjectivity follows since for any $G(gH_1, gH_2) \in ((G/H_1) \times (G/H_2))/G$, we can find a corresponding $(H_1 \times H_2)(g) \in G/(H_1 \times H_2)$. Injectivity follows from the fact that if $\phi_3((H_1 \times H_2)(g)) = \phi_3((H_1 \times H_2)(g'))$, then by the definition of $\phi_3$, we have $G(gH_1, gH_2) = G(g'H_1, g'H_2)$. This equality implies that the cosets $(gH_1, gH_2)$ and $(g'H_1, g'H_2)$ are identical under the action of $G$. Consequently, there exists some $h \in G$ such that $g' = hg$, where $h$ is an element of the group $G$ that relates $g$ and $g'$. Substituting this back, we see that $(H_1 \times H_2)(g) = (H_1 \times H_2)(g')$, which confirms that $\phi_3$ is injective because distinct elements in the domain map to distinct elements in the codomain.


\begin{problem} \\ 
    Fix $n$ and put $S_n$; for any $m \leq n$, let $H_m \subset S_n$ be the subgroup $S_m \times S_{n-m}$. %(permutations of the first $m$ elements and permutations of the last $n-m$ elements)`'
    The quotient $G / H_m$ can be identified with the set of  $m$-element subsets of the set $\{1, 2, \ldots, n\}$. (How?) Show that the double quotient $H_{m_1} \backslash G / H_{m_2}$ is a finite set with 
    \[
        \min(m_1, m_2) - \max(0, m_1 + m_2 - n) + 1
    \]
    elements. (Hint: the set counts the number of possible relative positions of two subsets of size $m_1$ and $m_2$.% in an $n$-element set.)
\end{problem}


\begin{proof}
    Let $G = S_n$ and consider the subgroups $H_{m_1} = S_{m_1} \times S_{n-m_1}$ and $H_{m_2} = S_{m_2} \times S_{n-m_2}$. The quotient $G / H_{m_1}$ can be identified with the set of $m_1$-element subsets of $\{1, 2, \ldots, n\}$, and similarly, $G / H_{m_2}$ can be identified with the set of $m_2$-element subsets of $\{1, 2, \ldots, n\}$. 

    The double quotient $H_{m_1} \backslash G / H_{m_2}$ represents the set of orbits of the action of $H_{m_1}$ on the left cosets of $H_{m_2}$ in $G$. Each orbit corresponds to a distinct way of positioning an $m_1$-element subset relative to an $m_2$-element subset within the $n$-element set.

    To determine the number of distinct relative positions, we analyze how many elements can be shared between the two subsets. Let $k$ denote the number of elements common to both subsets. The value of $k$ is constrained by the following:
    % 1. **Maximum possible overlap**: The maximum number of elements that can be shared between the two subsets is $\min(m_1, m_2)$, since neither subset can have more than $m_1$ or $m_2$ elements.
    % 2. **Minimum possible overlap**: The minimum number of elements that can be shared depends on the total number of elements in the set. If $m_1 + m_2 \leq n$, the subsets can be disjoint, so the minimum overlap is $k = 0$. However, if $m_1 + m_2 > n$, the subsets must share at least $m_1 + m_2 - n$ elements, since there are not enough elements in the set to avoid overlap.

    % Combining these constraints, the possible values of $k$ range from $\max(0, m_1 + m_2 - n)$ to $\min(m_1, m_2)$. The total number of possible values for $k$ is therefore given by:
    % \[
    % \min(m_1, m_2) - \max(0, m_1 + m_2 - n) + 1.
    % \]
    
    Suppose $A$ and $B$ are $m_1$-element and $m_2$-element subsets, respectively. The intersection $A \cap B$ can have at most $\min(|A|, |B|) = \min(m_1, m_2)$ elements, since the intersection cannot exceed the size of the smaller subset.

    Similarly, if $m_1 + m_2 \leq n$, the subsets can be disjoint, so the minimum overlap is $k = 0$. If $m_1 + m_2 > n$, the subsets must share at least $m_1 + m_2 - n$ elements, because there are only $n$ elements in total, and the subsets together contain $m_1 + m_2$ elements.

    Thus we have that the double quotient $H_{m_1} \backslash G / H_{m_2}$ has exactly $\min(m_1, m_2) - \max(0, m_1 + m_2 - n) + 1$ elements, corresponding to the possible values of $k$.
\end{proof}


\begin{problem} \\ 
    (A follow-up to II.5.9) Suppose $G$ is a finite group, and that $p$ is the smallest prime factor of $|G|$. Show that any subgroup $H \subset G$ of index $p$ is normal.% in $G$. (Hint: if $|G| = pm$, then $|S_p| = p!$ does not divide $m$.)
    (One possible way to prove this: consider the action of $H$ on $G/H$, and notice that the trivial coset $H$ is a fixed point.)
\end{problem}

\begin{proof}
    Let $S$ be the set of all left cosets of $H$ in $G$. Since $[G : H] = p$, the group $G$ acts on $S$ by left multiplication, and this action induces a homomorphism $\phi: G \to A(S)$, where $A(S)$ is the group of permutations of $S$. Since $|S| = p$, we have $A(S) \cong S_p$, the symmetric group on $p$ elements.

    Let $K = \ker(\phi)$ be the kernel of this homomorphism. By the First Isomorphism Theorem, $G/K \cong \operatorname{im}(\phi)$, which is a subgroup of $A(S) \cong S_p$. Hence, $|G/K|$ divides $|S_p| = p!$. Furthermore, since $K = \ker(\phi)$, it is a normal subgroup of $G$, and $K \subseteq H$ because $H$ is the stabilizer of the trivial coset $H$ under the action of $G$ on $S$.

    Now, since $|G| = pm$, we know that $p$ is the smallest prime factor of $|G|$. This implies that $p!$ cannot divide $m$, because $p!$ contains factors larger than $p$ that are not divisors of $|G|$. Therefore, the only divisors of $|G/K| = [G : K]$ that are consistent with $|G| = |K|[G : K]$ are $1$ and $p$.

    Since $[G : K] = |G/K| \geq p$, it follows that $|G/K| = p$. Thus, $[H : K] = \frac{|H|}{|K|} = 1$, which implies that $H = K$. Since $K$ is normal in $G$, we conclude that $H$ is normal in $G$.
\end{proof}

\end{document}