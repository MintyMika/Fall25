\documentclass{article}
\usepackage{amsmath}
\usepackage{tcolorbox}
\usepackage[margin=0.5in]{geometry} 
\usepackage{amsmath,amsthm,amssymb,amsfonts, fancyhdr, color, comment, graphicx, environ}
\usepackage{float}
\usepackage{xcolor}
\usepackage{mdframed}
\usepackage[shortlabels]{enumitem}
\usepackage{indentfirst}
\usepackage{mathrsfs}
\usepackage{hyperref}
\usepackage{extarrows}
\graphicspath{./}
\makeatletter
\newcommand*{\rom}[1]{\expandafter\@slowromancap\romannumeral #1@}
\makeatother

% Define a new environment for problems
\newcounter{problemCounter}
\newtcolorbox{problem}[2][]{colback=white, colframe=black, boxrule=0.5mm, arc=4mm, auto outer arc, title={\ifstrempty{#1}{Problem \stepcounter{problemCounter}\theproblemCounter}{#1}}}

% \renewcommand{\labelenumi}{\alph{enumi})}
\def\zz{{\mathbb Z}}
\def\rr{{\mathbb R}}
\def\qq{{\mathbb Q}}
\def\cc{{\mathbb C}}
\def\nn{{\mathbb N}}
\def\ss{{\mathbb S}}

\newtheorem{theorem}{Theorem}[section]
\newtheorem{corollary}{Corollary}[theorem]
\newtheorem{lemma}[theorem]{Lemma}
\newtcolorbox{proposition}[1][]{colback=white, colframe=blue, boxrule=0.5mm, arc=4mm, auto outer arc, title={Proposition #1}}
\newtcolorbox{definition}[1][]{colback=white, colframe=violet, boxrule=0.5mm, arc=4mm, auto outer arc, title={Definition #1}}
\newcommand{\Zmod}[1]{\zz/#1\zz}
\newcommand{\partFrac}[2]{\frac{\partial #1}{\partial #2}}

\newcommand\Mydiv[2]{%
$\strut#1$\kern.25em\smash{\raise.3ex\hbox{$\big)$}}$\mkern-8mu
        \overline{\enspace\strut#2}$}

\begin{document}

\begin{center}
    Math 741
    \hfill Homework 5
    \hfill \textit{Stephen Cornelius}
\end{center}


% Start of Problems

\begin{problem} \\ 
    Verify that the following construction is a contravariant functor $F$ from the category of sets to itself: for a set $A, F(A)$ is the power set of $A$ (that is, the set of all the subsets of $A$), while for a map $f : A \to B$, the induced map $F(f) : F(B) \to F(A)$ sends $X \subset B$ to $f^{-1}(X) \subset A$.
\end{problem}

\begin{proof}
    % We need to verify two properties of a contravariant functor:
    % \begin{enumerate}
    %     \item For any sets $A$ and $B$ and any functions $f : A \to B$ and $g : B \to C$, we have 
    %     \[
    %         F(g \circ f) = F(f) \circ F(g).
    %     \]
    %     \item For any set $A$, we have 
    %     \[
    %         F(\operatorname{id}_A) = \operatorname{id}_{F(A)}.
    %     \]
    % \end{enumerate}

    Let $X \subset C$. Then 
    \[
        F(g \circ f)(X) = (g \circ f)^{-1}(X) = f^{-1}(g^{-1}(X)) = F(f)(F(g)(X)).
    \]
    Thus, $F(g \circ f) = F(f) \circ F(g)$. \\
    Finally, we check the identity property. Let $X \subset A$. Then 
    \[
        F(\operatorname{id}_A)(X) = (\operatorname{id}_A)^{-1}(X) = X.
    \]
    Thus, $F(\operatorname{id}_A) = \operatorname{id}_{F(A)}$. \\
    Since both properties of \textbf{Definition X.1.2} are satisfied, $F$ is indeed a contravariant functor from the category of sets to itself.
\end{proof}



\begin{problem} \\ 
    Let $\mathcal{C}$ be any category. Fix an object $a \in \mathcal{C}$ and suppose that for any object $x \in \mathcal{C}$, the product $a \times x$ exists. Show that the correspondence
    \[
        x \mapsto a \times x
    \]
    is a functor. (Technically, there is a subtlety here: we  know that if a product exists, it is unique up to isomorphism, however, in order to construct a functor, we need to make a specific choice of $a \times x$ for all $x$. It would be better to say that the functor is defined up to a cononical isomorphism, but this detail is usually ignored.)
\end{problem}

\begin{proof}
    We need to verify two properties of a functor:
    \begin{enumerate}
        \item For any objects $x, y, z \in \mathcal{C}$ and any morphisms $f : x \to y$ and $g : y \to z$, we have 
        \[
            F(g \circ f) = F(g) \circ F(f).
        \]
        \item For any object $x \in \mathcal{C}$, we have 
        \[
            F(\operatorname{id}_x) = \operatorname{id}_{F(x)}.
        \]
    \end{enumerate}

    Let $f : x \to y$ and $g : y \to z$ be morphisms in $\mathcal{C}$. By the universal property of products, there exist unique morphisms $F(f) : a \times x \to a \times y$ and $F(g) : a \times y \to a \times z$ such that the following diagrams commute:
    \[
        \begin{array}{ccc}
            a \times x & \xrightarrow{F(f)} & a \times y \\
            \downarrow{\pi_a} & & \downarrow{\pi_a} \\
            a & = & a
        \end{array}
    \quad
        \begin{array}{ccc}
            a \times y & \xrightarrow{F(g)} & a \times z \\
            \downarrow{\pi_a} & & \downarrow{\pi_a} \\
            a & = & a
        \end{array}
    \]
    Now, consider the composition $g \circ f : x \to z$. By the universal property of products again, there exists a unique morphism $F(g \circ f) : a \times x \to a \times z$ such that the following diagram commutes:
    \[
        \begin{array}{ccc}
            a \times x & \xrightarrow{F(g \circ f)} & a \times z \\
            \downarrow{\pi_a} & & \downarrow{\pi_a} \\ 
            a & = & a
        \end{array}
    \]
    Since both $F(g) \circ F(f)$ and $F(g \circ f)$ satisfy the same universal property, by the uniqueness part of the universal property of products, we have 
    \[
        F(g \circ f) = F(g) \circ F(f).
    \] \\
    Finally, we check the identity property. For any object $x \in \mathcal{C}$, by the universal property of products, there exists a unique morphism $F(\operatorname{id}_x) : a \times x \to a \times x$ such that the following diagram commutes:
    \[
        \begin{array}{ccc}
            a \times x & \xrightarrow{F(\operatorname{id}_x)} & a \times x \\
            \downarrow{\pi_a} & & \downarrow{\pi_a} \\ 
            a & = & a
        \end{array}
    \]
    Since both $\operatorname{id}_{a \times x}$ and $F(\operatorname{id}_x)$ satisfy the same universal property, by the uniqueness part of the universal property of products, we have 
    \[
        F(\operatorname{id}_x) = \operatorname{id}_{a \times x}.
    \]
    Therefore we have that $x \mapsto a \times x$ is indeed a functor.
\end{proof}



\begin{problem} \\
    Let $V$ be a linear space and $P : V \to V$ be a linear operator such that $P^2 = P$. (Operators having this property are called \textit{projectors}.) Show that $V$ is the internal direct sum of subspaces $\mathrm{Im}(P)$ and $\mathrm{Ker}(P)$.
\end{problem}

\begin{proof}
    Notice that for any $x \in V$ we have that $x = x - P(x) + P(x)$. Then $P(x - P(x)) = P(x) - P^2(x) = 0$. So we have that $x - P(x) \in \operatorname{ker} (P)$ and clearly $P(x) \in \operatorname{im} (P)$. Now suppose $x \in \operatorname{ker} (P) \cap \operatorname{im} (P)$. Fix $y$ with $P(x) = y$ then we have that $0 = P(x) =P^2(x) = P(y)$. So $0 = P(x) =P(y) \iff x = y = 0$. Therefore we have that $V = \operatorname{ker} (P) \oplus \operatorname{im} (P)$.
\end{proof}



\begin{problem} \\ 
    Continuing with the previous problem, suppose that $\dim (V) = n$ Prove that there exists a basis of $V$ such that the matrix of $P$ is of the form $\operatorname{diag}(1,1, \dots, 1,0,0, \dots, 0)$. ($\operatorname{diag}$ denotes the diagonal matrix with given entries.)
\end{problem}


\begin{proof}
    If we fix bases $\mathscr{B}$ and $\mathscr{C}$ for $\operatorname{im} (P)$ and $\operatorname{ker} (P)$ respectively, then from question 3 we have that $\mathscr{B} \cup \mathscr{C}$ is a basis for $V$. Then for any $x \in \mathscr{B}$, fix $y_x \in V$ such that $x = P(y_x)$ so we have $P(x) = P^2(y_x) = P(y_x) = x$. Then for any $y \in \mathscr{C}$, we have $P(y) = 0$ by definition. Therefore the matrix representation of $P$ with respect to the basis $\mathscr{B} \cup \mathscr{C}$ is of the form $\operatorname{diag}(1,1, \dots, 1,0,0, \dots, 0)$.
\end{proof}


\begin{problem} \\ 
    Fix $n$, and consider the vector space: 
    \[
        V = \{ p(t) \in \rr[t] : \deg (p) \leq n \}
    \]
    over $\rr$. Fix $n + 1$ numbers $a_0, \dots, a_n \in \rr$ and consider the map
    \[
        \phi : V \to \rr^{n + 1} ; \quad \phi (p) = (p(a_0), \dots, p(a_n)).
    \]
    Show that $\phi$ is invertible (and therefore an isomorphism of vector spaces) if and only if $a_i$'s are all distinct.
\end{problem}


\begin{proof}
    ($\Leftarrow$) Suppose the $a_i$'s are all distinct. We first show that $\phi$ is injective. Suppose $\phi(p) = \phi(q)$ for some $p, q \in V$. Then we have that $p(a_i) = q(a_i)$ for all $0 \leq i \leq n$. Thus, the polynomial $r(t) = p(t) - q(t)$ has $n + 1$ distinct roots $a_0, a_1, \dots, a_n$. Since $\deg(r) \leq n$, we must have $r(t) \equiv 0$, which implies that $p(t) = q(t)$. Therefore, $\phi$ is injective. \\
    To show that $\phi$ is surjective, let $(b_0, b_1, \dots, b_n) \in \rr^{n + 1}$ be any vector. We need to find a polynomial $p(t) \in V$ such that $\phi(p) = (b_0, b_1, \dots, b_n)$. This is equivalent to solving the system of equations:
    \[
        p(a_i) = b_i \quad \text{for } i = 0, 1, \dots, n.
    \]
    Since the $a_i$'s are distinct, we have:
    \[
        p(t) = \sum_{i=0}^{n} b_i \prod_{\substack{0 \leq j \leq n \\ j \neq i}} \frac{t - a_j}{a_i - a_j}
    \]
    is a polynomial of degree at most $n$ that satisfies the above equations. Thus, $\phi$ is surjective. Since $\phi$ is both injective and surjective, it is invertible. \\

    ($\Rightarrow$) Now suppose that the $a_i$'s are not all distinct. Without loss of generality, assume that $a_0 = a_1$. We will show that $\phi$ is not injective. Consider the polynomial $p(t) = t - a_0$. Then we have:
    \[
        \phi(p) = (p(a_0), p(a_1), \dots, p(a_n)) = (0, 0, p(a_2), \dots, p(a_n)).
    \]
    Now consider the zero polynomial $q(t) = 0$. We have:
    \[
        \phi(q) = (q(a_0), q(a_1), \dots, q(a_n)) = (0, 0, 0, \dots, 0).
    \]
    Since $\phi(p) \neq \phi(q)$, we have found two distinct polynomials $p$ and $q$ such that $\phi(p) = \phi(q)$. Therefore, $\phi$ is not injective, and hence not invertible.
    
\end{proof}




\begin{problem} \\ 
    Let $V$ be a vector space over a field $K$. A (linear) functional on $V$ is a linear operator $\phi : V \to K$. Show that if two functionals $\phi, \psi : V \to K$ satisfy $\operatorname{ker}(\phi) \subset \operatorname{ker}(\psi)$, then there exists $a \in K$ such that $\psi = a \phi$. (If you need to, you can assume that $V$ is finite-dimensional, but it should not be necessary.)
\end{problem}


\begin{proof}
    If $\phi=0$ then $\ker\phi=V$, so $\ker\psi=V$ and hence $\psi=0$; take $a=0$.  
    Otherwise pick $v_0\in V$ with $\phi(v_0)=1$ and set $a=\psi(v_0)$. For any $v\in V$ put
    \[
        u:=v-\phi(v)v_0.
    \]
    Then $\phi(u)=\phi(v)-\phi(v)\phi(v_0)=0$, so $u\in\ker\phi \subset \ker\psi$, hence $\psi(u)=0$. Therefore
    \[
        0=\psi(u)=\psi(v)-\phi(v)\psi(v_0)=\psi(v)-a\phi(v),
    \]
    so $\psi(v)=a\phi(v)$ for all $v\in V$. Uniqueness of $a$ follows since $a=\psi(v_0)$.
\end{proof}


\begin{problem} \\ 
    Let $K$ be a field and $L \subset K$ be a smaller field (e.g., $L = \rr$ and $K = \cc$). Given a $K$-vector space $V$, we can also consider it as a $L$-vector space, so we get two notions of dimension: as a vector space over $K$ and as a vector space over $L$. Denote them by $\dim_K V$ and $\dim_L V$, respectively. Show that 
    \[
        \dim_L V = (\dim_K V) \cdot (\dim_L K).
    \]
\end{problem}


\begin{proof}
    Let $\{v_1, v_2, \dots, v_n\}$ be a basis for $V$ over $K$ and let $\{k_1, k_2, \dots, k_m\}$ be a basis for $K$ over $L$. We claim that the set 
    \[
        \{k_i v_j : 1 \leq i \leq m, 1 \leq j \leq n\}
    \]
    is a basis for $V$ over $L$. 

    First we show that this set spans $V$ over $L$. Take any $v \in V$. Since $\{v_1, v_2, \dots, v_n\}$ is a basis for $V$ over $K$, we can write 
    \[
        v = a_1 v_1 + a_2 v_2 + \dots + a_n v_n
    \]
    for some $a_i \in K$. Each $a_i$ can be expressed as a linear combination of the basis elements of $K$ over $L$, i.e., 
    \[
        a_i = b_{i1} k_1 + b_{i2} k_2 + \dots + b_{im} k_m
    \]
    for some $b_{ij} \in L$. Substituting this back into the expression for $v$, we get 
    \[
        v = \sum_{j=1}^n \left( \sum_{i=1}^m b_{ij} k_i \right) v_j = \sum_{i=1}^m \sum_{j=1}^n b_{ij} (k_i v_j),
    \]
    which shows that $v$ can be expressed as an $L$-linear combination of the elements in our set.

    Next, we show that the set is linearly independent over $L$. Suppose we have 
    \[
        \sum_{i=1}^m \sum_{j=1}^n c_{ij} (k_i v_j) = 0
    \]
    for some $c_{ij} \in L$. Rearranging gives 
    \[
        \sum_{j=1}^n \left( \sum_{i=1}^m c_{ij} k_i \right) v_j = 0.
    \]
    Since $\{v_1, v_2, \dots, v_n\}$ is a basis for $V$ over $K$, the coefficients must all be zero, i.e., 
    \[
        \sum_{i=1}^m c_{ij} k_i = 0 \quad \forall j.
    \]
    But since $\{k_1, k_2, \dots, k_m\}$ is a basis for $K$ over $L$, we have that $c_{ij} = 0$ for all $i,j$. This shows that our set is linearly independent over $L$.

    Since we have shown that the set spans $V$ over $L$ and is linearly independent over $L$, we conclude that it is a basis for $V$ over $L$. Therefore we have 
    \[
        \dim_L V = m \cdot n = (\dim_K V) \cdot (\dim_L K).
    \]
\end{proof}



\end{document}