\documentclass{article}
\usepackage{amsmath}
\usepackage{tcolorbox}
\usepackage[margin=0.5in]{geometry} 
\usepackage{amsmath,amsthm,amssymb,amsfonts, fancyhdr, color, comment, graphicx, environ}
\usepackage{float}
\usepackage{xcolor}
\usepackage{mdframed}
\usepackage[shortlabels]{enumitem}
\usepackage{indentfirst}
\usepackage{mathrsfs}
\usepackage{hyperref}
\graphicspath{./}
\makeatletter
\newcommand*{\rom}[1]{\expandafter\@slowromancap\romannumeral #1@}
\makeatother

% Define a new environment for problems
\newcounter{problemCounter}
\newtcolorbox{problem}[2][]{colback=white, colframe=black, boxrule=0.5mm, arc=4mm, auto outer arc, title={\ifstrempty{#1}{Problem \stepcounter{problemCounter}\theproblemCounter}{#1}}}

% \renewcommand{\labelenumi}{\alph{enumi})}
\def\zz{{\mathbb Z}}
\def\rr{{\mathbb R}}
\def\qq{{\mathbb Q}}
\def\cc{{\mathbb C}}
\def\nn{{\mathbb N}}
\def\ss{{\mathbb S}}

\newtheorem{theorem}{Theorem}[section]
\newtheorem{corollary}{Corollary}[theorem]
\newtheorem{lemma}[theorem]{Lemma}
\newtcolorbox{proposition}[1][]{colback=white, colframe=blue, boxrule=0.5mm, arc=4mm, auto outer arc, title={Proposition #1}}
\newtcolorbox{definition}[1][]{colback=white, colframe=violet, boxrule=0.5mm, arc=4mm, auto outer arc, title={Definition #1}}
\newcommand{\Zmod}[1]{\zz/#1\zz}
\newcommand{\partFrac}[2]{\frac{\partial #1}{\partial #2}}

\newcommand\Mydiv[2]{%
$\strut#1$\kern.25em\smash{\raise.3ex\hbox{$\big)$}}$\mkern-8mu
        \overline{\enspace\strut#2}$}

\begin{document}

\begin{center}
    Math 741
    \hfill Homework 1
    \hfill \textit{Stephen Cornelius}
\end{center}

\begin{problem}[Exercise 1.2.2] \\
    A group $G$ is abelian if and only if the map $G \to G$ given by $x \mapsto x^{-1}$ is an automorphism.
\end{problem}

\begin{proof}
    ($\Rightarrow$) Suppose $G$ is abelian. We want to show that the map $f: G \to G$ given by $f(x) = x^{-1}$ is an automorphism. First, we show that $f$ is a homomorphism. Let $a,b \in G$ and consider $f(ab)$. We have
    \[
        f(ab) = (ab)^{-1} = b^{-1}a^{-1} = a^{-1}b^{-1} = f(a)f(b),
    \]
    where the third equality follows from the fact that $G$ is abelian. Next, we show that $f$ is bijective. To see that $f$ is injective, suppose $f(a) = f(b)$ for some distinct $a,b \in G$. Then we have
    \[
        f(a) = a^{-1} = b^{-1} = f(b).
    \]
    Since inverses are unique in a group, we must have $a = b$, a contradiction. Thus, $f$ is injective. To see that $f$ is surjective, let $y \in G$. We want to find an $x \in G$ such that $f(x) = y$. Note that if we let $x = y^{-1}$, then we have
    \[
        f(x) = x^{-1} = y.
    \]
    Thus, $f$ is surjective. Since $f$ is a bijective homomorphism, it is an automorphism.

    ($\Leftarrow$) Suppose the map $f: G \to G$ given by $x \mapsto x^{-1}$ is an automorphism. We want to show that $G$ is abelian. Let $a,b \in G$. Since $f$ is a homomorphism, we have
    \[
        f(ab) = f(a)f(b).
    \]
    Expanding both sides, we have
    \[
        (ab)^{-1} = a^{-1}b^{-1}.
    \]
    Taking the inverse of both sides, we have
    \[
        ab = (a^{-1}b^{-1})^{-1} = ba,
    \]
    where the last equality follows from the property of inverses in a group. Thus, $G$ is abelian.
\end{proof}


\begin{problem}[Exercise 1.2.3] \\
    Let $Q_8$ be the group (under ordinary matrix multiplication) generated by the complex matrices $A = \begin{pmatrix}
        0 & 1 \\
        -1 & 0
    \end{pmatrix}$ and $B = \begin{pmatrix}
        0 & i \\
        i & 0
    \end{pmatrix}$, where $i^2 = -1$. Show that $Q_8$ is a nonabelian group of order $8$. $Q_8$ is called the \textbf{quaternion group}. [\textit{Hint:} Observe that $BA = A^3B$, whence every element of $Q_8$ is of the form $A^i B^j$. Note also that $A^4 = B^4 = I$, where $I = \begin{pmatrix}
        1 & 0 \\
        0 & 1
    \end{pmatrix}$ is the identity element of $Q_8$.]
\end{problem}

\begin{proof}
    Following the hint first we compute $BA$.
    \[
        BA = \begin{pmatrix}
            0 & 1 \\
            -1 & 0
        \end{pmatrix} \begin{pmatrix}
            0 & i \\
            i & 0 
        \end{pmatrix}
         = \begin{pmatrix}
            -i & 0 \\
            0 & i
         \end{pmatrix}.
    \]
    Next we compute $A^3 B$.
    \[
         A^3 B = 
         \begin{pmatrix}
            0 & 1 \\
            -1 & 0 
         \end{pmatrix}^3 \begin{pmatrix}
            0 & i \\
            i & 0
         \end{pmatrix} = \begin{pmatrix}
            -1 & 0 \\
            0 & -1
         \end{pmatrix} \begin{pmatrix}
            0 & 1 \\
            -1 & 0 
         \end{pmatrix} \begin{pmatrix}
            0 & i \\
            i & 0
         \end{pmatrix}
          = \begin{pmatrix}
            0 & -1 \\
            1 & 0
          \end{pmatrix} \begin{pmatrix}
            0 & i \\
            i & 0
         \end{pmatrix}
         = \begin{pmatrix}
            -i & 0 \\
            0 & i 
         \end{pmatrix}.
    \]
    Therefore we have that $BA = A^3 B$ therefore we have that every element is of the form $A^iB^j$. Notice next that $A^4 = B^4 = I$ where $I$ is the identity matrix. Thus, the possible values for $i$ and $j$ are $0,1,2,3$. This gives us a total of $4 \cdot 4 = 16$ possible combinations of $A^i B^j$. However, we can reduce this number by noting that $A^2 = B^2$. Thus, we have the following distinct elements of $Q_8$:
    \[
        I, A, A^2, A^3, B, AB, A^2B, A^3B.
    \]
    Thus, $|Q_8| = 8$. Finally, we show that $Q_8$ is nonabelian. To see this, we compute $AB$ and $BA$.
    \[
        AB = \begin{pmatrix}
            0 & 1 \\
            -1 & 0
        \end{pmatrix} \begin{pmatrix}
            0 & i \\
            i & 0
        \end{pmatrix} = \begin{pmatrix}
            i & 0 \\
            0 & -i
        \end{pmatrix}, \quad
        BA = \begin{pmatrix}
            0 & i \\
            i & 0
        \end{pmatrix} \begin{pmatrix}
            0 & 1 \\
            -1 & 0
        \end{pmatrix} = \begin{pmatrix}
            -i & 0 \\
            0 & i
        \end{pmatrix}.
    \]
    Since $AB \neq BA$, we conclude that $Q_8$ is nonabelian.
\end{proof}






\begin{problem}[Exercise 1.4.8] \\
    If $H$ and $K$ are subgroups of finite index of a group $G$ such that $\left[ G : H \right]$ and $\left[ G : K \right]$ are relatively prime, then $G = HK$.
\end{problem}

\begin{proof}
    Let $H$ and $K$ be subgroups of finite index of a group $G$ such that $[G:H]$ and $[G:K]$ are relatively prime. Let $[G:H] = m$ and $[G:K] = n$. We want to show that $G = HK$. Notice first that $H \cap K$ is a subgroup of both $H$ and $K$. So by Theorem 1.4.5 we have that
    \begin{align*}
        [G: H \cap K] = [G : H][H : H \cap K] & \iff [G: H \cap K] = m[H : H \cap K] \\
        [G: H \cap K] = [G : K][K : H \cap K] & \iff [G: H \cap K] = n[K : H \cap K] \\
    \end{align*}
    Then by substitution we have $m[H : H \cap K] = n[K : H \cap K]$. Since $(m,n) = 1$, we have $m \vert [K : H \cap K]$ and $n | [H : H \cap K]$. For brevity, let $[H:H\cap K] = a, [K:H\cap K] = b$. Then,
    \begin{align*}
        [G : H \cap K] = mna \\
        [G : H \cap K] = mnb. \\
    \end{align*}
    This implies that $a = b$. By Proposition 1.4.8, $[H:H\cap K] \leq [G : K]$. This yields $na \leq n$, which forces $a = 1$. Then $[G: H \cap K] = [G : H][G : K]$ so by Proposition 1.4.9 we have that $G = HK$ as desired. \\
\end{proof}


\begin{problem}[Exercise 1.4.12] \\
    If $H$ and $K$ are subgroups of a group $G$, then $\left[ H \vee K : H\right] \geq \left[ K : H \cap K\right]$.
\end{problem}

\begin{proof}
    Let $H$ and $K$ be subgroups of a group $G$. Notice then that $ H < H \vee K$ and $K < H \vee K$. From Theorem 1.4.8 we have that $[K:K\cap H] \leq [H\vee K : H]$, that is, $[H\vee K : H] \geq [K:K\cap H]$ as desired.
\end{proof}



\begin{problem}[Exercise 1.4.13] \\
    If $p > q$ are primes, a group of order $pq$ has at most one subgroup of order $p$. \newline
    [\textit{Hint:} Suppose $H,K$ are distinct subgroups of order $p$. Show that $H \cap K = \langle e \rangle$; use Exercise 1.2.12 to get a contradiction.]
\end{problem}


\begin{proof}
    Let $p > q$ be primes and let $G$ be a group of order $pq$. Suppose $H,K$ are distinct subgroups of order $p$. We want to show that $H \cap K = \langle e \rangle$. To see this, let $x \in H \cap K$. Since $H$ and $K$ are subgroups of order $p$, we have that the order of any element in $H$ or $K$ must divide $p$ by Lagrange's Theorem. Thus, the possible orders for $x$ are $1$ or $p$. If the order of $x$ is $1$, then we have that $x = e$. If the order of $x$ is $p$, then we have that $\langle x \rangle = H = K$, a contradiction since we assumed that $H$ and $K$ are distinct. Therefore, we must have that the order of $x$ is $1$, and thus we have that $H \cap K = \langle e \rangle$. \\
    Following the hint, we use Exercise 1.2.12 to get a contradiction. \\
    We can see that $[K : H \cap K] = [K : \langle e \rangle] = p$. Next, we note that $H \vee K$ is a subgroup of $G$ that contains both $H$ and $K$. Thus, we have that $|H \vee K|$ must be a multiple of both $|H|$ and $|K|$. Since $|H| = |K| = p$, we have that $|H \vee K|$ must be a multiple of $p$. The possible multiples of $p$ that are less than or equal to $pq$ are $p$ and $pq$. If $|H \vee K| = p$, then we have that $H \vee K = H = K$, a contradiction since we assumed that $H$ and $K$ are distinct. Thus, we must have that $|H \vee K| = pq$. Therefore, we have that $H \vee K = G$. We also have that $\vert G \vert = pq = [H \vee K : H] \cdot \vert H \vert = [H \vee K : H] \cdot p$. Dividing both sides by $p$, we have that $q = [H \vee K : H]$. Thus, we have that $[H \vee K : H] = q$. \\
    Finally, we note that since $p > q$, we have that $[H \vee K : H] = q < p = [K : H \cap K]$. This contradicts Exercise 1.4.12 which states that $[H \vee K : H] \geq [K : H \cap K]$. Therefore, we conclude that a group of order $pq$ has at most one subgroup of order $p$. \\
\end{proof}


 \newpage
\begin{problem}[Exercise 1.5.1] \\
    If $N$ is a subgroup of index $2$ in a group $G$, then $N$ is normal in $G$.
\end{problem}


\begin{proof}
    Let $N$ be a subgroup of index $2$ in a group $G$. We want to show that $N$ is normal in $G$. Choose $g \in G$ arbitrarily. If $g \in N$, then we have that $gN = N = Ng$. If $g \notin N$ then, since there are only two left cosets of $N$ in $G$, and $g \notin N$ we must have that the cosets are $gN$ and $N$. We also have that cosets partition $G$ so we have that $G = N \cup gN$. Similarly, we have that the right cosets of $N$ in $G$ are $Ng$ and $N$. Since cosets partition $G$, we have that $G = N \cup Ng$. Thus, we have that $gN = Ng$. Since $g \in G$ was arbitrarily chosen, we conclude that $N$ is normal in $G$. \\
\end{proof}



\begin{problem}[Exercise 1.5.6] \\
    Let $H < G$; then the set $aHa^{-1}$ is a subgroup for each $a \in G$, and $H \cong aHa^{-1}$.
\end{problem}

\begin{proof}
    Let $H < G$. We want to show that the set $aHa^{-1}$ is a subgroup for each $a \in G$, and that $H \cong aHa^{-1}$. First, we show that $aHa^{-1}$ is a subgroup of $G$. \\
    First we show that $aHa^{-1} < G$ for all $a \in G$. \\
    Let $a \in G$ be arbitrarily chosen. Then, by definition we have that $aHa^{-1} = \left\{ aha^{-1} \vert h \in H \right\}$. Let $x,y \in aHa^{-1}$ then we have that $x = ah_1a^{-1}$ and $y = ah_2a^{-1}$ for some $h_1,h_2 \in H$ by definition. Now consider $xy^{-1}$. We have
    \[
        xy^{-1} = (ah_1a^{-1})(ah_2a^{-1})^{-1} = (ah_1a^{-1})(ah_2^{-1}a^{-1}) = ah_1h_2^{-1}a^{-1}.
    \]
    Clearly, $ah_1h_2^{-1}a^{-1} \in aHa^{-1}$. Therefore, by Theorem 1.2.5 $aHa^{-1} < G$ for all $a \in G$ as $a$ was arbitrarily chosen. \\
    Next we show that $H \cong aHa^{-1}$. \\
    Let $\varphi : H \to aHa^{-1}$ be given by $x \mapsto axa^{-1}$. We first show $\varphi$ is a homomorphism. \\
    Let $x,y \in H$ and consider $\varphi(xy)$,
    \[
        \varphi(xy) = axya^{-1} = axeya^{-1} = axa^{-1}aya^{-1} = \varphi(x)\varphi(y).
    \]
    Therefore $\varphi$ is a homomorphism. \\
    Next we show injectivity. Let $x,y$ be distinct elements of $H$. For sake of contradiction, suppose $\varphi(x) = \varphi(y)$. Then we have
    \begin{multline*}
        \varphi(x) = \varphi(y) \\
        axa^{-1} = aya^{-1} \\
        \implies a^{-1}axa^{-1} = a^{-1}aya^{-1} \\
        \implies xa^{-1}a = ya^{-1}a \\
        \implies x = y. \\
    \end{multline*}
    This a contradiction, therefore we have that $\varphi$ is injective. \\
    Next we show surjectivity. Let $y \in aHa^{-1}$. We want to find an $x \in H$ such that $\varphi(x) = y$. Note that if we let $x = a^{-1}ya$, then we have
    \[
        \varphi(x) = axa^{-1} = a(a^{-1}ya)a^{-1} = y.
    \]
    Thus, $\varphi$ is surjective. \\
    Since $\varphi$ is a bijective homomorphism, we conclude that $H \cong aHa^{-1}$ as desired. \\
\end{proof}



\begin{problem}[Exercise 1.5.7] \\
    Let $G$ be a finite group and $H$ a subgroup of $G$ of order $n$. If $H$ is the only subgroup of $G$ of order $n$, then $H$ is normal in $G$.
\end{problem}

\begin{proof}
    Let $G$ be a finite group and $H$ a subgroup of $G$ of order $n$. Suppose $H$ is the only subgroup of $G$ of order $n$. We want to show that $H$ is normal in $G$. To see this, let $a \in G$ be arbitrarily chosen. We want to show that $aHa^{-1} = H$. First, we note that since $H$ is a subgroup of $G$, we have that $aHa^{-1}$ is also a subgroup of $G$. Then from Exercise 1.5.6, we have that $H \cong aHa^{-1}$. Thus, we have that $|H| = |aHa^{-1}| = n$. Since $H$ is the only subgroup of $G$ of order $n$, we must have that $aHa^{-1} = H$. Since $a \in G$ was arbitrarily chosen, we conclude that $H$ is normal in $G$.
\end{proof}



\begin{problem}[Exercise 1.6.3] \\
    If $\sigma = \left( i_1i_2\dots i_r \right) \in S_n$ and $\tau \in S_n$, then $\tau \sigma \tau^{-1}$ is the $r$-cycle $(\tau(i_1)\tau(i_2)\dots\tau(i_r))$.
\end{problem}

\begin{proof}
    Let $\sigma = (i_1 i_2 \dots i_r) \in S_n$ and $\tau \in S_n$. We want to show that $\tau \sigma \tau^{-1}$ is the $r$-cycle $(\tau(i_1) \tau(i_2) \dots \tau(i_r))$. To see this, let $x \in \left\{ 1,2,\dots,n \right\}$. We consider two cases. \\
    Case 1: Suppose $x = \tau(i_k)$ for some $k \in \left\{ 1,2,\dots,r \right\}$. Then we have
    \[
        (\tau \sigma \tau^{-1})(x) = (\tau \sigma)(\tau^{-1}(x)) = (\tau \sigma)(i_k) = \tau(i_{k+1}),
    \]
    where the last equality follows from the definition of $\sigma$ and we take $i_{r+1} = i_1$. Thus, we have that $(\tau \sigma \tau^{-1})(\tau(i_k)) = \tau(i_{k+1})$ for all $k \in \left\{ 1,2,\dots,r \right\}$. \\
    Case 2: Suppose $x \neq \tau(i_k)$ for all $k \in \left\{ 1,2,\dots,r \right\}$. Then we have
    \[
        (\tau \sigma \tau^{-1})(x) = (\tau \sigma)(\tau^{-1}(x)) = (\tau)(\tau^{-1}(x)) = x,
    \]
    where the second equality follows from the definition of $\sigma$ since $\tau^{-1}(x) \neq i_k$ for all $k$. Thus, we have that $(\tau \sigma \tau^{-1})(x) = x$ for all $x$ not in the set $\left\{ \tau(i_1),\tau(i_2),\dots,\tau(i_r) \right\}$.\\
    Combining both cases, we have that $\tau\sigma\tau^{-1}$ sends $\tau(i_k)$ to $\tau(i_{k+1})$ for all $k$ and fixes all other elements. Therefore, we conclude that $\tau \sigma \tau^{-1}$ is the $r$-cycle $(\tau(i_1) \tau(i_2) \dots \tau(i_r))$ as desired. \\
\end{proof}


\begin{problem}[Exercise 1.6.8] \\
    The group $A_4$ has no subgroup of order $6$.
\end{problem}

\begin{proof}
    Suppose for sake of contradiction that $A_4$ has a subgroup $H$ of order $6$. Since $|A_4| = 12$, we have that the index of $H$ in $A_4$ is given by
    \[
        [A_4 : H] = \frac{|A_4|}{|H|} = \frac{12}{6} = 2.
    \]
    Thus, we have that $H$ is a subgroup of index $2$ in $A_4$. From Exercise 1.5.1, we have that any subgroup of index $2$ in a group is normal. Therefore, we have that $H$ is normal in $A_4$. \\
    Next, we note that since $H$ is a subgroup of order $6$, it must contain an element of order $3$ by Cauchy's Theorem (Theorem 2.5.2). Then we have that since $H$ is normal in $A_4$ and contains an element of order $3$, we have that $H = A_4$ by Theorem 1.6.12 which is a contradiction since $|H| = 6$ and $|A_4| = 12$. Therefore, we conclude that $A_4$ has no subgroup of order $6$. \\
\end{proof}


\newpage
\begin{problem}[Exercise 1.6.12] \\
    The center (Exercise 1.2.11) of the group $D_n$ is $\langle e \rangle$ if $n$ is odd and isomorphic to $\zz_2$ if $n$ is even.
    
\end{problem}


\begin{proof}
    %Idea: When n is odd the center is trivial, when n is even the center is e and pi rotation
    Let $D_n$ be the dihedral group of order $2n$ with generators $\{r,s\}$ where $r$ is a rotation of order $n$ and $s$ is a reflection of order $2$. We want to show that the center of $D_n$ is $\langle e \rangle$ if $n$ is odd and isomorphic to $\zz_2$ if $n$ is even. First, we consider the case when $n$ is odd. \\
    Case 1: Suppose $n$ is odd. We want to show that the center of $D_n$ is $\langle e \rangle$. To see this, let $x \in Z(D_n)$. We want to show that $x = e$. Since $D_n$ is generated by a rotation $r$ of order $n$ and a reflection $s$ of order $2$, we have that any element in $D_n$ can be written as either $r^k$ or $r^ks$ for some integer $k$. We consider two cases. \\
    Subcase 1: Suppose $x = r^k$ for some integer $k$. Since $x \in Z(D_n)$, we have that $xr = rx$. Thus, we have
    \[
        r^k r = r r^k \implies r^{k+1} = r^{k+1},
    \]
    which is true for all integers $k$. Next, since $x \in Z(D_n)$, we have that $xs = sx$. Thus, we have
    \[
        r^k s = s r^k \implies r^{2k} = e,
    \]
    where the last equality follows from the relation $s r^k s = r^{-k}$. Since $n$ is odd, we have that $r^{2k} = e$ if and only if $k$ is a multiple of $n$. Thus, we have that $x = r^k = e$. \\
    Subcase 2: Suppose $x = r^k s$ for some integer $k$. Since $x \in Z(D_n)$, we have that $xr = rx$. Thus, we have
    \[
        r^k s r = r r^k s \implies r^{k-1} s = r^{k+1} s \implies r^{-2} = e,
    \]
    where the last equality follows from the relation $s r^k s = r^{-k}$. Since $n$ is odd, we have that $r^{-2} = e$ is a contradiction. Therefore, we must have that $x \neq r^k s$ for any integer $k$. \\
    Combining both subcases, we have that $x = e$. Since $x \in Z(D_n)$ was arbitrarily chosen, we conclude that $Z(D_n) = \langle e \rangle$ when $n$ is odd. \\
    Case 2: Suppose $n$ is even. We want to show that the center of $D_n$ is isomorphic to $\zz_2$. To see this, let $x \in Z(D_n)$. We want to show that $x$ is either $e$ or $r^{n/2}$. Since $D_n$ is generated by a rotation $r$ of order $n$ and a reflection $s$ of order $2$, we have that any element in $D_n$ can be written as either $r^k$ or $r^k s$ for some integer $k$. We consider two cases. \\
    Subcase 1: Suppose $x = r^k$ for some integer $k$. Since $x \in Z(D_n)$, we have that $xr = rx$. Thus, we have
    \[
        r^k r = r r^k \implies r^{k+1} = r^{k+1},
    \]
    which is true for all integers $k$. Next, since $x \in Z(D_n)$, we have that $xs = sx$. Thus, we have
    \[
        r^k s = s r^k \implies r^{2k} = e,
    \]
    where the last equality follows from the relation $s r^k s = r^{-k}$. Since $n$ is even, we have that $r^{2k} = e$ if and only if $k$ is a multiple of $n/2$. Thus, we have that $x = r^k$ is either $e$ or $r^{n/2}$. \\
    Subcase 2: Suppose $x = r^k s$ for some integer $k$. Since $x \in Z(D_n)$, we have that $xr = rx$. Thus, we have
    \[
        r^k s r = r r^k s \implies r^{k-1} s = r^{k+1} s \implies r^{-2} = e,
    \]
    where the last equality follows from the relation $s r^k s = r^{-k}$. Since $n$ is even, we have that $r^{-2} = e$ is a contradiction. Therefore, we must have that $x \neq r^k s$ for any integer $k$. \\
    Combining both subcases, we have that $x$ is either $e$ or $r^{n/2}$. Since $x \in Z(D_n)$ was arbitrarily chosen, we conclude that $Z(D_n) = \{e, r^{n/2}\}$ when $n$ is even. Finally, we note that the set $\{e, r^{n/2}\}$ is isomorphic to $\zz_2$ since it is a group of order $2$ under the operation of composition. Therefore, we conclude that the center of $D_n$ is $\langle e \rangle$ if $n$ is odd and isomorphic to $\zz_2$ if $n$ is even as desired. \\
\end{proof}

\end{document}