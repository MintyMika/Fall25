\documentclass{article}
\usepackage{amsmath}
\usepackage{tcolorbox}
\usepackage[margin=0.5in]{geometry} 
\usepackage{amsmath,amsthm,amssymb,amsfonts, fancyhdr, color, comment, graphicx, environ}
\usepackage{float}
\usepackage{xcolor}
\usepackage{mdframed}
\usepackage[shortlabels]{enumitem}
\usepackage{indentfirst}
\usepackage{mathrsfs}
\usepackage{hyperref}
\usepackage{extarrows}
\graphicspath{./}
\makeatletter
\newcommand*{\rom}[1]{\expandafter\@slowromancap\romannumeral #1@}
\makeatother

% Define a new environment for problems
\newcounter{problemCounter}
\newtcolorbox{problem}[2][]{colback=white, colframe=black, boxrule=0.5mm, arc=4mm, auto outer arc, title={\ifstrempty{#1}{Problem \stepcounter{problemCounter}\theproblemCounter}{#1}}}

% \renewcommand{\labelenumi}{\alph{enumi})}
\def\zz{{\mathbb Z}}
\def\rr{{\mathbb R}}
\def\qq{{\mathbb Q}}
\def\cc{{\mathbb C}}
\def\nn{{\mathbb N}}
\def\ss{{\mathbb S}}

\newtheorem{theorem}{Theorem}[section]
\newtheorem{corollary}{Corollary}[theorem]
\newtheorem{lemma}[theorem]{Lemma}
\newtcolorbox{proposition}[1][]{colback=white, colframe=blue, boxrule=0.5mm, arc=4mm, auto outer arc, title={Proposition #1}}
\newtcolorbox{definition}[1][]{colback=white, colframe=violet, boxrule=0.5mm, arc=4mm, auto outer arc, title={Definition #1}}
\newcommand{\Zmod}[1]{\zz/#1\zz}
\newcommand{\partFrac}[2]{\frac{\partial #1}{\partial #2}}

\newcommand\Mydiv[2]{%
$\strut#1$\kern.25em\smash{\raise.3ex\hbox{$\big)$}}$\mkern-8mu
        \overline{\enspace\strut#2}$}

\begin{document}

\begin{center}
    Math 741
    \hfill Homework 2
    \hfill \textit{Stephen Cornelius}
\end{center}

\begin{problem}[Exercise 1.5.14] \\
    If $N_1 \triangleleft G_1$, $N_2 \triangleleft G_2$, then $(N_1 \times N_2) \triangleleft (G_1 \times G_2)$ and $(G_1 \times G_2)/(N_1 \times N_2) \cong (G_1/N_1) \times (G_2/N_2)$.
\end{problem}



\begin{problem}[1.6.11] \\ 
    Find all normal subgroups of $D_n$.
\end{problem}



\begin{problem}[1.8.2] \\ 
    Give an example of groups $H_i, K_j$ such that $H_1 \times H_2 \cong K_1 \times K_2$ and no $H_i$ is isomorphic to any $K_j$.
\end{problem}



\begin{problem}[1.8.3] \\ 
    Let $G$ be an (additive) abelian group with subgroups $H$ and $K$. Show that $G \cong H \oplus K$ if and only if there are homomorphisms $H \mathrel{\mathop{\leftrightarrows}^{{\pi_1}}_{\iota_1}} G \mathrel{\mathop{\leftrightarrows}^{{\pi_2}}_{\iota_2}} K$ such that $\pi_1\iota_1 = 1_H, \pi_2\iota_2 = 1_K, \pi_1\iota_2 = 0,$ and $\pi_2\iota_1 = 0$, where $0$ is the map sending every element onto the zero (identity) element, and $\iota_1\pi_1(x) + \iota_2\pi_2(x) = x$ for all $x \in G$.
\end{problem}




\begin{problem}[1.8.5] \\ 
    Let $G, H$ be finite cyclic groups. Then $G \times H$ is cyclic if and only if $(|G|,|H|) = 1$.
\end{problem}



\begin{problem}[1.8.9] \\ 
    If a group $G$ is the (internal) direct product of its subgroups $H, K$, then $H \cong G/K$ and $G/H \cong K$.
\end{problem}




\begin{problem}[1.9.1] \\ 
    Every nonidentity element in a free group $F$ has infinite order.
\end{problem}




\begin{problem}[1.9.4] \\ 
    Let $F$ be the free group on the set $X$, and let $Y \subset X$. If $H$ is the smallest normal subgroup of $F$ containing $Y$, then $F/H$ is a free group.
\end{problem}




\end{document}