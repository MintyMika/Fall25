\documentclass{article}
\usepackage{amsmath}
\usepackage{tcolorbox}
\usepackage[margin=0.5in]{geometry} 
\usepackage{amsmath,amsthm,amssymb,amsfonts, fancyhdr, color, comment, graphicx, environ}
\usepackage{float}
\usepackage{xcolor}
\usepackage{mdframed}
\usepackage[shortlabels]{enumitem}
\usepackage{indentfirst}
\usepackage{mathrsfs}
\usepackage{hyperref}
\usepackage{extarrows}
\graphicspath{./}
\makeatletter
\newcommand*{\rom}[1]{\expandafter\@slowromancap\romannumeral #1@}
\makeatother

% Define a new environment for problems
\newcounter{problemCounter}
\newtcolorbox{problem}[2][]{colback=white, colframe=black, boxrule=0.5mm, arc=4mm, auto outer arc, title={\ifstrempty{#1}{Problem \stepcounter{problemCounter}\theproblemCounter}{#1}}}

% \renewcommand{\labelenumi}{\alph{enumi})}
\def\zz{{\mathbb Z}}
\def\rr{{\mathbb R}}
\def\qq{{\mathbb Q}}
\def\cc{{\mathbb C}}
\def\nn{{\mathbb N}}
\def\ss{{\mathbb S}}

\newtheorem{theorem}{Theorem}[section]
\newtheorem{corollary}{Corollary}[theorem]
\newtheorem{lemma}[theorem]{Lemma}
\newtcolorbox{proposition}[1][]{colback=white, colframe=blue, boxrule=0.5mm, arc=4mm, auto outer arc, title={Proposition #1}}
\newtcolorbox{definition}[1][]{colback=white, colframe=violet, boxrule=0.5mm, arc=4mm, auto outer arc, title={Definition #1}}
\newcommand{\Zmod}[1]{\zz/#1\zz}
\newcommand{\partFrac}[2]{\frac{\partial #1}{\partial #2}}

\newcommand\Mydiv[2]{%
$\strut#1$\kern.25em\smash{\raise.3ex\hbox{$\big)$}}$\mkern-8mu
        \overline{\enspace\strut#2}$}

\begin{document}

\begin{center}
    Math 741
    \hfill Homework 3
    \hfill \textit{Stephen Cornelius}
\end{center}



\begin{problem}[1.9.6] \\ 
    The cyclic group of order $6$ is the group defined by generators $a,b$ and relations $a^2 = b^3 = a^{-1}b^{-1}ab = e$.
\end{problem}

\begin{proof}
    Let $X = \{a, b\}$ and let $F(X)$ be the free group on $X$. Consider the normal subgroup $N$ of $F(X)$ generated by the relations $a^2 = e$, $b^3 = e$, and $a^{-1}b^{-1}ab = e$. Then $G \cong F(X)/N$.

    We claim that $G$ is cyclic of order $6$, i.e., $G \cong \zz/6\zz$. The relation $a^{-1}b^{-1}ab = e$ implies $ab = ba$, so $G$ is abelian. The relations $a^2 = e$ and $b^3 = e$ show that the orders of $a$ and $b$ are $2$ and $3$, respectively.

    Since $G$ is abelian and generated by $a$ and $b$, every element of $F(X)/N$ can be written as $a^i b^j N$ for $i \in \{0,1\}$ and $j \in \{0,1,2\}$, giving $2 \times 3 = 6$ elements, so $|G| \leq 6$.

    Define a homomorphism $\varphi: F(X)/N \to \zz/6\zz$ by $\varphi(a) = 3$ and $\varphi(b) = 2$. This is well-defined since $3 + 3 = 0$ and $2 + 2 + 2 = 0$ in $\zz/6\zz$, matching the relations. The commutativity also matches the structure of $\zz/6\zz$.

    Since $\varphi$ is surjective and $|F(X)/N| \leq 6$, it follows that $|F(X)/N| = 6$ and $\varphi$ is an isomorphism. Thus, $G \cong F(X)/N \cong \zz/6\zz$, as claimed.
\end{proof}


% Copy of proof for 1.9.6
% Let $G$ be the cyclic group of order $6$. Let $X = \{a,b\}$ be the generators with relations $a^2 = b^3 = a^{-1}b^{-1}ab = e$. Let $N$ be the normal subgroup of $F(X)$ generated by the relations. Since $G$ is cyclic of order $6$, we know that $G \cong \zz/6\zz$. We will show that $\zz/6\zz \cong F(X)/N$. Since $\zz/6\zz$ is generated by $2$ and $3$, with $3 + 3 = 2 + 2 + 2 = -3 + -2 + 3 + 2 = 0$, we have that by Theorem 1.9.5, there exists an epimorphism $\varphi: F(X)/N \to \zz/6\zz$. Hence, $\vert F(X)/N \vert \geq \vert \zz/6\zz \vert = 6$. Next we show that every element of $F(X)/N$ is of the form $a^i b^j N$ where $i \in \{0,1\}$ and $j \in \{0,1,2\}$. Consider any arbitrary element $\tilde{x} =\sum_{k=1}^n a^{i_k} b^{j_k} N \in F(X)/N$ with $i_k = 1$ and $j_k = 1$ or $2$. Assume $\tilde{x}$ is in reduced form.  Since $a^{-1}b^{-1}ab = e$, we have that the operation is commutative, so that $ab = ba$, and hence we may rearrange the coset and reduce the word to get it to the form $a^ib^jN, i = 0, 1$ and $j = 0, 1, 2$ as desired (where if, after rearranging, $i$ or $j$ exceed or equal $2$ or $3$ respectively, that $a^2$ or $b^3$ gets absorbed into $N$, as $a^2N = b^3N = N$). Thus, there are at most 6 elements in $F(X)/N$, meaning that $\varphi$ is an isomorphism.


\begin{problem}[1.9.10] \\ 
    The operation of free product is commutative and associative: for any groups $A,B,C, A * B \cong B * A$ and $(A * B) * C \cong A * (B * C)$.
\end{problem}

\begin{proof}
    Let $A$, $B$, $C$ be groups. First, we show that $A * B \cong B * A$. The free product $A * B$ consists of all reduced words formed by alternating elements from $A$ and $B$, and the same holds for $B * A$. Therefore we can define a homomorphism $\varphi : A * B \to B*A$ where $\varphi$ is the identity map. The correspondence that sends each reduced word in $A * B$ to the same word in $B * A$ (just swapping the roles of $A$ and $B$). Clearly we have that $\varphi$ is surjective. To see that $\varphi$ is injective, note that if $\varphi(w) = e$ in $B * A$, then $w$ must be the empty word in $A * B$, since the only way for a reduced word to map to the identity is if it is itself the identity. Thus, $\ker(\varphi) = \{e\}$, so $\varphi$ is injective. Thus, $A * B \cong B * A$. \\ 
    Next, we show that $(A * B) * C \cong A * (B * C)$. We know that $(A * B) * C$ consists of reduced words alternating among $A * B$ and $C$, but each letter from $A * B$ is itself a reduced word in $A$ and $B$. By flattening, every element of $(A * B) * C$ can be written as a reduced word in $A$, $B$, and $C$, with no two consecutive letters from the same group. Likewise, $A * (B * C)$ consists of reduced words alternating among $A$ and $B * C$, and each letter from $B * C$ is a reduced word in $B$ and $C$. Again, flattening yields reduced words in $A$, $B$, and $C$, with no two consecutive letters from the same group. Define a map $\Phi: (A * B) * C \to A * (B * C)$ as follows: Given a reduced word $w$ in $(A * B) * C$, write $w$ as a sequence of letters $x_1 x_2 \cdots x_n$, where each $x_i$ is either a non-identity element from $A * B$ or $C$. If $x_i$ is from $A * B$, further decompose it into its reduced word in $A$ and $B$. Concatenate all these letters to obtain a reduced word in $A$, $B$, and $C$. This word is then interpreted as an element of $A * (B * C)$. Similarly, define the inverse map $\Psi: A * (B * C) \to (A * B) * C$ by grouping consecutive letters from $A$ and $B$ into elements of $A * B$, and letters from $C$ as elements of $C$, thus forming a reduced word in $(A * B) * C$. To see that $\Phi$ and $\Psi$ are well-defined, note that the process of flattening and grouping preserves the reduced word property: no two consecutive letters come from the same group, and the group operation is respected. It is clear that $\Phi$ and $\Psi$ are inverses of each other, since flattening and grouping are mutually inverse operations. Therefore, $\Phi$ is a bijection. Hence, $(A * B) * C \cong A * (B * C)$. \\
    Hence, the free product is both commutative and associative.
\end{proof}




\begin{problem}[1.9.11] \\ 
    If $N$ is the normal subgroup of $A *B$ generated by $A$, then $(A * B)/N \cong B$.
\end{problem}

\begin{proof}
    Let $N$ be the normal subgroup of $A * B$ generated by $A$. We want to show that $(A * B)/N \cong B$.

    The subgroup $N$ is the smallest normal subgroup containing all elements of $A$, so in the quotient $(A * B)/N$, every element of $A$ becomes identified with the identity. In other words, for any $a \in A$, $a N = N$ in the quotient.

    Now, consider any element $w \in A * B$. We can write $w$ as a reduced word $x_1 x_2 \cdots x_n$, where each $x_i$ is either in $A$ or $B$. In the quotient $(A * B)/N$, any occurrence of $x_i \in A$ can be replaced by the identity, since $x_i N = N$. Therefore, $w N$ is equal to the product of all the $x_i$ that are in $B$, with all $A$-letters removed. This product is simply an element of $B$, since the group operation in $B$ is preserved.

    Thus, every coset in $(A * B)/N$ can be uniquely represented by an element $b N$ for some $b \in B$. The identity coset is $N$, which corresponds to the identity element $e_B$ in $B$.

    Define a map $\varphi : (A * B)/N \to B$ by $\varphi(b N) = b$ for $b \in B$. To see that $\varphi$ is well-defined, note that every element of $(A * B)/N$ is of the form $b N$ for a unique $b \in B$, as shown above.

    Next, we check that $\varphi$ is a homomorphism. For any $b_1, b_2 \in B$,
    \[
        \varphi(b_1 N \cdot b_2 N) = \varphi(b_1 b_2 N) = b_1 b_2 = \varphi(b_1 N)\varphi(b_2 N).
    \]
    So $\varphi$ preserves the group operation.

    To see that $\varphi$ is surjective, observe that for any $b \in B$, $b N$ is a coset in $(A * B)/N$, and $\varphi(b N) = b$.

    To check injectivity, suppose $\varphi(b_1 N) = \varphi(b_2 N)$. Then $b_1 = b_2$, so $b_1 N = b_2 N$.

    Therefore, $\varphi$ is a well-defined isomorphism. Thus we conclude that $(A * B)/N \cong B$.
\end{proof}


\begin{problem}[1.9.12] \\ 
    If $G$ and $H$ each have more than one element, then $G * H$ is an infinite group with center $<e>$.
\end{problem}


\begin{proof}
    Let $G$ and $H$ be groups, each with more than one element. We want to show that their free product $G * H$ is infinite and that its center is trivial.

    First, we show that $G * H$ is infinite. 
    %Recall that the elements of $G * H$ are represented by reduced words, which are finite sequences of elements from $G$ and $H$, alternating between the two groups, with no consecutive elements from the same group and no identity elements except possibly at the ends. 
    Since both $G$ and $H$ have more than one element, pick $g \in G$ and $h \in H$ with $g \neq e_G$ and $h \neq e_H$. Consider the sequence of words $w_n = (g h)^n$ for $n \geq 1$. Each $w_n$ is a reduced word of length $2n$, and no two such words are equal in $G * H$ because the free product imposes no relations between $g$ and $h$ other than those in their respective groups. Thus, for every $n$, $w_n$ is distinct, and we can construct arbitrarily long reduced words. Therefore, $G * H$ is infinite.

    Next, we show that the center of $G * H$ is trivial. Recall that the center $Z(G * H)$ consists of all elements $z \in G * H$ such that $z w = w z$ for all $w \in G * H$. Clearly, the identity element $e$ is in the center. Suppose $z$ is a nontrivial reduced word in $G * H$. We will show that $z$ cannot commute with all elements of $G * H$.

    Let $z$ be a reduced word of length $k \geq 1$, say $z = x_1 x_2 \cdots x_k$, where each $x_i$ is in $G$ or $H$, and consecutive $x_i$ are from different groups. Without loss of generality, suppose $x_k \in H$. Pick $h \in H$ with $h \neq e_H$ and $h \neq x_k^{-1}$. Consider the element $w = h$. Then,
    \[
    z w = x_1 x_2 \cdots x_k h
    \]
    is a reduced word ending with $x_k h$ (which is not the identity since $h \neq x_k^{-1}$). On the other hand,
    \[
    w z = h x_1 x_2 \cdots x_k
    \]
    is a reduced word starting with $h$, which is distinct from $z w$ because the reduced word structure is different. Thus, $z w \neq w z$. A similar argument applies if $x_k \in G$ by choosing $g \in G$ with $g \neq e_G$ and $g \neq x_k^{-1}$.

    Therefore, the only element that commutes with all elements of $G * H$ is the identity. Thus, the center of $G * H$ is trivial:
    \[
    Z(G * H) = \langle e \rangle.
    \]
\end{proof}



\begin{problem}[1.9.15] \\ 
    If $f : G_1 \to G_2$ and $g : H_1 \to H_2$ are homomorphisms of groups, then there is a unique homomorphism $h : G_1 * H_1 \to G_2 * H_2$ such that $h|_{G_1} = f$ and $h|_{H_1} = g$.
\end{problem}



\begin{proof}
    Let $f : G_1 \to G_2$ and $g : H_1 \to H_2$ be group homomorphisms. We wish to construct a homomorphism $h : G_1 * H_1 \to G_2 * H_2$ such that $h|_{G_1} = f$ and $h|_{H_1} = g$, and show that it is unique.

    Recall that every element of the free product $G_1 * H_1$ can be written uniquely as a reduced word $a_1 a_2 \cdots a_n$, where each $a_i$ is a non-identity element from either $G_1$ or $H_1$, and consecutive $a_i$ are from different groups. The empty word corresponds to the identity element.

    Define $h : G_1 * H_1 \to G_2 * H_2$ as follows:
    \begin{itemize}
        \item For $g \in G_1$, set $h(g) = f(g)$.
        \item For $h_1 \in H_1$, set $h(h_1) = g(h_1)$.
        \item For a reduced word $a_1 a_2 \cdots a_n$ in $G_1 * H_1$, where each $a_i$ is in $G_1$ or $H_1$, define
        \[
            h(a_1 a_2 \cdots a_n) = h(a_1) h(a_2) \cdots h(a_n).
        \]
        \item For the identity element (the empty word), set $h(e) = e$.
    \end{itemize}

    We first verify that $h$ is a homomorphism. Let $w = a_1 a_2 \cdots a_n$ and $w' = b_1 b_2 \cdots b_m$ be reduced words in $G_1 * H_1$. The product $ww'$ is obtained by concatenating the words, and if the last letter of $w$ and the first letter of $w'$ are from the same group, their product is taken in that group and the result is reduced accordingly. Since $f$ and $g$ are homomorphisms, $h$ respects the group operations within $G_1$ and $H_1$, and the concatenation of images under $h$ corresponds to the product in $G_2 * H_2$, with reduction occurring in the same way. Thus,
    \[
        h(ww') = h(a_1 a_2 \cdots a_n b_1 b_2 \cdots b_m) = h(a_1) h(a_2) \cdots h(a_n) h(b_1) h(b_2) \cdots h(b_m) = h(w) h(w').
    \]
    Therefore, $h$ is a homomorphism.

    Next, we check that $h$ restricts to $f$ on $G_1$ and to $g$ on $H_1$. For any $g \in G_1$, $h(g) = f(g)$ by definition, and for any $h_1 \in H_1$, $h(h_1) = g(h_1)$. Thus, $h|_{G_1} = f$ and $h|_{H_1} = g$.

    Finally, we show that $h$ is unique with these properties. Suppose $h' : G_1 * H_1 \to G_2 * H_2$ is another homomorphism such that $h'|_{G_1} = f$ and $h'|_{H_1} = g$. For any reduced word $a_1 a_2 \cdots a_n$ in $G_1 * H_1$, we have
    \[
        h'(a_1 a_2 \cdots a_n) = h'(a_1) h'(a_2) \cdots h'(a_n).
    \]
    But $h'(a_i) = f(a_i)$ if $a_i \in G_1$, and $h'(a_i) = g(a_i)$ if $a_i \in H_1$, which matches the definition of $h(a_i)$. Therefore,
    \[
        h'(a_1 a_2 \cdots a_n) = h(a_1) h(a_2) \cdots h(a_n) = h(a_1 a_2 \cdots a_n).
    \]
    Thus, $h' = h$ on all elements of $G_1 * H_1$, so $h$ is unique.

    In summary, there exists a unique homomorphism $h : G_1 * H_1 \to G_2 * H_2$ such that $h|_{G_1} = f$ and $h|_{H_1} = g$.
\end{proof}




\begin{problem}[2.1.10] \\
    \begin{enumerate}[(a)]
        \item Show that the additive group of rationals $\qq$ is not finitely genertated.
        \item Show that $\qq$ is not free.
        \item Conclude that Exercise 9 is false if the hypothesis "finitely generated" is omitted.
    \end{enumerate}
\end{problem}



\begin{enumerate}[(a)]
    \item \begin{proof}
        Suppose, for contradiction, that $\qq$ is finitely generated as an abelian group. That is, there exist finitely many elements $q_1, q_2, \ldots, q_n \in \qq$ such that every rational number can be written as an integer linear combination of these generators. Write each $q_i$ in lowest terms as $q_i = \frac{a_i}{b_i}$, where $a_i \in \zz$, $b_i \in \nn$, and $\gcd(a_i, b_i) = 1$.

        Let $k = b_1 b_2 \cdots b_n$ be the product of all denominators. Consider the subgroup $H = \left\langle q_1, q_2, \ldots, q_n \right\rangle \leq \qq$. Any element $q \in H$ can be written as an integer linear combination:
        \[
        q = d_1 q_1 + d_2 q_2 + \cdots + d_n q_n = \frac{d_1 a_1}{b_1} + \frac{d_2 a_2}{b_2} + \cdots + \frac{d_n a_n}{b_n}
        \]
        for some $d_1, \ldots, d_n \in \zz$. By clearing denominators, we can write this sum as a single fraction with denominator $k$:
        \[
        q = \frac{d_1 a_1 \frac{k}{b_1} + d_2 a_2 \frac{k}{b_2} + \cdots + d_n a_n \frac{k}{b_n}}{k}
        \]
        Thus, every element of $H$ is a rational number whose denominator divides $k$; in other words, $H \subseteq \left\langle \frac{1}{k} \right\rangle$, the subgroup of $\qq$ consisting of all rational numbers with denominator dividing $k$.

        However, $\qq$ contains elements such as $\frac{1}{k+1}$, which cannot be written as an integer linear combination of elements with denominator $k$. Therefore, $\left\langle q_1, \ldots, q_n \right\rangle$ cannot be all of $\qq$, contradicting our assumption that $\qq$ is finitely generated.

        Hence, the additive group of rationals $\qq$ is not finitely generated.
    \end{proof}
    \item \begin{proof}
        Assume $\qq$ were free, say with a generating set $X$. Let $\iota:  X \to \qq$ be the inclusion map. Define $f: X \to \zz$ by $f(x) = 1$ for all $x \in X$. By the universal property of free abelian groups, there exists a unique homomorphism $\varphi: \qq \to \zz$ such that $\varphi \circ \iota = f$. Then, see that $\varphi(\iota(x)) = f(x) = 1$ for all $x \in X$. Since $\varphi$ is a homomorphism, for any $q \in \qq$, which can be expressed as a finite integer linear combination of elements from $X$, we have
        \[
            \varphi(q) = \varphi\left(\sum_{i=1}^n d_i x_i\right) = \sum_{i=1}^n d_i \varphi(x_i) = \sum_{i=1}^n d_i.
        \]
        However, this implies that $\varphi(q)$ is always an integer, which contradicts the fact that $\qq$ contains elements that cannot be mapped to integers in a way that preserves the group structure. For example, consider $q = \frac{1}{2}$. There is no integer $n$ such that $\varphi\left(\frac{1}{2}\right) = n$ while still satisfying the homomorphism property for all elements of $\qq$. Thus, $\varphi$ cannot be well-defined for all of $\qq$, contradicting the assumption that $\qq$ is free. Therefore, $\qq$ is not a free abelian group.
    \end{proof}
    \item Sine $\qq$ is an abelian group where no element (except $0$) has finite order, exercise 9 does not hold. This is the case as in (a) we showed that $\qq$ is not finitely generated, and in (b) we showed that $\qq$ is not free. Thus, the hypothesis "finitely generated" is necessary for exercise 9 to hold.
\end{enumerate}


\begin{problem} \\ 
    (Algebra Qual, Jan 2016) Let $D_k$ be the dihedral group of order $2k$, where $k \geq 3$.
    \begin{enumerate}[(a)]
        \item Show that the number of automorphisms of the grouop $D_k$ is equal to $k \cdot \varphi(k)$. Here $\varphi$ is the Euler $\varphi$-function.
        \item Automorphisms of $D_k$ form a group; let us denote it by $\text{Aut}(D_k)$. What is the structure of $\text{Aut}(D_k)$? Describe the group as explicitly as you can.
    \end{enumerate}
\end{problem}

\begin{enumerate}[(a)]
    \item \begin{proof}
        Recall that $D_k$ is generated by two elements $r$ and $s$ with relations $r^k = s^2 = e$ and $srs = r^{-1}$. The element $r$ represents a rotation by $\frac{2\pi}{k}$ radians, and $s$ represents a reflection.

        An automorphism $\varphi \in \text{Aut}(D_k)$ is determined by its action on the generators $r$ and $s$. Since $\varphi$ must preserve the order of elements, we have:
        - $\varphi(r)$ must be an element of order $k$. The elements of order $k$ in $D_k$ are precisely the powers of $r$, i.e., $\{r^m : 1 \leq m < k, \gcd(m, k) = 1\}$. There are $\varphi(k)$ such elements.
        - $\varphi(s)$ must be an element of order $2$. The elements of order $2$ in $D_k$ are the reflections, which can be written as $sr^j$ for $0 \leq j < k$. There are exactly $k$ such elements.

        Therefore, for each choice of $\varphi(r) = r^m$ (with $\gcd(m, k) = 1$), there are $k$ choices for $\varphi(s)$. Thus, the total number of automorphisms is given by:
        \[
        |\text{Aut}(D_k)| = k \cdot \varphi(k).
        \]
    \end{proof}
    \item The automorphism group $\text{Aut}(D_k)$ of the dihedral group $D_k$ can be understood by analyzing how automorphisms act on the generators of $D_k$. Recall that $D_k$ is generated by a rotation $r$ of order $k$ and a reflection $s$ of order $2$, with the relation $s r s^{-1} = r^{-1}$.

    Any automorphism must send $r$ to another element of order $k$, which must be some power $r^a$ where $a$ is coprime to $k$ (i.e., $a \in (\mathbb{Z}/k\mathbb{Z})^\times$). Similarly, $s$ can be sent to $r^b s$ for some $b \in \mathbb{Z}/k\mathbb{Z}$, since $r^b s$ is also a reflection.

    The set of possible choices for $a$ forms the group $(\mathbb{Z}/k\mathbb{Z})^\times$, and the choices for $b$ form the group $\mathbb{Z}/k\mathbb{Z}$. However, the way $a$ and $b$ interact is not independent: the choice of $a$ affects how $b$ acts, so the automorphism group is not a direct product, but a semidirect product.

    Therefore, we have:
    \[
    \text{Aut}(D_k) \cong (\mathbb{Z}/k\mathbb{Z})^\times \ltimes \mathbb{Z}/k\mathbb{Z}
    \]
    where $(\mathbb{Z}/k\mathbb{Z})^\times$ acts on $\mathbb{Z}/k\mathbb{Z}$ by multiplication.  
\end{enumerate}

% Copy of proof for 2.1.10
% \begin{proof}
%         Recall that $D_k$ is generated by two elements $r$ and $s$ with relations $r^k = s^2 = e$ and $srs = r^{-1}$. The element $r$ represents a rotation by $\frac{2\pi}{k}$ radians, and $s$ represents a reflection.

%         An automorphism $\varphi \in \text{Aut}(D_k)$ is determined by its action on the generators $r$ and $s$. Since $\varphi$ must preserve the order of elements, we have:
%         - $\varphi(r)$ must be an element of order $k$. The elements of order $k$ in $D_k$ are precisely the powers of $r$, i.e., $\{r^m : 1 \leq m < k, \gcd(m, k) = 1\}$. There are $\varphi(k)$ such elements.
%         - $\varphi(s)$ must be an element of order $2$. The elements of order $2$ in $D_k$ are the reflections, which can be written as $sr^j$ for $0 \leq j < k$. There are exactly $k$ such elements.

%         Therefore, for each choice of $\varphi(r) = r^m$ (with $\gcd(m, k) = 1$), there are $k$ choices for $\varphi(s)$. Thus, the total number of automorphisms is given by:
%         \[
%         |\text{Aut}(D_k)| = k \cdot \varphi(k).
%         \]
%     \end{proof}

\begin{problem} \\ 
    (Algebra Qual, Aug 2018) For a finite group $G$, denote by $s(G)$ the number of its subgroups.
    \begin{enumerate}[(a)]
        \item Show that $s(G)$ is finite.
        \item Show that if $H$ is a nontrivial subgroup of $G$, then $s(G/H) < s(G)$.
        \item Show that $s(g) = 2$ if and only if $G$ is a cyclic of prime order.
        \item Show that $s(G) = 3$ if and only if $G$ is cyclic group whose order is a square of a prime.
    \end{enumerate}
\end{problem}


Let $G$ be a finite group.
\begin{enumerate}[(a)]
    \item \begin{proof}
        Since $G$ is finite, it has a finite number of elements. Any subgroup $H \leq G$ is determined by a subset of $G$ that is closed under the group operation and taking inverses. The number of subsets of a finite set with $n$ elements is $2^n$, which is finite. Since not all subsets are subgroups, the number of subgroups $s(G)$ is at most $2^{|G|}$, which is finite. Therefore, $s(G)$ is finite.
    \end{proof}
    \item \begin{proof}
        Let $H$ be a nontrivial subgroup of $G$. Consider the quotient group $G/H$. There is a natural correspondence between the subgroups of $G/H$ and the subgroups of $G$ that contain $H$. Specifically, if $K/H$ is a subgroup of $G/H$, then $K$ is a subgroup of $G$ containing $H$. Conversely, if $K$ is a subgroup of $G$ containing $H$, then $K/H$ is a subgroup of $G/H$. This correspondence is bijective.

        Since $H$ is nontrivial, there exists at least one subgroup of $G$ that contains $H$, namely $H$ itself. However, not all subgroups of $G$ contain $H$. Therefore, the number of subgroups of $G/H$ is strictly less than the number of subgroups of $G$. Hence, we have:
        \[
        s(G/H) < s(G).
        \]
    \end{proof}
    \item \begin{proof}
        ($\Rightarrow$) Suppose $s(G) = 2$. The only subgroups of $G$ are the trivial subgroup $\langle e\rangle$ and $G$ itself. Then for any $g \in G$ with $g \neq e$, the subgroup $\langle g \rangle$ generated by $g$ must be either $\langle e \rangle$ or $G$. Since $g \neq e$, we have $\langle g \rangle = G$. Thus, $G$ is cyclic and generated by any of its non-identity elements. Now, if the order of $G$ were composite, say $|G| = mn$ with $m, n > 1$, then $G$ would have a subgroup of order $m$ (by Cauchy's theorem), contradicting the assumption that $s(G) = 2$. Therefore, the order of $G$ must be prime. Hence, $G \cong \zz/p\zz$ for some prime $p$.

        ($\Leftarrow$) Conversely, if $G \cong \zz/p\zz$ for some prime $p$, then the only subgroups of $G$ are $\langle e\rangle$ and $G$ itself. Thus, $s(G) = 2$.
        
        Hence, we conclude that $s(G) = 2$ if and only if $G$ is cyclic of prime order.
    \end{proof}
    \item \begin{proof}
        We prove both directions.

        ($\Rightarrow$) Suppose $G$ is a cyclic group of order $p^2$ for some prime $p$. Then $G \cong \zz/p^2\zz$, and every subgroup of $G$ is cyclic. The subgroups of a cyclic group of order $n$ correspond to the divisors of $n$. For $n = p^2$, the divisors are $1$, $p$, and $p^2$. Thus, the subgroups are:
        \begin{itemize}
            \item The trivial subgroup $\langle e \rangle$ of order $1$,
            \item The subgroup $\langle a^p \rangle$ of order $p$, where $a$ is a generator of $G$,
            \item The whole group $G$ itself, of order $p^2$.
        \end{itemize}
        There are no other divisors of $p^2$, so these are the only subgroups. Therefore, $s(G) = 3$.

        ($\Leftarrow$) Now suppose $G$ is a finite group with $s(G) = 3$. That is, $G$ has exactly three subgroups: the trivial subgroup, $G$ itself, and one proper nontrivial subgroup $H$. We claim that $G$ must be cyclic of order $p^2$ for some prime $p$.

        First, note that every group has the trivial subgroup and itself as subgroups, so the only possibility for $s(G) = 3$ is that there is exactly one proper nontrivial subgroup $H$. Consider any $a \in G$ with $a \neq e$. The subgroup $\langle a \rangle$ generated by $a$ is a subgroup of $G$. Since $s(G) = 3$, every non-identity element must generate either $G$ or $H$. If $a$ generates $G$, then $G$ is cyclic. If $a$ generates $H$, then $H$ must be cyclic as well.

        Suppose $G$ is not cyclic. Then for every $a \neq e$, $\langle a \rangle$ is a proper subgroup, so must be $H$. But then $H$ contains all non-identity elements of $G$, so $H = G$, which is a contradiction. Therefore, $G$ must be cyclic.

        Let $|G| = n$. Suppose $n = pq$ for distinct primes $p$ and $q$. Then $G$ would have subgroups of orders $p$ and $q$, contradicting the assumption that there is only one proper nontrivial subgroup. Thus, $n$ must be a power of a single prime, say $n = p^k$. If $k \geq 3$, then $G$ would have subgroups of orders $p$ and $p^2$, again contradicting the assumption. Hence, $k$ must be $1$ or $2$. If $k = 1$, then $G$ is cyclic of prime order, which has $s(G) = 2$. Thus, $k$ must be $2$.

        Therefore, $G$ is cyclic of order $p^2$ for some prime $p$.

        Thus, as desired, $s(G) = 3$ if and only if $G$ is cyclic of order $p^2$ for some prime $p$.
    \end{proof}

\end{enumerate}





\end{document}