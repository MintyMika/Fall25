\documentclass{article}
\usepackage{amsmath}
\usepackage{tcolorbox}
\usepackage[margin=0.5in]{geometry} 
\usepackage{amsmath,amsthm,amssymb,amsfonts, fancyhdr, color, comment, graphicx, environ}
\usepackage{float}
\usepackage{xcolor}
\usepackage{mdframed}
\usepackage[shortlabels]{enumitem}
\usepackage{indentfirst}
\usepackage{mathrsfs}
\usepackage{hyperref}
\usepackage{extarrows}
\graphicspath{./}
\makeatletter
\newcommand*{\rom}[1]{\expandafter\@slowromancap\romannumeral #1@}
\makeatother

% Define a new environment for problems
\newcounter{problemCounter}
\newtcolorbox{problem}[2][]{colback=white, colframe=black, boxrule=0.5mm, arc=4mm, auto outer arc, title={\ifstrempty{#1}{Problem \stepcounter{problemCounter}\theproblemCounter}{#1}}}

% \renewcommand{\labelenumi}{\alph{enumi})}
\def\zz{{\mathbb Z}}
\def\rr{{\mathbb R}}
\def\qq{{\mathbb Q}}
\def\cc{{\mathbb C}}
\def\nn{{\mathbb N}}
\def\ss{{\mathbb S}}

\newtheorem{theorem}{Theorem}[section]
\newtheorem{corollary}{Corollary}[theorem]
\newtheorem{lemma}[theorem]{Lemma}
\newtcolorbox{proposition}[1][]{colback=white, colframe=blue, boxrule=0.5mm, arc=4mm, auto outer arc, title={Proposition #1}}
\newtcolorbox{definition}[1][]{colback=white, colframe=violet, boxrule=0.5mm, arc=4mm, auto outer arc, title={Definition #1}}
\newcommand{\Zmod}[1]{\zz/#1\zz}
\newcommand{\partFrac}[2]{\frac{\partial #1}{\partial #2}}

\newcommand\Mydiv[2]{%
$\strut#1$\kern.25em\smash{\raise.3ex\hbox{$\big)$}}$\mkern-8mu
        \overline{\enspace\strut#2}$}

\begin{document}

\begin{center}
    Math 741
    \hfill Homework 3
    \hfill \textit{Stephen Cornelius}
\end{center}



\begin{problem}[II.2.1] \\ 
    Show that a finite abelian group that is not cyclic contains a subgroup which is isomorphic to $\zz_p \oplus \zz_p$ for some prime $p$.
\end{problem}


\begin{proof}
    Let $G$ be a finite abelian group that is not cyclic. By the Fundamental Theorem of Finite Abelian Groups, we can write $G$ as a direct sum of cyclic groups of prime power order:
    \[
        G \cong \zz_{p_1^{k_1}} \oplus \zz_{p_2^{k_2}} \oplus \dots \oplus \zz_{p_n^{k_n}}
    \]
    where $p_i$ are primes and $k_i$ are positive integers. Since $G$ is not cyclic we have that the direct sum has at least two cyclic components with the same prime base, i.e., there exist $i \neq j$ such that $p_i = p_j = p$. In this case, the decomposition contains $\zz_{p^a}\oplus \zz_{p^b}$ for some $a, b \geq 1$. So $G$ contains a subgroup isomorphic to $\zz_{p^a} \oplus \zz_{p^b}$. Since $\zz_{p^a}$ and $\zz_{p^b}$ both have subgroups isomorphic to $\zz_p$, we can find a subgroup of $G$ isomorphic to $\zz_p \oplus \zz_p$. Thus, we conclude that any finite abelian group that is not cyclic contains a subgroup isomorphic to $\zz_p \oplus \zz_p$ for some prime $p$.
\end{proof}


\begin{problem}[II.2.7] \\
    A (sub)group in which every element has order a power of a fixed prime $p$ is called a $p$-(sub)group (\textit{note:} $\vert 0 \vert = 1 = p^0$). Let $G$ be an abelian torsion group. 
    \begin{enumerate}[(a)]
        \item $G(p)$ is the unique maximum $p$-subgroup of $G$ (that is, every $p$-subgroup of $G$ is contained in $G(p)$).
        \item $G = \sum G(p)$, where the sum is over all primes $p$ such that $G(p) \neq 0$. \\
        {[}\textit{Hint:} If $|u| = p_1^{n_1} \dots p_t^{n_t}$. There exist $c_i \in \zz$ uch that $c_1m_1 + \dots + c_tm_t = 1$, whence $u = c_1m_1u + \dots + c_tm_tu$; but $C_im_iu \in G(p_i)$.{]}
        \item If $H$ is another abelian torsion group, then $G \cong H$ if and only if $G(p) \cong H(p)$ for all primes $p$.
    \end{enumerate}
\end{problem}



\begin{enumerate}[(a)]
    \item \begin{proof}
        Recall that $G(p) = \{ u \in G \quad |\quad |u| = p^n \text{ for some } n \geq 0 \}$. Suppose, for a contradiction, that there exists a $p$-subgroup $H$ of $G$ such that $H \not\subseteq G(p)$. Then there exists an element $h \in H$ such that $h \notin G(p)$. This means that the order of $h$ is not a power of $p$, contradicting the definition of a $p$-subgroup. Therefore, every $p$-subgroup of $G$ is contained in $G(p)$, making $G(p)$ the unique maximum $p$-subgroup of $G$.
    \end{proof}
    \item \begin{proof}
        Let $u \in G$ be an arbitrary element. Since $G$ is a torsion group, the order of $u$ is finite, say $|u| = m$. We can factor $m$ into its prime power decomposition:
        \[
            m = p_1^{n_1} p_2^{n_2} \dots p_t^{n_t}
        \]
        for distinct primes $p_i$ and positive integers $n_i$. Define $m_i = m/p_i^{n_i}$ for each $i$. By the Extended Euclidean Algorithm, there exist integers $c_1, c_2, \dots, c_t$ such that:
        \[
            c_1 m_1 + c_2 m_2 + \dots + c_t m_t = 1
        \]
        Multiplying both sides by $u$, we have:
        \[
            u = c_1 m_1 u + c_2 m_2 u + \dots + c_t m_t u
        \]
        Note that each term $c_i m_i u$ has order dividing $p_i^{n_i}$, hence belongs to $G(p_i)$. Therefore, we can express $u$ as a sum of elements from different $p$-subgroups:
        \[
            u \in G(p_1) + G(p_2) + \dots + G(p_t)
        \]
        Since $u$ was arbitrary, it follows that:
        \[
            G = \sum_{p} G(p)
        \]
        where the sum is over all primes $p$ such that $G(p) \neq 0$.
    \end{proof}
    \newpage
    \item \begin{proof}
        ($\Rightarrow$) Suppose $G \cong H$. Then there exists an isomorphism $\phi: G \to H$. For any prime $p$, consider the restriction of $\phi$ to $G(p)$:
        \[
            \phi|_{G(p)}: G(p) \to H(p)
        \]
        Since $\phi$ is an isomorphism, it preserves the order of elements. Thus, $\phi|_{G(p)}$ is an isomorphism from $G(p)$ to $H(p)$, implying that $G(p) \cong H(p)$ for all primes $p$.

        ($\Leftarrow$) Conversely, suppose that $G(p) \cong H(p)$ for all primes $p$. By part (b), we have:
        \[
            G = \sum_{p} G(p) \quad \text{and} \quad H = \sum_{p} H(p)
        \]
        Since each corresponding $p$-subgroup is isomorphic, we can construct an isomorphism $\psi: G \to H$ by defining it on each $G(p)$ and extending linearly. Specifically, for each prime $p$, let $\psi_p: G(p) \to H(p)$ be the isomorphism. Then define:
        \[
            \psi(u) = \sum_{p} \psi_p(u_p)
        \]
        where $u = \sum_{p} u_p$ with $u_p \in G(p)$. This map $\psi$ is well-defined and bijective, hence an isomorphism. Therefore, $G \cong H$.
    \end{proof}
\end{enumerate}



\begin{problem}[II.2.9] \\ 
    How many subgroups of order $p^2$ does the abelian group $\zz_{p^3} \oplus \zz_{p^2}$ have?
\end{problem}

All subgroups of order $p^2$ are isomorphic to either $\zz_{p^2}$ or $\zz_p \oplus \zz_p$. We will count the number of subgroups of each type and then sum them to get the total number of subgroups of order $p^2$.

\textbf{Subgroups isomorphic to $\zz_{p^2}$:} \\
All of these subgroups are generated by elements $(a,b) \in \zz_{p^3} \oplus \zz_{p^2}$ of order $p^2$. We know that the order of $(a,b)$ is given by $\operatorname{lcm}(|a|, |b|)$. This necessitates that $\operatorname{lcm}(|a|, |b|) = p^2$. Thus we can have the following cases:
\begin{enumerate}[(i)]
    \item $|a| = p^2$ and $|b| = \{ 1, p, p^2 \}$.
    \item $|a| = p$ and $|b| = p^2$.
    \item $|a| = 1$ and $|b| = p^2$.
\end{enumerate}

So we can count the number of elements in each case:
\begin{enumerate}[(i)]
    \item The number of elements of order $p^2$ in $\zz_{p^3}$ is $\phi(p^2) = p^2 - p$. 
    \item The number of elements of order $p$ in $\zz_{p^3}$ is $\phi(p) = p - 1$.
    \item The number of elements of order $1$ in $\zz_{p^3}$ is $1$.
    \item All elements in $\zz_{p^2}$ have order $1, p,$ or $p^2$, so there are $p^2$ choices for $b$ in case (i).
    \item The number of elements of order $p^2$ in $\zz_{p^2}$ is $\phi(p^2) = p^2 - p$. Thus, there are $p^2 - p$ choices for $b$ in cases (ii) and (iii).
\end{enumerate}
Since each subgroup $\langle (a,b) \rangle$ has $\phi(p^2) = p^2 - p$ generators, we can count the number of distinct subgroups isomorphic to $\zz_{p^2}$ as follows:
\begin{align*}
    \text{Number of subgroups isomorphic to } \zz_{p^2} &= \frac{(p^2 - p)(p^2) + (p - 1)(p^2 - p) + 1(p^2 - p)}{p^2 - p} \\
    &= p^2 + (p - 1) + 1 \\
    &= p^2 + p.
\end{align*}
The division by $p^2 - p$ accounts for the fact that each subgroup has $p^2 - p$ generators. \\
\textbf{Subgroups isomorphic to $\zz_p \oplus \zz_p$:} \\
To count the number of subgroups isomorphic to $\zz_p \oplus \zz_p$, we note that such a subgroup is generated by two elements of order $p$. The elements of order $p$ in $\zz_{p^3}$ are those of the form $kp^2$ for $k = 1, 2, \ldots, p-1$, giving us $p-1$ choices. Then, the elements of order $p$ in $\zz_{p^2}$ are those of the form $lp$ for $l = 1, 2, \ldots, p-1$, giving us another $p-1$ choices. To form a subgroup isomorphic to $\zz_p \oplus \zz_p$, we need to select two linearly independent elements of order $p$. The number of ways to choose two linearly independent elements is:
\[
\binom{p^2 - 1}{2} = \frac{(p^2 - 1)(p^2 - 2)}{2}.
\]
Thus, the total number of subgroups isomorphic to $\zz_p \oplus \zz_p$ is $\frac{p^2 + p}{2}$.
Thus we have that the total number of subgroups of order $p^2$ in $\zz_{p^3} \oplus \zz_{p^2}$ is $(p^2 + p) + \frac{p(p-1)}{2} = \frac{3}{2}\left(p^2 + p\right)$. \\



\newpage
\begin{problem}[II.4.1] \\
    Let $G$ be a group and $A$ a normal abelian subgroup. Show that $G/A$ operates on $A$ by conjugation and obtain a homomorphism $G/A \to \operatorname{Aut}(A)$.
\end{problem}



\begin{proof}
    Let $G$ be a group and $A$ a normal abelian subgroup of $G$. We want to show that the factor group $G/A$ acts on $A$ by conjugation.

    Define the action of an element $gA \in G/A$ on an element $a \in A$ by:
    \[
        (gA) \cdot a = gag^{-1}
    \]
    for any representative $g \in G$ of the coset $gA$. Since $A$ is normal in $G$, the conjugation $gag^{-1}$ is indeed an element of $A$.
    To see this defines an action we see that for the identity element $eA \in G/A$, we have:
    \[
        (eA) \cdot a = eae^{-1} = a
    \]
    for all $a \in A$ as well as for any $gA, hA \in G/A$ and $a \in A$, we have:
    \[
        (gA)(hA) \cdot a = (gh)A \cdot a = (gh)a(gh)^{-1} = g(ha h^{-1})g^{-1} = gA \cdot (hA \cdot a)
    \]

    Thus, we have shown that $G/A$ acts on $A$ by conjugation.

    Next, we define a homomorphism $\varphi: G/A \to \operatorname{Aut}(A)$ by:
    \[
        \varphi(gA)(a) = gag^{-1}
    \]
    for all $a \in A$. 

    To see that $\varphi$ is a homomorphism, notice that for any $gA, hA \in G/A$, we have that
    \[
        \varphi((gA)(hA)) = \varphi(gA) \circ \varphi(hA).
    \]
    In fact, for any $a \in A$, we have:
    \[
        \varphi((gA)(hA))(a) = (gh)a(gh)^{-1} = g(ha h^{-1})g^{-1} = \varphi(gA)(\varphi(hA)(a)).
    \]
    Thus, we have shown that $\varphi$ is a homomorphism. Therefore, we conclude that $G/A$ operates on $A$ by conjugation and there exists a homomorphism $\varphi: G/A \to \operatorname{Aut}(A)$.
\end{proof}



\newpage
\begin{problem}[II.4.5] \\
    If $H$ is a subgroup of $G$, the factor group $N_G(H)/C_G(H)$ (see Exercise 4) is isomorphic to a subgroup of $\operatorname{Aut}(H)$.
\end{problem}


First we prove a lemma: \\
\begin{lemma}
    Let $H$ be a normal subgroup of the group $G$. Then $G$ acts by conjugation on $H$ as automorphisms of $H$. For each $g \in G$, conjugation by $g$ is an automorphism of $H$. The permutation representation afforded by this action is a homomorphism of $G$ into $\operatorname{Aut}(H)$ with kernel $C_G(H)$. In particular, $G / C_G(H)$ is isomorphic to a subgroup of $\operatorname{Aut}(H)$.
\end{lemma}

\begin{proof}
    Let $\varphi_g$ be conjugation by $g$. Note that since $g$ normalizes $A$, $\varphi_g$ maps $A$ to itself. Since conjugation defines an action, we have that $\varphi_1 = 1$ is the identity map on $A$ and $\varphi_a \circ \varphi_b = \varphi_{ab}$ for all $a, b \in G$. So each $\varphi_g$ gives a bijection from $A$ to itself since it has a two-sided inverse $\varphi_{g^{-1}}$. Each $\varphi_g$ is a homomorphism since for all $x, y \in A$,
   \[
        \varphi_g(xy) = g(xy)g^{-1} = gx(g^{-1}g)yg^{-1} = (gxg^{-1})(gyg^{-1}) = \varphi_g(x)\varphi_g(y).
   \]
    This shows that conjugation by any fixed element of $G$ defines an automorphism of $A$. Then we have that the permutation representation $\psi : G \to S_H$ defined by $\psi(g) = \varphi_g$ (which is a homomorphism) has image contained in the subgroup $\operatorname{Aut}(H)$ of $S_H$. Then,
    \begin{align*}
        \ker \psi &= \{ g \in G \quad | \quad \varphi_g = \text{id} \} \\
        &= \{ g \in G \quad | \quad ghg^{-1} = h \text{ for all } h \in H \} \\
        &= C_G(H).
    \end{align*}
    Then, by the First Isomorphism Theorem, we have that $G / C_G(H) \cong \operatorname{Im} \psi \leq \operatorname{Aut}(H)$.
\end{proof}

Now we can prove the problem statement: \\
\begin{proof}
    Since we have that $H$ is a normal subgroup of $N_G(H)$, we can apply the lemma with $G = N_G(H)$ to obtain that $N_G(H) / C_G(H)$ is isomorphic to a subgroup of $\operatorname{Aut}(H)$.
\end{proof}



\begin{problem}[II.4.7] \\
    Let $G$ be a group and let $\operatorname{In} G$ be the set of all inner automorphisms of $G$. Show that $\operatorname{In} G$ is a normal subgroup of $\operatorname{Aut} G$.
\end{problem}

\begin{proof}
    Let $\sigma \in \operatorname{Aut} G$ and let $\varphi_g \in \operatorname{In} G$ be an inner automorphism defined by conjugation by $g \in G$. We want to show that $\sigma \varphi_g \sigma^{-1} = \varphi_{\sigma(g)}$. Then we have that for any $x \in G$,
    \begin{align*}
        (\sigma \varphi_g \sigma^{-1})(x) &= \sigma(\varphi_g(\sigma^{-1}(x))) \\
        &= \sigma(g \sigma^{-1}(x) g^{-1}) \\
        &= \sigma(g) \sigma(\sigma^{-1}(x)) \sigma(g^{-1}) \\
        &= \sigma(g) x \sigma(g)^{-1} \\
        &= \varphi_{\sigma(g)}(x).
    \end{align*}
    So we have that $\sigma \operatorname{In} G \sigma^{-1} \subseteq \operatorname{In} G$. Since $\sigma$ was arbitrary, we conclude that $\operatorname{In} G$ is a normal subgroup of $\operatorname{Aut} G$.
\end{proof}

\newpage
\begin{problem}[II.4.9] \\
    If $G/C(G)$ is cyclic, then $G$ is abelian.
\end{problem}


\begin{proof}
    Suppose that $G/C(G)$ is cyclic. Then there exists an element $g \in G$ such that $G/C(G) = \langle gC(G) \rangle$. This means that for any element $x \in G$, there exists an integer $n$ such that:
    \[
        xC(G) = (gC(G))^n = g^n C(G).
    \]
    Therefore, we can write:
    \[
        x = g^n c
    \]
    for some $c \in C(G)$.

    Now, consider any two elements $x, y \in G$. We can express them as:
    \[
        x = g^m c_1, \quad y = g^n c_2
    \]
    for some integers $m, n$ and elements $c_1, c_2 \in C(G)$.

    Now, we compute the product $xy$:
    \[
        xy = (g^m c_1)(g^n c_2) = g^{m+n} (c_1 c_2).
    \]
    Since $c_1$ and $c_2$ are in the center of $G$, they commute with all elements of $G$, including $g$. Thus, we have:
    \[
        yx = (g^n c_2)(g^m c_1) = g^{n+m} (c_2 c_1) = g^{m+n} (c_1 c_2) = xy.
    \]
    Therefore, for any two elements $x, y \in G$, we have shown that $xy = yx$. This implies that $G$ is abelian.
\end{proof}


%\newpage
\begin{problem} \\
    Suppose $G = \zz \times (\zz/10\zz) \times (\zz/100\zz)$, and $H$ is the subgroup generated by elements $(1,1,1), (1,2,3)$.
    \begin{enumerate}[(a)]
        \item What is the isomorphism class of $H$? (That is, what is $H$'s standard form, as in Theorem 2.6(ii) or Theorem 2.6(iii)?)
        \item What is the isomorphism class of $G/H$?
    \end{enumerate}
\end{problem}


\begin{enumerate}[(a)]
    \item To determine the isomorphism class of $H$, we first find the relations among the generators $(1,1,1)$ and $(1,2,3)$. We can represent these generators as rows in a matrix:
     \begin{align*}
        M &= \begin{pmatrix}
            1 & 1 & 1 \\
            1 & 2 & 3
        \end{pmatrix} \\
        (\text{Row reduce } M) &\sim \begin{pmatrix}
            1 & 0 & -1 \\
            0 & 1 & 2
        \end{pmatrix}
    \end{align*}
    From the row-reduced form, we can express the generators in terms of new generators:
    \[
        H \cong \langle (1,0,-1), (0,1,2) \rangle 
    \]
    Since $\langle (1,0,-1) \rangle \cap \langle (0,1,2) \rangle = \{(0,0,0)\}$, we have that $H \cong \langle (1,0,-1) \rangle \oplus \langle (0,1,2) \rangle \cong \zz \oplus \zz/50\zz$ as the element $(0,1,2)$ has order $50$ in $G$.
    
    \item To find the isomorphism class of $G/H$, we use the fact that $G \cong \zz \oplus \zz/10\zz \oplus \zz/100\zz$ and $H \cong \zz \oplus \zz/50\zz$. The quotient $G/H$ can be computed by dividing each component of $G$ by the corresponding component of $H$. Specifically:
        - The $\zz$ component of $G$ is entirely contained in the $\zz$ component of $H$, so the quotient of these components is trivial: $\zz / \zz = 0$.
        - The $\zz/10\zz$ component of $G$ is unaffected by $H$ because $H$ does not contribute anything to this component. Thus, the quotient of this component is $\zz/10\zz$.
        - The $\zz/100\zz$ component of $G$ is partially "covered" by the $\zz/50\zz$ component of $H$. The quotient of these components is $\zz/100\zz / \zz/50\zz \cong \zz/2\zz$ because dividing $\zz/100\zz$ by $\zz/50\zz$ reduces the order by a factor of $50$.

        Combining these results, we have:
        \[
            G/H \cong (\zz / \zz) \oplus (\zz/10\zz / 0) \oplus (\zz/100\zz / \zz/50\zz) \cong 0 \oplus \zz/10\zz \oplus \zz/2\zz.
        \]
        Simplifying further, we conclude that:
        \[
            G/H \cong \zz/10\zz \oplus \zz/2\zz.
        \]
    
    

\end{enumerate}


\end{document}