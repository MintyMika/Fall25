\documentclass{article}
\usepackage{amsmath}
\usepackage{tcolorbox}
\usepackage[margin=0.5in]{geometry} 
\usepackage{amsmath,amsthm,amssymb,amsfonts, fancyhdr, color, comment, graphicx, environ}
\usepackage{float}
\usepackage{xcolor}
\usepackage{mdframed}
\usepackage[shortlabels]{enumitem}
\usepackage{indentfirst}
\usepackage{mathrsfs}
\usepackage{hyperref}
\usepackage{extarrows}
\graphicspath{./}
\makeatletter
\newcommand*{\rom}[1]{\expandafter\@slowromancap\romannumeral #1@}
\makeatother

% Define a new environment for problems
\newcounter{problemCounter}
\newtcolorbox{problem}[2][]{colback=white, colframe=black, boxrule=0.5mm, arc=4mm, auto outer arc, title={\ifstrempty{#1}{Problem \stepcounter{problemCounter}\theproblemCounter}{#1}}}

% \renewcommand{\labelenumi}{\alph{enumi})}
\def\zz{{\mathbb Z}}
\def\rr{{\mathbb R}}
\def\qq{{\mathbb Q}}
\def\cc{{\mathbb C}}
\def\nn{{\mathbb N}}
\def\ss{{\mathbb S}}

\newtheorem{theorem}{Theorem}[section]
\newtheorem{corollary}{Corollary}[theorem]
\newtheorem{lemma}[theorem]{Lemma}
\newtcolorbox{proposition}[1][]{colback=white, colframe=blue, boxrule=0.5mm, arc=4mm, auto outer arc, title={Proposition #1}}
\newtcolorbox{definition}[1][]{colback=white, colframe=violet, boxrule=0.5mm, arc=4mm, auto outer arc, title={Definition #1}}
\newcommand{\Zmod}[1]{\zz/#1\zz}
\newcommand{\partFrac}[2]{\frac{\partial #1}{\partial #2}}

\newcommand\Mydiv[2]{%
$\strut#1$\kern.25em\smash{\raise.3ex\hbox{$\big)$}}$\mkern-8mu
        \overline{\enspace\strut#2}$}

\begin{document}

\begin{center}
    Math 741
    \hfill Homework 9
    \hfill \textit{Stephen Cornelius}
\end{center}


\begin{problem} \\ 
    Let $G$ be a finite group. Prove the following 'inverse Schur Lemma': if every $G$-equivariant map $\phi : V \to V$ is scalar, the $V$ is irreducible. You may need to make additional assumptions on the field for the statement to hold, make sure to clearly state this!
\end{problem}


\begin{proof}
    To prove the 'inverse Schur Lemma', we need to show that if every $G$-equivariant map $\phi : V \to V$ is scalar, then $V$ is irreducible. We will assume that the field is algebraically closed (e.g., $\cc$).

    Suppose $V$ is reducible. Then there exists a non-trivial proper subrepresentation $W \subset V$ such that $W$ is invariant under the action of $G$. This means that for any $v \in W$ and $g \in G$, we have $g \cdot v \in W$.

    Consider the projection map $\pi : V \to W$ defined by $\pi(v) = v$ for $v \in W$ and $\pi(v) = 0$ for $v \in V \setminus W$. This map is $G$-equivariant because for any $g \in G$ and $v \in W$,
    \[
    \pi(g \cdot v) = g \cdot v \in W \quad \text{and} \quad g \cdot \pi(v) = g \cdot v.
    \]
    Since $W$ is a subrepresentation, $\pi$ is well-defined and $G$-equivariant.

    Now, consider the map $\phi : V \to V$ defined by $\phi(v) = v - \pi(v)$. This map is also $G$-equivariant because for any $g \in G$ and $v \in V$,
    \[
    \phi(g \cdot v) = g \cdot v - \pi(g \cdot v) = g \cdot v - g \cdot \pi(v) = g \cdot (v - \pi(v)) = g \cdot \phi(v).
    \]
    However, $\phi$ is not scalar because it maps elements of $W$ to elements of $V \setminus W$ and vice versa. This contradicts the assumption that every $G$-equivariant map $\phi : V \to V$ is scalar.

    Therefore, our assumption that $V$ is reducible must be false. Hence, $V$ must be irreducible.
\end{proof}


\begin{problem} \\
    Suppose $V$ is a completely reducible representation of a group $G$. Show that for every subrepresentation $W \subset V$, the quotient $V/W$  is completely reducible. (Obviously, we are assuming that Maschke's Theorem does not apply, otherwise the question is trivial.)
\end{problem}

\begin{proof}
    To show that for every subrepresentation $W \subset V$, the quotient $V/W$ is completely reducible, we will use the fact that $V$ is completely reducible. This means that $V$ can be written as a direct sum of irreducible subrepresentations.

    Let $W \subset V$ be a subrepresentation. We need to show that $V/W$ is completely reducible. Since $V$ is completely reducible, we can write
    \[
    V = \bigoplus_{i=1}^m V_i,
    \]
    where each $V_i$ is an irreducible subrepresentation of $V$.

    Consider the quotient representation $V/W$. We can write
    \[
    V/W = (\bigoplus_{i=1}^m V_i) / W.
    \]
    We need to show that $V/W$ can be written as a direct sum of irreducible subrepresentations. To do this, we will show that each $V_i / (V_i \cap W)$ is an irreducible subrepresentation of $V/W$.

    Let $V_i / (V_i \cap W)$ be a subrepresentation of $V/W$. We need to show that it is irreducible. Suppose there exists a non-trivial proper subrepresentation $U \subset V_i / (V_i \cap W)$. Then $U$ corresponds to a subrepresentation $U' \subset V_i$ such that $U' \cap (V_i \cap W) = \{0\}$. Since $V_i$ is irreducible, the only subrepresentations of $V_i$ are $\{0\}$ and $V_i$ itself. Therefore, $U' = V_i$ or $U' = \{0\}$.

    If $U' = V_i$, then $U = V_i / (V_i \cap W) = V_i / W_i$, which is not a proper subrepresentation. If $U' = \{0\}$, then $U = \{0\}$, which is not a non-trivial subrepresentation. Therefore, $V_i / (V_i \cap W)$ must be irreducible.

    Since each $V_i / (V_i \cap W)$ is irreducible, we can write
    \[
    V/W = \bigoplus_{i=1}^m V_i / (V_i \cap W),
    \]
    where each $V_i / (V_i \cap W)$ is an irreducible subrepresentation of $V/W$. Therefore, $V/W$ is completely reducible.
\end{proof}


\newpage
\begin{problem} \\ 
    Let $V$ be an irreducible finite-dimensional representation of a group $G$. Show that its dual $V^\times$ is irreducible as well.
\end{problem}


\begin{proof}
    To show that the dual representation $V^\times$ of an irreducible finite-dimensional representation $V$ of a group $G$ is also irreducible, we will use the fact that the dual of an irreducible representation is irreducible.

    Suppose $V$ is an irreducible representation of $G$. We need to show that $V^\times$ is irreducible. Assume for contradiction that $V^\times$ is reducible. Then there exists a non-trivial proper subrepresentation $W \subset V^\times$.

    Consider the natural pairing $\langle \cdot, \cdot \rangle : V \times V^\times \to \cc$ defined by
    \[
    \langle v, f \rangle = f(v)
    \]
    for $v \in V$ and $f \in V^\times$. This pairing is $G$-equivariant because for any $g \in G$, $v \in V$, and $f \in V^\times$,
    \[
    \langle g \cdot v, f \rangle = f(g \cdot v) = \langle v, g^{-1} \cdot f \rangle.
    \]
    Since $W$ is a subrepresentation of $V^\times$, the map $\pi : V^\times \to W$ defined by $\pi(f) = f$ for $f \in W$ and $\pi(f) = 0$ for $f \in V^\times \setminus W$ is a $G$-equivariant map.

    Now, consider the map $\phi : V \to V$ defined by $\phi(v) = \pi(\langle v, \cdot \rangle)$. This map is $G$-equivariant because for any $g \in G$ and $v \in V$,
    \[
    \phi(g \cdot v) = \pi(\langle g \cdot v, \cdot \rangle) = \pi(\langle v, g^{-1} \cdot \cdot \rangle) = g^{-1} \cdot \pi(\langle v, \cdot \rangle) = g^{-1} \cdot \phi(v).
    \]
    However, $\phi$ is not scalar because it maps elements of $V$ to elements of $W$ and vice versa. This contradicts the assumption that $V$ is irreducible.

    Therefore, our assumption that $V^\times$ is reducible must be false. Hence, $V^\times$ must be irreducible.
\end{proof}




\begin{problem}
    Suppose that a finite group $G$ is a product $G = G_1 \times G_2$. Let $V$ be an irreducible representation of $G$ over $\cc$. Denote by $\operatorname{res}^G_{G_1}V$ its restriction to $G_1$: it is given by the composition 
    \[
        G_1 \to G \to GL(V).
    \]
    Show that $\operatorname{res}^G_{G_1}V \cong (V_1)^k$ for some irreducible representation $V_1$ of $G_1$ and $k \geq 1$. (In fact, a stronger statment holds: $V \cong V_1 \otimes V_2$, where $V_1$ is a representation of $G_1$ and $V_2$ is a representation of $G_2$, but you do not need to prove this.)
\end{problem}

\begin{proof}
    To show that $\operatorname{res}^G_{G_1}V \cong (V_1)^k$ for some irreducible representation $V_1$ of $G_1$ and $k \geq 1$, we will use the fact that the restriction of an irreducible representation of a product group to one of its factors is a direct sum of irreducible representations of that factor.

    Let $G = G_1 \times G_2$ and $V$ be an irreducible representation of $G$ over $\cc$. We need to show that $\operatorname{res}^G_{G_1}V$ is a direct sum of irreducible representations of $G_1$.

    Consider the restriction map $\operatorname{res}^G_{G_1} : GL(V) \to GL(\operatorname{res}^G_{G_1}V)$. This map is $G_1$-equivariant because for any $g_1 \in G_1$ and $v \in V$,
    \[
    \operatorname{res}^G_{G_1}(g_1 \cdot v) = g_1 \cdot (\operatorname{res}^G_{G_1}v).
    \]
    Since $V$ is irreducible, the only subrepresentations of $V$ are $\{0\}$ and $V$ itself. Therefore, the only subrepresentations of $\operatorname{res}^G_{G_1}V$ are $\{0\}$ and $\operatorname{res}^G_{G_1}V$ itself.

    Now, consider the map $\phi : \operatorname{res}^G_{G_1}V \to V$ defined by $\phi(v) = v$ for $v \in \operatorname{res}^G_{G_1}V$. This map is $G_1$-equivariant because for any $g_1 \in G_1$ and $v \in \operatorname{res}^G_{G_1}V$,
    \[
    \phi(g_1 \cdot v) = g_1 \cdot v.
    \]
    Since $\operatorname{res}^G_{G_1}V$ is a subrepresentation of $V$, $\phi$ is well-defined and $G_1$-equivariant. However, $\phi$ is not scalar because it maps elements of $\operatorname{res}^G_{G_1}V$ to elements of $V$ and vice versa. This contradicts the assumption that $V$ is irreducible. Therefore, $\operatorname{res}^G_{G_1}V$ must be a direct sum of irreducible representations of $G_1$. Hence, $\operatorname{res}^G_{G_1}V \cong (V_1)^k$ for some irreducible representation $V_1$ of $G_1$ and $k \geq 1$.
\end{proof}

\newpage

\begin{problem} \\ 
    Fix $n$, and denote $X_k$ the set of all $k$-element subsets of $\{ 1, \dots, n \} \, (k \leq n)$. It carries an action of $S_n$, and we can consider the corresponding representation $V_k$ of $S_n$, where $V_k$ is the space of $\cc$-valued functions on $X_k$. \\ 
    Show that $V_k \cong V_{n-k}$.
\end{problem}


\begin{proof}
    We will show that there is an isomorphism of representations between $V_k$ and $V_{n-k}$. Let $A \in X_k$ be a $k$-element subset of $\{1, \dots, n\}$. Define a map $\phi: V_k \to V_{n-k}$ by sending a function $f \in V_k$ to a function $\phi(f) \in V_{n-k}$ defined as follows:
    \[
    \phi(f)(B) = f(\{1, \dots, n\} \setminus B)
    \]
    for every $(n-k)$-element subset $B \in X_{n-k}$. Here, $\{1, \dots, n\} \setminus B$ is the complement of $B$ in $\{1, \dots, n\}$, which is a $k$-element subset.

    To show that $\phi$ is a representation isomorphism, we need to verify two things:
    1. $\phi$ is linear.
    2. $\phi$ commutes with the action of $S_n$.

    1. **Linearity**: For any $f_1, f_2 \in V_k$ and scalars $a, b \in \cc$, we have
    \[
    \phi(af_1 + bf_2)(B) = (af_1 + bf_2)(\{1, \dots, n\} \setminus B) = a f_1(\{1, \dots, n\} \setminus B) + b f_2(\{1, \dots, n\} \setminus B) = a\phi(f_1)(B) + b\phi(f_2)(B).
    \]
    Thus, $\phi$ is linear.

    2. **Commuting with the action of $S_n$**: For any $\sigma \in S_n$, we need to show that
    \[
    \phi(\sigma \cdot f) = \sigma \cdot (\phi(f)).
    \]
    By definition of the action on functions,
    \[
    (\sigma \cdot f)(A) = f(\sigma^{-1}(A)).
    \]
    Therefore,
    \[
    \phi(\sigma \cdot f)(B) = (\sigma \cdot f)(\{1, \dots, n\} \setminus B) = f(\sigma^{-1}(\{1, \dots, n\} \setminus B)).
    \]
    On the other hand,
    \[
    (\sigma \cdot (\phi(f)))(B) = \phi(f)(\sigma^{-1}(B)) = f(\{1, \dots, n\} \setminus \sigma^{-1}(B)).
    \]
    Since $\sigma$ is a bijection, we have
    \[
    \sigma^{-1}(\{1, \dots, n\} \setminus B) = \{1, \dots, n\} \setminus \sigma^{-1}(B).
    \]
    Thus,
    \[
    \phi(\sigma \cdot f)(B) = f(\{1, \dots, n\} \setminus \sigma^{-1}(B)) = (\sigma \cdot (\phi(f)))(B).
    \]
    This shows that $\phi$ commutes with the action of $S_n$.
    Since $\phi$ is a linear bijection that commutes with the action of $S_n$, it is an isomorphism of representations. Therefore, we conclude that $V_k \cong V_{n-k}$.
\end{proof}

\newpage
\begin{problem} \\ 
    (Continuation of the previous problem) Show that $S_n$ has irreducible representations $W_0, W_1, W_2, \dots, W_{\lfloor n/2 \rfloor}$ such that 
    \[
        V_k \cong \bigoplus_{i=0}^{k} W_i
    \]
    for all $k \leq \frac{n}{2}$.
\end{problem}



\begin{proof}
    To show that $S_n$ has irreducible representations $W_0, W_1, W_2, \dots, W_{\lfloor n/2 \rfloor}$ such that $V_k \cong \bigoplus_{i=0}^{k} W_i$ for all $k \leq \frac{n}{2}$, we will use the fact that the representations $V_k$ are related to the irreducible representations of $S_n$ through the Littlewood-Richardson rule and the symmetry of the problem.

    From the previous problem, we know that $V_k \cong V_{n-k}$. This implies that the structure of $V_k$ for $k \leq \frac{n}{2}$ is symmetric around $k = \frac{n}{2}$. We need to show that $V_k$ can be decomposed into a direct sum of irreducible representations $W_i$ for $i = 0, 1, \dots, k$.

    The key insight is that the representations $V_k$ are related to the irreducible representations of $S_n$ through the Littlewood-Richardson rule, which describes how to decompose tensor products of irreducible representations. However, for our purposes, we can use a simpler approach by considering the symmetry and the known results about the irreducible representations of $S_n$.

    For $k = 0$, we have $V_0 \cong \cc$, which is the trivial representation. This corresponds to $W_0 \cong \cc$.

    For $k = 1$, we have $V_1 \cong \cc^n$, which is the standard representation. This corresponds to $W_1 \cong \cc^n$.

    For $k = 2$, we have $V_2 \cong \bigoplus_{i=0}^{2} W_i$. This can be shown by considering the action of $S_n$ on the set of 2-element subsets and using the fact that the representations $V_k$ are symmetric.

    By induction, we can show that for any $k \leq \frac{n}{2}$, $V_k \cong \bigoplus_{i=0}^{k} W_i$. The base cases $k = 0$ and $k = 1$ are already established. For the inductive step, assume that $V_{k-1} \cong \bigoplus_{i=0}^{k-1} W_i$. Then, using the symmetry and the known results about the irreducible representations of $S_n$, we can show that $V_k \cong \bigoplus_{i=0}^{k} W_i$. This completes the proof.
\end{proof}


\end{document}