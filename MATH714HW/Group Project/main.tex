\documentclass{article}
\usepackage[utf8]{inputenc}
\usepackage{graphicx}
\usepackage{fullpage,mathpazo,amsmath,amssymb,amsfonts,nicefrac,graphicx}

\usepackage{hyperref,color,textcomp}
\usepackage[skip=-2pt,font=footnotesize]{caption}

\graphicspath{ {images/} }
\usepackage{caption}
\usepackage{subcaption}
\usepackage{float}
\usepackage[width=150mm,top=35mm,bottom=25mm,bindingoffset=6mm]{geometry}
\usepackage{fancyhdr}
\usepackage{setspace}
% \pagestyle{fancy}% <- use fancyplain instead fancy
% \fancyhf{}
% \fancyhead[R]{\thepage}
% \renewcommand{\headrulewidth}{0pt}
% \setlength{\headheight}{14pt}

% \usepackage[style=authoryear,sorting=none]{biblatex}
% \addbibresource{references.bib}

\title{Evaluation of the Black--Scholes equation via Crank--Nicolson and SINDy}
\author{Stephen Cornelius \& Andy Mapes \& Bill Xu}
\date{\today}

\begin{document}

\maketitle

\doublespacing
\section{Introduction}

The objective of this project was to empirically validate the Black--Scholes partial differential equation (PDE) using observed option market data. In particular, we sought to recover the governing PDE directly from data via Sparse Identification of Nonlinear Dynamics (SINDy), a framework designed to infer parsimonious differential equations from time-dependent observations. Such an approach offers a data-driven alternative to traditional model verification and provides a natural test of whether classical pricing equations are supported by empirical evidence.

A central challenge of this task is that the Black--Scholes PDE depends on local derivatives of the option price with respect to both the underlying asset price and time to maturity. Accurate estimation of these derivatives requires smooth, high-frequency observations of the option price surface. However, the empirical data available for this study consisted of option prices sampled at a monthly frequency, with additional noise arising from market microstructure effects, discrete strike grids, and volatility smiles. As a result, direct numerical differentiation produced unstable and unreliable derivative estimates.

To address these limitations, we explored a wide range of interpolation schemes and feature-based filtering strategies aimed at constructing a smoother approximation of the option price surface. These methods sought to reduce extraneous variability while preserving the essential dynamics implied by the Black--Scholes framework. Despite these efforts, the resulting surfaces remained too irregular for SINDy to consistently recover meaningful governing equations, highlighting fundamental identifiability constraints imposed by the data resolution.

To separate methodological limitations from data limitations, we supplemented the empirical analysis with synthetic option price data generated using a Crank--Nicolson finite difference scheme applied to the Black--Scholes PDE. In this controlled setting, SINDy successfully recovered the expected PDE structure, thereby validating the system identification pipeline itself. This contrast demonstrates that while data-driven PDE discovery is feasible in principle, its application to empirical option data is strongly constrained by sampling frequency and noise.

Overall, this study illustrates both the promise and the limitations of data-driven model discovery in quantitative finance, emphasizing the critical role of data quality and resolution in validating continuous-time pricing models.



\section{Black--Scholes Derivation}

The Black--Scholes equation links the stochastic dynamics of a traded asset to the
deterministic pricing of derivatives written on that asset. Its derivation rests on
a single central idea: by dynamically hedging a derivative with the underlying
asset, all market risk can be eliminated over an infinitesimal time interval.
The absence of arbitrage then forces the hedged portfolio to grow at the risk-free
rate, yielding a partial differential equation for the option price.

\subsection{Asset Dynamics and Option Dependence}

We assume the underlying asset price $S_t$ follows a geometric Brownian motion,
\begin{equation}
    dS_t = \mu S_t\,dt + \sigma S_t\,dW_t,
\end{equation}
where $\mu$ is the expected return, $\sigma$ is the volatility, and $W_t$ is a standard
Wiener process. Let $V(S,t)$ denote the value of a derivative written on the asset.
Since $V$ depends on the random variable $S_t$, its evolution is itself stochastic.

\subsection{Propagation of Randomness}

Applying It\^o’s Lemma to $V(S,t)$ gives
\begin{equation}
    dV =
    \left(
    V_t + \frac{1}{2}\sigma^2 S^2 V_{SS} + \mu S V_S
    \right)dt
    + \sigma S V_S\,dW_t.
\end{equation}

The key observation is that both the asset and the derivative are driven by the
same Brownian motion. This shared source of randomness makes it possible to
construct a hedging strategy that removes risk.

\subsection{Delta Hedging and Risk Elimination}

Consider a portfolio that is short one derivative and long $\Delta$ shares of the
underlying asset,
\begin{equation}
    \Pi = -V + \Delta S.
\end{equation}
Choosing
\[
\Delta = V_S
\]
eliminates the stochastic term proportional to $dW_t$, rendering the portfolio
locally risk-free. This step is the conceptual core of the Black--Scholes argument:
risk is neutralized through trading rather than priced directly.

\subsection{No-Arbitrage and the Black--Scholes PDE}

A risk-free portfolio must earn the risk-free interest rate $r$. Enforcing this
condition yields the Black--Scholes partial differential equation,
\begin{equation}
    \boxed{
    V_t
    + \frac{1}{2}\sigma^2 S^2 V_{SS}
    + rS V_S
    - rV = 0.
    }
\end{equation}

Notably, the expected return $\mu$ does not appear in the equation, reflecting
the risk-neutral nature of derivative pricing.

\subsection{Interpretation and Boundary Conditions}

Introducing the option Greeks
\[
\theta = V_t, \qquad \Delta = V_S, \qquad \Gamma = V_{SS},
\]
the equation may be written as
\begin{equation}
    \theta + \frac{1}{2}\sigma^2 S^2 \Gamma = r\bigl(V - S\Delta\bigr),
\end{equation}
which equates the intrinsic evolution of the option to the required return on the
delta-hedged position.

To price a specific derivative, the PDE is solved backward in time subject to a
terminal condition at expiration $t = T$,
\[
V(S,T) = \Phi(S),
\]
where $\Phi(S)$ is the payoff function. This yields the unique arbitrage-free option
price.




\section{Crank-Nicolson Simulation}

\subsection{Theoretical Framework}

The Crank--Nicolson method is a widely-used finite difference scheme for solving parabolic partial differential equations. It combines implicit and explicit updates to achieve second-order accuracy in both space and time while maintaining unconditional stability. This section describes the application of Crank--Nicolson to the Black--Scholes equation.

\subsection{Discretization of the Black--Scholes PDE}

We discretize the spatial domain $[0, S_{\max}]$ into $M+1$ points with spacing $\Delta S$, and the temporal domain $[0, T]$ into $N+1$ points with spacing $\Delta t$. Let $V_{n,j} \approx V(t_n, S_j)$ denote the numerical approximation at time $t_n = n\Delta t$ and asset price $S_j = j\Delta S$.

The Black--Scholes PDE is rewritten in the form
\[
V_t = \frac{1}{2}\sigma^2 S^2 V_{SS} + rS V_S - rV.
\]

\subsection{Crank--Nicolson Scheme}

The Crank--Nicolson method evaluates the right-hand side at both time levels $n$ and $n+1$, averaging to achieve balanced accuracy:
\[
\frac{V_{n+1,j} - V_{n,j}}{\Delta t} = \frac{1}{2}\left[\mathcal{L}(V_{n+1,j}) + \mathcal{L}(V_{n,j})\right],
\]
where $\mathcal{L}$ denotes the spatial differential operator. Rearranging yields a system of linear equations at each time step:
\[
(I - \tfrac{\Delta t}{2}\mathcal{L})\,V^{n+1} = (I + \tfrac{\Delta t}{2}\mathcal{L})\,V^{n}.
\]

For the Black--Scholes equation, this becomes a tridiagonal system that can be solved efficiently using the Thomas algorithm.

\subsection{Boundary Conditions and Initial Setup}

At each time level, boundary conditions are enforced at $S=0$ and $S=S_{\max}$. For a European call option, these are typically
\[
V(t, 0) = 0, \qquad V(t, S_{\max}) \approx S_{\max} - K e^{-r(T-t)},
\]
where $K$ is the strike price and $T$ is the expiration time.

The scheme is initialized with the option payoff at maturity:
\[
V(T, S) = \max(S - K, 0) \quad \text{for a call option}.
\]
Time integration then proceeds backward from $t = T$ to $t = 0$, generating a complete spatio-temporal solution matrix.

\subsection{Advantages for SINDy}

The Crank--Nicolson scheme produces smooth, accurate numerical solutions with minimal oscillation. This smoothness is essential for SINDy, since derivative estimation is highly sensitive to noise and irregularity. By solving the known Black--Scholes PDE numerically, we generate synthetic data with known ground truth, enabling validation of the SINDy recovery process before application to empirical market data.

\section{Construction of a Smooth Option Surface from Sparse Market Data}

The empirical option data used in this study consists of prices sampled at a
monthly frequency across a discrete grid of underlying prices and times to
maturity. While sufficient for descriptive analysis, this data presents
substantial challenges for derivative-based system identification methods such
as SINDy. The Black--Scholes PDE depends on local derivatives of the option price
surface, which are highly sensitive to noise, irregular sampling, and market
microstructure effects.

To make the data more amenable to analysis, we explored interpolation and
filtering strategies aimed at constructing a smoother approximation to the
option price surface
\[
V = V(S,t),
\]
where $S$ denotes the underlying price and $t$ denotes time to maturity.

\subsection{Interpolation in the Underlying Price}

At each fixed maturity $t_j$, option prices are observed only at a discrete set of
underlying prices $\{S_i\}$. To enable stable estimation of spatial derivatives
$V_S$ and $V_{SS}$, we interpolated across $S$ using smooth basis
representations. The approaches considered were:

\begin{itemize}
    \item \textbf{Spline interpolation:} Cubic splines were fit to $V(S,t_j)$ for
    each maturity slice, ensuring continuity of the function and its first two
    derivatives with respect to $S$.

    \item \textbf{Local polynomial regression:} Low-order local polynomial fits
    were applied in $S$ to reduce sensitivity to outliers and irregular spacing.

    \item \textbf{Implied-volatility–space interpolation:} In some cases,
    interpolation was performed on the implied volatility surface
    $\sigma_{\mathrm{imp}}(S,t)$ rather than directly on prices, leveraging its
    empirically smoother structure.
\end{itemize}

Each method imposes different smoothness assumptions on the option surface,
directly affecting the accuracy of higher-order derivative estimates.

\subsection{Interpolation in Time to Maturity}

The monthly sampling frequency is insufficient to directly approximate the time
derivative $V_t$. To mitigate this limitation, intermediate surfaces were
introduced between observed maturities using temporal interpolation. The
strategies examined were:

\begin{itemize}
    \item \textbf{Linear interpolation:} Option prices were interpolated linearly
    in time for fixed $S$, imposing minimal temporal structure.

    \item \textbf{Higher-order interpolation:} Polynomial and spline-based
    interpolants were also considered to produce smoother estimates of $V_t$,
    at the cost of introducing additional modeling assumptions.
\end{itemize}

These interpolations create a pseudo-continuous time dimension, enabling
finite-difference approximations of temporal derivatives.

\subsection{Feature-Based Filtering}

Rather than smoothing the option surface directly, we applied feature-based
filtering to reduce variance in variables not explicitly represented in the
Black--Scholes PDE. Specifically, filtering was performed over combinations of
\[
S - K, \quad K, \quad \sigma_{\mathrm{imp}}, \quad t,
\]
where $K$ denotes the strike price.

By restricting attention to subsets of the data with limited variation in these
features—such as narrow moneyness bands, strike ranges, implied volatility
windows, or maturity intervals—we aimed to suppress cross-sectional
heterogeneity arising from volatility smiles, term structure effects, and
discrete quoting behavior. This conditioning reduces extraneous variability while
preserving the raw structure of the data.

\subsection{Modeling Implications}

Interpolation and filtering introduce implicit assumptions about smoothness and
local behavior of the option surface. While necessary for derivative-based
analysis, these regularization steps constrain the extent to which governing
equations can be recovered from empirical data. In practice, aggressive
regularization suppresses genuine market structure, while insufficient
regularization leads to unstable derivative estimates.

Accordingly, the constructed surfaces should be viewed as regularized
approximations suitable for exploratory PDE identification rather than exact
representations of the underlying market dynamics.



\section{SINDy Method}
% Section 1
\subsection{Introduction and Motivation}
\label{sec:sindy-intro}

In this section we introduce the \emph{Sparse Identification of
Nonlinear Dynamics} (SINDy) framework and describe how to validate our 
numerical solutions of the Black--Scholes equation. SINDy is a data-driven
method for discovering governing differential equations directly from
time-resolved measurements, under the assumption that the true
dynamics can be represented by a small number of active terms in a
larger candidate library of functions
(for example, 
%\cite{brunton2016discovering,rudy2017data}
).

In our setting, a Crank--Nicolson scheme is used to compute
approximate option values \(V(t,S)\) on a discrete grid of asset
prices and times to maturity. The resulting spatio-temporal data set
serves as input to SINDy, which attempts to recover the underlying
partial differential equation from the data alone. By comparing the
identified model to the classical Black--Scholes equation, we are able to
obtain a data-driven consistency check on both the numerical scheme and 
the model assumptions.


%Section 2
\subsection{Overview of the SINDy Framework}
\label{sec:sindy-overview}

SINDy provides a systematic approach for uncovering the 
governing equations of a dynamical system directly from data. 
The central idea is to represent the unknown dynamics in terms of a 
library of candidate functions and to determine a parsimonious model 
by selecting only the few terms that actively contribute to the evolution
of the system 
%\cite{brunton2016discovering}. 
This sparsity assumption reflects the structure of many physical and financial 
models, where the underlying equations typically involve only a small 
subset of all possible functional combinations.

In the context of partial differential equations (PDEs), SINDy is extended to
the PDE-FIND methodology introduced in 
%\cite{rudy2017data},
where spatio-temporal data are used to estimate both temporal and spatial
derivatives. A function library consisting of candidate terms such as
\(u\), \(u_x\), \(u_{xx}\), and nonlinear products is constructed, and
the algorithm identifies the subset of terms that best explains the
observed dynamics.



%Section 3
\subsection{Mathematical Foundations of SINDy}
\label{sec:sindy-theory}

SINDy is built on the assumption that dynamical systems---including those 
governed by partial differential equations---has simple underlying 
structure. Despite the large number of admissible functional 
forms, the true governing equation typically contains only a small subset of
all possible terms. The goal of SINDy is to exploit this sparsity to
recover the governing equation directly from data
%\cite{brunton2016discovering}.

\subsubsection{Representation of Dynamics Using a Function Library}

Consider a dynamical system with state variable \(x(t)\in\mathbb{R}^d\)
governed by an unknown differential equation
\[
\dot{x}(t) = f(x(t)).
\]
SINDy approximates \(f(x)\) by expressing it as a linear combination of its
candidate nonlinear functions. To do so, one constructs a library
\(\Theta(x)\) consisting of functions such as constants, polynomial
terms, trig functions, or some other nonlinearities:
\[
\Theta(x) = 
\big[
1,\; x,\; x^2,\; x^3,\; \sin(x),\; \ldots
\big].
\]
The dynamics are then assumed to satisfy
\[
\dot{x}(t) \approx \Theta(x(t))\,\Xi,
\]
where \(\Xi\in\mathbb{R}^{K\times d}\) is a matrix of coefficients. Here, 
SINDy seeks a \emph{sparse} coefficient matrix: most rows of \(\Xi\) are zero.

\subsubsection{Formulation as a Sparse Regression Problem}

Given measurement data \(\{x(t_j)\}_{j=1}^N\), one computes numerical
estimates of the time derivative \(\dot{x}(t_j)\). Evaluating the
library at all data samples yields the matrix equation
\[
\dot{X} = \Theta(X)\,\Xi,
\]
where \(\dot{X}\in\mathbb{R}^{N\times d}\) and
\(\Theta(X)\in\mathbb{R}^{N\times K}\). Recovering the model becomes a
sparse regression problem: identify the few active columns of
\(\Theta(X)\) that best explain the observed dynamics.

A common approach is \emph{sequential thresholded least squares}
(STLSQ), introduced in 
%\cite{brunton2016discovering}. This iterative
method alternates between:
\begin{enumerate}
    \item solving a standard least-squares regression, and
    \item thresholding (setting to zero) coefficients whose magnitudes
          fall below a prescribed sparsity level.
\end{enumerate}
This procedure is analogous to performing an \(\ell_0\)-type model
selection step but avoids the combinatorial complexity of exhaustive
search. Connections to LASSO and other sparse regression techniques
%\cite{tibshirani1996lasso,hastie2009elements}
further justify the stability and parsimony properties of the method.

\subsubsection{Extension to Partial Differential Equations (PDE-FIND)}

To identify partial differential equations, SINDy is extended 
to incorporate spatial derivatives. Let \(u(t,x)\) denote the
solution of an unknown PDE. The PDE-FIND approach introduced in
%\cite{rudy2017data} 
estimates temporal and spatial derivatives from a
spatio-temporal data set and constructs a function library of candidate
terms such as
\[
u,\quad u_x,\quad u_{xx},\quad u^2,\quad u\,u_x,\quad x\,u_x,\quad
\text{etc.}
\]
The governing equation is assumed to have the form
\[
u_t = \Theta(u, u_x, u_{xx}, \ldots)\,\Xi,
\]
and sparse regression is again used to determine the active terms.

\subsubsection{Conditions for Successful Model Recovery}

The accuracy of SINDy depends on several identifiability conditions:
\begin{itemize}
    \item \textbf{Rich data:} the solution must sufficiently explore the
          dynamics so that active terms in \(\Theta\) can be
          distinguished.
    \item \textbf{Derivative accuracy:} numerical differentiation errors
          should be small relative to the magnitudes of the terms in the
          PDE.
    \item \textbf{Low correlation in the function library:} highly
          collinear candidate terms make sparse regression unstable.
    \item \textbf{Correct library specification:} the true terms must be
          included in the candidate dictionary.
    \item \textbf{Appropriate sparsity thresholding:} prevents
          overfitting and guards against noise amplification.
\end{itemize}

When these conditions are met, SINDy provides a mathematically justified
and computationally efficient framework for discovering interpretable
governing equations from data.



%Section 4
\subsection{Why SINDy is Well Suited for the Black--Scholes Equation}
\label{sec:sindy-bs}

SINDy is well suited for our case because the Black--Scholes
equation is sparse in its natural variables and can be expressed using
a small number of linear terms involving \(V\), \(V_S\), and
\(V_{SS}\). By SINDy, our goal is not to derive the equation
from first principles, but to demonstrate that the numerical solution
generated by the Crank--Nicolson method is consistent with the
equation identified by SINDy.

The Black--Scholes equation possesses a structure that makes it
especially compatible with the SINDy framework. The PDE is linear in
the option value \(V\) and involves only a small number of active terms:
\[
V_t = -\frac{1}{2}\sigma^2 S^2 V_{SS} - r S V_S + rV.
\]
Although financial models can be highly nonlinear, the Black--Scholes
equation depends only on three fundamental components: \(V\), \(V_S\),
and \(V_{SS}\), along with multiplicative factors involving the asset
price \(S\). This sparsity matches one of the core assumptions of SINDy:
that the governing equation can be represented using only a few
elements from a larger library of candidate terms.

Moreover, the solution \(V(t,S)\) of the Black--Scholes equation is
smooth in both variables under standard boundary and payoff conditions,
ensuring that numerical derivatives such as \(V_S\), \(V_{SS}\), and
\(V_t\) can be estimated with sufficient accuracy. This smoothness is an
important assumption for the PDE-FIND extension of SINDy
%\cite{rudy2017data}, 
since derivative estimation is typically the most sensitive part 
of the identification process.

Crucially, the terms in the Black--Scholes PDE can be expressed using a
simple candidate library of the form
\[
\big\{\, V,\; V_S,\; V_{SS},\; S V_S,\; S^2 V_{SS} \,\big\},
\]
all of which fall within the class of polynomial and product terms that
SINDy handles efficiently. Therefore, SINDy can be expected to
identify the correct combination of active terms and recover coefficients
that correspond to the parameters \(r\) and \(\sigma\).

From a practical perspective, SINDy provides a data-driven mechanism for
verifying that the numerical solution generated by the Crank--Nicolson
scheme is consistent with the theoretical Black--Scholes model. Instead
of assuming the PDE, we attempt to rediscover it directly from the
computed solution surface. Agreement between the identified PDE and the
classical equation serves as a validation of both the numerical method
and the suitability of the Black--Scholes assumptions for the given
input data.



%Section 5
\subsection{Application of SINDy to Crank--Nicolson Data}
\label{sec:sindy-application}

To apply the SINDy framework to the Black--Scholes equation, we will
convert the numerical solution produced by the Crank--Nicolson method
into a form suitable for sparse regression. This section summarizes the
steps required to construct the data matrices, compute numerical
derivatives, assemble the function library, and perform the regression.

\subsubsection{Discrete Data Generated by the Crank--Nicolson Scheme}

Let the asset price domain \([0,S_{\max}]\) be discretized into
\(M+1\) grid points
\[
S_0,\, S_1,\,\ldots,\, S_M,
\]
and let the time interval \([0,T]\) be discretized into \(N+1\) time
levels
\[
t_0,\, t_1,\,\ldots,\, t_N.
\]
The Crank--Nicolson scheme produces numerical approximations of the
option price at each grid point, yielding a matrix
\[
V \in \mathbb{R}^{(N+1)\times (M+1)},
\]
where
\[
V_{n,j} \approx V(t_n, S_j).
\]
Each row corresponds to a fixed time level and each column corresponds
to a fixed asset price.

\subsubsection{Numerical Derivatives}

To construct the regression matrix for SINDy, we must estimate the
temporal and spatial derivatives of \(V\).

\paragraph{Temporal derivative.}
For each spatial index \(j\), we approximate \(V_t\) using a backward
difference:
\[
(V_t)_{n,j} \approx \frac{V_{n,j} - V_{n-1,j}}{\Delta t},
\qquad n = 1, \ldots, N.
\]
This results in a derivative matrix
\[
V_t \in \mathbb{R}^{N\times (M+1)}.
\]

\paragraph{First spatial derivative.}
Using centered differences for interior points,
\[
(V_S)_{n,j} \approx
\frac{V_{n, j+1} - V_{n, j-1}}{2\,\Delta S},
\qquad j = 1, \ldots, M-1.
\]
At boundaries, one-sided differences are used. The resulting matrix
lies in
\[
V_S \in \mathbb{R}^{(N+1)\times (M+1)}.
\]

\paragraph{Second spatial derivative.}
For interior points,
\[
(V_{SS})_{n,j} \approx
\frac{V_{n, j+1} - 2V_{n, j} + V_{n, j-1}}{(\Delta S)^2}.
\]
Again, boundary handling is applied where necessary. The matrix
satisfies
\[
V_{SS} \in \mathbb{R}^{(N+1)\times (M+1)}.
\]

\subsubsection{Vectorization and Data Alignment}

SINDy expects the data in vectorized form. For each quantity \(Q\in
\{V, V_S, V_{SS}, V_t\}\), we reshape the matrix into a column vector
by stacking rows or columns:
\[
q = \mathrm{vec}(Q) \in \mathbb{R}^{L},
\]
where
\[
L = N \times (M+1)
\]
for quantities defined only on time levels \(1,\ldots,N\), such as
\(V_t\), and
\[
L = (N+1)\times (M+1)
\]
for quantities defined at all time levels. To ensure consistency, we
truncate all variables to use the same \(L=N(M+1)\) data points
corresponding to time levels where \(V_t\) is defined.

\subsubsection{Construction of the Library Matrix}

For each data point \((t_n, S_j)\), we evaluate the candidate functions
in the SINDy library. A typical library for the Black--Scholes equation
includes
\[
\Theta = 
\big[
V,\;
V_S,\;
V_{SS},\;
S V_S,\;
S^2 V_{SS},\;
V_t \text{ (target)}
\big].
\]

To build the regression matrix, we vectorize each candidate term and
form the column-stacked library matrix
\[
\Theta(X) \in \mathbb{R}^{L \times K},
\]
where each column corresponds to a candidate function and \(K\) is the
number of functions in the library.

For the Black--Scholes example above, \(K = 5\), and thus
\[
\Theta(X) =
\begin{bmatrix}
| & | & | & | & | \\
\theta_1 & \theta_2 & \theta_3 & \theta_4 & \theta_5 \\
| & | & | & | & |
\end{bmatrix},
\]
where \(\theta_k = \mathrm{vec}(\text{candidate term}_k)\).

The target vector is
\[
u_t = \mathrm{vec}(V_t) \in \mathbb{R}^{L}.
\]

\subsubsection{Sparse Regression}

SINDy seeks a sparse coefficient vector \(\Xi \in \mathbb{R}^{K}\) such
that
\[
u_t \approx \Theta(X)\,\Xi.
\]
Sequential thresholded least squares (STLSQ)
%\cite{brunton2016discovering} 
is used to determine which columns of
\(\Theta(X)\) contribute significantly to the dynamics:
\[
\Xi^{(k+1)} = \mathcal{T}_{\lambda}
\Big(
  (\Theta^T \Theta)^{-1}\Theta^T u_t
\Big),
\]
where \(\mathcal{T}_{\lambda}\) is a hard-thresholding operator that
zeros out coefficients smaller than a prescribed level \(\lambda\).

\subsubsection{Interpretation of the Identified PDE}

Once the sparse vector \(\Xi\) is obtained, the corresponding active
terms reconstruct the PDE:
\[
V_t
= \xi_1 V
+ \xi_2 V_S
+ \xi_3 V_{SS}
+ \xi_4 (S V_S)
+ \xi_5 (S^2 V_{SS}).
\]
If the Crank--Nicolson solution is consistent with the theoretical
Black--Scholes model, SINDy should identify
\[
\xi_4 \approx r, \qquad
\xi_5 \approx \tfrac{1}{2}\sigma^2,
\qquad
\xi_1 \approx -r,
\]
with all other coefficients close to zero.

This provides a fully data-driven verification of the numerical
solution and the governing dynamics.


\section{Conclusion}

This study investigated the application of data-driven methods for discovering and validating the Black--Scholes partial differential equation from both synthetic and empirical option price data. We employed two complementary approaches: (i) numerical solution via the Crank--Nicolson finite difference scheme, and (ii) sparse system identification via the SINDy framework.

Our key findings are as follows:

\begin{enumerate}
    \item \textbf{Synthetic Data Validation:} When applied to option price surfaces generated by solving the Black--Scholes PDE numerically with Crank--Nicolson, the SINDy algorithm successfully recovered the governing equation with high fidelity. The identified coefficients matched the theoretical parameters $r$ and $\sigma$, demonstrating that the system identification pipeline is sound and that SINDy can reliably discover PDE structure from smooth, high-resolution spatio-temporal data.

    \item \textbf{Empirical Data Limitations:} Application to real market option data proved substantially more challenging. Despite employing multiple interpolation schemes, smoothing filters, and feature-based conditioning strategies, the empirical option price surface remained too noisy and irregular to support consistent PDE recovery. The monthly sampling frequency and inherent market microstructure effects (volatility smiles, discrete strike grids, bid-ask spreads) limited the accuracy of derivative estimation, which is the most sensitive step in the SINDy procedure.

    \item \textbf{Methodological Insights:} The contrast between successful recovery from synthetic data and poor recovery from market data highlights a fundamental distinction: data-driven PDE discovery is sensitive not only to algorithmic choices but critically to data quality, resolution, and smoothness. High-frequency, low-noise observations are essential for identifying continuous-time dynamics from discrete samples.

    \item \textbf{Practical Implications:} For practitioners seeking to validate or discover pricing models in quantitative finance, these results underscore the importance of data preprocessing and the limitations of inference from sparse, noisy observations. Model validation in finance may require either higher-frequency data acquisition or the integration of domain knowledge to regularize the inverse problem.
\end{enumerate}

Future work could explore advanced smoothing techniques such as Gaussian process regression or kernel methods to construct smoother synthetic surfaces from market data, examine alternative sparse regression formulations robust to noise amplification, or investigate whether SINDy can identify modified or alternative pricing models when the classical Black--Scholes assumptions are relaxed. Additionally, extending this framework to higher-dimensional problems (e.g., multi-asset options) or time-varying parameters may reveal new challenges and opportunities in financial model discovery.


\end{document}