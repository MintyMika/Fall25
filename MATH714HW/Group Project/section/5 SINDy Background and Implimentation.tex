\section{SINDy Method}
% Section 1
\subsection{Introduction and Motivation}
\label{sec:sindy-intro}

In this section we introduce the \emph{Sparse Identification of
Nonlinear Dynamics} (SINDy) framework and describe how to validate our 
numerical solutions of the Black--Scholes equation. SINDy is a data-driven
method for discovering governing differential equations directly from
time-resolved measurements, under the assumption that the true
dynamics can be represented by a small number of active terms in a
larger candidate library of functions
(for example, 
%\cite{brunton2016discovering,rudy2017data}
).

In our setting, a Crank--Nicolson scheme is used to compute
approximate option values \(V(t,S)\) on a discrete grid of asset
prices and times to maturity. The resulting spatio-temporal data set
serves as input to SINDy, which attempts to recover the underlying
partial differential equation from the data alone. By comparing the
identified model to the classical Black--Scholes equation, we are able to
obtain a data-driven consistency check on both the numerical scheme and 
the model assumptions.


%Section 2
\subsection{Overview of the SINDy Framework}
\label{sec:sindy-overview}

SINDy provides a systematic approach for uncovering the 
governing equations of a dynamical system directly from data. 
The central idea is to represent the unknown dynamics in terms of a 
library of candidate functions and to determine a parsimonious model 
by selecting only the few terms that actively contribute to the evolution
of the system 
%\cite{brunton2016discovering}. 
This sparsity assumption reflects the structure of many physical and financial 
models, where the underlying equations typically involve only a small 
subset of all possible functional combinations.

In the context of partial differential equations (PDEs), SINDy is extended to
the PDE-FIND methodology introduced in 
%\cite{rudy2017data},
where spatio-temporal data are used to estimate both temporal and spatial
derivatives. A function library consisting of candidate terms such as
\(u\), \(u_x\), \(u_{xx}\), and nonlinear products is constructed, and
the algorithm identifies the subset of terms that best explains the
observed dynamics.



%Section 3
\subsection{Mathematical Foundations of SINDy}
\label{sec:sindy-theory}

SINDy is built on the assumption that dynamical systems---including those 
governed by partial differential equations---has simple underlying 
structure. Despite the large number of admissible functional 
forms, the true governing equation typically contains only a small subset of
all possible terms. The goal of SINDy is to exploit this sparsity to
recover the governing equation directly from data
%\cite{brunton2016discovering}.

\subsubsection{Representation of Dynamics Using a Function Library}

Consider a dynamical system with state variable \(x(t)\in\mathbb{R}^d\)
governed by an unknown differential equation
\[
\dot{x}(t) = f(x(t)).
\]
SINDy approximates \(f(x)\) by expressing it as a linear combination of its
candidate nonlinear functions. To do so, one constructs a library
\(\Theta(x)\) consisting of functions such as constants, polynomial
terms, trig functions, or some other nonlinearities:
\[
\Theta(x) = 
\big[
1,\; x,\; x^2,\; x^3,\; \sin(x),\; \ldots
\big].
\]
The dynamics are then assumed to satisfy
\[
\dot{x}(t) \approx \Theta(x(t))\,\Xi,
\]
where \(\Xi\in\mathbb{R}^{K\times d}\) is a matrix of coefficients. Here, 
SINDy seeks a \emph{sparse} coefficient matrix: most rows of \(\Xi\) are zero.

\subsubsection{Formulation as a Sparse Regression Problem}

Given measurement data \(\{x(t_j)\}_{j=1}^N\), one computes numerical
estimates of the time derivative \(\dot{x}(t_j)\). Evaluating the
library at all data samples yields the matrix equation
\[
\dot{X} = \Theta(X)\,\Xi,
\]
where \(\dot{X}\in\mathbb{R}^{N\times d}\) and
\(\Theta(X)\in\mathbb{R}^{N\times K}\). Recovering the model becomes a
sparse regression problem: identify the few active columns of
\(\Theta(X)\) that best explain the observed dynamics.

A common approach is \emph{sequential thresholded least squares}
(STLSQ), introduced in 
%\cite{brunton2016discovering}. This iterative
method alternates between:
\begin{enumerate}
    \item solving a standard least-squares regression, and
    \item thresholding (setting to zero) coefficients whose magnitudes
          fall below a prescribed sparsity level.
\end{enumerate}
This procedure is analogous to performing an \(\ell_0\)-type model
selection step but avoids the combinatorial complexity of exhaustive
search. Connections to LASSO and other sparse regression techniques
%\cite{tibshirani1996lasso,hastie2009elements}
further justify the stability and parsimony properties of the method.

\subsubsection{Extension to Partial Differential Equations (PDE-FIND)}

To identify partial differential equations, SINDy is extended 
to incorporate spatial derivatives. Let \(u(t,x)\) denote the
solution of an unknown PDE. The PDE-FIND approach introduced in
%\cite{rudy2017data} 
estimates temporal and spatial derivatives from a
spatio-temporal data set and constructs a function library of candidate
terms such as
\[
u,\quad u_x,\quad u_{xx},\quad u^2,\quad u\,u_x,\quad x\,u_x,\quad
\text{etc.}
\]
The governing equation is assumed to have the form
\[
u_t = \Theta(u, u_x, u_{xx}, \ldots)\,\Xi,
\]
and sparse regression is again used to determine the active terms.

\subsubsection{Conditions for Successful Model Recovery}

The accuracy of SINDy depends on several identifiability conditions:
\begin{itemize}
    \item \textbf{Rich data:} the solution must sufficiently explore the
          dynamics so that active terms in \(\Theta\) can be
          distinguished.
    \item \textbf{Derivative accuracy:} numerical differentiation errors
          should be small relative to the magnitudes of the terms in the
          PDE.
    \item \textbf{Low correlation in the function library:} highly
          collinear candidate terms make sparse regression unstable.
    \item \textbf{Correct library specification:} the true terms must be
          included in the candidate dictionary.
    \item \textbf{Appropriate sparsity thresholding:} prevents
          overfitting and guards against noise amplification.
\end{itemize}

When these conditions are met, SINDy provides a mathematically justified
and computationally efficient framework for discovering interpretable
governing equations from data.



%Section 4
\subsection{Why SINDy is Well Suited for the Black--Scholes Equation}
\label{sec:sindy-bs}

SINDy is well suited for our case because the Black--Scholes
equation is sparse in its natural variables and can be expressed using
a small number of linear terms involving \(V\), \(V_S\), and
\(V_{SS}\). By SINDy, our goal is not to derive the equation
from first principles, but to demonstrate that the numerical solution
generated by the Crank--Nicolson method is consistent with the
equation identified by SINDy.

The Black--Scholes equation possesses a structure that makes it
especially compatible with the SINDy framework. The PDE is linear in
the option value \(V\) and involves only a small number of active terms:
\[
V_t = -\frac{1}{2}\sigma^2 S^2 V_{SS} - r S V_S + rV.
\]
Although financial models can be highly nonlinear, the Black--Scholes
equation depends only on three fundamental components: \(V\), \(V_S\),
and \(V_{SS}\), along with multiplicative factors involving the asset
price \(S\). This sparsity matches one of the core assumptions of SINDy:
that the governing equation can be represented using only a few
elements from a larger library of candidate terms.

Moreover, the solution \(V(t,S)\) of the Black--Scholes equation is
smooth in both variables under standard boundary and payoff conditions,
ensuring that numerical derivatives such as \(V_S\), \(V_{SS}\), and
\(V_t\) can be estimated with sufficient accuracy. This smoothness is an
important assumption for the PDE-FIND extension of SINDy
%\cite{rudy2017data}, 
since derivative estimation is typically the most sensitive part 
of the identification process.

Crucially, the terms in the Black--Scholes PDE can be expressed using a
simple candidate library of the form
\[
\big\{\, V,\; V_S,\; V_{SS},\; S V_S,\; S^2 V_{SS} \,\big\},
\]
all of which fall within the class of polynomial and product terms that
SINDy handles efficiently. Therefore, SINDy can be expected to
identify the correct combination of active terms and recover coefficients
that correspond to the parameters \(r\) and \(\sigma\).

From a practical perspective, SINDy provides a data-driven mechanism for
verifying that the numerical solution generated by the Crank--Nicolson
scheme is consistent with the theoretical Black--Scholes model. Instead
of assuming the PDE, we attempt to rediscover it directly from the
computed solution surface. Agreement between the identified PDE and the
classical equation serves as a validation of both the numerical method
and the suitability of the Black--Scholes assumptions for the given
input data.



%Section 5
\subsection{Application of SINDy to Crank--Nicolson Data}
\label{sec:sindy-application}

To apply the SINDy framework to the Black--Scholes equation, we will
convert the numerical solution produced by the Crank--Nicolson method
into a form suitable for sparse regression. This section summarizes the
steps required to construct the data matrices, compute numerical
derivatives, assemble the function library, and perform the regression.

\subsubsection{Discrete Data Generated by the Crank--Nicolson Scheme}

Let the asset price domain \([0,S_{\max}]\) be discretized into
\(M+1\) grid points
\[
S_0,\, S_1,\,\ldots,\, S_M,
\]
and let the time interval \([0,T]\) be discretized into \(N+1\) time
levels
\[
t_0,\, t_1,\,\ldots,\, t_N.
\]
The Crank--Nicolson scheme produces numerical approximations of the
option price at each grid point, yielding a matrix
\[
V \in \mathbb{R}^{(N+1)\times (M+1)},
\]
where
\[
V_{n,j} \approx V(t_n, S_j).
\]
Each row corresponds to a fixed time level and each column corresponds
to a fixed asset price.

\subsubsection{Numerical Derivatives}

To construct the regression matrix for SINDy, we must estimate the
temporal and spatial derivatives of \(V\).

\paragraph{Temporal derivative.}
For each spatial index \(j\), we approximate \(V_t\) using a backward
difference:
\[
(V_t)_{n,j} \approx \frac{V_{n,j} - V_{n-1,j}}{\Delta t},
\qquad n = 1, \ldots, N.
\]
This results in a derivative matrix
\[
V_t \in \mathbb{R}^{N\times (M+1)}.
\]

\paragraph{First spatial derivative.}
Using centered differences for interior points,
\[
(V_S)_{n,j} \approx
\frac{V_{n, j+1} - V_{n, j-1}}{2\,\Delta S},
\qquad j = 1, \ldots, M-1.
\]
At boundaries, one-sided differences are used. The resulting matrix
lies in
\[
V_S \in \mathbb{R}^{(N+1)\times (M+1)}.
\]

\paragraph{Second spatial derivative.}
For interior points,
\[
(V_{SS})_{n,j} \approx
\frac{V_{n, j+1} - 2V_{n, j} + V_{n, j-1}}{(\Delta S)^2}.
\]
Again, boundary handling is applied where necessary. The matrix
satisfies
\[
V_{SS} \in \mathbb{R}^{(N+1)\times (M+1)}.
\]

\subsubsection{Vectorization and Data Alignment}

SINDy expects the data in vectorized form. For each quantity \(Q\in
\{V, V_S, V_{SS}, V_t\}\), we reshape the matrix into a column vector
by stacking rows or columns:
\[
q = \mathrm{vec}(Q) \in \mathbb{R}^{L},
\]
where
\[
L = N \times (M+1)
\]
for quantities defined only on time levels \(1,\ldots,N\), such as
\(V_t\), and
\[
L = (N+1)\times (M+1)
\]
for quantities defined at all time levels. To ensure consistency, we
truncate all variables to use the same \(L=N(M+1)\) data points
corresponding to time levels where \(V_t\) is defined.

\subsubsection{Construction of the Library Matrix}

For each data point \((t_n, S_j)\), we evaluate the candidate functions
in the SINDy library. A typical library for the Black--Scholes equation
includes
\[
\Theta = 
\big[
V,\;
V_S,\;
V_{SS},\;
S V_S,\;
S^2 V_{SS},\;
V_t \text{ (target)}
\big].
\]

To build the regression matrix, we vectorize each candidate term and
form the column-stacked library matrix
\[
\Theta(X) \in \mathbb{R}^{L \times K},
\]
where each column corresponds to a candidate function and \(K\) is the
number of functions in the library.

For the Black--Scholes example above, \(K = 5\), and thus
\[
\Theta(X) =
\begin{bmatrix}
| & | & | & | & | \\
\theta_1 & \theta_2 & \theta_3 & \theta_4 & \theta_5 \\
| & | & | & | & |
\end{bmatrix},
\]
where \(\theta_k = \mathrm{vec}(\text{candidate term}_k)\).

The target vector is
\[
u_t = \mathrm{vec}(V_t) \in \mathbb{R}^{L}.
\]

\subsubsection{Sparse Regression}

SINDy seeks a sparse coefficient vector \(\Xi \in \mathbb{R}^{K}\) such
that
\[
u_t \approx \Theta(X)\,\Xi.
\]
Sequential thresholded least squares (STLSQ)
%\cite{brunton2016discovering} 
is used to determine which columns of
\(\Theta(X)\) contribute significantly to the dynamics:
\[
\Xi^{(k+1)} = \mathcal{T}_{\lambda}
\Big(
  (\Theta^T \Theta)^{-1}\Theta^T u_t
\Big),
\]
where \(\mathcal{T}_{\lambda}\) is a hard-thresholding operator that
zeros out coefficients smaller than a prescribed level \(\lambda\).

\subsubsection{Interpretation of the Identified PDE}

Once the sparse vector \(\Xi\) is obtained, the corresponding active
terms reconstruct the PDE:
\[
V_t
= \xi_1 V
+ \xi_2 V_S
+ \xi_3 V_{SS}
+ \xi_4 (S V_S)
+ \xi_5 (S^2 V_{SS}).
\]
If the Crank--Nicolson solution is consistent with the theoretical
Black--Scholes model, SINDy should identify
\[
\xi_4 \approx r, \qquad
\xi_5 \approx \tfrac{1}{2}\sigma^2,
\qquad
\xi_1 \approx -r,
\]
with all other coefficients close to zero.

This provides a fully data-driven verification of the numerical
solution and the governing dynamics.


