\section{Construction of a Smooth Option Surface from Sparse Market Data}

The empirical option data used in this study consists of prices sampled at a
monthly frequency across a discrete grid of underlying prices and times to
maturity. While sufficient for descriptive analysis, this data presents
substantial challenges for derivative-based system identification methods such
as SINDy. The Black--Scholes PDE depends on local derivatives of the option price
surface, which are highly sensitive to noise, irregular sampling, and market
microstructure effects.

To make the data more amenable to analysis, we explored interpolation and
filtering strategies aimed at constructing a smoother approximation to the
option price surface
\[
V = V(S,t),
\]
where $S$ denotes the underlying price and $t$ denotes time to maturity.

\subsection{Interpolation in the Underlying Price}

At each fixed maturity $t_j$, option prices are observed only at a discrete set of
underlying prices $\{S_i\}$. To enable stable estimation of spatial derivatives
$V_S$ and $V_{SS}$, we interpolated across $S$ using smooth basis
representations. The approaches considered were:

\begin{itemize}
    \item \textbf{Spline interpolation:} Cubic splines were fit to $V(S,t_j)$ for
    each maturity slice, ensuring continuity of the function and its first two
    derivatives with respect to $S$.

    \item \textbf{Local polynomial regression:} Low-order local polynomial fits
    were applied in $S$ to reduce sensitivity to outliers and irregular spacing.

    \item \textbf{Implied-volatility–space interpolation:} In some cases,
    interpolation was performed on the implied volatility surface
    $\sigma_{\mathrm{imp}}(S,t)$ rather than directly on prices, leveraging its
    empirically smoother structure.
\end{itemize}

Each method imposes different smoothness assumptions on the option surface,
directly affecting the accuracy of higher-order derivative estimates.

\subsection{Interpolation in Time to Maturity}

The monthly sampling frequency is insufficient to directly approximate the time
derivative $V_t$. To mitigate this limitation, intermediate surfaces were
introduced between observed maturities using temporal interpolation. The
strategies examined were:

\begin{itemize}
    \item \textbf{Linear interpolation:} Option prices were interpolated linearly
    in time for fixed $S$, imposing minimal temporal structure.

    \item \textbf{Higher-order interpolation:} Polynomial and spline-based
    interpolants were also considered to produce smoother estimates of $V_t$,
    at the cost of introducing additional modeling assumptions.
\end{itemize}

These interpolations create a pseudo-continuous time dimension, enabling
finite-difference approximations of temporal derivatives.

\subsection{Feature-Based Filtering}

Rather than smoothing the option surface directly, we applied feature-based
filtering to reduce variance in variables not explicitly represented in the
Black--Scholes PDE. Specifically, filtering was performed over combinations of
\[
S - K, \quad K, \quad \sigma_{\mathrm{imp}}, \quad t,
\]
where $K$ denotes the strike price.

By restricting attention to subsets of the data with limited variation in these
features—such as narrow moneyness bands, strike ranges, implied volatility
windows, or maturity intervals—we aimed to suppress cross-sectional
heterogeneity arising from volatility smiles, term structure effects, and
discrete quoting behavior. This conditioning reduces extraneous variability while
preserving the raw structure of the data.

\subsection{Modeling Implications}

Interpolation and filtering introduce implicit assumptions about smoothness and
local behavior of the option surface. While necessary for derivative-based
analysis, these regularization steps constrain the extent to which governing
equations can be recovered from empirical data. In practice, aggressive
regularization suppresses genuine market structure, while insufficient
regularization leads to unstable derivative estimates.

Accordingly, the constructed surfaces should be viewed as regularized
approximations suitable for exploratory PDE identification rather than exact
representations of the underlying market dynamics.
