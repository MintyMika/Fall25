\section{Introduction}

The objective of this project was to empirically validate the Black--Scholes partial differential equation (PDE) using observed option market data. In particular, we sought to recover the governing PDE directly from data via Sparse Identification of Nonlinear Dynamics (SINDy), a framework designed to infer parsimonious differential equations from time-dependent observations. Such an approach offers a data-driven alternative to traditional model verification and provides a natural test of whether classical pricing equations are supported by empirical evidence.

A central challenge of this task is that the Black--Scholes PDE depends on local derivatives of the option price with respect to both the underlying asset price and time to maturity. Accurate estimation of these derivatives requires smooth, high-frequency observations of the option price surface. However, the empirical data available for this study consisted of option prices sampled at a monthly frequency, with additional noise arising from market microstructure effects, discrete strike grids, and volatility smiles. As a result, direct numerical differentiation produced unstable and unreliable derivative estimates.

To address these limitations, we explored a wide range of interpolation schemes and feature-based filtering strategies aimed at constructing a smoother approximation of the option price surface. These methods sought to reduce extraneous variability while preserving the essential dynamics implied by the Black--Scholes framework. Despite these efforts, the resulting surfaces remained too irregular for SINDy to consistently recover meaningful governing equations, highlighting fundamental identifiability constraints imposed by the data resolution.

To separate methodological limitations from data limitations, we supplemented the empirical analysis with synthetic option price data generated using a Crank--Nicolson finite difference scheme applied to the Black--Scholes PDE. In this controlled setting, SINDy successfully recovered the expected PDE structure, thereby validating the system identification pipeline itself. This contrast demonstrates that while data-driven PDE discovery is feasible in principle, its application to empirical option data is strongly constrained by sampling frequency and noise.

Overall, this study illustrates both the promise and the limitations of data-driven model discovery in quantitative finance, emphasizing the critical role of data quality and resolution in validating continuous-time pricing models.
