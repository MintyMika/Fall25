\section{Black--Scholes Derivation}

The Black--Scholes equation links the stochastic dynamics of a traded asset to the
deterministic pricing of derivatives written on that asset. Its derivation rests on
a single central idea: by dynamically hedging a derivative with the underlying
asset, all market risk can be eliminated over an infinitesimal time interval.
The absence of arbitrage then forces the hedged portfolio to grow at the risk-free
rate, yielding a partial differential equation for the option price.

\subsection{Asset Dynamics and Option Dependence}

We assume the underlying asset price $S_t$ follows a geometric Brownian motion,
\begin{equation}
    dS_t = \mu S_t\,dt + \sigma S_t\,dW_t,
\end{equation}
where $\mu$ is the expected return, $\sigma$ is the volatility, and $W_t$ is a standard
Wiener process. Let $V(S,t)$ denote the value of a derivative written on the asset.
Since $V$ depends on the random variable $S_t$, its evolution is itself stochastic.

\subsection{Propagation of Randomness}

Applying It\^o’s Lemma to $V(S,t)$ gives
\begin{equation}
    dV =
    \left(
    V_t + \frac{1}{2}\sigma^2 S^2 V_{SS} + \mu S V_S
    \right)dt
    + \sigma S V_S\,dW_t.
\end{equation}

The key observation is that both the asset and the derivative are driven by the
same Brownian motion. This shared source of randomness makes it possible to
construct a hedging strategy that removes risk.

\subsection{Delta Hedging and Risk Elimination}

Consider a portfolio that is short one derivative and long $\Delta$ shares of the
underlying asset,
\begin{equation}
    \Pi = -V + \Delta S.
\end{equation}
Choosing
\[
\Delta = V_S
\]
eliminates the stochastic term proportional to $dW_t$, rendering the portfolio
locally risk-free. This step is the conceptual core of the Black--Scholes argument:
risk is neutralized through trading rather than priced directly.

\subsection{No-Arbitrage and the Black--Scholes PDE}

A risk-free portfolio must earn the risk-free interest rate $r$. Enforcing this
condition yields the Black--Scholes partial differential equation,
\begin{equation}
    \boxed{
    V_t
    + \frac{1}{2}\sigma^2 S^2 V_{SS}
    + rS V_S
    - rV = 0.
    }
\end{equation}

Notably, the expected return $\mu$ does not appear in the equation, reflecting
the risk-neutral nature of derivative pricing.

\subsection{Interpretation and Boundary Conditions}

Introducing the option Greeks
\[
\theta = V_t, \qquad \Delta = V_S, \qquad \Gamma = V_{SS},
\]
the equation may be written as
\begin{equation}
    \theta + \frac{1}{2}\sigma^2 S^2 \Gamma = r\bigl(V - S\Delta\bigr),
\end{equation}
which equates the intrinsic evolution of the option to the required return on the
delta-hedged position.

To price a specific derivative, the PDE is solved backward in time subject to a
terminal condition at expiration $t = T$,
\[
V(S,T) = \Phi(S),
\]
where $\Phi(S)$ is the payoff function. This yields the unique arbitrage-free option
price.
