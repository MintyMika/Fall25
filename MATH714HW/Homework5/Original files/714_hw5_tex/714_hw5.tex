\documentclass[12pt]{article}
\usepackage{fullpage,mathpazo,amsmath,amssymb,amsfonts,nicefrac,microtype,graphicx}

\usepackage{hyperref,color,textcomp}
\usepackage[skip=-2pt,font=footnotesize]{caption}

\definecolor{webgreen}{rgb}{0,.35,0}
\definecolor{webbrown}{rgb}{.6,0,0}
\definecolor{RoyalBlue}{rgb}{0,0,0.9}
\definecolor{purp}{rgb}{0.6,0.3,0.9}
\hypersetup{
   colorlinks=true, linktocpage=true, pdfstartpage=3, pdfstartview=FitV,
   breaklinks=true, pdfpagemode=UseNone, pageanchor=true, pdfpagemode=UseOutlines,
   plainpages=false, bookmarksnumbered, bookmarksopen=true, bookmarksopenlevel=1,
   hypertexnames=true, pdfhighlight=/O,
   urlcolor=webbrown, linkcolor=RoyalBlue, citecolor=webgreen,
   pdfauthor={Chris H. Rycroft, Yue Sun},
   pdfsubject={UW--Madison Math/CS 714 (Fall 2025)},
   pdfkeywords={},
   pdfcreator={pdfLaTeX},
   pdfproducer={LaTeX with hyperref}
}
\hypersetup{pdftitle={Math/CS 714: Homework 5}}

\newcommand{\N}{\mathbb{N}}
\newcommand{\Z}{\mathbb{Z}}
\newcommand{\Q}{\mathbb{Q}}
\newcommand{\R}{\mathbb{R}}
\newcommand{\p}{\partial}
\renewcommand{\vec}[1]{\mathbf{#1}}
\newcommand{\vx}{\vec{x}}
\newcommand{\vy}{\vec{y}}
\newcommand{\calC}{\mathcal{C}}
\newcommand{\calF}{\mathcal{F}}
\newcommand{\Lag}{\mathcal{L}}
\newcommand{\sep}{\,:\,}
\setlength{\unitlength}{0.5mm}
\renewcommand{\labelitemii}{$\diamond$}
\DeclareMathOperator{\Real}{Re}

\graphicspath{{./}{gr/}}

\begin{document}
\section*{Math/CS 714: Assignment 5}

\begin{enumerate}
  \item \textbf{Red light, green light (8 points).} In this problem, we will consider the behavior of traffic on a stretch of road $x\in[-1,1]$ due to a traffic light at $x=1$.
        \begin{figure}[h!]
          \centering
          \vspace{-1em}
          \includegraphics[width=0.65\textwidth]{traffic}
        \end{figure}
        \begin{enumerate}
          \vspace{-1em}
          \item Consider the 1D LWR traffic model with car density $q\in[0,1]$ and max car speed $u_\text{max}=1$, traffic flow velocity $u(q)=1-q \in[0,1]$, and the flux $f(q)=q(1-q)$. What is the analytical expression for the characteristic velocity $c(q)$ for this model? For what values of $q$ is the characteristic velocity negative?

          \item Consider the discretization
                \begin{equation}
                  \label{cons_law}
                  \frac{Q_j^{n+1}-Q_j^n}{\Delta{t}} + \frac{F^n_{j+\frac{1}{2}}-F^n_{j-\frac{1}{2}}}{\Delta{x}} = 0
                \end{equation}
                with
                \begin{equation*}
                  F^n_{j+\frac{1}{2}}=U^n_{j+\frac{1}{2}}Q^n_{j+\frac{1}{2}},\quad F^n_{j-\frac{1}{2}}=U^n_{j-\frac{1}{2}}Q^n_{j-\frac{1}{2}}.
                \end{equation*}
                Here, $F^n_{j+\frac{1}{2}}$, $U^n_{j+\frac{1}{2}}$, and $Q^n_{j+\frac{1}{2}}$ are numerical approximations of the flux $f(x_{j+\frac{1}{2}},t_n)$, velocity $u(x_{j+\frac{1}{2}},t_n)$, and density $q(x_{j+\frac{1}{2}},t_n)$, respectively, with $x_{j+\frac{1}{2}}=-1+\big(j+\frac{1}{2}\big)\Delta{x}$ and $t_n=n\Delta{t}$. Show that Eq.~\eqref{cons_law} can be rewritten as
                \begin{equation}
                  \label{traffic_model}
                  \frac{Q_j^{n+1}-Q_j^n}{\Delta{t}} + C^n_j\frac{Q^n_{j+\frac{1}{2}}-Q^n_{j-\frac{1}{2}}}{\Delta{x}} = 0
                \end{equation}
                where the discrete characteristic velocity
                \begin{equation}
                  \label{char_vel}
                  C^n_j=1-Q^n_{j+\frac{1}{2}}-Q^n_{j-\frac{1}{2}}.
                \end{equation}

          \item Implement the WENO scheme discussed in class to solve for the density in time using the discretization in Eq.~\eqref{traffic_model}. Discretize the domain $x\in[-1,1]$ by $m=201$ evenly-spaced points with spacing $\Delta{x}=2/(m-1)$. Use a timestep of $\Delta{t}=0.001$ and integrate to a final time $T=1.5$, outputting $150$ snapshots after the initial state ($151$ outputs total). As initial conditions, use the step function
                \begin{equation}
                  Q^0_j=\begin{cases}
                    q_l, & x<0    \\
                    q_r, & x\geq0
                  \end{cases}
                \end{equation}
                for fixed values $q_l$ and $q_r$. When the interpolant for $Q^n_{j+\frac{1}{2}}$ or $Q^n_{j-\frac{1}{2}}$ relies on points outside the domain, use the ghost node approach and treat the density on these nodes as fixed at $q_l$ on the left and $q_r$ on the right, respectively.

                At each grid cell, the sign of the characteristic velocity $C^n_j$ must be determined to select the appropriate upwind condition. At the start of the step, compute the intermediate value
                \begin{equation}
                  \widehat{C}^n_j=1-Q^n_{j+1}-Q^n_{j-1}
                \end{equation}
                from known values $Q^n_{j+1}$ and $Q^n_{j-1}$, and use its sign to select the proper set of upwind interpolants and compute $Q^n_{j+\frac{1}{2}}$ and $Q^n_{j-\frac{1}{2}}$. Then compute the final characteristic velocity $C^n_j$ from Eq.~\eqref{char_vel} and use it in the update rule given by Eq.~\eqref{traffic_model}.
                You can consult \href{https://github.com/chr1shr/am205_g_activities/blob/master/eno_methods/lin_adv_eno_solutions.ipynb}{this Jupyter notebook} for implementing the WENO scheme, but not that the notations for density and velocity is different.

                Run your program for the following two cases:
                \begin{enumerate}
                  \item \textbf{Red light:} $q_l=0.4, q_r=1.0$. Physically, we imagine that the traffic light at $x=1$ is red, and the traffic is initially backed up and standing still from $[0,1]$ while more vehicles approach from the left.
                  \item \textbf{Green light:} $q_l=1.0, q_r=0.4$. Now, the light at $x=1$ is green, and traffic on $[0,1]$ has reduced, while cars on $[-1,0]$ are initially still at rest.
                \end{enumerate}
                For each case, plot snapshots of the density at the four times $t=0,0.5,1,1.5$.
        \end{enumerate}

  \item \textbf{Primers on spectral methods (6 points).}
      \begin{enumerate}
      \item Consider the infinite grid $h\Z$ with $h=1$. By considering the second derivative
        of the constant function $v(x)=1$ on $h\Z$, show that
        \begin{equation}
          \frac{\pi^2}{12} = 1 - \frac{1}{2^2} + \frac{1}{3^2} - \frac{1}{4^2} + \ldots.
        \end{equation}
      \item Consider the function $v(x)=\sin \tfrac{\pi x}{2}$ on the infinite grid $h\Z$
        with $h=1$. Introduce the function
        \begin{equation}
          p_\alpha(x) = \sum_{m=-\alpha}^\alpha v_m S_h(x-x_m)
        \end{equation}
        where $S_h$ is the sinc function introduced in the lectures, $x_m=mh$,
        and $v_m=v(x_m)$. Write a program to compute\footnote{You will need to
        use numerical integration to do this. You can use a library function,
        trapezoid rule, or other quadrature rule.} $E_\alpha=\|p_\alpha -
        v\|_2$ over the range $[-5,5]$ for $\alpha=1,2,4,8,\ldots,512$. Fit the
        data to a power law
        \begin{equation}
          E_\alpha = C \alpha^q
        \end{equation}
        and determine the parameters $C$ and $q$.
      \item The band-limited interpolant is $p(x) = \lim_{\alpha\to \infty}
        p_\alpha(x)$. Part (b) shows that $E_\alpha \to 0$ as $\alpha \to \infty$,
        and hence $p(x)=v(x)$. Explain from the theory of band-limited
        interpolants why this must be the case.\footnote{It may be helpful to
        know that the Fourier transform of $e^{i\lambda x}$ is
        $\delta(k-\lambda)$ where $\delta$ is the Dirac delta function.}
      \item By considering the first derivative of $v$ from part (b), show that
        \begin{equation}
          \frac{\pi}{4} = 1 - \frac{1}{3} + \frac{1}{5} - \frac{1}{7} + \ldots.
        \end{equation}
      \item By using an appropriate function choice, show that
        \begin{equation}
          \frac{\pi^2}{6} = 1 + \frac{1}{2^2} + \frac{1}{3^2} + \frac{1}{4^2} + \ldots.
        \end{equation}
    \end{enumerate}

  \item \textbf{Chebyshev spectral method (6 points).}
        Consider the boundary value problem
        \begin{equation}
          u_{xx} + u^5 = f \label{eq:cbvp}
        \end{equation}
        for the function $u(x)$ on the non-periodic interval $[-1,1]$, with a
        source term $f(x)$. Use the Dirichlet conditions $u(-1)=u(1)=0$. Write
        a program that can find $u$ represented on a Chebyshev grid with $N+1$ grid
        points. Since Eq.~\eqref{eq:cbvp} is nonlinear, you will need to solve this
        using the Newton method.\footnote{This is similar to Question 3 on Homework
          1.}
        \begin{enumerate}
          \item Use the method of manufactured solutions, with the solution
                \begin{equation}
                  u(x)=e^x(x^2-1). \label{eq:uexact}
                \end{equation}
                Calculate what $f$ will be in order for $u$ to satify
                Eq.~\eqref{eq:cbvp}.
          \item For a range of $N$ from 4 to 64, calculate the numerical solution
                $p_N(x)$ to Eq.~\eqref{eq:cbvp}. Start your Newton method from the
                function $(x+1)^2(x-1)$ as an initial guess.\footnote{The nonlinear
                  system has multiple solutions. This function is close enough to
                  Eq.~\eqref{eq:uexact} that your Newton method will reliably converge to it.}
                Make a semilog plot of $\|p_N-u\|_2$ as a function of $N$.
        \end{enumerate}
\end{enumerate}
\end{document}
