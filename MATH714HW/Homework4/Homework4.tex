\documentclass[11pt]{article}
% Common LaTeX preamble. Save as preamble.tex and % Common LaTeX preamble. Save as preamble.tex and % Common LaTeX preamble. Save as preamble.tex and \input{preamble} after \documentclass.
% Engine detection
\usepackage{iftex}
\ifPDFTeX
    \usepackage[T1]{fontenc}
    \usepackage[utf8]{inputenc} % harmless on newer LaTeX
    \usepackage{lmodern}
    \usepackage[english]{babel}
    \usepackage{inconsolata} % better mono font
\else
    \usepackage{fontspec}
    \defaultfontfeatures{Ligatures=TeX,Scale=MatchLowercase}
    \setmainfont{Latin Modern Roman}
    \setsansfont{Latin Modern Sans}
    \setmonofont{Latin Modern Mono}
    \usepackage{polyglossia}
    \setmainlanguage{english}
\fi

% Page geometry and layout
\usepackage[margin=1in]{geometry}
\usepackage{microtype}
\usepackage{setspace}
% \onehalfspacing % uncomment for 1.5 line spacing

% Graphics and color
\usepackage{xcolor}
\usepackage{graphicx}
\graphicspath{{figures/}}
\usepackage{tikz}
\usetikzlibrary{arrows.meta,positioning,calc,fit,patterns,decorations.pathmorphing}
\usepackage{pgfplots}
\pgfplotsset{compat=1.18}

% Tables
\usepackage{booktabs}
\usepackage{tabularx}
\usepackage{array}
\newcolumntype{Y}{>{\centering\arraybackslash}X}

% Lists
\usepackage{enumitem}
\setlist{itemsep=0.25em, topsep=0.5em}

% Math
\usepackage{amsmath,amssymb,mathtools,bm}
\allowdisplaybreaks
\usepackage{siunitx}
\sisetup{detect-all=true, per-mode=symbol, group-minimum-digits=4}

% Theorems
\usepackage{amsthm}
\numberwithin{equation}{section}
\newtheorem{theorem}{Theorem}[section]
\newtheorem{lemma}[theorem]{Lemma}
\newtheorem{proposition}[theorem]{Proposition}
\theoremstyle{definition}
\newtheorem{definition}[theorem]{Definition}
\theoremstyle{remark}
\newtheorem{remark}[theorem]{Remark}
\newtheorem{example}[theorem]{Example}

% Code listings
\usepackage{listings}
\lstdefinestyle{code}{
    basicstyle=\ttfamily\small,
    numbers=left,
    numberstyle=\scriptsize\color{gray},
    stepnumber=1,
    numbersep=8pt,
    showstringspaces=false,
    breaklines=true,
    frame=lines,
    tabsize=2,
    captionpos=b,
    keywordstyle=\color{blue!70!black}\bfseries,
    commentstyle=\color{green!50!black}\itshape,
    stringstyle=\color{orange!60!black},
}
\lstset{style=code}

% Captions and subfigures
\usepackage[labelfont=bf]{caption}
\usepackage{subcaption}

% Hyperlinks and clever references
\usepackage[hidelinks]{hyperref}
\hypersetup{
    pdftitle={},
    pdfauthor={},
    pdfsubject={},
    pdfcreator={LaTeX},
    pdfkeywords={}
}
\usepackage[capitalize,noabbrev]{cleveref}

% Citations (BibLaTeX)
\usepackage{csquotes}
\usepackage[
    backend=biber,
    style=authoryear,
    natbib=true,
    giveninits=true,
    maxbibnames=99,
    uniquename=init
]{biblatex}
% \addbibresource{references.bib}

% Headers/footers
\usepackage{fancyhdr}
\pagestyle{fancy}
\fancyhf{}
\lhead{\leftmark}
\rhead{\thepage}

% Utility packages
\usepackage{lastpage}
\usepackage[textsize=footnotesize]{todonotes} % \todo{...}

% Paragraph spacing
\setlength{\parskip}{0.5em}
\setlength{\parindent}{0pt}

% TOC and numbering depth
\setcounter{secnumdepth}{3}
\setcounter{tocdepth}{2}

% Common macros
\newcommand{\R}{\mathbb{R}}
\newcommand{\N}{\mathbb{N}}
\newcommand{\Z}{\mathbb{Z}}
\newcommand{\Q}{\mathbb{Q}}
\newcommand{\E}{\mathbb{E}}
\newcommand{\F}{\mathbb{F}}
\newcommand{\Var}{\operatorname{Var}}
\newcommand{\Cov}{\operatorname{Cov}}
\newcommand{\argmin}{\operatorname*{arg\,min}}
\newcommand{\argmax}{\operatorname*{arg\,max}}
\newcommand{\norm}[1]{\left\lVert #1 \right\rVert}
\newcommand{\abs}[1]{\left\lvert #1 \right\rvert}
\newcommand{\set}[1]{\left\{ #1 \right\}}
\newcommand{\ip}[2]{\left\langle #1,\, #2 \right\rangle}
\DeclarePairedDelimiter{\ceil}{\lceil}{\rceil}
\DeclarePairedDelimiter{\floor}{\lfloor}{\rfloor}
\newcommand{\Zmod}[1]{\zz/#1\zz}
\newcommand{\partFrac}[2]{\frac{\partial #1}{\partial #2}}

% Draft helpers
\newif\ifdraft
\draftfalse
% \drafttrue
\ifdraft
    \overfullrule=2pt
    \usepackage[inline]{showlabels}
    \usepackage{refcheck}
\fi

% Usage:
% \documentclass{article}
% \input{preamble}
% \title{Title}\author{Author}\date{\today}
% \begin{document}\maketitle
% ...
% \printbibliography % if using BibLaTeX
% \end{document} after \documentclass.
% Engine detection
\usepackage{iftex}
\ifPDFTeX
    \usepackage[T1]{fontenc}
    \usepackage[utf8]{inputenc} % harmless on newer LaTeX
    \usepackage{lmodern}
    \usepackage[english]{babel}
    \usepackage{inconsolata} % better mono font
\else
    \usepackage{fontspec}
    \defaultfontfeatures{Ligatures=TeX,Scale=MatchLowercase}
    \setmainfont{Latin Modern Roman}
    \setsansfont{Latin Modern Sans}
    \setmonofont{Latin Modern Mono}
    \usepackage{polyglossia}
    \setmainlanguage{english}
\fi

% Page geometry and layout
\usepackage[margin=1in]{geometry}
\usepackage{microtype}
\usepackage{setspace}
% \onehalfspacing % uncomment for 1.5 line spacing

% Graphics and color
\usepackage{xcolor}
\usepackage{graphicx}
\graphicspath{{figures/}}
\usepackage{tikz}
\usetikzlibrary{arrows.meta,positioning,calc,fit,patterns,decorations.pathmorphing}
\usepackage{pgfplots}
\pgfplotsset{compat=1.18}

% Tables
\usepackage{booktabs}
\usepackage{tabularx}
\usepackage{array}
\newcolumntype{Y}{>{\centering\arraybackslash}X}

% Lists
\usepackage{enumitem}
\setlist{itemsep=0.25em, topsep=0.5em}

% Math
\usepackage{amsmath,amssymb,mathtools,bm}
\allowdisplaybreaks
\usepackage{siunitx}
\sisetup{detect-all=true, per-mode=symbol, group-minimum-digits=4}

% Theorems
\usepackage{amsthm}
\numberwithin{equation}{section}
\newtheorem{theorem}{Theorem}[section]
\newtheorem{lemma}[theorem]{Lemma}
\newtheorem{proposition}[theorem]{Proposition}
\theoremstyle{definition}
\newtheorem{definition}[theorem]{Definition}
\theoremstyle{remark}
\newtheorem{remark}[theorem]{Remark}
\newtheorem{example}[theorem]{Example}

% Code listings
\usepackage{listings}
\lstdefinestyle{code}{
    basicstyle=\ttfamily\small,
    numbers=left,
    numberstyle=\scriptsize\color{gray},
    stepnumber=1,
    numbersep=8pt,
    showstringspaces=false,
    breaklines=true,
    frame=lines,
    tabsize=2,
    captionpos=b,
    keywordstyle=\color{blue!70!black}\bfseries,
    commentstyle=\color{green!50!black}\itshape,
    stringstyle=\color{orange!60!black},
}
\lstset{style=code}

% Captions and subfigures
\usepackage[labelfont=bf]{caption}
\usepackage{subcaption}

% Hyperlinks and clever references
\usepackage[hidelinks]{hyperref}
\hypersetup{
    pdftitle={},
    pdfauthor={},
    pdfsubject={},
    pdfcreator={LaTeX},
    pdfkeywords={}
}
\usepackage[capitalize,noabbrev]{cleveref}

% Citations (BibLaTeX)
\usepackage{csquotes}
\usepackage[
    backend=biber,
    style=authoryear,
    natbib=true,
    giveninits=true,
    maxbibnames=99,
    uniquename=init
]{biblatex}
% \addbibresource{references.bib}

% Headers/footers
\usepackage{fancyhdr}
\pagestyle{fancy}
\fancyhf{}
\lhead{\leftmark}
\rhead{\thepage}

% Utility packages
\usepackage{lastpage}
\usepackage[textsize=footnotesize]{todonotes} % \todo{...}

% Paragraph spacing
\setlength{\parskip}{0.5em}
\setlength{\parindent}{0pt}

% TOC and numbering depth
\setcounter{secnumdepth}{3}
\setcounter{tocdepth}{2}

% Common macros
\newcommand{\R}{\mathbb{R}}
\newcommand{\N}{\mathbb{N}}
\newcommand{\Z}{\mathbb{Z}}
\newcommand{\Q}{\mathbb{Q}}
\newcommand{\E}{\mathbb{E}}
\newcommand{\F}{\mathbb{F}}
\newcommand{\Var}{\operatorname{Var}}
\newcommand{\Cov}{\operatorname{Cov}}
\newcommand{\argmin}{\operatorname*{arg\,min}}
\newcommand{\argmax}{\operatorname*{arg\,max}}
\newcommand{\norm}[1]{\left\lVert #1 \right\rVert}
\newcommand{\abs}[1]{\left\lvert #1 \right\rvert}
\newcommand{\set}[1]{\left\{ #1 \right\}}
\newcommand{\ip}[2]{\left\langle #1,\, #2 \right\rangle}
\DeclarePairedDelimiter{\ceil}{\lceil}{\rceil}
\DeclarePairedDelimiter{\floor}{\lfloor}{\rfloor}
\newcommand{\Zmod}[1]{\zz/#1\zz}
\newcommand{\partFrac}[2]{\frac{\partial #1}{\partial #2}}

% Draft helpers
\newif\ifdraft
\draftfalse
% \drafttrue
\ifdraft
    \overfullrule=2pt
    \usepackage[inline]{showlabels}
    \usepackage{refcheck}
\fi

% Usage:
% \documentclass{article}
% % Common LaTeX preamble. Save as preamble.tex and \input{preamble} after \documentclass.
% Engine detection
\usepackage{iftex}
\ifPDFTeX
    \usepackage[T1]{fontenc}
    \usepackage[utf8]{inputenc} % harmless on newer LaTeX
    \usepackage{lmodern}
    \usepackage[english]{babel}
    \usepackage{inconsolata} % better mono font
\else
    \usepackage{fontspec}
    \defaultfontfeatures{Ligatures=TeX,Scale=MatchLowercase}
    \setmainfont{Latin Modern Roman}
    \setsansfont{Latin Modern Sans}
    \setmonofont{Latin Modern Mono}
    \usepackage{polyglossia}
    \setmainlanguage{english}
\fi

% Page geometry and layout
\usepackage[margin=1in]{geometry}
\usepackage{microtype}
\usepackage{setspace}
% \onehalfspacing % uncomment for 1.5 line spacing

% Graphics and color
\usepackage{xcolor}
\usepackage{graphicx}
\graphicspath{{figures/}}
\usepackage{tikz}
\usetikzlibrary{arrows.meta,positioning,calc,fit,patterns,decorations.pathmorphing}
\usepackage{pgfplots}
\pgfplotsset{compat=1.18}

% Tables
\usepackage{booktabs}
\usepackage{tabularx}
\usepackage{array}
\newcolumntype{Y}{>{\centering\arraybackslash}X}

% Lists
\usepackage{enumitem}
\setlist{itemsep=0.25em, topsep=0.5em}

% Math
\usepackage{amsmath,amssymb,mathtools,bm}
\allowdisplaybreaks
\usepackage{siunitx}
\sisetup{detect-all=true, per-mode=symbol, group-minimum-digits=4}

% Theorems
\usepackage{amsthm}
\numberwithin{equation}{section}
\newtheorem{theorem}{Theorem}[section]
\newtheorem{lemma}[theorem]{Lemma}
\newtheorem{proposition}[theorem]{Proposition}
\theoremstyle{definition}
\newtheorem{definition}[theorem]{Definition}
\theoremstyle{remark}
\newtheorem{remark}[theorem]{Remark}
\newtheorem{example}[theorem]{Example}

% Code listings
\usepackage{listings}
\lstdefinestyle{code}{
    basicstyle=\ttfamily\small,
    numbers=left,
    numberstyle=\scriptsize\color{gray},
    stepnumber=1,
    numbersep=8pt,
    showstringspaces=false,
    breaklines=true,
    frame=lines,
    tabsize=2,
    captionpos=b,
    keywordstyle=\color{blue!70!black}\bfseries,
    commentstyle=\color{green!50!black}\itshape,
    stringstyle=\color{orange!60!black},
}
\lstset{style=code}

% Captions and subfigures
\usepackage[labelfont=bf]{caption}
\usepackage{subcaption}

% Hyperlinks and clever references
\usepackage[hidelinks]{hyperref}
\hypersetup{
    pdftitle={},
    pdfauthor={},
    pdfsubject={},
    pdfcreator={LaTeX},
    pdfkeywords={}
}
\usepackage[capitalize,noabbrev]{cleveref}

% Citations (BibLaTeX)
\usepackage{csquotes}
\usepackage[
    backend=biber,
    style=authoryear,
    natbib=true,
    giveninits=true,
    maxbibnames=99,
    uniquename=init
]{biblatex}
% \addbibresource{references.bib}

% Headers/footers
\usepackage{fancyhdr}
\pagestyle{fancy}
\fancyhf{}
\lhead{\leftmark}
\rhead{\thepage}

% Utility packages
\usepackage{lastpage}
\usepackage[textsize=footnotesize]{todonotes} % \todo{...}

% Paragraph spacing
\setlength{\parskip}{0.5em}
\setlength{\parindent}{0pt}

% TOC and numbering depth
\setcounter{secnumdepth}{3}
\setcounter{tocdepth}{2}

% Common macros
\newcommand{\R}{\mathbb{R}}
\newcommand{\N}{\mathbb{N}}
\newcommand{\Z}{\mathbb{Z}}
\newcommand{\Q}{\mathbb{Q}}
\newcommand{\E}{\mathbb{E}}
\newcommand{\F}{\mathbb{F}}
\newcommand{\Var}{\operatorname{Var}}
\newcommand{\Cov}{\operatorname{Cov}}
\newcommand{\argmin}{\operatorname*{arg\,min}}
\newcommand{\argmax}{\operatorname*{arg\,max}}
\newcommand{\norm}[1]{\left\lVert #1 \right\rVert}
\newcommand{\abs}[1]{\left\lvert #1 \right\rvert}
\newcommand{\set}[1]{\left\{ #1 \right\}}
\newcommand{\ip}[2]{\left\langle #1,\, #2 \right\rangle}
\DeclarePairedDelimiter{\ceil}{\lceil}{\rceil}
\DeclarePairedDelimiter{\floor}{\lfloor}{\rfloor}
\newcommand{\Zmod}[1]{\zz/#1\zz}
\newcommand{\partFrac}[2]{\frac{\partial #1}{\partial #2}}

% Draft helpers
\newif\ifdraft
\draftfalse
% \drafttrue
\ifdraft
    \overfullrule=2pt
    \usepackage[inline]{showlabels}
    \usepackage{refcheck}
\fi

% Usage:
% \documentclass{article}
% \input{preamble}
% \title{Title}\author{Author}\date{\today}
% \begin{document}\maketitle
% ...
% \printbibliography % if using BibLaTeX
% \end{document}
% \title{Title}\author{Author}\date{\today}
% \begin{document}\maketitle
% ...
% \printbibliography % if using BibLaTeX
% \end{document} after \documentclass.
% Engine detection
\usepackage{iftex}
\ifPDFTeX
    \usepackage[T1]{fontenc}
    \usepackage[utf8]{inputenc} % harmless on newer LaTeX
    \usepackage{lmodern}
    \usepackage[english]{babel}
    \usepackage{inconsolata} % better mono font
\else
    \usepackage{fontspec}
    \defaultfontfeatures{Ligatures=TeX,Scale=MatchLowercase}
    \setmainfont{Latin Modern Roman}
    \setsansfont{Latin Modern Sans}
    \setmonofont{Latin Modern Mono}
    \usepackage{polyglossia}
    \setmainlanguage{english}
\fi

% Page geometry and layout
\usepackage[margin=1in]{geometry}
\usepackage{microtype}
\usepackage{setspace}
% \onehalfspacing % uncomment for 1.5 line spacing

% Graphics and color
\usepackage{xcolor}
\usepackage{graphicx}
\graphicspath{{figures/}}
\usepackage{tikz}
\usetikzlibrary{arrows.meta,positioning,calc,fit,patterns,decorations.pathmorphing}
\usepackage{pgfplots}
\pgfplotsset{compat=1.18}

% Tables
\usepackage{booktabs}
\usepackage{tabularx}
\usepackage{array}
\newcolumntype{Y}{>{\centering\arraybackslash}X}

% Lists
\usepackage{enumitem}
\setlist{itemsep=0.25em, topsep=0.5em}

% Math
\usepackage{amsmath,amssymb,mathtools,bm}
\allowdisplaybreaks
\usepackage{siunitx}
\sisetup{detect-all=true, per-mode=symbol, group-minimum-digits=4}

% Theorems
\usepackage{amsthm}
\numberwithin{equation}{section}
\newtheorem{theorem}{Theorem}[section]
\newtheorem{lemma}[theorem]{Lemma}
\newtheorem{proposition}[theorem]{Proposition}
\theoremstyle{definition}
\newtheorem{definition}[theorem]{Definition}
\theoremstyle{remark}
\newtheorem{remark}[theorem]{Remark}
\newtheorem{example}[theorem]{Example}

% Code listings
\usepackage{listings}
\lstdefinestyle{code}{
    basicstyle=\ttfamily\small,
    numbers=left,
    numberstyle=\scriptsize\color{gray},
    stepnumber=1,
    numbersep=8pt,
    showstringspaces=false,
    breaklines=true,
    frame=lines,
    tabsize=2,
    captionpos=b,
    keywordstyle=\color{blue!70!black}\bfseries,
    commentstyle=\color{green!50!black}\itshape,
    stringstyle=\color{orange!60!black},
}
\lstset{style=code}

% Captions and subfigures
\usepackage[labelfont=bf]{caption}
\usepackage{subcaption}

% Hyperlinks and clever references
\usepackage[hidelinks]{hyperref}
\hypersetup{
    pdftitle={},
    pdfauthor={},
    pdfsubject={},
    pdfcreator={LaTeX},
    pdfkeywords={}
}
\usepackage[capitalize,noabbrev]{cleveref}

% Citations (BibLaTeX)
\usepackage{csquotes}
\usepackage[
    backend=biber,
    style=authoryear,
    natbib=true,
    giveninits=true,
    maxbibnames=99,
    uniquename=init
]{biblatex}
% \addbibresource{references.bib}

% Headers/footers
\usepackage{fancyhdr}
\pagestyle{fancy}
\fancyhf{}
\lhead{\leftmark}
\rhead{\thepage}

% Utility packages
\usepackage{lastpage}
\usepackage[textsize=footnotesize]{todonotes} % \todo{...}

% Paragraph spacing
\setlength{\parskip}{0.5em}
\setlength{\parindent}{0pt}

% TOC and numbering depth
\setcounter{secnumdepth}{3}
\setcounter{tocdepth}{2}

% Common macros
\newcommand{\R}{\mathbb{R}}
\newcommand{\N}{\mathbb{N}}
\newcommand{\Z}{\mathbb{Z}}
\newcommand{\Q}{\mathbb{Q}}
\newcommand{\E}{\mathbb{E}}
\newcommand{\F}{\mathbb{F}}
\newcommand{\Var}{\operatorname{Var}}
\newcommand{\Cov}{\operatorname{Cov}}
\newcommand{\argmin}{\operatorname*{arg\,min}}
\newcommand{\argmax}{\operatorname*{arg\,max}}
\newcommand{\norm}[1]{\left\lVert #1 \right\rVert}
\newcommand{\abs}[1]{\left\lvert #1 \right\rvert}
\newcommand{\set}[1]{\left\{ #1 \right\}}
\newcommand{\ip}[2]{\left\langle #1,\, #2 \right\rangle}
\DeclarePairedDelimiter{\ceil}{\lceil}{\rceil}
\DeclarePairedDelimiter{\floor}{\lfloor}{\rfloor}
\newcommand{\Zmod}[1]{\zz/#1\zz}
\newcommand{\partFrac}[2]{\frac{\partial #1}{\partial #2}}

% Draft helpers
\newif\ifdraft
\draftfalse
% \drafttrue
\ifdraft
    \overfullrule=2pt
    \usepackage[inline]{showlabels}
    \usepackage{refcheck}
\fi

% Usage:
% \documentclass{article}
% % Common LaTeX preamble. Save as preamble.tex and % Common LaTeX preamble. Save as preamble.tex and \input{preamble} after \documentclass.
% Engine detection
\usepackage{iftex}
\ifPDFTeX
    \usepackage[T1]{fontenc}
    \usepackage[utf8]{inputenc} % harmless on newer LaTeX
    \usepackage{lmodern}
    \usepackage[english]{babel}
    \usepackage{inconsolata} % better mono font
\else
    \usepackage{fontspec}
    \defaultfontfeatures{Ligatures=TeX,Scale=MatchLowercase}
    \setmainfont{Latin Modern Roman}
    \setsansfont{Latin Modern Sans}
    \setmonofont{Latin Modern Mono}
    \usepackage{polyglossia}
    \setmainlanguage{english}
\fi

% Page geometry and layout
\usepackage[margin=1in]{geometry}
\usepackage{microtype}
\usepackage{setspace}
% \onehalfspacing % uncomment for 1.5 line spacing

% Graphics and color
\usepackage{xcolor}
\usepackage{graphicx}
\graphicspath{{figures/}}
\usepackage{tikz}
\usetikzlibrary{arrows.meta,positioning,calc,fit,patterns,decorations.pathmorphing}
\usepackage{pgfplots}
\pgfplotsset{compat=1.18}

% Tables
\usepackage{booktabs}
\usepackage{tabularx}
\usepackage{array}
\newcolumntype{Y}{>{\centering\arraybackslash}X}

% Lists
\usepackage{enumitem}
\setlist{itemsep=0.25em, topsep=0.5em}

% Math
\usepackage{amsmath,amssymb,mathtools,bm}
\allowdisplaybreaks
\usepackage{siunitx}
\sisetup{detect-all=true, per-mode=symbol, group-minimum-digits=4}

% Theorems
\usepackage{amsthm}
\numberwithin{equation}{section}
\newtheorem{theorem}{Theorem}[section]
\newtheorem{lemma}[theorem]{Lemma}
\newtheorem{proposition}[theorem]{Proposition}
\theoremstyle{definition}
\newtheorem{definition}[theorem]{Definition}
\theoremstyle{remark}
\newtheorem{remark}[theorem]{Remark}
\newtheorem{example}[theorem]{Example}

% Code listings
\usepackage{listings}
\lstdefinestyle{code}{
    basicstyle=\ttfamily\small,
    numbers=left,
    numberstyle=\scriptsize\color{gray},
    stepnumber=1,
    numbersep=8pt,
    showstringspaces=false,
    breaklines=true,
    frame=lines,
    tabsize=2,
    captionpos=b,
    keywordstyle=\color{blue!70!black}\bfseries,
    commentstyle=\color{green!50!black}\itshape,
    stringstyle=\color{orange!60!black},
}
\lstset{style=code}

% Captions and subfigures
\usepackage[labelfont=bf]{caption}
\usepackage{subcaption}

% Hyperlinks and clever references
\usepackage[hidelinks]{hyperref}
\hypersetup{
    pdftitle={},
    pdfauthor={},
    pdfsubject={},
    pdfcreator={LaTeX},
    pdfkeywords={}
}
\usepackage[capitalize,noabbrev]{cleveref}

% Citations (BibLaTeX)
\usepackage{csquotes}
\usepackage[
    backend=biber,
    style=authoryear,
    natbib=true,
    giveninits=true,
    maxbibnames=99,
    uniquename=init
]{biblatex}
% \addbibresource{references.bib}

% Headers/footers
\usepackage{fancyhdr}
\pagestyle{fancy}
\fancyhf{}
\lhead{\leftmark}
\rhead{\thepage}

% Utility packages
\usepackage{lastpage}
\usepackage[textsize=footnotesize]{todonotes} % \todo{...}

% Paragraph spacing
\setlength{\parskip}{0.5em}
\setlength{\parindent}{0pt}

% TOC and numbering depth
\setcounter{secnumdepth}{3}
\setcounter{tocdepth}{2}

% Common macros
\newcommand{\R}{\mathbb{R}}
\newcommand{\N}{\mathbb{N}}
\newcommand{\Z}{\mathbb{Z}}
\newcommand{\Q}{\mathbb{Q}}
\newcommand{\E}{\mathbb{E}}
\newcommand{\F}{\mathbb{F}}
\newcommand{\Var}{\operatorname{Var}}
\newcommand{\Cov}{\operatorname{Cov}}
\newcommand{\argmin}{\operatorname*{arg\,min}}
\newcommand{\argmax}{\operatorname*{arg\,max}}
\newcommand{\norm}[1]{\left\lVert #1 \right\rVert}
\newcommand{\abs}[1]{\left\lvert #1 \right\rvert}
\newcommand{\set}[1]{\left\{ #1 \right\}}
\newcommand{\ip}[2]{\left\langle #1,\, #2 \right\rangle}
\DeclarePairedDelimiter{\ceil}{\lceil}{\rceil}
\DeclarePairedDelimiter{\floor}{\lfloor}{\rfloor}
\newcommand{\Zmod}[1]{\zz/#1\zz}
\newcommand{\partFrac}[2]{\frac{\partial #1}{\partial #2}}

% Draft helpers
\newif\ifdraft
\draftfalse
% \drafttrue
\ifdraft
    \overfullrule=2pt
    \usepackage[inline]{showlabels}
    \usepackage{refcheck}
\fi

% Usage:
% \documentclass{article}
% \input{preamble}
% \title{Title}\author{Author}\date{\today}
% \begin{document}\maketitle
% ...
% \printbibliography % if using BibLaTeX
% \end{document} after \documentclass.
% Engine detection
\usepackage{iftex}
\ifPDFTeX
    \usepackage[T1]{fontenc}
    \usepackage[utf8]{inputenc} % harmless on newer LaTeX
    \usepackage{lmodern}
    \usepackage[english]{babel}
    \usepackage{inconsolata} % better mono font
\else
    \usepackage{fontspec}
    \defaultfontfeatures{Ligatures=TeX,Scale=MatchLowercase}
    \setmainfont{Latin Modern Roman}
    \setsansfont{Latin Modern Sans}
    \setmonofont{Latin Modern Mono}
    \usepackage{polyglossia}
    \setmainlanguage{english}
\fi

% Page geometry and layout
\usepackage[margin=1in]{geometry}
\usepackage{microtype}
\usepackage{setspace}
% \onehalfspacing % uncomment for 1.5 line spacing

% Graphics and color
\usepackage{xcolor}
\usepackage{graphicx}
\graphicspath{{figures/}}
\usepackage{tikz}
\usetikzlibrary{arrows.meta,positioning,calc,fit,patterns,decorations.pathmorphing}
\usepackage{pgfplots}
\pgfplotsset{compat=1.18}

% Tables
\usepackage{booktabs}
\usepackage{tabularx}
\usepackage{array}
\newcolumntype{Y}{>{\centering\arraybackslash}X}

% Lists
\usepackage{enumitem}
\setlist{itemsep=0.25em, topsep=0.5em}

% Math
\usepackage{amsmath,amssymb,mathtools,bm}
\allowdisplaybreaks
\usepackage{siunitx}
\sisetup{detect-all=true, per-mode=symbol, group-minimum-digits=4}

% Theorems
\usepackage{amsthm}
\numberwithin{equation}{section}
\newtheorem{theorem}{Theorem}[section]
\newtheorem{lemma}[theorem]{Lemma}
\newtheorem{proposition}[theorem]{Proposition}
\theoremstyle{definition}
\newtheorem{definition}[theorem]{Definition}
\theoremstyle{remark}
\newtheorem{remark}[theorem]{Remark}
\newtheorem{example}[theorem]{Example}

% Code listings
\usepackage{listings}
\lstdefinestyle{code}{
    basicstyle=\ttfamily\small,
    numbers=left,
    numberstyle=\scriptsize\color{gray},
    stepnumber=1,
    numbersep=8pt,
    showstringspaces=false,
    breaklines=true,
    frame=lines,
    tabsize=2,
    captionpos=b,
    keywordstyle=\color{blue!70!black}\bfseries,
    commentstyle=\color{green!50!black}\itshape,
    stringstyle=\color{orange!60!black},
}
\lstset{style=code}

% Captions and subfigures
\usepackage[labelfont=bf]{caption}
\usepackage{subcaption}

% Hyperlinks and clever references
\usepackage[hidelinks]{hyperref}
\hypersetup{
    pdftitle={},
    pdfauthor={},
    pdfsubject={},
    pdfcreator={LaTeX},
    pdfkeywords={}
}
\usepackage[capitalize,noabbrev]{cleveref}

% Citations (BibLaTeX)
\usepackage{csquotes}
\usepackage[
    backend=biber,
    style=authoryear,
    natbib=true,
    giveninits=true,
    maxbibnames=99,
    uniquename=init
]{biblatex}
% \addbibresource{references.bib}

% Headers/footers
\usepackage{fancyhdr}
\pagestyle{fancy}
\fancyhf{}
\lhead{\leftmark}
\rhead{\thepage}

% Utility packages
\usepackage{lastpage}
\usepackage[textsize=footnotesize]{todonotes} % \todo{...}

% Paragraph spacing
\setlength{\parskip}{0.5em}
\setlength{\parindent}{0pt}

% TOC and numbering depth
\setcounter{secnumdepth}{3}
\setcounter{tocdepth}{2}

% Common macros
\newcommand{\R}{\mathbb{R}}
\newcommand{\N}{\mathbb{N}}
\newcommand{\Z}{\mathbb{Z}}
\newcommand{\Q}{\mathbb{Q}}
\newcommand{\E}{\mathbb{E}}
\newcommand{\F}{\mathbb{F}}
\newcommand{\Var}{\operatorname{Var}}
\newcommand{\Cov}{\operatorname{Cov}}
\newcommand{\argmin}{\operatorname*{arg\,min}}
\newcommand{\argmax}{\operatorname*{arg\,max}}
\newcommand{\norm}[1]{\left\lVert #1 \right\rVert}
\newcommand{\abs}[1]{\left\lvert #1 \right\rvert}
\newcommand{\set}[1]{\left\{ #1 \right\}}
\newcommand{\ip}[2]{\left\langle #1,\, #2 \right\rangle}
\DeclarePairedDelimiter{\ceil}{\lceil}{\rceil}
\DeclarePairedDelimiter{\floor}{\lfloor}{\rfloor}
\newcommand{\Zmod}[1]{\zz/#1\zz}
\newcommand{\partFrac}[2]{\frac{\partial #1}{\partial #2}}

% Draft helpers
\newif\ifdraft
\draftfalse
% \drafttrue
\ifdraft
    \overfullrule=2pt
    \usepackage[inline]{showlabels}
    \usepackage{refcheck}
\fi

% Usage:
% \documentclass{article}
% % Common LaTeX preamble. Save as preamble.tex and \input{preamble} after \documentclass.
% Engine detection
\usepackage{iftex}
\ifPDFTeX
    \usepackage[T1]{fontenc}
    \usepackage[utf8]{inputenc} % harmless on newer LaTeX
    \usepackage{lmodern}
    \usepackage[english]{babel}
    \usepackage{inconsolata} % better mono font
\else
    \usepackage{fontspec}
    \defaultfontfeatures{Ligatures=TeX,Scale=MatchLowercase}
    \setmainfont{Latin Modern Roman}
    \setsansfont{Latin Modern Sans}
    \setmonofont{Latin Modern Mono}
    \usepackage{polyglossia}
    \setmainlanguage{english}
\fi

% Page geometry and layout
\usepackage[margin=1in]{geometry}
\usepackage{microtype}
\usepackage{setspace}
% \onehalfspacing % uncomment for 1.5 line spacing

% Graphics and color
\usepackage{xcolor}
\usepackage{graphicx}
\graphicspath{{figures/}}
\usepackage{tikz}
\usetikzlibrary{arrows.meta,positioning,calc,fit,patterns,decorations.pathmorphing}
\usepackage{pgfplots}
\pgfplotsset{compat=1.18}

% Tables
\usepackage{booktabs}
\usepackage{tabularx}
\usepackage{array}
\newcolumntype{Y}{>{\centering\arraybackslash}X}

% Lists
\usepackage{enumitem}
\setlist{itemsep=0.25em, topsep=0.5em}

% Math
\usepackage{amsmath,amssymb,mathtools,bm}
\allowdisplaybreaks
\usepackage{siunitx}
\sisetup{detect-all=true, per-mode=symbol, group-minimum-digits=4}

% Theorems
\usepackage{amsthm}
\numberwithin{equation}{section}
\newtheorem{theorem}{Theorem}[section]
\newtheorem{lemma}[theorem]{Lemma}
\newtheorem{proposition}[theorem]{Proposition}
\theoremstyle{definition}
\newtheorem{definition}[theorem]{Definition}
\theoremstyle{remark}
\newtheorem{remark}[theorem]{Remark}
\newtheorem{example}[theorem]{Example}

% Code listings
\usepackage{listings}
\lstdefinestyle{code}{
    basicstyle=\ttfamily\small,
    numbers=left,
    numberstyle=\scriptsize\color{gray},
    stepnumber=1,
    numbersep=8pt,
    showstringspaces=false,
    breaklines=true,
    frame=lines,
    tabsize=2,
    captionpos=b,
    keywordstyle=\color{blue!70!black}\bfseries,
    commentstyle=\color{green!50!black}\itshape,
    stringstyle=\color{orange!60!black},
}
\lstset{style=code}

% Captions and subfigures
\usepackage[labelfont=bf]{caption}
\usepackage{subcaption}

% Hyperlinks and clever references
\usepackage[hidelinks]{hyperref}
\hypersetup{
    pdftitle={},
    pdfauthor={},
    pdfsubject={},
    pdfcreator={LaTeX},
    pdfkeywords={}
}
\usepackage[capitalize,noabbrev]{cleveref}

% Citations (BibLaTeX)
\usepackage{csquotes}
\usepackage[
    backend=biber,
    style=authoryear,
    natbib=true,
    giveninits=true,
    maxbibnames=99,
    uniquename=init
]{biblatex}
% \addbibresource{references.bib}

% Headers/footers
\usepackage{fancyhdr}
\pagestyle{fancy}
\fancyhf{}
\lhead{\leftmark}
\rhead{\thepage}

% Utility packages
\usepackage{lastpage}
\usepackage[textsize=footnotesize]{todonotes} % \todo{...}

% Paragraph spacing
\setlength{\parskip}{0.5em}
\setlength{\parindent}{0pt}

% TOC and numbering depth
\setcounter{secnumdepth}{3}
\setcounter{tocdepth}{2}

% Common macros
\newcommand{\R}{\mathbb{R}}
\newcommand{\N}{\mathbb{N}}
\newcommand{\Z}{\mathbb{Z}}
\newcommand{\Q}{\mathbb{Q}}
\newcommand{\E}{\mathbb{E}}
\newcommand{\F}{\mathbb{F}}
\newcommand{\Var}{\operatorname{Var}}
\newcommand{\Cov}{\operatorname{Cov}}
\newcommand{\argmin}{\operatorname*{arg\,min}}
\newcommand{\argmax}{\operatorname*{arg\,max}}
\newcommand{\norm}[1]{\left\lVert #1 \right\rVert}
\newcommand{\abs}[1]{\left\lvert #1 \right\rvert}
\newcommand{\set}[1]{\left\{ #1 \right\}}
\newcommand{\ip}[2]{\left\langle #1,\, #2 \right\rangle}
\DeclarePairedDelimiter{\ceil}{\lceil}{\rceil}
\DeclarePairedDelimiter{\floor}{\lfloor}{\rfloor}
\newcommand{\Zmod}[1]{\zz/#1\zz}
\newcommand{\partFrac}[2]{\frac{\partial #1}{\partial #2}}

% Draft helpers
\newif\ifdraft
\draftfalse
% \drafttrue
\ifdraft
    \overfullrule=2pt
    \usepackage[inline]{showlabels}
    \usepackage{refcheck}
\fi

% Usage:
% \documentclass{article}
% \input{preamble}
% \title{Title}\author{Author}\date{\today}
% \begin{document}\maketitle
% ...
% \printbibliography % if using BibLaTeX
% \end{document}
% \title{Title}\author{Author}\date{\today}
% \begin{document}\maketitle
% ...
% \printbibliography % if using BibLaTeX
% \end{document}
% \title{Title}\author{Author}\date{\today}
% \begin{document}\maketitle
% ...
% \printbibliography % if using BibLaTeX
% \end{document}

% Graphics path
\graphicspath{{./}{gr/}}

\begin{document}
\section*{Math/CS 714: Assignment 4}

\begin{problem} \\ 
    \textbf{Beam--Warming method (4 points).}
        The Beam--Warming method for the linear advection equation $u_t + au_x =0$
        is given by
        \begin{equation}
          U^{n+1}_j = U^n_j - \frac{ak}{2h} (3 U^n_j - 4U^n_{j-1} + U^n_{j-2} ) +
          \frac{a^2 k^2}{2h^2} (U^n_j - 2U^n_{j-1} + U^n_{j-2}),
          \label{eq:beam_warming}
        \end{equation}
        where $U^n_j$ is the approximation of $u(jh,nk)$.
        \begin{enumerate}
          \item Use Taylor series to show that this method is second-order accurate.
          \item For a given plane wave solution $U^0_j = e^{ij h \xi}$, compute the
                amplification factor $g(\xi)$, and hence determine the stability
                restriction for this method.
        \end{enumerate}
\end{problem}


\begin{problem} \\ 
    \textbf{(9 points).}
        Dropping the last term in the Beam--Warming method from Eq.~\eqref{eq:beam_warming}
        gives
        \begin{equation}
          U^{n+1}_j = U^n_j - \frac{ak}{2h} (3 U^n_j - 4U^n_{j-1} + U^n_{j-2} ), \label{eq:sec_ord}
        \end{equation}
        which corresponds to forward Euler method in time, and a second-order
        one-sided derivative in space. Define $\nu = ak/h$.
        \begin{enumerate}
          \item Calculate the amplification factor $g(\xi)$ for a plane wave solution
                $U^0_j = e^{ijh\xi}$.
          \item Define $A(\xi)= |g(\xi)|^2$ and calculate a Taylor series for $A$ at
                $\xi=0$ up to second order. Using the Taylor series, explain why we
                consider the numerical scheme of Eq.~\eqref{eq:sec_ord} to be unstable
                regardless of the choice of timestep.
          \item Make two plots of $A(\xi)$ for $\nu = \nicefrac1{100}$ using two different
                axis ranges:
                \begin{itemize}
                  \item $0\le h\xi \le 2\pi$ and $0.91 \le A \le 1.01$,
                  \item $0\le h\xi \le 0.17$ and $1-10^{-6} \le A \le 1+10^{-6}$.
                \end{itemize}
          \item Write a program to simulate Eq.~\eqref{eq:sec_ord} on a periodic interval $[0,2\pi)$ using $N=40$ grid points
                and a grid spacing of $h = 2\pi/N$. Use the initial condition $u= \exp (2\sin x)$
                and $\nu =\nicefrac1{100}$. Plot the solution for $n=0,1000,2000,4000$. Define
                the root mean squared value of the solution,
                \begin{equation}
                  R(n) = \sqrt{\frac{1}{N} \sum_{j=0}^{N-1} (U^n_j)^2}.
                \end{equation}
                Make a plot of $R$ over the range from $n=0$ to $n=10000$. You should find that $R$
                does not grow over time, indicating that the method is stable.
          \item Using the discrete Fourier transform, it can be shown that an arbitrary initial
                condition on the periodic interval can be written as
                \begin{equation}
                  U^0_j = \sum_{l=0}^{N-1} \alpha_l e^{ijlh}
                \end{equation}
                for some constants $\alpha_l$. Write down an expression for the general
                solution $U^n_j$. Using your answer, explain why your result in part (d)
                does not contradict the result in part (b).
        \end{enumerate}
\end{problem}

\begin{problem} \\ 
    \textbf{Lax--Wendroff method (7 points).}
        Consider the hyperbolic conservation equation
        \begin{equation}
          q_t + [A(x) q]_x =0 \label{eq:cons}
        \end{equation}
        for a function on $q(x,t)$ on the periodic interval $[0,2\pi)$. Let
        \smash{$A(x)=2+\tfrac{4}{3}\sin x$}. Following the finite volume approach,
        divide the intervals into $m$ domains $\calC_i$ of length
        $h=\tfrac{2\pi}{m}$, for $i=\{0,1,\ldots,m-1\}$. Let $Q_i^n \approx q(
          (i+\nicefrac12)h, n\Delta t)$ be the discretized solution at the center of
        each $\calC_i$. The generalized Lax--Wendroff scheme for this equation is
        given by
        \begin{equation}
          Q^{n+1}_i = Q_i - \frac{\Delta t}{h} \left[ \calF_{i+1/2}^n - \calF_{i-1/2}^n\right]
        \end{equation}
        where the fluxes are
        \begin{equation}
          \calF_{i-1/2}^n = \frac{A_{i-1} Q_{i-1}^n + A_i Q_i^n}{2} - \frac{A_{i-1/2}\Delta t}{2h} \left[A_iQ^n_i - A_{i-1}Q_{i-1}^n\right]. \label{eq:lw}
        \end{equation}
        Here, $A_i = A( (i+\nicefrac12)h)$ and $A_{i-1/2}=A(ih)$. It can be shown
        that the solution to Eq.~\eqref{eq:cons} is time-periodic so that
        $q(x,t+T)=q(x,t)$ where $T=3\pi/\sqrt{5}$.
        \begin{enumerate}
          \item The CFL condition requires that \smash{$\Delta t \le \tfrac{h}{c}$}
                for stability. What is $c$ in this case?
          \item Implement Eq.~\eqref{eq:lw} and set $\Delta t = \tfrac{h}{3c}$. Use
                the initial condition
                \begin{equation}
                  q(x,0) = \exp\left(\sin x + \tfrac{1}{2}\sin 4x\right). \label{eq:lwic}
                \end{equation}
                For $m=512$, plot snapshots of the solution for
                $t=0,\tfrac{T}{4},\tfrac{T}{2},\tfrac{3T}{4},T$.\footnote{Since
                  multiples of $\Delta t$ do not exactly match these snapshot times, you
                  may need to make a small adjustment to the timestep.}
          \item By considering a range of $m$ (\textit{e.g.}~256 and upward) with
                the initial condition in Eq.~\eqref{eq:lwic}, calculate the $L_2$ norm
                between the numerical solution at $t=T$ and the exact answer. Determine
                the order of convergence.\footnote{When determining the order of
                  convergence, you are interested in the asymptotic properites of error
                  as $m$ gets large. You can ignore initial transients in error.}
          \item Repeat parts (b) and (c) for the initial condition
                \begin{equation}
                  q(x,0) = \max\{ \tfrac{\pi}{2}-|x-\pi|,0\}.
                \end{equation}
          \item \textbf{Optional.} By the considering the characteristics, or otherwise,
                derive the result that $q$ is time-periodic with period $T$.
        \end{enumerate}
\end{problem}


\end{document}