\documentclass{article}
\usepackage{amsmath}
\usepackage{tcolorbox}
\usepackage[margin=0.5in]{geometry} 
\usepackage{amsmath,amsthm,amssymb,amsfonts, fancyhdr, color, comment, environ, nicefrac}
\usepackage{graphicx}
% \usepackage{caption}
% \usepackage{subcaption}
\usepackage{float}
\usepackage{xcolor}
\usepackage{mdframed}
\usepackage[shortlabels]{enumitem}
\usepackage{indentfirst}
\usepackage{mathrsfs}
\usepackage{hyperref}
\graphicspath{{./}{gr/}}
\makeatletter
\newcommand*{\rom}[1]{\expandafter\@slowromancap\romannumeral #1@}
\makeatother
% Change enumerate labels to (a), (b), (c), ...
% Define a new environment for problems
\newcounter{problemCounter}
\newtcolorbox{problem}[2][]{colback=white, colframe=black, boxrule=0.5mm, arc=4mm, auto outer arc, title={\ifstrempty{#1}{Problem \stepcounter{problemCounter}\theproblemCounter}{#1}}}

% \renewcommand{\labelenumi}{\alph{enumi})}
\def\zz{{\mathbb Z}}
\def\R{{\mathbb R}}
\def\qq{{\mathbb Q}}
\def\cc{{\mathbb C}}
\def\N{{\mathbb N}}
\def\ss{{\mathbb S}}
\def\calC{{\mathcal C}}
\def\calF{{\mathcal F}}

\newcommand{\p}{\partial}
\renewcommand{\vec}[1]{\mathbf{#1}}
\newcommand{\vx}{\vec{x}}
\newcommand{\Lag}{\mathcal{L}}
\newcommand{\sep}{\,:\,}


\newtheorem{theorem}{Theorem}[section]
\newtheorem{corollary}{Corollary}[theorem]
\newtheorem{lemma}[theorem]{Lemma}
\newtcolorbox{proposition}[1][]{colback=white, colframe=blue, boxrule=0.5mm, arc=4mm, auto outer arc, title={Proposition #1}}
\newtcolorbox{definition}[1][]{colback=white, colframe=violet, boxrule=0.5mm, arc=4mm, auto outer arc, title={Definition #1}}
\newcommand{\Zmod}[1]{\zz/#1\zz}
\newcommand{\partFrac}[2]{\frac{\partial #1}{\partial #2}}

\newcommand\Mydiv[2]{%
$\strut#1$\kern.25em\smash{\raise.3ex\hbox{$\big)$}}$\mkern-8mu
        \overline{\enspace\strut#2}$}

\begin{document}
\section*{Math/CS 714: Assignment 4}

\begin{problem} \\ 
    \textbf{Beam--Warming method (4 points).}
        The Beam--Warming method for the linear advection equation $u_t + au_x =0$
        is given by
        \begin{equation}
          U^{n+1}_j = U^n_j - \frac{ak}{2h} (3 U^n_j - 4U^n_{j-1} + U^n_{j-2} ) +
          \frac{a^2 k^2}{2h^2} (U^n_j - 2U^n_{j-1} + U^n_{j-2}),
          \label{eq:beam_warming}
        \end{equation}
        where $U^n_j$ is the approximation of $u(jh,nk)$.
        \begin{enumerate}[a)]
          \item Use Taylor series to show that this method is second-order accurate.
          \item For a given plane wave solution $U^0_j = e^{ij h \xi}$, compute the
                amplification factor $g(\xi)$, and hence determine the stability
                restriction for this method.
        \end{enumerate}
\end{problem}


\begin{enumerate}[a)]
  \item For brevity, define $U^n_j := U$. Then we can Taylor expand $U$ about $(j,n)$ and we have
  \begin{align*}
    U^n_{j - 1} &= U  -h U_x + \frac{h^2}{2} U_{xx}\\
    U^n_{j - 2} &= U - 2h U_x + 2h^2 U_{xx}\\
    U^{n+1}_j &= U + k U_t + \frac{k^2}{2} U_{tt} + O(k^3)
  \end{align*}
  Substituting these into the Beam-Warming method gives
  \begin{align*}
    U + k U_t + \frac{k^2}{2} U_{tt} + O(k^3) &= U - \frac{ak}{2h} \left( 3U - 4\left( U - h U_x + \frac{h^2}{2} U_{xx}\right) + \left( U - 2h U_x + 2h^2 U_{xx}\right) \right) \\
    &+ \frac{a^2 k^2}{2h^2} \left( U - 2\left( U - h U_x + \frac{h^2}{2} U_{xx}\right) + \left( U - 2h U_x + 2h^2 U_{xx}\right) \right)  + O(h^3)\\
    &= U - \frac{ak}{2h} \left( 3U - 4U + 4h U_x - 2h^2 U_{xx} + U - 2h U_x + 2h^2 U_{xx} \right) \\
    &+ \frac{a^2 k^2}{2h^2} \left( U - 2U + 2h U_x - h^2 U_{xx} + U - 2h U_x + 2h^2 U_{xx} \right) + O(h^3)\\
    &= U - \frac{ak}{2h} (2h U_x) + \frac{a^2 k^2}{2h^2} (h^2 U_{xx}) + O(h^3)\\
    &= U - ak U_x + \frac{a^2 k^2}{2} U_{xx} + O(h^3)
  \end{align*}
  To reiterate, we have
  \[
    k U_t + \frac{k^2}{2} U_{tt} + O(k^3) = - ak U_x + \frac{a^2 k^2}{2} U_{xx} + O(h^3)
  \]
  Dividing by $k$ gives
  \[
    U_t + \frac{k}{2} U_{tt} + O(k^2) = - a U_x + \frac{a^2 k}{2} U_{xx} + O(h^2)
  \]
  Then we can use the fact that $u_t = -au_x$ and $u_{tt} = a^2 u_{xx}$, and substitue for $U_{tt} = a^2 U_{xx}$ which gives
  \[
    U_t + a U_x = O(k^2) + O(h^2)
  \]
  Thus the Beam-Warming method is second-order accurate in both space and time.

  \item Given that $U^0_j = e^{ij h \xi}$, we can calculate the amplification factor with $g(\xi) e^{ijh\xi} = U^{n+1}_j = U^n_j - \frac{ak}{2h} (3 U^n_j - 4U^n_{j-1} + U^n_{j-2} ) + \frac{a^2 k^2}{2h^2} (U^n_j - 2U^n_{j-1} + U^n_{j-2})$. Plugging in the plane wave solution and letting $\nu = \frac{ak}{h}$ we have:
  \begin{align*}
    g(\xi)e^{ijh\xi} &= e^{ijh\xi} - \frac{\nu}{2}\left( 3e^{ijh\xi} - 4e^{i(j-1)h\xi} + e^{i(j-2)h\xi}  \right) + \frac{\nu^2}{2} \left( 
      e^{ijh\xi} - 2e^{i(j-1)h\xi} + e^{i(j-1)h\xi}
    \right) \\
    \implies g(\xi) &= \frac{e^{ijh\xi} - \frac{\nu}{2}\left( 3e^{ijh\xi} - 4e^{i(j-1)h\xi} + e^{i(j-2)h\xi}  \right) + \frac{\nu^2}{2} \left( 
      e^{ijh\xi} - 2e^{i(j-1)h\xi} + e^{i(j-1)h\xi}
    \right)}{e^{ijh\xi}} \\
    &= 1 - \frac{\nu}{2} \left(
      3 - 4e^{-ih\xi} + e^{-2ih\xi}
    \right) + \frac{\nu^2}{2} \left(
      1 -2e^{-ih\xi} + e^{-2ih\xi}
    \right)
  \end{align*}
  Then, noticing that $e^x = 1 + x + \frac{x^2}{2} + \cdots$ we have: 
  \begin{align*}
    g(\xi) &= 1 - \frac{\nu}{2} \left(
      3 - 4\left( 1 - ih\xi + \frac{(ih\xi)^2}{2} \right) + \left( 1 - 2ih\xi + 2(ih\xi)^2 \right)
    \right) + \frac{\nu^2}{2} \left(
      1 - 2\left( 1 - ih\xi + \frac{(ih\xi)^2}{2} \right) + \left( 1 - 2ih\xi + 2(ih\xi)^2 \right)
    \right) \\
    &= 1 - \frac{\nu}{2} \left(
      0 + 2ih\xi - (ih\xi)^2
    \right) + \frac{\nu^2}{2} \left(
      0 + 0 + (ih\xi)^2
    \right) + O((h\xi)^3)\\
    &= 1 - i\nu h\xi + \frac{\nu}{2}(h\xi)^2 + \frac{\nu^2}{2}(h\xi)^2 + O((h\xi)^3)\\
    &= 1 - i\nu h\xi + \frac{\nu(1+\nu)}{2}(h\xi)^2 + O((h\xi)^3)
  \end{align*}
  Then we can compute $|g(\xi)|^2$:
  \begin{align*}
    |g(\xi)|^2 &= \left( 1 + \frac{\nu(1+\nu)}{2}(h\xi)^2 + O((h\xi)^3) \right)^2 + \left( -\nu h\xi + O((h\xi)^3) \right)^2 \\
    &= 1 + \nu(1+\nu)(h\xi)^2 + \nu^2 (h\xi)^2 + O((h\xi)^3)\\
    &= 1 + \nu(1+2\nu)(h\xi)^2 + O((h\xi)^3)
  \end{align*}
  For stability, we require that $|g(\xi)|^2 \leq 1$ for all $\xi$. Therefore we need $\nu(1+2\nu) \leq 0$. This gives the stability restriction of $- \frac{1}{2} \leq \nu \leq 0$.
\end{enumerate}


\begin{problem} \\ 
    \textbf{(9 points).}
        Dropping the last term in the Beam--Warming method from Eq.~\eqref{eq:beam_warming}
        gives
        \begin{equation}
          U^{n+1}_j = U^n_j - \frac{ak}{2h} (3 U^n_j - 4U^n_{j-1} + U^n_{j-2} ), \label{eq:sec_ord}
        \end{equation}
        which corresponds to forward Euler method in time, and a second-order
        one-sided derivative in space. Define $\nu = ak/h$.
        \begin{enumerate}[a)]
          \item Calculate the amplification factor $g(\xi)$ for a plane wave solution
                $U^0_j = e^{ijh\xi}$.
          \item Define $A(\xi)= |g(\xi)|^2$ and calculate a Taylor series for $A$ at
                $\xi=0$ up to second order. Using the Taylor series, explain why we
                consider the numerical scheme of Eq.~\eqref{eq:sec_ord} to be unstable
                regardless of the choice of timestep.
          \item Make two plots of $A(\xi)$ for $\nu = \nicefrac1{100}$ using two different
                axis ranges:
                \begin{itemize}
                  \item $0\le h\xi \le 2\pi$ and $0.91 \le A \le 1.01$,
                  \item $0\le h\xi \le 0.17$ and $1-10^{-6} \le A \le 1+10^{-6}$.
                \end{itemize}
          \item Write a program to simulate Eq.~\eqref{eq:sec_ord} on a periodic interval $[0,2\pi)$ using $N=40$ grid points
                and a grid spacing of $h = 2\pi/N$. Use the initial condition $u= \exp (2\sin x)$
                and $\nu =\nicefrac1{100}$. Plot the solution for $n=0,1000,2000,4000$. Define
                the root mean squared value of the solution,
                \begin{equation}
                  R(n) = \sqrt{\frac{1}{N} \sum_{j=0}^{N-1} (U^n_j)^2}.
                \end{equation}
                Make a plot of $R$ over the range from $n=0$ to $n=10000$. You should find that $R$
                does not grow over time, indicating that the method is stable.
          \item Using the discrete Fourier transform, it can be shown that an arbitrary initial
                condition on the periodic interval can be written as
                \begin{equation}
                  U^0_j = \sum_{l=0}^{N-1} \alpha_l e^{ijlh}
                \end{equation}
                for some constants $\alpha_l$. Write down an expression for the general
                solution $U^n_j$. Using your answer, explain why your result in part (d)
                does not contradict the result in part (b).
        \end{enumerate}
\end{problem}



\begin{enumerate}[a)]
  \item We can calculate the amplification factor with $g(\xi) e^{ijh\xi} = U^{n+1}_j = U^n_j - \frac{ak}{2h} (3 U^n_j - 4U^n_{j-1} + U^n_{j-2} )$. Plugging in the plane wave solution and simplifying as in the previous problem gives $g(\xi) = 1 - \frac{\nu}{2} (3 - 4e^{-ih\xi} + e^{-2ih\xi})$.
  \item Taking $A(\xi) = |g(\xi)|^2$ and expanding in a Taylor series about $\xi=0$, we can use the fact that $e^{x} = 1 + x + \frac{x^2}{2} + \dots$ to find that 
  \begin{align*}
    A(\xi) &= \left| 1 - \frac{\nu}{2} (3 - 4(1 - ih\xi + \frac{(ih\xi)^2}{2}) + (1 - 2ih\xi + 2(ih\xi)^2)) \right|^2 \\
    &= \left| 1 - \frac{\nu}{2} (0 + 2ih\xi - (ih\xi)^2) \right|^2 \\
  \end{align*}
  Thus we have
  \begin{align*}
    A(\xi) &= \left( 1 + \frac{\nu}{2}(h\xi)^2 \right)^2 + \left( -\nu h\xi \right)^2 \\
    &= 1 + \nu (h\xi)^2 + \frac{\nu^2}{4} (h\xi)^4 + \nu^2 (h\xi)^2 \\
    &= 1 + \nu(1+\nu)(h\xi)^2 + O((h\xi)^4)
  \end{align*}
  Since $\nu(1+\nu) > 0$ for all $\nu > 0$, we have that $A(\xi) > 1$ for sufficiently small but nonzero $\xi$. This indicates that the method is unstable regardless of the choice of timestep.
  


  \item In Figure \ref{fig:Q2C1} and Firgure \ref{fig:Q2C2} we have the two plots of $A(\xi)$ for $\nu = \frac{1}{100}$.
  \begin{figure}[H]
    \centering
    \includegraphics[width=1.0\textwidth]{gr/HW4Q2C1.png}
    \caption{$A(\xi)$ for $0\le h\xi \le 2\pi$ and $0.91 \le A \le 1.01$.}
    \label{fig:Q2C1}
  \end{figure}
  \begin{figure}[H]
    \centering
    \includegraphics[width=1.0\textwidth]{gr/HW4Q2C2.png}
    \caption{$A(\xi)$ for $0\le h\xi \le 0.17$ and $1-10^{-6} \le A \le 1+10^{-6}$.}
    \label{fig:Q2C2}
  \end{figure}

  \newpage
  \item See Figure \ref{fig:Q2D1} for the plots of the solution at $n=0,1000,2000,4000$ and Figure \ref{fig:Q2D2} for the plot of $R$ over the range from $n=0$ to $n=10000$.
  
  \begin{figure}[H]
    \centering
    \includegraphics[width=1.0\textwidth]{gr/HW4Q2D1.png}
    \caption{Plots of the solution at $n=0,1000,2000,4000$.}
    \label{fig:Q2D1}
  \end{figure}
  \begin{figure}[H]
    \centering
    \includegraphics[width=1.0\textwidth]{gr/HW4Q2D2.png}
    \caption{Plot of $R$ over the range from $n=0$ to $n=10000$.}
    \label{fig:Q2D2}
  \end{figure}

  \item The discrete Fourier representation gives the general solution. If the initial data has Fourier coefficients \(\{\alpha_l\}_{l=0}^{N-1}\) then for the allowed discrete wave-numbers
  \[
  h\xi_l=\frac{2\pi l}{N}\qquad(l=0,\dots,N-1)
  \]
  we have the mode-wise evolution
  \[
  U^n_j=\sum_{l=0}^{N-1}\alpha_l\bigl(g(\xi_l)\bigr)^n e^{ijh\xi_l}.
  \]
  Thus each Fourier mode evolves independently:
  \[
  U^n_j=\sum_{l=0}^{N-1}\alpha_l\bigl(g(\xi_l)\bigr)^n e^{ijh\xi_l},
  \]
  so mode \(l\) is multiplied by \(g(\xi_l)\) at every time step.

  The calculation in part (b) gives the local expansion near \(\xi=0\)
  \[
  A(\xi)=|g(\xi)|^2=1+C(h\xi)^2+\cdots\qquad(C>0),
  \]
  so for the continuous problem there are arbitrarily small nonzero wavenumbers \(\xi\) with \(A(\xi)>1\). That is the von Neumann instability: some very long wavelengths are amplified.

  A discrete periodic grid, however, only admits the wavenumbers
  \[
  h\xi_l=\frac{2\pi l}{N},\qquad l=0,\dots,N-1,
  \]
  so the smallest nonzero wavenumber is \(h\xi_1=2\pi/N\). If \(N\) is not large (or for the chosen \(\nu\)), none of the sampled \(\xi_l\) lie in the asymptotically unstable region, and all sampled modes satisfy \(|g(\xi_l)|\le1\); the numerical solution then remains bounded. Even when a sampled mode has \(A(\xi_l)>1\), the excess \(A-1\) can be extremely small, so growth per step is tiny and may be imperceptible over the finite number of time steps used in the simulation (since a mode grows like \(A^{n/2}\approx\exp\big(\tfrac{n}{2}(A-1)\big)\) for small \(A-1\)). 

  So we have that in part (b) we showed the instability in the continuous/von Neumann sense (existence of arbitrarily long unstable wavelengths), whereas the finite discrete simulation can appear stable because only a discrete set of wavenumbers is present and those sampled may not include the unstable, very-small-\(\xi\) modes or may grow too slowly to be observed.

\end{enumerate}




\begin{problem} \\ 
    \textbf{Lax--Wendroff method (7 points).}
        Consider the hyperbolic conservation equation
        \begin{equation}
          q_t + [A(x) q]_x =0 \label{eq:cons}
        \end{equation}
        for a function on $q(x,t)$ on the periodic interval $[0,2\pi)$. Let
        \smash{$A(x)=2+\tfrac{4}{3}\sin x$}. Following the finite volume approach,
        divide the intervals into $m$ domains $\calC_i$ of length
        $h=\tfrac{2\pi}{m}$, for $i=\{0,1,\ldots,m-1\}$. Let $Q_i^n \approx q(
          (i+\nicefrac12)h, n\Delta t)$ be the discretized solution at the center of
        each $\calC_i$. The generalized Lax--Wendroff scheme for this equation is
        given by
        \begin{equation}
          Q^{n+1}_i = Q_i - \frac{\Delta t}{h} \left[ \calF_{i+1/2}^n - \calF_{i-1/2}^n\right]
        \end{equation}
        where the fluxes are
        \begin{equation}
          \calF_{i-1/2}^n = \frac{A_{i-1} Q_{i-1}^n + A_i Q_i^n}{2} - \frac{A_{i-1/2}\Delta t}{2h} \left[A_iQ^n_i - A_{i-1}Q_{i-1}^n\right]. \label{eq:lw}
        \end{equation}
        Here, $A_i = A( (i+\nicefrac12)h)$ and $A_{i-1/2}=A(ih)$. It can be shown
        that the solution to Eq.~\eqref{eq:cons} is time-periodic so that
        $q(x,t+T)=q(x,t)$ where $T=3\pi/\sqrt{5}$.
        \begin{enumerate}[a)]
          \item The CFL condition requires that \smash{$\Delta t \le \tfrac{h}{c}$}
                for stability. What is $c$ in this case?
          \item Implement Eq.~\eqref{eq:lw} and set $\Delta t = \tfrac{h}{3c}$. Use
                the initial condition
                \begin{equation}
                  q(x,0) = \exp\left(\sin x + \tfrac{1}{2}\sin 4x\right). \label{eq:lwic}
                \end{equation}
                For $m=512$, plot snapshots of the solution for
                $t=0,\tfrac{T}{4},\tfrac{T}{2},\tfrac{3T}{4},T$.\footnote{Since
                  multiples of $\Delta t$ do not exactly match these snapshot times, you
                  may need to make a small adjustment to the timestep.}
          \item By considering a range of $m$ (\textit{e.g.}~256 and upward) with
                the initial condition in Eq.~\eqref{eq:lwic}, calculate the $L_2$ norm
                between the numerical solution at $t=T$ and the exact answer. Determine
                the order of convergence.\footnote{When determining the order of
                  convergence, you are interested in the asymptotic properites of error
                  as $m$ gets large. You can ignore initial transients in error.}
          \item Repeat parts (b) and (c) for the initial condition
                \begin{equation}
                  q(x,0) = \max\{ \tfrac{\pi}{2}-|x-\pi|,0\}.
                \end{equation}
          \item \textbf{Optional.} By the considering the characteristics, or otherwise,
                derive the result that $q$ is time-periodic with period $T$.
        \end{enumerate}
\end{problem}

\begin{enumerate}[a)]
  \item The wave (characteristic) speed is $A(x)$, so 
  \[
  c=\max_{x\in[0,2\pi)}|A(x)|=\max_{x}\Big(2+\tfrac{4}{3}\sin x\Big)=2+\tfrac{4}{3}=\tfrac{10}{3}.
  \]
  (Here $A(x)\ge2-\tfrac{4}{3}=\tfrac{2}{3}>0$, hence the maximum absolute value is $10/3$.) 
  Thus the CFL condition is $\Delta t\le h/c= \tfrac{3h}{10}$.

  \item See Figure \ref{fig:Q3B} for the plots of the solution at $t=0,\tfrac{T}{4},\tfrac{T}{2},\tfrac{3T}{4},T$ with $m=512$.
  \begin{figure}[H]
    \centering
    \includegraphics[width=1.0\textwidth]{gr/HW4Q3B.png}
    \caption{Plots of the solution at $t=0,\tfrac{T}{4},\tfrac{T}{2},\tfrac{3T}{4},T$ with $m=512$.}
    \label{fig:Q3B}
  \end{figure}

  \item See Figure \ref{fig:Q3C} for the log-log plot of the $L_2$ norm between the numerical solution at $t=T$ and the exact answer for various $m$. The slope of the line of best fit is approximately $1$ indicating first-order convergence.
  \begin{figure}[H]
    \centering
    \includegraphics[width=0.7\textwidth]{gr/HW4Q3C.png}
    \caption{Log-log plot of the $L_2$ norm between the numerical solution at $t=T$ and the exact answer for various $m$.}
    \label{fig:Q3C}
  \end{figure}

  \item See Figure \ref{fig:Q3D1} for the plots of the solution at $t=0,\tfrac{T}{4},\tfrac{T}{2},\tfrac{3T}{4},T$ with $m=512$ and Figure \ref{fig:Q3D2} for the log-log plot of the $L_2$ norm between the numerical solution at $t=T$ and the exact answer for various $m$. The slope of the line of best fit is approximately $1$ indicating first-order convergence.
  % Side by side figures
  \begin{figure}[H]
    \centering
    \begin{minipage}{0.5\textwidth}
      \centering
      \includegraphics[width=.9\textwidth]{gr/HW4Q3D1.png}
      \caption{Plots of the solutions with $m=512$.}
      \label{fig:Q3D1}
    \end{minipage}%
    \begin{minipage}{0.5\textwidth}
      \centering
      \includegraphics[width=0.9\textwidth]{gr/HW4Q3D2.png}
      \caption{Log-log plot of the $L_2$ norm between the numerical solution at $t=T$ and the exact answer for various $m$.}
      \label{fig:Q3D2}
    \end{minipage}
  \end{figure}

    

\end{enumerate}



\end{document}